\documentclass[11pt]{book}
\setlength{\paperheight}{11in}
\setlength{\paperwidth}{8.5in}
\setlength{\topmargin}{0pt}
\setlength{\voffset}{-28.908pt}
\setlength{\evensidemargin}{36.135000000000005pt}
\setlength{\oddsidemargin}{36.135000000000005pt}
\setlength{\textwidth}{433.62pt}
\setlength{\textheight}{645.0097499999999pt}
\setlength{\headheight}{13.5pt}
\setlength{\headsep}{14.10125pt}
\setlength{\footskip}{.5in}
\DeclareTextSymbol{\textsquarebracketleft}{EU1}{91}
\DeclareTextSymbol{\textsquarebracketright}{EU1}{93}
\usepackage{xltxtra}
\usepackage{setspace}
\usepackage[normalem]{ulem}
\usepackage{color}
\usepackage{colortbl}
\usepackage{tabularx}
\usepackage{longtable}
\usepackage{lscape}
\usepackage{multirow}
\usepackage{booktabs}
\usepackage{calc}
\usepackage{fancyhdr}
\usepackage{fontspec}
\usepackage{hyperref}
\hypersetup{colorlinks=true, citecolor=black, filecolor=black, linkcolor=black, urlcolor=blue, bookmarksopen=true, pdfauthor={Hugh Paterson III}, pdfcreator={XLingPaper version 2.31.0 (www.xlingpaper.org)}, pdftitle={Text input issues for under-resourced languages: with special notes on Eastern Dan}}
\fancypagestyle{frontmattertitle}
{\fancyhf{}
\renewcommand{\headrulewidth}{0pt}
\renewcommand{\footrulewidth}{0pt}
}\fancypagestyle{frontmatterfirstpage}
{\fancyhf{}
\fancyfoot[C]{{\XLingPaperTimesZNewZRomanFontFamily{\thepage}}}
\renewcommand{\headrulewidth}{0pt}
\renewcommand{\footrulewidth}{0pt}
}\fancypagestyle{frontmatter}
{\fancyhf{}
\fancyfoot[CE]{{\XLingPaperTimesZNewZRomanFontFamily{\thepage}}}
\fancyfoot[CO]{{\XLingPaperTimesZNewZRomanFontFamily{\thepage}}}
\renewcommand{\headrulewidth}{0pt}
\renewcommand{\footrulewidth}{0pt}
}\fancypagestyle{bodyfirstpage}
{\fancyhf{}
\fancyfoot[C]{{\XLingPaperTimesZNewZRomanFontFamily{\thepage}}}
\renewcommand{\headrulewidth}{0pt}
\renewcommand{\footrulewidth}{0pt}
}\fancypagestyle{body}
{\fancyhf{}
\fancyfoot[CE]{{\XLingPaperTimesZNewZRomanFontFamily{\thepage}}}
\fancyfoot[CO]{{\XLingPaperTimesZNewZRomanFontFamily{\thepage}}}
\renewcommand{\headrulewidth}{0pt}
\renewcommand{\footrulewidth}{0pt}
}\setmainfont{Charis SIL}
\font\MainFont="Charis SIL" at 11pt
\newfontfamily{\XLingPaperCambriaZMathFontFamily}{Cambria Math}
\newfontfamily{\XLingPaperSymbolaFontFamily}{Symbola}
\newfontfamily{\XLingPaperJomolhariFontFamily}{Jomolhari}
\newfontfamily{\XLingPaperSimSunFontFamily}{SimSun}
\newfontfamily{\XLingPaperTavirajFontFamily}{Taviraj}
\newfontfamily{\XLingPaperLateefFontFamily}{Lateef}
\newfontfamily{\XLingPaperCharisZSILFontFamily}{Charis SIL}
\newfontfamily{\XLingPaperCambriaFontFamily}{Cambria}
\newfontfamily{\XLingPaperCharisZSILZSmallZCapsFontFamily}{Charis SIL Small Caps}
\newfontfamily{\XLingPaperDoulosZSILFontFamily}{Doulos SIL}
\newfontfamily{\XLingPaperDejaVuZSerifFontFamily}{DejaVu Serif}
\newfontfamily{\XLingPaperMonospacedFontFamily}{Courier New}
\newfontfamily{\XLingPaperKeyboardZKeysExZExpandedFontFamily}{Keyboard KeysEx Expanded}
\newfontfamily{\XLingPaperaaZQWERTZXTastenFontFamily}{aa QWERTZ-Tasten}
\newfontfamily{\XLingPaperLinuxZBiolinumZKeyboardFontFamily}{Linux Biolinum Keyboard}
\newfontfamily{\XLingPaperxkcdZScriptFontFamily}{xkcd Script}
\newfontfamily{\XLingPaperTavirajZBlackFontFamily}{Taviraj Black}
\newfontfamily{\XLingPaperDejaVuZSansFontFamily}{DejaVu Sans}
\newfontfamily{\XLingPaperTimesZNewZRomanFontFamily}{Times New Roman}
\setlength{\parindent}{.3in}
\doublespacing
\catcode`^^^^200b=\active
\def^^^^200b{\hskip0pt}
\let\origdoublepage\cleardoublepage
\newcommand{\clearemptydoublepage}{\clearpage{\pagestyle{empty}\origdoublepage}}\renewenvironment{quotation}{\list{}{\leftmargin=.25in\rightmargin=.25in}\item[]{}}{\endlist}
\clubpenalty=10000
\widowpenalty=10000\newlength{\XLingPaperabbrbaselineskip}
\begin{document}
\baselineskip=\glueexpr\baselineskip + 0pt plus 2pt minus 1pt\relax
\renewcommand{\footnotesize}{\fontsize{9}{10.799999999999999}\selectfont }
\newlength{\leveloneindent}
\newlength{\levelonewidth}
\newlength{\leveltwoindent}
\newlength{\leveltwowidth}
\newlength{\levelthreeindent}
\newlength{\levelthreewidth}
\newlength{\levelfourindent}
\newlength{\levelfourwidth}
\newlength{\levelfiveindent}
\newlength{\levelfivewidth}
\newlength{\levelsixindent}
\newlength{\levelsixwidth}
\newdimen\XLingPapertempdim
                \newdimen\XLingPapertempdimletter
                \newcommand{\XLingPapertableofcontents}{\immediate\openout8 = \jobname.toc\relax
\immediate\write8{<toc>}}
\newcommand{\XLingPaperaddtocontents}[1]{\write8{<tocline ref="#1" page="\thepage"/>}}
\newcommand{\XLingPaperendtableofcontents}{\immediate\write8{</toc>}\closeout8\relax
}
\newcommand{\XLingPaperdotfill}{\leaders\hbox{$\mathsurround 0pt\mkern 4.5 mu\hbox{.}\mkern 4.5 mu$}\hfill}
\newcommand{\XLingPaperdottedtocline}[4]{
\newdimen\XLingPapertempdim
\vskip0pt plus .2pt{
\leftskip#1\relax% left glue for indent
\rightskip\XLingPapertocrmarg% right glue for for right margin
\parfillskip-\rightskip% so can go into margin if need be???
\parindent#1\relax
\interlinepenalty10000
\leavevmode
\XLingPapertempdim#2\relax% numwidth
\advance\leftskip\XLingPapertempdim\null\nobreak\hskip-\leftskip{#3}\nobreak
\XLingPaperdotfill\nobreak
\hbox to\XLingPaperpnumwidth{\hfil\normalfont\normalcolor#4}
\par}}
\newlength{\XLingPaperpnumwidth}
\newlength{\XLingPapertocrmarg}
\setlength{\XLingPaperpnumwidth}{2.55em}\setlength{\XLingPapertocrmarg}{\XLingPaperpnumwidth+1em}
\newsavebox{\XLingPapertempbox}
\newlength{\XLingPapertemplen}
\newlength{\XLingPaperavailabletablewidth}
\newlength{\XLingPapertableminwidth}
\newlength{\XLingPapertablemaxwidth}
\newlength{\XLingPapertablewidthminustableminwidth}
\newlength{\XLingPapertablemaxwidthminusminwidth}
\newlength{\XLingPapertablewidthratio}
\newlength{\XLingPapermincola}\newlength{\XLingPapermaxcola}\newlength{\XLingPapercolawidth}
\newlength{\XLingPapermincolb}\newlength{\XLingPapermaxcolb}\newlength{\XLingPapercolbwidth}
\newlength{\XLingPapermincolc}\newlength{\XLingPapermaxcolc}\newlength{\XLingPapercolcwidth}
\newlength{\XLingPapermincold}\newlength{\XLingPapermaxcold}\newlength{\XLingPapercoldwidth}
\newlength{\XLingPapermincole}\newlength{\XLingPapermaxcole}\newlength{\XLingPapercolewidth}
\newlength{\XLingPapermincolf}\newlength{\XLingPapermaxcolf}\newlength{\XLingPapercolfwidth}
\newlength{\XLingPapermincolg}\newlength{\XLingPapermaxcolg}\newlength{\XLingPapercolgwidth}
\newlength{\XLingPapermincolh}\newlength{\XLingPapermaxcolh}\newlength{\XLingPapercolhwidth}
\newlength{\XLingPapermincoli}\newlength{\XLingPapermaxcoli}\newlength{\XLingPapercoliwidth}
\newlength{\XLingPapermincolj}\newlength{\XLingPapermaxcolj}\newlength{\XLingPapercoljwidth}
\newlength{\XLingPapermincolk}\newlength{\XLingPapermaxcolk}\newlength{\XLingPapercolkwidth}
\newlength{\XLingPapermincoll}\newlength{\XLingPapermaxcoll}\newlength{\XLingPapercollwidth}
\newlength{\XLingPapermincolm}\newlength{\XLingPapermaxcolm}\newlength{\XLingPapercolmwidth}
\newlength{\XLingPapermincoln}\newlength{\XLingPapermaxcoln}\newlength{\XLingPapercolnwidth}
\newlength{\XLingPapermincolo}\newlength{\XLingPapermaxcolo}\newlength{\XLingPapercolowidth}
\newlength{\XLingPapermincolp}\newlength{\XLingPapermaxcolp}\newlength{\XLingPapercolpwidth}
\newlength{\XLingPapermincolq}\newlength{\XLingPapermaxcolq}\newlength{\XLingPapercolqwidth}
\newlength{\XLingPapermincolr}\newlength{\XLingPapermaxcolr}\newlength{\XLingPapercolrwidth}
\newlength{\XLingPapermincols}\newlength{\XLingPapermaxcols}\newlength{\XLingPapercolswidth}
\newlength{\XLingPapermincolt}\newlength{\XLingPapermaxcolt}\newlength{\XLingPapercoltwidth}
\newlength{\XLingPapermincolu}\newlength{\XLingPapermaxcolu}\newlength{\XLingPapercoluwidth}
\newlength{\XLingPapermincolv}\newlength{\XLingPapermaxcolv}\newlength{\XLingPapercolvwidth}
\newlength{\XLingPapermincolw}\newlength{\XLingPapermaxcolw}\newlength{\XLingPapercolwwidth}
\newlength{\XLingPapermincolx}\newlength{\XLingPapermaxcolx}\newlength{\XLingPapercolxwidth}
\newlength{\XLingPapermincoly}\newlength{\XLingPapermaxcoly}\newlength{\XLingPapercolywidth}
\newlength{\XLingPapermincolz}\newlength{\XLingPapermaxcolz}\newlength{\XLingPapercolzwidth}
\newcommand{\XLingPaperlongestcell}[2]{
\ifdim#1>#2
#2=#1
\fi
}
\newcommand{\XLingPaperminmaxcellincolumn}[5]{
\savebox{\XLingPapertempbox}{#3}
\settowidth{\XLingPapertemplen}{\usebox{\XLingPapertempbox}}
\addtolength{\XLingPapertemplen}{#5}
\XLingPaperlongestcell{\XLingPapertemplen}{#4}
\setlength{\XLingPapertemplen}{\widthof{#1}}
\addtolength{\XLingPapertemplen}{#5}
\ifdim\XLingPapertemplen>#4
\XLingPapertemplen=#4
\fi
\XLingPaperlongestcell{\XLingPapertemplen}{#2}}
\newcommand{\XLingPapersetcolumnwidth}[4]{
\ifdim\XLingPapertableminwidth>\XLingPaperavailabletablewidth
#1=#2
\else
\ifdim\XLingPapertableminwidth=\XLingPaperavailabletablewidth
#1=#2
\else
\ifdim\XLingPapertablemaxwidth<\XLingPaperavailabletablewidth
#1=#3
\else
\setlength{\XLingPapertemplen}{#3-#2}
\divide\XLingPapertemplen by 100
\multiply\XLingPapertemplen by \XLingPapertablewidthratio
#1=#2
\addtolength{#1}{\XLingPapertemplen}
\addtolength{#1}{#4}
\fi
\fi
\fi
}
\newcommand{\XLingPapercalculatetablewidthratio}{
\setlength{\XLingPapertablewidthminustableminwidth}{\XLingPaperavailabletablewidth-\XLingPapertableminwidth}
\setlength{\XLingPapertablemaxwidthminusminwidth}{\XLingPapertablemaxwidth-\XLingPapertableminwidth}
\ifdim\XLingPapertablemaxwidthminusminwidth=0sp
\XLingPapertablemaxwidthminusminwidth=10000sp
\fi
\setlength{\XLingPapertablewidthratio}{\XLingPapertablewidthminustableminwidth}
\divide\XLingPapertablemaxwidthminusminwidth by 100
\divide\XLingPapertablewidthratio by \XLingPapertablemaxwidthminusminwidth}
\newlength{\XLingPaperlistinexampleindent}
\newlength{\XLingPaperisocodewidth}\setlength{\XLingPaperlistinexampleindent}{0in+ 2.75em}
\newlength{\XLingPaperlistitemindent}
\newlength{\XLingPaperbulletlistitemwidth}\settowidth{\XLingPaperbulletlistitemwidth}{•\ }\newlength{\XLingPapersingledigitlistitemwidth}
\settowidth{\XLingPapersingledigitlistitemwidth}{8.\ }\newlength{\XLingPaperdoubledigitlistitemwidth}
\settowidth{\XLingPaperdoubledigitlistitemwidth}{88.\ }\newlength{\XLingPapertripledigitlistitemwidth}
\settowidth{\XLingPapertripledigitlistitemwidth}{888.\ }\newlength{\XLingPapersingleletterlistitemwidth}
\settowidth{\XLingPapersingleletterlistitemwidth}{m.\ }\newlength{\XLingPaperdoubleletterlistitemwidth}
\settowidth{\XLingPaperdoubleletterlistitemwidth}{mm.\ }\newlength{\XLingPapertripleletterlistitemwidth}
\settowidth{\XLingPapertripleletterlistitemwidth}{mmm.\ }\newlength{\XLingPaperromanviilistitemwidth}
\settowidth{\XLingPaperromanviilistitemwidth}{vii.\ }\newlength{\XLingPaperromanviiilistitemwidth}
\settowidth{\XLingPaperromanviiilistitemwidth}{viii.\ }\newlength{\XLingPaperromanxviiilistitemwidth}
\settowidth{\XLingPaperromanxviiilistitemwidth}{xviii.\ }\newlength{\XLingPaperspacewidth}
\settowidth{\XLingPaperspacewidth}{\ }\newlength{\XLingPapersingledigitlistofwidth}
\settowidth{\XLingPapersingledigitlistofwidth}{8.  }\newlength{\XLingPaperdoubledigitlistofwidth}
\settowidth{\XLingPaperdoubledigitlistofwidth}{88.  }\newlength{\XLingPapertripledigitlistofwidth}
\settowidth{\XLingPapertripledigitlistofwidth}{888.  }
\newcommand{\XLingPaperneedspace}[1]{\penalty-100\begingroup
\newdimen{\XLingPaperspaceneeded}
\newdimen{\XLingPaperspaceavailable}
\setlength{\XLingPaperspaceneeded}{#1}%
\XLingPaperspaceavailable\pagegoal \advance\XLingPaperspaceavailable-\pagetotal
\ifdim \XLingPaperspaceneeded>\XLingPaperspaceavailable
\ifdim \XLingPaperspaceavailable>0pt
\vfil
\fi
\break
\fi\endgroup}
\newcommand{\XLingPaperlistitem}[4]{
\newdimen\XLingPapertempdim
\vskip0pt plus .2pt{
\leftskip#1\relax% left glue for indent
\parindent#1\relax
\interlinepenalty10000
\leavevmode
\XLingPapertempdim#2\relax% label width
\advance\leftskip\XLingPapertempdim\null\nobreak\hskip-\leftskip\hbox to\XLingPapertempdim{\hfil\normalfont\normalcolor#3\ }{#4}\nobreak
\par}}
\newcommand{\XLingPaperblockquote}[4]{\vskip#3{
\leftskip#1\relax% left glue for indent\relax
\rightskip#1\relax% right glue for indent
\interlinepenalty10000
\leavevmode\hskip-\parindent{#2}\nobreak
}\vskip#4}
\newcommand{\XLingPaperexample}[5]{
\newdimen\XLingPapertempdim
\vskip0pt plus .2pt{
\leftskip#1\relax% left glue for indent
\hspace*{#1}\relax
\rightskip#2\relax% right glue for indent
\interlinepenalty10000
\leavevmode
\XLingPapertempdim#3\relax% example number width
\advance\leftskip\XLingPapertempdim\null\nobreak\hskip-\leftskip\hbox to\XLingPapertempdim{\normalfont\normalcolor#4\hfil}{#5}\nobreak
\par}}
\newcommand{\XLingPaperexampleintable}[5]{
\newdimen\XLingPapertempdim
\leftskip#1\relax% left glue for indent
\hspace*{#1}\relax
\rightskip#2\relax% right glue for indent
\interlinepenalty10000
\leavevmode
\XLingPapertempdim#3\relax% example number width
\hbox to\XLingPapertempdim{\normalfont\normalcolor#4\hfil}{
\begin{tabular}
[t]{@{}l@{}}#5\end{tabular}
}\nobreak
}
\newcommand{\XLingPaperfree}[2]{\vskip0pt plus .2pt{
\leftskip#1\relax% left glue for indent
\parindent#1\relax
\interlinepenalty10000
\leavevmode{#2}\nobreak
\par}}
\newcommand{\XLingPaperlistinterlinear}[5]{\vskip0pt plus .2pt{\hspace*{#1}\hspace*{#2}
\XLingPapertempdimletter#3\relax% letter width
\advance\leftskip\XLingPapertempdimletter\null\nobreak\hskip-\leftskip\hspace*{-.3em}\hbox to\XLingPapertempdimletter{\normalfont\normalcolor#4\ \hfil}{#5}\nobreak
\par}}
\newcommand{\XLingPaperlistinterlinearintable}[5]{
\XLingPapertempdimletter#3\relax% letter width
\hspace*{-.3em}\hbox to\XLingPapertempdimletter{\normalfont\normalcolor#4\ \hfil}{
\begin{tabular}
[t]{@{}l@{}}#5\end{tabular}
}\nobreak
}

\newlength{\XLingPaperexamplefreeindent}\setlength{\XLingPaperexamplefreeindent}{-.3 em}\newskip\XLingPaperinterwordskip
\XLingPaperinterwordskip=6.66666pt plus 3.33333pt minus 2.22222pt
\def\XLingPaperintspace{\hskip\XLingPaperinterwordskip}
\def\XLingPaperraggedright{\rightskip=0pt plus1fil\pretolerance=10000}\raggedbottom
\pagestyle{fancy}
\begin{MainFont}
\XLingPapertableofcontents\pagenumbering{roman}
\pagestyle{frontmattertitle}\pagestyle{frontmattertitle}{\clearpage
\vspace*{1.25in}\XLingPaperneedspace{3\baselineskip}\noindent
\fontsize{14}{16.8}\selectfont \textbf{{\centering
Text input issues for under-resourced languages: with special notes on Eastern Dan\protect\\}}}\par{}
{\clearpage
\vspace*{1.25in}\XLingPaperneedspace{3\baselineskip}\noindent
\fontsize{14}{16.8}\selectfont \textbf{{\centering
Text input issues for under-resourced languages: with special notes on Eastern Dan\protect\\}}}\par{}
{\XLingPaperneedspace{3\baselineskip}\noindent
\textit{{\centering
Hugh Paterson III\protect\\}}}\par{}
{\XLingPaperneedspace{3\baselineskip}\noindent
\textit{{\centering
University of North Dakota\protect\\}}}\par{}
{\XLingPaperneedspace{3\baselineskip}sil.linguist@gmail.com}\par{}
\clearpage
\pagestyle{frontmatter}\thispagestyle{frontmatterfirstpage}\clearpage
\thispagestyle{frontmatterfirstpage}{\vspace*{.65in}\noindent
\raisebox{\baselineskip}[0pt]{\pdfbookmark[1]{Contents}{rXLingPapContents}}\raisebox{\baselineskip}[0pt]{\protect\hypertarget{rXLingPapContents}{}}\XLingPaperneedspace{3\baselineskip}\noindent
{\MakeUppercase{{\protect\centering
Contents\protect\\}}}\markboth{Contents}{Contents}
\XLingPaperaddtocontents{rXLingPapContents}}\penalty10000\par{}
\vspace{10.8pt}{\singlespacing
\hyperlink{rXLingPapPreface1}{\XLingPaperdottedtocline{0pt}{0pt}{D͙̖̖̋̋ȉ͖̭̌͛̂â̖͓͙̗͛̽c̖̭̖̭̽̋̌́r͕̭͓͐̌̀î̭̖̽̏̋̀̌t͔͓̽͐̽͐̏i̬̗͖̬̽͐̽ć̬̖̖̌́̀î̖͕͙̭͕̏́́s͖̗̖͓̏̌m̭͙̗̌͛}{vi}
}}{\singlespacing
\hyperlink{rXLingPapPreface2}{\XLingPaperdottedtocline{0pt}{0pt}{Tables, Figures, and Abbreviations}{vii}
}}{\singlespacing
\hyperlink{rXLingPapPreface3}{\XLingPaperdottedtocline{0pt}{0pt}{Typographical Conventions}{xv}
}}{\singlespacing
\hyperlink{rXLingPapAcknowledgements}{\XLingPaperdottedtocline{0pt}{0pt}{Acknowledgements}{xvii}
}}{\singlespacing
\hyperlink{rXLingPapAbstract0}{\XLingPaperdottedtocline{0pt}{0pt}{Abstract}{xviii}
}}{\singlespacing{}\noindent{}CHAPTER\par{}}\settowidth{\levelonewidth}{1. \ }{\singlespacing
\hyperlink{c1-Introduction}{\XLingPaperdottedtocline{.3in}{\levelonewidth}{1.  \ Introduction}{1}
}}\settowidth{\leveltwoindent}{{1. }\ }\settowidth{\leveltwowidth}{{1.1. }\thinspace\thinspace}\addtolength{\leveltwoindent}{.3in}{\singlespacing
\hyperlink{sKBD-underresourced-languages}{\XLingPaperdottedtocline{\leveltwoindent}{\leveltwowidth}{{1.1. } The keyboard layout in language vitality and the under-resourced language}{7}
}}\settowidth{\leveltwoindent}{{1. }\ }\settowidth{\leveltwowidth}{{1.2. }\thinspace\thinspace}\addtolength{\leveltwoindent}{.3in}{\singlespacing
\hyperlink{sCongnitiveFriction}{\XLingPaperdottedtocline{\leveltwoindent}{\leveltwowidth}{{1.2. } Cognitive Friction and the Linguistic Enterprise}{9}
}}\settowidth{\leveltwoindent}{{1. }\ {1.2. }\ }\settowidth{\leveltwowidth}{{1.2.1. }\thinspace\thinspace}\addtolength{\leveltwoindent}{.3in}{\singlespacing
\hyperlink{sPlaceOfDigitalCommunication}{\XLingPaperdottedtocline{\leveltwoindent}{\leveltwowidth}{{1.2.1. } The {\XLingPaperCharisZSILFontFamily{\textit{place}}} of digital communication}{13}
}}\settowidth{\leveltwoindent}{{1. }\ {1.2. }\ }\settowidth{\leveltwowidth}{{1.2.2. }\thinspace\thinspace}\addtolength{\leveltwoindent}{.3in}{\singlespacing
\hyperlink{sCognitiveLoad}{\XLingPaperdottedtocline{\leveltwoindent}{\leveltwowidth}{{1.2.2. } Cognitive Load}{16}
}}\settowidth{\leveltwoindent}{{1. }\ }\settowidth{\leveltwowidth}{{1.3. }\thinspace\thinspace}\addtolength{\leveltwoindent}{.3in}{\singlespacing
\hyperlink{sDefinitions}{\XLingPaperdottedtocline{\leveltwoindent}{\leveltwowidth}{{1.3. } Linguistic orientation}{17}
}}\settowidth{\leveltwoindent}{{1. }\ {1.3. }\ }\settowidth{\leveltwowidth}{{1.3.1. }\thinspace\thinspace}\addtolength{\leveltwoindent}{.3in}{\singlespacing
\hyperlink{sTerms}{\XLingPaperdottedtocline{\leveltwoindent}{\leveltwowidth}{{1.3.1. } Tell of characters}{17}
}}\settowidth{\leveltwoindent}{{1. }\ {1.3. }\ }\settowidth{\leveltwowidth}{{1.3.2. }\thinspace\thinspace}\addtolength{\leveltwoindent}{.3in}{\singlespacing
\hyperlink{sTemp}{\XLingPaperdottedtocline{\leveltwoindent}{\leveltwowidth}{{1.3.2. } To Delete}{29}
}}\settowidth{\leveltwoindent}{{1. }\ {1.3. }\ }\settowidth{\leveltwowidth}{{1.3.3. }\thinspace\thinspace}\addtolength{\leveltwoindent}{.3in}{\singlespacing
\hyperlink{sPhonology}{\XLingPaperdottedtocline{\leveltwoindent}{\leveltwowidth}{{1.3.3. } Phonology}{29}
}}\settowidth{\leveltwoindent}{{1. }\ {1.3. }\ }\settowidth{\leveltwowidth}{{1.3.4. }\thinspace\thinspace}\addtolength{\leveltwoindent}{.3in}{\singlespacing
\hyperlink{sTone}{\XLingPaperdottedtocline{\leveltwoindent}{\leveltwowidth}{{1.3.4. } Tone}{30}
}}\settowidth{\leveltwoindent}{{1. }\ {1.3. }\ }\settowidth{\leveltwowidth}{{1.3.5. }\thinspace\thinspace}\addtolength{\leveltwoindent}{.3in}{\singlespacing
\hyperlink{sFunctionalLoad}{\XLingPaperdottedtocline{\leveltwoindent}{\leveltwowidth}{{1.3.5. } Functional Load}{30}
}}\settowidth{\leveltwoindent}{{1. }\ {1.3. }\ }\settowidth{\leveltwowidth}{{1.3.6. }\thinspace\thinspace}\addtolength{\leveltwoindent}{.3in}{\singlespacing
\hyperlink{sOrthography}{\XLingPaperdottedtocline{\leveltwoindent}{\leveltwowidth}{{1.3.6. } Orthography}{31}
}}\settowidth{\leveltwoindent}{{1. }\ {1.3. }\ }\settowidth{\leveltwowidth}{{1.3.7. }\thinspace\thinspace}\addtolength{\leveltwoindent}{.3in}{\singlespacing
\hyperlink{sTypingIsLingusitics}{\XLingPaperdottedtocline{\leveltwoindent}{\leveltwowidth}{{1.3.7. } Typing as a linguistic activity}{33}
}}\settowidth{\levelonewidth}{2. \ }{\singlespacing
\hyperlink{c2CS}{\XLingPaperdottedtocline{.3in}{\levelonewidth}{2.  \ Multiple perspectives on the keyboard layout problem}{34}
}}\settowidth{\leveltwoindent}{{2. }\ }\settowidth{\leveltwowidth}{{2.1. }\thinspace\thinspace}\addtolength{\leveltwoindent}{.3in}{\singlespacing
\hyperlink{sKeyboardsGenreal}{\XLingPaperdottedtocline{\leveltwoindent}{\leveltwowidth}{{2.1. } Keyboards}{36}
}}\settowidth{\leveltwoindent}{{2. }\ }\settowidth{\leveltwowidth}{{2.2. }\thinspace\thinspace}\addtolength{\leveltwoindent}{.3in}{\singlespacing
\hyperlink{sCS}{\XLingPaperdottedtocline{\leveltwoindent}{\leveltwowidth}{{2.2. } Keyboards in Computer Science}{41}
}}\settowidth{\leveltwoindent}{{2. }\ }\settowidth{\leveltwowidth}{{2.3. }\thinspace\thinspace}\addtolength{\leveltwoindent}{.3in}{\singlespacing
\hyperlink{sAsseingKeyboards}{\XLingPaperdottedtocline{\leveltwoindent}{\leveltwowidth}{{2.3. } Current state of assessing comparing keyboard efficiency}{45}
}}\settowidth{\leveltwoindent}{{2. }\ }\settowidth{\leveltwowidth}{{2.4. }\thinspace\thinspace}\addtolength{\leveltwoindent}{.3in}{\singlespacing
\hyperlink{sLingusiticActivity}{\XLingPaperdottedtocline{\leveltwoindent}{\leveltwowidth}{{2.4. } Typing as a linguistic activity}{47}
}}\settowidth{\leveltwoindent}{{2. }\ }\settowidth{\leveltwowidth}{{2.5. }\thinspace\thinspace}\addtolength{\leveltwoindent}{.3in}{\singlespacing
\hyperlink{sMappingLtoT}{\XLingPaperdottedtocline{\leveltwoindent}{\leveltwowidth}{{2.5. } Mapping linguistics to technology: orthographies in keyboards}{48}
}}\settowidth{\leveltwoindent}{{2. }\ {2.5. }\ }\settowidth{\leveltwowidth}{{2.5.1. }\thinspace\thinspace}\addtolength{\leveltwoindent}{.3in}{\singlespacing
\hyperlink{sDigraphs}{\XLingPaperdottedtocline{\leveltwoindent}{\leveltwowidth}{{2.5.1. } Digraphs}{48}
}}\settowidth{\leveltwoindent}{{2. }\ {2.5. }\ }\settowidth{\leveltwowidth}{{2.5.2. }\thinspace\thinspace}\addtolength{\leveltwoindent}{.3in}{\singlespacing
\hyperlink{sTonalMelodies}{\XLingPaperdottedtocline{\leveltwoindent}{\leveltwowidth}{{2.5.2. } Tonal Melodies}{50}
}}\settowidth{\levelonewidth}{3. \ }{\singlespacing
\hyperlink{sEDWritingSystem}{\XLingPaperdottedtocline{.3in}{\levelonewidth}{3.  \ Eastern Dan writing system}{52}
}}\settowidth{\leveltwoindent}{{3. }\ }\settowidth{\leveltwowidth}{{3.1. }\thinspace\thinspace}\addtolength{\leveltwoindent}{.3in}{\singlespacing
\hyperlink{sEasternDanBib}{\XLingPaperdottedtocline{\leveltwoindent}{\leveltwowidth}{{3.1. } Previous work on Eastern Dan}{57}
}}\settowidth{\leveltwoindent}{{3. }\ }\settowidth{\leveltwowidth}{{3.2. }\thinspace\thinspace}\addtolength{\leveltwoindent}{.3in}{\singlespacing
\hyperlink{sEasternDanPhonology}{\XLingPaperdottedtocline{\leveltwoindent}{\leveltwowidth}{{3.2. } Phonology}{57}
}}\settowidth{\leveltwoindent}{{3. }\ {3.2. }\ }\settowidth{\leveltwowidth}{{3.2.1. }\thinspace\thinspace}\addtolength{\leveltwoindent}{.3in}{\singlespacing
\hyperlink{sSylabis}{\XLingPaperdottedtocline{\leveltwoindent}{\leveltwowidth}{{3.2.1. } Syllabics}{57}
}}\settowidth{\leveltwoindent}{{3. }\ {3.2. }\ }\settowidth{\leveltwowidth}{{3.2.2. }\thinspace\thinspace}\addtolength{\leveltwoindent}{.3in}{\singlespacing
\hyperlink{sConsonants}{\XLingPaperdottedtocline{\leveltwoindent}{\leveltwowidth}{{3.2.2. } Consonants}{58}
}}\settowidth{\leveltwoindent}{{3. }\ {3.2. }\ }\settowidth{\leveltwowidth}{{3.2.3. }\thinspace\thinspace}\addtolength{\leveltwoindent}{.3in}{\singlespacing
\hyperlink{sVowels}{\XLingPaperdottedtocline{\leveltwoindent}{\leveltwowidth}{{3.2.3. } Vowels}{58}
}}\settowidth{\leveltwoindent}{{3. }\ {3.2. }\ }\settowidth{\leveltwowidth}{{3.2.4. }\thinspace\thinspace}\addtolength{\leveltwoindent}{.3in}{\singlespacing
\hyperlink{sTone-Marking}{\XLingPaperdottedtocline{\leveltwoindent}{\leveltwowidth}{{3.2.4. } Patterns of pitch}{60}
}}\settowidth{\leveltwoindent}{{3. }\ }\settowidth{\leveltwowidth}{{3.3. }\thinspace\thinspace}\addtolength{\leveltwoindent}{.3in}{\singlespacing
\hyperlink{sEasternDanOrthography}{\XLingPaperdottedtocline{\leveltwoindent}{\leveltwowidth}{{3.3. } Sound visualizations}{60}
}}\settowidth{\leveltwoindent}{{3. }\ {3.3. }\ }\settowidth{\leveltwowidth}{{3.3.1. }\thinspace\thinspace}\addtolength{\leveltwoindent}{.3in}{\singlespacing
\hyperlink{sSegmental}{\XLingPaperdottedtocline{\leveltwoindent}{\leveltwowidth}{{3.3.1. } Segmental}{61}
}}\settowidth{\leveltwoindent}{{3. }\ {3.3. }\ }\settowidth{\leveltwowidth}{{3.3.2. }\thinspace\thinspace}\addtolength{\leveltwoindent}{.3in}{\singlespacing
\hyperlink{sSupraSegmental}{\XLingPaperdottedtocline{\leveltwoindent}{\leveltwowidth}{{3.3.2. } Suprasegmental}{63}
}}\settowidth{\leveltwoindent}{{3. }\ }\settowidth{\leveltwowidth}{{3.4. }\thinspace\thinspace}\addtolength{\leveltwoindent}{.3in}{\singlespacing
\hyperlink{sUnitsAndOrders}{\XLingPaperdottedtocline{\leveltwoindent}{\leveltwowidth}{{3.4. } Units and orders}{64}
}}\settowidth{\leveltwoindent}{{3. }\ {3.4. }\ }\settowidth{\leveltwowidth}{{3.4.1. }\thinspace\thinspace}\addtolength{\leveltwoindent}{.3in}{\singlespacing
\hyperlink{sAlphabet}{\XLingPaperdottedtocline{\leveltwoindent}{\leveltwowidth}{{3.4.1. } Alphabet}{64}
}}\settowidth{\leveltwoindent}{{3. }\ {3.4. }\ }\settowidth{\leveltwowidth}{{3.4.2. }\thinspace\thinspace}\addtolength{\leveltwoindent}{.3in}{\singlespacing
\hyperlink{sFunctional-Units}{\XLingPaperdottedtocline{\leveltwoindent}{\leveltwowidth}{{3.4.2. } Functional units}{67}
}}\settowidth{\leveltwoindent}{{3. }\ }\settowidth{\leveltwowidth}{{3.5. }\thinspace\thinspace}\addtolength{\leveltwoindent}{.3in}{\singlespacing
\hyperlink{Numbers}{\XLingPaperdottedtocline{\leveltwoindent}{\leveltwowidth}{{3.5. } Numbers}{69}
}}\settowidth{\leveltwoindent}{{3. }\ }\settowidth{\leveltwowidth}{{3.6. }\thinspace\thinspace}\addtolength{\leveltwoindent}{.3in}{\singlespacing
\hyperlink{Punctuation}{\XLingPaperdottedtocline{\leveltwoindent}{\leveltwowidth}{{3.6. } Punctuation}{71}
}}\settowidth{\leveltwoindent}{{3. }\ }\settowidth{\leveltwowidth}{{3.7. }\thinspace\thinspace}\addtolength{\leveltwoindent}{.3in}{\singlespacing
\hyperlink{sInternetUnicodeCharacters}{\XLingPaperdottedtocline{\leveltwoindent}{\leveltwowidth}{{3.7. } Internet characters}{75}
}}\settowidth{\leveltwoindent}{{3. }\ }\settowidth{\leveltwowidth}{{3.8. }\thinspace\thinspace}\addtolength{\leveltwoindent}{.3in}{\singlespacing
\hyperlink{sCasing}{\XLingPaperdottedtocline{\leveltwoindent}{\leveltwowidth}{{3.8. } Casing rules}{76}
}}\settowidth{\leveltwoindent}{{3. }\ }\settowidth{\leveltwowidth}{{3.9. }\thinspace\thinspace}\addtolength{\leveltwoindent}{.3in}{\singlespacing
\hyperlink{sWordBreaks}{\XLingPaperdottedtocline{\leveltwoindent}{\leveltwowidth}{{3.9. } Word breaking behavior}{77}
}}\settowidth{\leveltwoindent}{{3. }\ }\settowidth{\leveltwowidth}{{3.10. }\thinspace\thinspace}\addtolength{\leveltwoindent}{.3in}{\singlespacing
\hyperlink{sLoanWords}{\XLingPaperdottedtocline{\leveltwoindent}{\leveltwowidth}{{3.10. } Loan Words and borrowed letters}{78}
}}\settowidth{\leveltwoindent}{{3. }\ {3.10. }\ }\settowidth{\leveltwowidth}{{3.10.1. }\thinspace\thinspace}\addtolength{\leveltwoindent}{.3in}{\singlespacing
\hyperlink{sAuxiliaryCharacters}{\XLingPaperdottedtocline{\leveltwoindent}{\leveltwowidth}{{3.10.1. } Auxiliary characters}{80}
}}\settowidth{\levelonewidth}{4. \ }{\singlespacing
\hyperlink{cMethods}{\XLingPaperdottedtocline{.3in}{\levelonewidth}{4.  \ Methodology}{83}
}}\settowidth{\leveltwoindent}{{4. }\ }\settowidth{\leveltwowidth}{{4.1. }\thinspace\thinspace}\addtolength{\leveltwoindent}{.3in}{\singlespacing
\hyperlink{sCorpus}{\XLingPaperdottedtocline{\leveltwoindent}{\leveltwowidth}{{4.1. } Corpus creation and description}{84}
}}\settowidth{\leveltwoindent}{{4. }\ }\settowidth{\leveltwowidth}{{4.2. }\thinspace\thinspace}\addtolength{\leveltwoindent}{.3in}{\singlespacing
\hyperlink{sKeyboards}{\XLingPaperdottedtocline{\leveltwoindent}{\leveltwowidth}{{4.2. } Keyboards}{86}
}}\settowidth{\leveltwoindent}{{4. }\ }\settowidth{\leveltwowidth}{{4.3. }\thinspace\thinspace}\addtolength{\leveltwoindent}{.3in}{\singlespacing
\hyperlink{sSoftware}{\XLingPaperdottedtocline{\leveltwoindent}{\leveltwowidth}{{4.3. } Software}{87}
}}\settowidth{\levelonewidth}{5. \ }{\singlespacing
\hyperlink{cExperiments}{\XLingPaperdottedtocline{.3in}{\levelonewidth}{5.  \ Experiments}{89}
}}\settowidth{\levelonewidth}{6. \ }{\singlespacing
\hyperlink{cResults}{\XLingPaperdottedtocline{.3in}{\levelonewidth}{6.  \ Results}{91}
}}\settowidth{\levelonewidth}{7. \ }{\singlespacing
\hyperlink{cDiscussion}{\XLingPaperdottedtocline{.3in}{\levelonewidth}{7.  \ Discussion}{93}
}}\settowidth{\leveltwoindent}{{7. }\ }\settowidth{\leveltwowidth}{{7.1. }\thinspace\thinspace}\addtolength{\leveltwoindent}{.3in}{\singlespacing
\hyperlink{sConclusion}{\XLingPaperdottedtocline{\leveltwoindent}{\leveltwowidth}{{7.1. } Conclusion}{93}
}}\settowidth{\leveltwoindent}{{7. }\ }\settowidth{\leveltwowidth}{{7.2. }\thinspace\thinspace}\addtolength{\leveltwoindent}{.3in}{\singlespacing
\hyperlink{sApplications}{\XLingPaperdottedtocline{\leveltwoindent}{\leveltwowidth}{{7.2. } Implications for future work}{94}
}}\settowidth{\leveltwoindent}{{7. }\ {7.2. }\ }\settowidth{\leveltwowidth}{{7.2.1. }\thinspace\thinspace}\addtolength{\leveltwoindent}{.3in}{\singlespacing
\hyperlink{sLiveType}{\XLingPaperdottedtocline{\leveltwoindent}{\leveltwowidth}{{7.2.1. } Implications for live typing experiments}{94}
}}\settowidth{\leveltwoindent}{{7. }\ {7.2. }\ }\settowidth{\leveltwowidth}{{7.2.2. }\thinspace\thinspace}\addtolength{\leveltwoindent}{.3in}{\singlespacing
\hyperlink{sAlphabets}{\XLingPaperdottedtocline{\leveltwoindent}{\leveltwowidth}{{7.2.2. } Some thoughts on Alphabets}{94}
}}\settowidth{\leveltwoindent}{{7. }\ {7.2. }\ }\settowidth{\leveltwowidth}{{7.2.3. }\thinspace\thinspace}\addtolength{\leveltwoindent}{.3in}{\singlespacing
\hyperlink{sSome}{\XLingPaperdottedtocline{\leveltwoindent}{\leveltwowidth}{{7.2.3. } Some thoughts on}{95}
}}\settowidth{\levelthreeindent}{{7. }\ {7.2. }\ {7.2.3. }\ }\settowidth{\levelthreewidth}{{7.2.3.1. }\thinspace\thinspace}\addtolength{\levelthreeindent}{.3in}{\singlespacing
\hyperlink{s}{\XLingPaperdottedtocline{\levelthreeindent}{\levelthreewidth}{{7.2.3.1. } Text input and security}{95}
}}{\singlespacing
\hyperlink{rXLingPapAppendiciesPage}{\XLingPaperdottedtocline{0pt}{0pt}{APPENDICES}{96}
}}{\singlespacing
\hyperlink{rXLingPapReferences}{\XLingPaperdottedtocline{0pt}{0pt}{References}{131}
}}\clearpage
\thispagestyle{frontmatterfirstpage}{\vspace*{.65in}\noindent
\raisebox{\baselineskip}[0pt]{\pdfbookmark[1]{D͙̖̖̋̋ȉ͖̭̌͛̂â̖͓͙̗͛̽c̖̭̖̭̽̋̌́r͕̭͓͐̌̀î̭̖̽̏̋̀̌t͔͓̽͐̽͐̏i̬̗͖̬̽͐̽ć̬̖̖̌́̀î̖͕͙̭͕̏́́s͖̗̖͓̏̌m̭͙̗̌͛}{rXLingPapPreface1}}\raisebox{\baselineskip}[0pt]{\protect\hypertarget{rXLingPapPreface1}{}}\XLingPaperneedspace{3\baselineskip}\noindent
{\MakeUppercase{{\protect\centering
D͙̖̖̋̋ȉ͖̭̌͛̂â̖͓͙̗͛̽c̖̭̖̭̽̋̌́r͕̭͓͐̌̀î̭̖̽̏̋̀̌t͔͓̽͐̽͐̏i̬̗͖̬̽͐̽ć̬̖̖̌́̀î̖͕͙̭͕̏́́s͖̗̖͓̏̌m̭͙̗̌͛\protect\\}}}\markboth{D͙̖̖̋̋ȉ͖̭̌͛̂â̖͓͙̗͛̽c̖̭̖̭̽̋̌́r͕̭͓͐̌̀î̭̖̽̏̋̀̌t͔͓̽͐̽͐̏i̬̗͖̬̽͐̽ć̬̖̖̌́̀î̖͕͙̭͕̏́́s͖̗̖͓̏̌m̭͙̗̌͛}{D͙̖̖̋̋ȉ͖̭̌͛̂â̖͓͙̗͛̽c̖̭̖̭̽̋̌́r͕̭͓͐̌̀î̭̖̽̏̋̀̌t͔͓̽͐̽͐̏i̬̗͖̬̽͐̽ć̬̖̖̌́̀î̖͕͙̭͕̏́́s͖̗̖͓̏̌m̭͙̗̌͛}
\XLingPaperaddtocontents{rXLingPapPreface1}}\penalty10000\par{}
\vspace{10.8pt}\XLingPaperblockquote{.25in}{{\singlespacing
\vspace{-1.3\baselineskip}"Clearly, reading is inherent in most writing. All writing is a putting down on paper of one's own thoughts or the thoughts of others; by its very nature, writing should entail reading." From The role of writing in developmental reading by Sandra Stotsky in Journal of Reading, January, 1982. \par{}– as quoted in READ 17.2:21\par{}}}{\baselineskip}{\baselineskip}\XLingPaperblockquote{.25in}{{\singlespacing
\vspace{-1.3\baselineskip}The irony of literacy pedagogy is that it is usually people writing about reading, to people who are already readers, not people reading about writing from people who are already writers; but in either case, the nature of both literatures is to propagate an ideal that persons of letters are the esteemed essence of a society. \par{}Whether knowledge of letters is truly the basis of an esteemed society is perhaps a matter of debate. But what is certain is that most, if not all people, have encountered letters in some form or fashion. And as objects in our cultures, some sort of connection with them is imposed upon us. We can choose to embrace them, to ignore them, or to have an affinity with some letters and not others. Our choice is formed based on the perceived costs and relative benefits that those letters afford us. Part of the cost of using letters is never spoken. It is only felt.\par{} And while we used to feel this pain alone as if we were in some sort of anguish and required rite of passage as we endeavor to express ourselves, now with the interconnectedness of mobile devices and social media upon us we are essentially screamed at by the words of others as they emplore us to suffer with them and write them back. \par{}– Hugh Paterson III\par{}}}{\baselineskip}{\baselineskip}\clearpage
\thispagestyle{frontmatterfirstpage}{\vspace*{.65in}\noindent
\raisebox{\baselineskip}[0pt]{\pdfbookmark[1]{Tables, Figures, and Abbreviations}{rXLingPapPreface2}}\raisebox{\baselineskip}[0pt]{\protect\hypertarget{rXLingPapPreface2}{}}\XLingPaperneedspace{3\baselineskip}\noindent
{\MakeUppercase{{\protect\centering
Tables, Figures, and Abbreviations\protect\\}}}\markboth{Tables, Figures, and Abbreviations}{Tables, Figures, and Abbreviations}
\XLingPaperaddtocontents{rXLingPapPreface2}}\penalty10000\par{}
\vspace{10.8pt}\noindent{}{Table }\hfill{}Page\par{}
{\singlespacing
{\singlespacing
\hyperlink{ntLanguages}{\XLingPaperdottedtocline{0pt}{\XLingPapersingledigitlistofwidth{}}{1.  Languages referenced and their \hyperlink{gtISO639-3}{{\textit{ISO 639-3}}} codes}{xi}
}}}{\singlespacing
{\singlespacing
\hyperlink{ntKhmer}{\XLingPaperdottedtocline{0pt}{\XLingPapersingledigitlistofwidth{}}{2.  Construction of the Khmer autonym}{12}
}}}{\singlespacing
{\singlespacing
\hyperlink{ntPressAndHold}{\XLingPaperdottedtocline{0pt}{\XLingPapersingledigitlistofwidth{}}{3.  Press and hold on Apple products}{14}
}}}{\singlespacing
{\singlespacing
\hyperlink{ntThaiTone}{\XLingPaperdottedtocline{0pt}{\XLingPapersingledigitlistofwidth{}}{4.  Thai Tone Marks}{23}
}}}{\singlespacing
{\singlespacing
\hyperlink{ntEthopic}{\XLingPaperdottedtocline{0pt}{\XLingPapersingledigitlistofwidth{}}{5.  Diacritics in the Ge'ez script}{24}
}}}{\singlespacing
{\singlespacing
\hyperlink{ntISOJISANSI}{\XLingPaperdottedtocline{0pt}{\XLingPapersingledigitlistofwidth{}}{6.  Different physical keybard layouts}{40}
}}}{\singlespacing
{\singlespacing
\hyperlink{ntHandPositions}{\XLingPaperdottedtocline{0pt}{\XLingPapersingledigitlistofwidth{}}{7.  Bad hand positions}{40}
}}}{\singlespacing
{\singlespacing
\hyperlink{ntKeyTypes}{\XLingPaperdottedtocline{0pt}{\XLingPapersingledigitlistofwidth{}}{8.  Key types on Physical Keyboards}{42}
}}}{\singlespacing
{\singlespacing
\hyperlink{ntKeypressTypes}{\XLingPaperdottedtocline{0pt}{\XLingPapersingledigitlistofwidth{}}{9.  }{43}
}}}{\singlespacing
{\singlespacing
\hyperlink{ntKeyboardOptimization}{\XLingPaperdottedtocline{0pt}{\XLingPaperdoubledigitlistofwidth{}}{10.  Keyboard Optimization Methods}{44}
}}}{\singlespacing
{\singlespacing
\hyperlink{ntHistoryOfDanOrthography}{\XLingPaperdottedtocline{0pt}{\XLingPaperdoubledigitlistofwidth{}}{11.  Evolutionary stages of the Eastern Dan writing system}{55}
}}}{\singlespacing
{\singlespacing
\hyperlink{Consonants}{\XLingPaperdottedtocline{0pt}{\XLingPaperdoubledigitlistofwidth{}}{12.  List of consonants in Eastern Dan}{58}
}}}{\singlespacing
{\singlespacing
\hyperlink{ntVowelsOral}{\XLingPaperdottedtocline{0pt}{\XLingPaperdoubledigitlistofwidth{}}{13.  List of oral vowels in Eastern Dan}{58}
}}}{\singlespacing
{\singlespacing
\hyperlink{ntVowelsNasal}{\XLingPaperdottedtocline{0pt}{\XLingPaperdoubledigitlistofwidth{}}{14.  List of nasal vowels in Eastern Dan}{59}
}}}{\singlespacing
{\singlespacing
\hyperlink{ntVowelsDipthongs}{\XLingPaperdottedtocline{0pt}{\XLingPaperdoubledigitlistofwidth{}}{15.  List of diphthongs vowels in Eastern Dan}{60}
}}}{\singlespacing
{\singlespacing
\hyperlink{ntOrthographyVowels}{\XLingPaperdottedtocline{0pt}{\XLingPaperdoubledigitlistofwidth{}}{16.  Functional units organized by phoneme}{61}
}}}{\singlespacing
{\singlespacing
\hyperlink{ntLanguagesWithTonePunctuation}{\XLingPaperdottedtocline{0pt}{\XLingPaperdoubledigitlistofwidth{}}{17.  List of languages reported to use punctuation marks according to the Ivorian tradition.}{63}
}}}{\singlespacing
{\singlespacing
\hyperlink{tEasternDanCharacters}{\XLingPaperdottedtocline{0pt}{\XLingPaperdoubledigitlistofwidth{}}{18.  List of letters used in Eastern Dan}{66}
}}}{\singlespacing
{\singlespacing
\hyperlink{tEasternDanCharactersdiacritics}{\XLingPaperdottedtocline{0pt}{\XLingPaperdoubledigitlistofwidth{}}{19.  List of characters whose composition includes diacritics}{67}
}}}{\singlespacing
{\singlespacing
\hyperlink{ntFuntionalUnitsList}{\XLingPaperdottedtocline{0pt}{\XLingPaperdoubledigitlistofwidth{}}{20.  List of functional units}{68}
}}}{\singlespacing
{\singlespacing
\hyperlink{NumberCharacters}{\XLingPaperdottedtocline{0pt}{\XLingPaperdoubledigitlistofwidth{}}{21.  List of number characters used in Eastern Dan}{69}
}}}{\singlespacing
{\singlespacing
\hyperlink{PunctuationCharacters}{\XLingPaperdottedtocline{0pt}{\XLingPaperdoubledigitlistofwidth{}}{22.  List of punctuation characters used in Eastern Dan}{71}
}}}{\singlespacing
{\singlespacing
\hyperlink{InternetUnicodeCharacters}{\XLingPaperdottedtocline{0pt}{\XLingPaperdoubledigitlistofwidth{}}{23.  List of Internet characters which need to be accessible in Eastern Dan}{76}
}}}{\singlespacing
{\singlespacing
\hyperlink{ntLoanLetters}{\XLingPaperdottedtocline{0pt}{\XLingPaperdoubledigitlistofwidth{}}{24.  Loan word and print borrwing examples and examples with borrowed letters}{79}
}}}{\singlespacing
{\singlespacing
\hyperlink{ntUnicodeCatagories}{\XLingPaperdottedtocline{0pt}{\XLingPaperdoubledigitlistofwidth{}}{25.  Unicode Categories}{80}
}}}{\singlespacing
{\singlespacing
\hyperlink{ntAuxCharacters}{\XLingPaperdottedtocline{0pt}{\XLingPaperdoubledigitlistofwidth{}}{26.  List of auxiliary characters used in Eastern Dan}{81}
}}}{\singlespacing
{\singlespacing
\hyperlink{ntFrenchCharacters}{\XLingPaperdottedtocline{0pt}{\XLingPaperdoubledigitlistofwidth{}}{27.  French Characters}{81}
}}}{\singlespacing
{\singlespacing
\hyperlink{ntJulaCharacters}{\XLingPaperdottedtocline{0pt}{\XLingPaperdoubledigitlistofwidth{}}{28.  Jula Characters}{81}
}}}{\singlespacing
{\singlespacing
\hyperlink{ntEnglishCharacters}{\XLingPaperdottedtocline{0pt}{\XLingPaperdoubledigitlistofwidth{}}{29.  English Characters}{82}
}}}{\singlespacing
{\singlespacing
\hyperlink{ntHausaCharacters}{\XLingPaperdottedtocline{0pt}{\XLingPaperdoubledigitlistofwidth{}}{30.  Hausa Characters}{82}
}}}{\singlespacing
{\singlespacing
\hyperlink{ntCorpusStats}{\XLingPaperdottedtocline{0pt}{\XLingPaperdoubledigitlistofwidth{}}{31.  Corpora Statistics}{85}
}}}{\singlespacing
{\singlespacing
\hyperlink{ntKeyboardsLanguage}{\XLingPaperdottedtocline{0pt}{\XLingPaperdoubledigitlistofwidth{}}{32.  Keyboard layouts by language}{86}
}}}{\singlespacing
{\singlespacing
\hyperlink{ntBaseline}{\XLingPaperdottedtocline{0pt}{\XLingPaperdoubledigitlistofwidth{}}{33.  Baseline Scores}{89}
}}}{\singlespacing
{\singlespacing
\hyperlink{ntResults}{\XLingPaperdottedtocline{0pt}{\XLingPaperdoubledigitlistofwidth{}}{34.  Comparison of English Fitness Scores}{91}
}}}{\singlespacing
{\singlespacing
\hyperlink{ntBaselineAndOptimize}{\XLingPaperdottedtocline{0pt}{\XLingPaperdoubledigitlistofwidth{}}{35.  Baseline versus Optimized Keyboard Scores}{93}
}}}{\singlespacing
{\singlespacing
\hyperlink{ntVadrinCorpus}{\XLingPaperdottedtocline{0pt}{\XLingPaperdoubledigitlistofwidth{}}{36.  The half a million word corpus.}{109}
}}}{\singlespacing
{\singlespacing
\hyperlink{EDCitations}{\XLingPaperdottedtocline{0pt}{\XLingPaperdoubledigitlistofwidth{}}{37.  A List of works on Eastern Dan}{122}
}}}{\singlespacing
{\singlespacing
\hyperlink{ntFontsUsed}{\XLingPaperdottedtocline{0pt}{\XLingPaperdoubledigitlistofwidth{}}{38.  Fonts used}{125}
}}}{\singlespacing
{\singlespacing
\hyperlink{ntProgramingTools}{\XLingPaperdottedtocline{0pt}{\XLingPaperdoubledigitlistofwidth{}}{39.  Programming tools used}{125}
}}}{\singlespacing
{\singlespacing
\hyperlink{ntOpenSourceToolsForOptimization}{\XLingPaperdottedtocline{0pt}{\XLingPaperdoubledigitlistofwidth{}}{40.  Open source tools for keyboard layout analysis}{126}
}}}\indent  \par{}\noindent{}{Figure }\hfill{}Page\par{}
{\singlespacing
{\singlespacing
\hyperlink{fDiacritics}{\XLingPaperdottedtocline{0pt}{\XLingPapersingledigitlistofwidth{}}{1.  Diacritics are commonly understood to be additional marks above or below a letter.}{2}
}}}{\singlespacing
{\singlespacing
\hyperlink{fMapOfDan}{\XLingPaperdottedtocline{0pt}{\XLingPapersingledigitlistofwidth{}}{2.  A map of the Dan speaking area }{4}
}}}{\singlespacing
{\singlespacing
\hyperlink{CI;MobileSubscriptions}{\XLingPaperdottedtocline{0pt}{\XLingPapersingledigitlistofwidth{}}{3.  Mobile subscriptions by year in Ivory Coast}{6}
}}}{\singlespacing
{\singlespacing
\hyperlink{fHiarchyoffriction}{\XLingPaperdottedtocline{0pt}{\XLingPapersingledigitlistofwidth{}}{4.  Hierarchy of user friction}{10}
}}}{\singlespacing
{\singlespacing
\hyperlink{fConceptualCharacter}{\XLingPaperdottedtocline{0pt}{\XLingPapersingledigitlistofwidth{}}{5.  Conceptual and graphical components of a Unicode character}{19}
}}}{\singlespacing
{\singlespacing
\hyperlink{fMultiGraphs}{\XLingPaperdottedtocline{0pt}{\XLingPapersingledigitlistofwidth{}}{6.  Visual demonstration of Multi-graphs – three tri-graphs}{20}
}}}{\singlespacing
{\singlespacing
\hyperlink{fComplexCharacter}{\XLingPaperdottedtocline{0pt}{\XLingPapersingledigitlistofwidth{}}{7.  Conceptual breakdown of complex characters}{21}
}}}{\singlespacing
{\singlespacing
\hyperlink{fHelvetica-CSIL}{\XLingPaperdottedtocline{0pt}{\XLingPapersingledigitlistofwidth{}}{8.  Demonstrating a vareity of font, and normalization varieties}{21}
}}}{\singlespacing
{\singlespacing
\hyperlink{fWritingSystemsNoCommnet}{\XLingPaperdottedtocline{0pt}{\XLingPapersingledigitlistofwidth{}}{9.  Relationship between languages, Writing Systems and Orthographies}{28}
}}}{\singlespacing
{\singlespacing
\hyperlink{fAtreus}{\XLingPaperdottedtocline{0pt}{\XLingPaperdoubledigitlistofwidth{}}{10.  {\XLingPaperCharisZSILFontFamily{\textit{Atreus}}} an example of a vertically aligned split keyboard.}{37}
}}}{\singlespacing
{\singlespacing
\hyperlink{ISO9995Grid}{\XLingPaperdottedtocline{0pt}{\XLingPaperdoubledigitlistofwidth{}}{11.  Sections, zones, and reference grid of a keyboard according to ISO/IEC 9995-1:2009.}{38}
}}}{\singlespacing
{\singlespacing
\hyperlink{Keyheight}{\XLingPaperdottedtocline{0pt}{\XLingPaperdoubledigitlistofwidth{}}{12.  Illustration of different key profiles..}{38}
}}}{\singlespacing
{\singlespacing
\hyperlink{ftcfHeatMap}{\XLingPaperdottedtocline{0pt}{\XLingPaperdoubledigitlistofwidth{}}{13.  Heatmap of {\textit{Me'phaa}} Keyboard while typing James}{46}
}}}{\singlespacing
{\singlespacing
\hyperlink{fspaHeatMap}{\XLingPaperdottedtocline{0pt}{\XLingPaperdoubledigitlistofwidth{}}{14.  Heatmap of Spanish Keyboard while typing James}{47}
}}}{\singlespacing
{\singlespacing
\hyperlink{fWith-without-toneMarks}{\XLingPaperdottedtocline{0pt}{\XLingPaperdoubledigitlistofwidth{}}{15.  The Same sentence – with and without tone marks.}{50}
}}}{\singlespacing
{\singlespacing
\hyperlink{fReadingtone}{\XLingPaperdottedtocline{0pt}{\XLingPaperdoubledigitlistofwidth{}}{16.  Tonal Melodies}{51}
}}}{\singlespacing
{\singlespacing
\hyperlink{fReadingtwoAccross}{\XLingPaperdottedtocline{0pt}{\XLingPaperdoubledigitlistofwidth{}}{17.  Reading each line twice}{51}
}}}{\singlespacing
{\singlespacing
\hyperlink{fReadingUpDown}{\XLingPaperdottedtocline{0pt}{\XLingPaperdoubledigitlistofwidth{}}{18.  General logical flow for the typing process for typing tonal languages}{51}
}}}{\singlespacing
{\singlespacing
\hyperlink{fOrthographyMap}{\XLingPaperdottedtocline{0pt}{\XLingPaperdoubledigitlistofwidth{}}{19.  A map of the languages near Dan reported to be using similar orthographic devices for tone. }{64}
}}}{\singlespacing
{\singlespacing
\hyperlink{fMethodProcess1}{\XLingPaperdottedtocline{0pt}{\XLingPaperdoubledigitlistofwidth{}}{20.  Research Process}{83}
}}}{\singlespacing
{\singlespacing
\hyperlink{fEasternDanCharacterFrequency}{\XLingPaperdottedtocline{0pt}{\XLingPaperdoubledigitlistofwidth{}}{21.  Eastern Dan Character Frequency}{86}
}}}{\singlespacing
{\singlespacing
\hyperlink{fAFU}{\XLingPaperdottedtocline{0pt}{\XLingPaperdoubledigitlistofwidth{}}{22.  AFU Experiments}{90}
}}}{\singlespacing
{\singlespacing
\hyperlink{fTrans-Mande}{\XLingPaperdottedtocline{0pt}{\XLingPaperdoubledigitlistofwidth{}}{23.  Trans-Mande Experiments}{90}
}}}{\singlespacing
{\singlespacing
\hyperlink{fResultsComparison}{\XLingPaperdottedtocline{0pt}{\XLingPaperdoubledigitlistofwidth{}}{24.  Results of the two evaluations}{92}
}}}{\singlespacing
{\singlespacing
\hyperlink{TotalResults}{\XLingPaperdottedtocline{0pt}{\XLingPaperdoubledigitlistofwidth{}}{25.  Results of all experiments by ordered by Experiment ID}{99}
}}}{\singlespacing
{\singlespacing
\hyperlink{fHelvetica-CSIL2}{\XLingPaperdottedtocline{0pt}{\XLingPaperdoubledigitlistofwidth{}}{26.  Helvetica and Charis SIL}{128}
}}}{\singlespacing
{\singlespacing
\hyperlink{fNTR-CSIL}{\XLingPaperdottedtocline{0pt}{\XLingPaperdoubledigitlistofwidth{}}{27.  New Times Roman and Charis SIL}{129}
}}}{\singlespacing
{\singlespacing
\hyperlink{fUnkeyboardinated}{\XLingPaperdottedtocline{0pt}{\XLingPaperdoubledigitlistofwidth{}}{28.  Unkeyboardinated}{129}
}}}{\singlespacing
{\singlespacing
\hyperlink{fZalgo}{\XLingPaperdottedtocline{0pt}{\XLingPaperdoubledigitlistofwidth{}}{29.  An example of \hyperlink{gtZalgo}{{\textit{Zalgo}}}}{130}
}}}\noindent \\\par{}\vspace{11pt plus 2pt minus 1pt}\XLingPaperneedspace{3\baselineskip}\protect\hypertarget{ntLanguages}{}\XLingPaperaddtocontents{ntLanguages}{\protect\raggedright{\singlespacing
{Table }{1.}{  Languages referenced and their \hyperlink{gtISO639-3}{{\textit{ISO 639-3}}} codes\\}}}\vspace{0pt}{\singlespacing
\hspace*{.25in}{
\XLingPaperminmaxcellincolumn{Language}{\XLingPapermincola}{\textbf{Language Name}}{\XLingPapermaxcola}{+0\tabcolsep}
\XLingPaperminmaxcellincolumn{639-3}{\XLingPapermincolb}{\textbf{ISO 639-3 Code}}{\XLingPapermaxcolb}{+0\tabcolsep}
\XLingPaperminmaxcellincolumn{Eastern}{\XLingPapermincola}{Eastern Dan}{\XLingPapermaxcola}{+0\tabcolsep}
\XLingPaperminmaxcellincolumn{dnj}{\XLingPapermincolb}{dnj}{\XLingPapermaxcolb}{+0\tabcolsep}
\XLingPaperminmaxcellincolumn{Western}{\XLingPapermincola}{Western Dan}{\XLingPapermaxcola}{+0\tabcolsep}
\XLingPaperminmaxcellincolumn{dnj}{\XLingPapermincolb}{dnj}{\XLingPapermaxcolb}{+0\tabcolsep}
\XLingPaperminmaxcellincolumn{Gio}{\XLingPapermincola}{Gio}{\XLingPapermaxcola}{+0\tabcolsep}
\XLingPaperminmaxcellincolumn{dnj}{\XLingPapermincolb}{dnj}{\XLingPapermaxcolb}{+0\tabcolsep}
\XLingPaperminmaxcellincolumn{Kla}{\XLingPapermincola}{Kla}{\XLingPapermaxcola}{+0\tabcolsep}
\XLingPaperminmaxcellincolumn{lda}{\XLingPapermincolb}{lda}{\XLingPapermaxcolb}{+0\tabcolsep}
\XLingPaperminmaxcellincolumn{Spanish}{\XLingPapermincola}{Spanish}{\XLingPapermaxcola}{+0\tabcolsep}
\XLingPaperminmaxcellincolumn{spa}{\XLingPapermincolb}{spa}{\XLingPapermaxcolb}{+0\tabcolsep}
\XLingPaperminmaxcellincolumn{French}{\XLingPapermincola}{French}{\XLingPapermaxcola}{+0\tabcolsep}
\XLingPaperminmaxcellincolumn{fra}{\XLingPapermincolb}{fra}{\XLingPapermaxcolb}{+0\tabcolsep}
\XLingPaperminmaxcellincolumn{German}{\XLingPapermincola}{German}{\XLingPapermaxcola}{+0\tabcolsep}
\XLingPaperminmaxcellincolumn{deu}{\XLingPapermincolb}{deu}{\XLingPapermaxcolb}{+0\tabcolsep}
\XLingPaperminmaxcellincolumn{Romanian}{\XLingPapermincola}{Romanian}{\XLingPapermaxcola}{+0\tabcolsep}
\XLingPaperminmaxcellincolumn{ron}{\XLingPapermincolb}{ron}{\XLingPapermaxcolb}{+0\tabcolsep}
\XLingPaperminmaxcellincolumn{Polish}{\XLingPapermincola}{Polish}{\XLingPapermaxcola}{+0\tabcolsep}
\XLingPaperminmaxcellincolumn{pol}{\XLingPapermincolb}{pol}{\XLingPapermaxcolb}{+0\tabcolsep}
\XLingPaperminmaxcellincolumn{English}{\XLingPapermincola}{English}{\XLingPapermaxcola}{+0\tabcolsep}
\XLingPaperminmaxcellincolumn{eng}{\XLingPapermincolb}{eng}{\XLingPapermaxcolb}{+0\tabcolsep}
\XLingPaperminmaxcellincolumn{Finnish}{\XLingPapermincola}{Finnish}{\XLingPapermaxcola}{+0\tabcolsep}
\XLingPaperminmaxcellincolumn{fin}{\XLingPapermincolb}{fin}{\XLingPapermaxcolb}{+0\tabcolsep}
\XLingPaperminmaxcellincolumn{Swedish}{\XLingPapermincola}{Swedish}{\XLingPapermaxcola}{+0\tabcolsep}
\XLingPaperminmaxcellincolumn{swe}{\XLingPapermincolb}{swe}{\XLingPapermaxcolb}{+0\tabcolsep}
\XLingPaperminmaxcellincolumn{Italian}{\XLingPapermincola}{Italian}{\XLingPapermaxcola}{+0\tabcolsep}
\XLingPaperminmaxcellincolumn{ita}{\XLingPapermincolb}{ita}{\XLingPapermaxcolb}{+0\tabcolsep}
\XLingPaperminmaxcellincolumn{Greek}{\XLingPapermincola}{Greek}{\XLingPapermaxcola}{+0\tabcolsep}
\XLingPaperminmaxcellincolumn{ell}{\XLingPapermincolb}{ell}{\XLingPapermaxcolb}{+0\tabcolsep}
\XLingPaperminmaxcellincolumn{Thai}{\XLingPapermincola}{Thai}{\XLingPapermaxcola}{+0\tabcolsep}
\XLingPaperminmaxcellincolumn{tha}{\XLingPapermincolb}{tha}{\XLingPapermaxcolb}{+0\tabcolsep}
\XLingPaperminmaxcellincolumn{Tibetan}{\XLingPapermincola}{Tibetan}{\XLingPapermaxcola}{+0\tabcolsep}
\XLingPaperminmaxcellincolumn{bod}{\XLingPapermincolb}{bod}{\XLingPapermaxcolb}{+0\tabcolsep}
\XLingPaperminmaxcellincolumn{Japanese}{\XLingPapermincola}{Japanese}{\XLingPapermaxcola}{+0\tabcolsep}
\XLingPaperminmaxcellincolumn{jpn}{\XLingPapermincolb}{jpn}{\XLingPapermaxcolb}{+0\tabcolsep}
\XLingPaperminmaxcellincolumn{Chuxnabán}{\XLingPapermincola}{Chuxnabán Mixe}{\XLingPapermaxcola}{+0\tabcolsep}
\XLingPaperminmaxcellincolumn{pxm}{\XLingPapermincolb}{pxm or mis}{\XLingPapermaxcolb}{+0\tabcolsep}
\XLingPaperminmaxcellincolumn{Amazigh}{\XLingPapermincola}{Amazigh}{\XLingPapermaxcola}{+0\tabcolsep}
\XLingPaperminmaxcellincolumn{txm}{\XLingPapermincolb}{txm}{\XLingPapermaxcolb}{+0\tabcolsep}
\XLingPaperminmaxcellincolumn{Moore}{\XLingPapermincola}{Moore}{\XLingPapermaxcola}{+0\tabcolsep}
\XLingPaperminmaxcellincolumn{mos}{\XLingPapermincolb}{mos}{\XLingPapermaxcolb}{+0\tabcolsep}
\XLingPaperminmaxcellincolumn{Natügu}{\XLingPapermincola}{Natügu}{\XLingPapermaxcola}{+0\tabcolsep}
\XLingPaperminmaxcellincolumn{ntu}{\XLingPapermincolb}{ntu}{\XLingPapermaxcolb}{+0\tabcolsep}
\XLingPaperminmaxcellincolumn{Meꞌphaa}{\XLingPapermincola}{Malinaltepec Meꞌphaa}{\XLingPapermaxcola}{+0\tabcolsep}
\XLingPaperminmaxcellincolumn{tcf}{\XLingPapermincolb}{tcf}{\XLingPapermaxcolb}{+0\tabcolsep}
\XLingPaperminmaxcellincolumn{Arabic}{\XLingPapermincola}{Arabic}{\XLingPapermaxcola}{+0\tabcolsep}
\XLingPaperminmaxcellincolumn{ara}{\XLingPapermincolb}{ara}{\XLingPapermaxcolb}{+0\tabcolsep}
\XLingPaperminmaxcellincolumn{Egyptian}{\XLingPapermincola}{Egyptian Arabic}{\XLingPapermaxcola}{+0\tabcolsep}
\XLingPaperminmaxcellincolumn{arz}{\XLingPapermincolb}{arz}{\XLingPapermaxcolb}{+0\tabcolsep}
\XLingPaperminmaxcellincolumn{Jordanian}{\XLingPapermincola}{Jordanian Arabic}{\XLingPapermaxcola}{+0\tabcolsep}
\XLingPaperminmaxcellincolumn{}{\XLingPapermincolb}{}{\XLingPapermaxcolb}{+0\tabcolsep}
\XLingPaperminmaxcellincolumn{Syrian}{\XLingPapermincola}{Syrian Arabic}{\XLingPapermaxcola}{+0\tabcolsep}
\XLingPaperminmaxcellincolumn{}{\XLingPapermincolb}{}{\XLingPapermaxcolb}{+0\tabcolsep}
\XLingPaperminmaxcellincolumn{Kalsha}{\XLingPapermincola}{Kalsha}{\XLingPapermaxcola}{+0\tabcolsep}
\XLingPaperminmaxcellincolumn{kls}{\XLingPapermincolb}{kls}{\XLingPapermaxcolb}{+0\tabcolsep}
\XLingPaperminmaxcellincolumn{Mafea}{\XLingPapermincola}{Mafea}{\XLingPapermaxcola}{+0\tabcolsep}
\XLingPaperminmaxcellincolumn{mkv}{\XLingPapermincolb}{mkv}{\XLingPapermaxcolb}{+0\tabcolsep}
\XLingPaperminmaxcellincolumn{Filipino}{\XLingPapermincola}{Filipino}{\XLingPapermaxcola}{+0\tabcolsep}
\XLingPaperminmaxcellincolumn{fil}{\XLingPapermincolb}{fil}{\XLingPapermaxcolb}{+0\tabcolsep}
\XLingPaperminmaxcellincolumn{Hindi}{\XLingPapermincola}{Hindi}{\XLingPapermaxcola}{+0\tabcolsep}
\XLingPaperminmaxcellincolumn{hin}{\XLingPapermincolb}{hin}{\XLingPapermaxcolb}{+0\tabcolsep}
\XLingPaperminmaxcellincolumn{Chinese}{\XLingPapermincola}{Chinese (Mandarin)}{\XLingPapermaxcola}{+0\tabcolsep}
\XLingPaperminmaxcellincolumn{cmn}{\XLingPapermincolb}{cmn}{\XLingPapermaxcolb}{+0\tabcolsep}
\XLingPaperminmaxcellincolumn{Farsi}{\XLingPapermincola}{Farsi}{\XLingPapermaxcola}{+0\tabcolsep}
\XLingPaperminmaxcellincolumn{fsa}{\XLingPapermincolb}{fsa}{\XLingPapermaxcolb}{+0\tabcolsep}
\XLingPaperminmaxcellincolumn{Latvian}{\XLingPapermincola}{Latvian}{\XLingPapermaxcola}{+0\tabcolsep}
\XLingPaperminmaxcellincolumn{lav}{\XLingPapermincolb}{lav}{\XLingPapermaxcolb}{+0\tabcolsep}
\XLingPaperminmaxcellincolumn{Khmer}{\XLingPapermincola}{Khmer}{\XLingPapermaxcola}{+0\tabcolsep}
\XLingPaperminmaxcellincolumn{khm}{\XLingPapermincolb}{khm}{\XLingPapermaxcolb}{+0\tabcolsep}
\XLingPaperminmaxcellincolumn{Twi}{\XLingPapermincola}{Twi}{\XLingPapermaxcola}{+0\tabcolsep}
\XLingPaperminmaxcellincolumn{twi}{\XLingPapermincolb}{twi}{\XLingPapermaxcolb}{+0\tabcolsep}
\XLingPaperminmaxcellincolumn{Southern}{\XLingPapermincola}{Southern Dagaare}{\XLingPapermaxcola}{+0\tabcolsep}
\XLingPaperminmaxcellincolumn{dga}{\XLingPapermincolb}{dga}{\XLingPapermaxcolb}{+0\tabcolsep}
\XLingPaperminmaxcellincolumn{Highland}{\XLingPapermincola}{Western Highland Chatino}{\XLingPapermaxcola}{+0\tabcolsep}
\XLingPaperminmaxcellincolumn{ctp}{\XLingPapermincolb}{ctp}{\XLingPapermaxcolb}{+0\tabcolsep}
\XLingPaperminmaxcellincolumn{Western}{\XLingPapermincola}{Western Krahn}{\XLingPapermaxcola}{+0\tabcolsep}
\XLingPaperminmaxcellincolumn{krw}{\XLingPapermincolb}{krw}{\XLingPapermaxcolb}{+0\tabcolsep}
\XLingPaperminmaxcellincolumn{Tsamakko}{\XLingPapermincola}{Tsamakko}{\XLingPapermaxcola}{+0\tabcolsep}
\XLingPaperminmaxcellincolumn{tsb}{\XLingPapermincolb}{tsb}{\XLingPapermaxcolb}{+0\tabcolsep}
\XLingPaperminmaxcellincolumn{Mbugwe}{\XLingPapermincola}{Mbugwe}{\XLingPapermaxcola}{+0\tabcolsep}
\XLingPaperminmaxcellincolumn{mgz}{\XLingPapermincolb}{mgz}{\XLingPapermaxcolb}{+0\tabcolsep}
\XLingPaperminmaxcellincolumn{Aguaruna}{\XLingPapermincola}{Aguaruna}{\XLingPapermaxcola}{+0\tabcolsep}
\XLingPaperminmaxcellincolumn{agr}{\XLingPapermincolb}{agr}{\XLingPapermaxcolb}{+0\tabcolsep}
\XLingPaperminmaxcellincolumn{Attié}{\XLingPapermincola}{Attié}{\XLingPapermaxcola}{+0\tabcolsep}
\XLingPaperminmaxcellincolumn{ati}{\XLingPapermincolb}{ati}{\XLingPapermaxcolb}{+0\tabcolsep}
\XLingPaperminmaxcellincolumn{Bakwé}{\XLingPapermincola}{Bakwé}{\XLingPapermaxcola}{+0\tabcolsep}
\XLingPaperminmaxcellincolumn{bjw}{\XLingPapermincolb}{bjw}{\XLingPapermaxcolb}{+0\tabcolsep}
\XLingPaperminmaxcellincolumn{Guiberoua}{\XLingPapermincola}{Bété Guiberoua}{\XLingPapermaxcola}{+0\tabcolsep}
\XLingPaperminmaxcellincolumn{bet}{\XLingPapermincolb}{bet}{\XLingPapermaxcolb}{+0\tabcolsep}
\XLingPaperminmaxcellincolumn{Yocoboué}{\XLingPapermincola}{Dida Yocoboué}{\XLingPapermaxcola}{+0\tabcolsep}
\XLingPaperminmaxcellincolumn{gud}{\XLingPapermincolb}{gud}{\XLingPapermaxcolb}{+0\tabcolsep}
\XLingPaperminmaxcellincolumn{Godié}{\XLingPapermincola}{Godié}{\XLingPapermaxcola}{+0\tabcolsep}
\XLingPaperminmaxcellincolumn{god}{\XLingPapermincolb}{god}{\XLingPapermaxcolb}{+0\tabcolsep}
\XLingPaperminmaxcellincolumn{Kouya}{\XLingPapermincola}{Kouya}{\XLingPapermaxcola}{+0\tabcolsep}
\XLingPaperminmaxcellincolumn{kyf}{\XLingPapermincolb}{kyf}{\XLingPapermaxcolb}{+0\tabcolsep}
\XLingPaperminmaxcellincolumn{Kroumen}{\XLingPapermincola}{Tépo Kroumen}{\XLingPapermaxcola}{+0\tabcolsep}
\XLingPaperminmaxcellincolumn{ted}{\XLingPapermincolb}{ted}{\XLingPapermaxcolb}{+0\tabcolsep}
\XLingPaperminmaxcellincolumn{Mwan}{\XLingPapermincola}{Mwan}{\XLingPapermaxcola}{+0\tabcolsep}
\XLingPaperminmaxcellincolumn{moa}{\XLingPapermincolb}{moa}{\XLingPapermaxcolb}{+0\tabcolsep}
\XLingPaperminmaxcellincolumn{Néyo}{\XLingPapermincola}{Néyo}{\XLingPapermaxcola}{+0\tabcolsep}
\XLingPaperminmaxcellincolumn{ney}{\XLingPapermincolb}{ney}{\XLingPapermaxcolb}{+0\tabcolsep}
\XLingPaperminmaxcellincolumn{(Nyabwa)}{\XLingPapermincola}{Nyaboua (Nyabwa)}{\XLingPapermaxcola}{+0\tabcolsep}
\XLingPaperminmaxcellincolumn{nwb}{\XLingPapermincolb}{nwb}{\XLingPapermaxcolb}{+0\tabcolsep}
\XLingPaperminmaxcellincolumn{Toura}{\XLingPapermincola}{Toura}{\XLingPapermaxcola}{+0\tabcolsep}
\XLingPaperminmaxcellincolumn{neb}{\XLingPapermincolb}{neb}{\XLingPapermaxcolb}{+0\tabcolsep}
\XLingPaperminmaxcellincolumn{Northern}{\XLingPapermincola}{Northern Wè}{\XLingPapermaxcola}{+0\tabcolsep}
\XLingPaperminmaxcellincolumn{wob}{\XLingPapermincolb}{wob}{\XLingPapermaxcolb}{+0\tabcolsep}
\XLingPaperminmaxcellincolumn{Wan}{\XLingPapermincola}{Wan}{\XLingPapermaxcola}{+0\tabcolsep}
\XLingPaperminmaxcellincolumn{wan}{\XLingPapermincolb}{wan}{\XLingPapermaxcolb}{+0\tabcolsep}
\XLingPaperminmaxcellincolumn{Northern}{\XLingPapermincola}{Northern Grebo}{\XLingPapermaxcola}{+0\tabcolsep}
\XLingPaperminmaxcellincolumn{gbo}{\XLingPapermincolb}{gbo}{\XLingPapermaxcolb}{+0\tabcolsep}
\XLingPaperminmaxcellincolumn{Yaouré}{\XLingPapermincola}{Yaouré}{\XLingPapermaxcola}{+0\tabcolsep}
\XLingPaperminmaxcellincolumn{yre}{\XLingPapermincolb}{yre}{\XLingPapermaxcolb}{+0\tabcolsep}
\XLingPaperminmaxcellincolumn{Southern}{\XLingPapermincola}{Southern Wè}{\XLingPapermaxcola}{+0\tabcolsep}
\XLingPaperminmaxcellincolumn{gxx}{\XLingPapermincolb}{gxx}{\XLingPapermaxcolb}{+0\tabcolsep}
\setlength{\XLingPaperavailabletablewidth}{433.62pt}
\setlength{\XLingPapertableminwidth}{\XLingPapermincola+\XLingPapermincolb}
\setlength{\XLingPapertablemaxwidth}{\XLingPapermaxcola+\XLingPapermaxcolb}
\XLingPapercalculatetablewidthratio{}
\XLingPapersetcolumnwidth{\XLingPapercolawidth}{\XLingPapermincola}{\XLingPapermaxcola}{-0\tabcolsep}
\XLingPapersetcolumnwidth{\XLingPapercolbwidth}{\XLingPapermincolb}{\XLingPapermaxcolb}{-2\tabcolsep}\singlespacing\vspace*{-3\baselineskip}
\begin{longtable}
[l]{@{}p{\XLingPapercolawidth}p{\XLingPapercolbwidth}@{}}\multicolumn{1}{@{}p{\XLingPapercolawidth}}{\textbf{Language Name}}&\multicolumn{1}{p{\XLingPapercolbwidth}@{}}{\textbf{ISO 639-3 Code}}\\%
\multicolumn{1}{@{}p{\XLingPapercolawidth}}{Eastern Dan}&\multicolumn{1}{p{\XLingPapercolbwidth}@{}}{dnj}\\%
\multicolumn{1}{@{}p{\XLingPapercolawidth}}{Western Dan}&\multicolumn{1}{p{\XLingPapercolbwidth}@{}}{dnj}\\%
\multicolumn{1}{@{}p{\XLingPapercolawidth}}{Gio}&\multicolumn{1}{p{\XLingPapercolbwidth}@{}}{dnj}\\%
\multicolumn{1}{@{}p{\XLingPapercolawidth}}{Kla}&\multicolumn{1}{p{\XLingPapercolbwidth}@{}}{lda}\\%
\multicolumn{1}{@{}p{\XLingPapercolawidth}}{Spanish}&\multicolumn{1}{p{\XLingPapercolbwidth}@{}}{spa}\\%
\multicolumn{1}{@{}p{\XLingPapercolawidth}}{French}&\multicolumn{1}{p{\XLingPapercolbwidth}@{}}{fra}\\%
\multicolumn{1}{@{}p{\XLingPapercolawidth}}{German}&\multicolumn{1}{p{\XLingPapercolbwidth}@{}}{deu}\\%
\multicolumn{1}{@{}p{\XLingPapercolawidth}}{Romanian}&\multicolumn{1}{p{\XLingPapercolbwidth}@{}}{ron}\\%
\multicolumn{1}{@{}p{\XLingPapercolawidth}}{Polish}&\multicolumn{1}{p{\XLingPapercolbwidth}@{}}{pol}\\%
\multicolumn{1}{@{}p{\XLingPapercolawidth}}{English}&\multicolumn{1}{p{\XLingPapercolbwidth}@{}}{eng}\\%
\multicolumn{1}{@{}p{\XLingPapercolawidth}}{Finnish}&\multicolumn{1}{p{\XLingPapercolbwidth}@{}}{fin}\\%
\multicolumn{1}{@{}p{\XLingPapercolawidth}}{Swedish}&\multicolumn{1}{p{\XLingPapercolbwidth}@{}}{swe}\\%
\multicolumn{1}{@{}p{\XLingPapercolawidth}}{Italian}&\multicolumn{1}{p{\XLingPapercolbwidth}@{}}{ita}\\%
\multicolumn{1}{@{}p{\XLingPapercolawidth}}{Greek}&\multicolumn{1}{p{\XLingPapercolbwidth}@{}}{ell}\\%
\multicolumn{1}{@{}p{\XLingPapercolawidth}}{Thai}&\multicolumn{1}{p{\XLingPapercolbwidth}@{}}{tha}\\%
\multicolumn{1}{@{}p{\XLingPapercolawidth}}{Tibetan}&\multicolumn{1}{p{\XLingPapercolbwidth}@{}}{bod}\\%
\multicolumn{1}{@{}p{\XLingPapercolawidth}}{Japanese}&\multicolumn{1}{p{\XLingPapercolbwidth}@{}}{jpn}\\%
\multicolumn{1}{@{}p{\XLingPapercolawidth}}{Chuxnabán Mixe}&\multicolumn{1}{p{\XLingPapercolbwidth}@{}}{pxm or mis}\\%
\multicolumn{1}{@{}p{\XLingPapercolawidth}}{Amazigh}&\multicolumn{1}{p{\XLingPapercolbwidth}@{}}{txm}\\%
\multicolumn{1}{@{}p{\XLingPapercolawidth}}{Moore}&\multicolumn{1}{p{\XLingPapercolbwidth}@{}}{mos}\\%
\multicolumn{1}{@{}p{\XLingPapercolawidth}}{Natügu}&\multicolumn{1}{p{\XLingPapercolbwidth}@{}}{ntu}\\%
\multicolumn{1}{@{}p{\XLingPapercolawidth}}{Malinaltepec Meꞌphaa}&\multicolumn{1}{p{\XLingPapercolbwidth}@{}}{tcf}\\%
\multicolumn{1}{@{}p{\XLingPapercolawidth}}{Arabic}&\multicolumn{1}{p{\XLingPapercolbwidth}@{}}{ara}\\%
\multicolumn{1}{@{}p{\XLingPapercolawidth}}{Egyptian Arabic}&\multicolumn{1}{p{\XLingPapercolbwidth}@{}}{arz}\\%
\multicolumn{1}{@{}p{\XLingPapercolawidth}}{Jordanian Arabic}&\multicolumn{1}{p{\XLingPapercolbwidth}@{}}{}\\%
\multicolumn{1}{@{}p{\XLingPapercolawidth}}{Syrian Arabic}&\multicolumn{1}{p{\XLingPapercolbwidth}@{}}{}\\%
\multicolumn{1}{@{}p{\XLingPapercolawidth}}{Kalsha}&\multicolumn{1}{p{\XLingPapercolbwidth}@{}}{kls}\\%
\multicolumn{1}{@{}p{\XLingPapercolawidth}}{Mafea}&\multicolumn{1}{p{\XLingPapercolbwidth}@{}}{mkv}\\%
\multicolumn{1}{@{}p{\XLingPapercolawidth}}{Filipino}&\multicolumn{1}{p{\XLingPapercolbwidth}@{}}{fil}\\%
\multicolumn{1}{@{}p{\XLingPapercolawidth}}{Hindi}&\multicolumn{1}{p{\XLingPapercolbwidth}@{}}{hin}\\%
\multicolumn{1}{@{}p{\XLingPapercolawidth}}{Chinese (Mandarin)}&\multicolumn{1}{p{\XLingPapercolbwidth}@{}}{cmn}\\%
\multicolumn{1}{@{}p{\XLingPapercolawidth}}{Farsi}&\multicolumn{1}{p{\XLingPapercolbwidth}@{}}{fsa}\\%
\multicolumn{1}{@{}p{\XLingPapercolawidth}}{Latvian}&\multicolumn{1}{p{\XLingPapercolbwidth}@{}}{lav}\\%
\multicolumn{1}{@{}p{\XLingPapercolawidth}}{Khmer}&\multicolumn{1}{p{\XLingPapercolbwidth}@{}}{khm}\\%
\multicolumn{1}{@{}p{\XLingPapercolawidth}}{Twi}&\multicolumn{1}{p{\XLingPapercolbwidth}@{}}{twi}\\%
\multicolumn{1}{@{}p{\XLingPapercolawidth}}{Southern Dagaare}&\multicolumn{1}{p{\XLingPapercolbwidth}@{}}{dga}\\%
\multicolumn{1}{@{}p{\XLingPapercolawidth}}{Western Highland Chatino}&\multicolumn{1}{p{\XLingPapercolbwidth}@{}}{ctp}\\%
\multicolumn{1}{@{}p{\XLingPapercolawidth}}{Western Krahn}&\multicolumn{1}{p{\XLingPapercolbwidth}@{}}{krw}\\%
\multicolumn{1}{@{}p{\XLingPapercolawidth}}{Tsamakko}&\multicolumn{1}{p{\XLingPapercolbwidth}@{}}{tsb}\\%
\multicolumn{1}{@{}p{\XLingPapercolawidth}}{Mbugwe}&\multicolumn{1}{p{\XLingPapercolbwidth}@{}}{mgz}\\%
\multicolumn{1}{@{}p{\XLingPapercolawidth}}{Aguaruna}&\multicolumn{1}{p{\XLingPapercolbwidth}@{}}{agr}\\%
\multicolumn{1}{@{}p{\XLingPapercolawidth}}{Attié}&\multicolumn{1}{p{\XLingPapercolbwidth}@{}}{ati}\\%
\multicolumn{1}{@{}p{\XLingPapercolawidth}}{Bakwé}&\multicolumn{1}{p{\XLingPapercolbwidth}@{}}{bjw}\\%
\multicolumn{1}{@{}p{\XLingPapercolawidth}}{Bété Guiberoua}&\multicolumn{1}{p{\XLingPapercolbwidth}@{}}{bet}\\%
\multicolumn{1}{@{}p{\XLingPapercolawidth}}{Dida Yocoboué}&\multicolumn{1}{p{\XLingPapercolbwidth}@{}}{gud}\\%
\multicolumn{1}{@{}p{\XLingPapercolawidth}}{Godié}&\multicolumn{1}{p{\XLingPapercolbwidth}@{}}{god}\\%
\multicolumn{1}{@{}p{\XLingPapercolawidth}}{Kouya}&\multicolumn{1}{p{\XLingPapercolbwidth}@{}}{kyf}\\%
\multicolumn{1}{@{}p{\XLingPapercolawidth}}{Tépo Kroumen}&\multicolumn{1}{p{\XLingPapercolbwidth}@{}}{ted}\\%
\multicolumn{1}{@{}p{\XLingPapercolawidth}}{Mwan}&\multicolumn{1}{p{\XLingPapercolbwidth}@{}}{moa}\\%
\multicolumn{1}{@{}p{\XLingPapercolawidth}}{Néyo}&\multicolumn{1}{p{\XLingPapercolbwidth}@{}}{ney}\\%
\multicolumn{1}{@{}p{\XLingPapercolawidth}}{Nyaboua (Nyabwa)}&\multicolumn{1}{p{\XLingPapercolbwidth}@{}}{nwb}\\%
\multicolumn{1}{@{}p{\XLingPapercolawidth}}{Toura}&\multicolumn{1}{p{\XLingPapercolbwidth}@{}}{neb}\\%
\multicolumn{1}{@{}p{\XLingPapercolawidth}}{Northern Wè}&\multicolumn{1}{p{\XLingPapercolbwidth}@{}}{wob}\\%
\multicolumn{1}{@{}p{\XLingPapercolawidth}}{Wan}&\multicolumn{1}{p{\XLingPapercolbwidth}@{}}{wan}\\%
\multicolumn{1}{@{}p{\XLingPapercolawidth}}{Northern Grebo}&\multicolumn{1}{p{\XLingPapercolbwidth}@{}}{gbo}\\%
\multicolumn{1}{@{}p{\XLingPapercolawidth}}{Yaouré}&\multicolumn{1}{p{\XLingPapercolbwidth}@{}}{yre}\\%
\multicolumn{1}{@{}p{\XLingPapercolawidth}}{Southern Wè}&\multicolumn{1}{p{\XLingPapercolbwidth}@{}}{gxx}\\%
\end{longtable}
}
}\noindent \\\par{}\noindent Abbreviations used\par{}{\setlength{\XLingPaperabbrbaselineskip}{\baselineskip}
\begin{longtable}
[l]{@{\hspace*{0pt}}ll}\setlength{\baselineskip}{\XLingPaperabbrbaselineskip}\raisebox{\baselineskip}[0pt]{\protect\hypertarget{vANSI}{}}{ANSI}&American National Standards Institute\\
\setlength{\baselineskip}{\XLingPaperabbrbaselineskip}\raisebox{\baselineskip}[0pt]{\protect\hypertarget{vBCP}{}}{BCP}&Best Common Practice\\
\setlength{\baselineskip}{\XLingPaperabbrbaselineskip}\raisebox{\baselineskip}[0pt]{\protect\hypertarget{vCf}{}}{cf}&Conferatur (compare with cited material)\\
\setlength{\baselineskip}{\XLingPaperabbrbaselineskip}\raisebox{\baselineskip}[0pt]{\protect\hypertarget{vCLDR}{}}{CLDR}&Common Locale Data Repository\\
\setlength{\baselineskip}{\XLingPaperabbrbaselineskip}\raisebox{\baselineskip}[0pt]{\protect\hypertarget{vCMC}{}}{CMC}&Computer Mitigated Communication\\
\setlength{\baselineskip}{\XLingPaperabbrbaselineskip}\raisebox{\baselineskip}[0pt]{\protect\hypertarget{vGKAP}{}}{GKAP}&General Keyboard Arrangement Problem\\
\setlength{\baselineskip}{\XLingPaperabbrbaselineskip}\raisebox{\baselineskip}[0pt]{\protect\hypertarget{vHMS}{}}{H:M:S}&Hours : Minutes : Seconds\\
\setlength{\baselineskip}{\XLingPaperabbrbaselineskip}\raisebox{\baselineskip}[0pt]{\protect\hypertarget{vHLT}{}}{HLT}&\hyperlink{gtHLT}{{\textit{Human Language Technologies}}}\\
\setlength{\baselineskip}{\XLingPaperabbrbaselineskip}\raisebox{\baselineskip}[0pt]{\protect\hypertarget{vISO}{}}{ISO}&International Standards Organization\\
\setlength{\baselineskip}{\XLingPaperabbrbaselineskip}\raisebox{\baselineskip}[0pt]{\protect\hypertarget{vJIS}{}}{JIS}&Japanese Industrial Standards (Committee)\\
\setlength{\baselineskip}{\XLingPaperabbrbaselineskip}\raisebox{\baselineskip}[0pt]{\protect\hypertarget{vKAP}{}}{KAP}&Keyboard Arrangement Problem\\
\setlength{\baselineskip}{\XLingPaperabbrbaselineskip}\raisebox{\baselineskip}[0pt]{\protect\hypertarget{vLWC}{}}{LWC}&\hyperlink{gtLWC}{{\textit{Language of Wider Communication}}}\\
\setlength{\baselineskip}{\XLingPaperabbrbaselineskip}\raisebox{\baselineskip}[0pt]{\protect\hypertarget{vMSKLC}{}}{MSKLC}&Microsoft Keyboard Layout Creator\\
\setlength{\baselineskip}{\XLingPaperabbrbaselineskip}\raisebox{\baselineskip}[0pt]{\protect\hypertarget{vNFC}{}}{NFC}&\hyperlink{gtNFC}{{\textit{Normalization Form C}}}\\
\setlength{\baselineskip}{\XLingPaperabbrbaselineskip}\raisebox{\baselineskip}[0pt]{\protect\hypertarget{vNFD}{}}{NFD}&\hyperlink{gtNFD}{{\textit{Normalization Form D}}}\\
\setlength{\baselineskip}{\XLingPaperabbrbaselineskip}\raisebox{\baselineskip}[0pt]{\protect\hypertarget{vNRSI}{}}{NRSI}&\hyperlink{gtNRSI}{{\textit{Non-Roman Script Initiative}}}\\
\setlength{\baselineskip}{\XLingPaperabbrbaselineskip}\raisebox{\baselineskip}[0pt]{\protect\hypertarget{vpc}{}}{p.c.}&Personal Communication\\
\setlength{\baselineskip}{\XLingPaperabbrbaselineskip}\raisebox{\baselineskip}[0pt]{\protect\hypertarget{vPSO}{}}{PSO}&\hyperlink{gtparticleswarmoptimization}{{\textit{Particle Swarm Optimization}}}\\
\setlength{\baselineskip}{\XLingPaperabbrbaselineskip}\raisebox{\baselineskip}[0pt]{\protect\hypertarget{vQAP}{}}{QAP}&Quadratic Assignment Problem\\
\setlength{\baselineskip}{\XLingPaperabbrbaselineskip}\raisebox{\baselineskip}[0pt]{\protect\hypertarget{vSMS}{}}{SMS}&\hyperlink{gtSMS}{{\textit{Short Message Service}}}\\
\setlength{\baselineskip}{\XLingPaperabbrbaselineskip}\raisebox{\baselineskip}[0pt]{\protect\hypertarget{vTAM}{}}{TAM}&Tense Aspect Mood\\
\setlength{\baselineskip}{\XLingPaperabbrbaselineskip}\raisebox{\baselineskip}[0pt]{\protect\hypertarget{vUCD}{}}{UCD}&Unicode Character Database\\
\setlength{\baselineskip}{\XLingPaperabbrbaselineskip}\raisebox{\baselineskip}[0pt]{\protect\hypertarget{vUI}{}}{UI}&\hyperlink{gtUI}{{\textit{User Interface}}}\\
\setlength{\baselineskip}{\XLingPaperabbrbaselineskip}\raisebox{\baselineskip}[0pt]{\protect\hypertarget{vUTF-8}{}}{UTF-8}&\hyperlink{gtUTF-8}{{\textit{Unicode Transformation Format – 8-bit}}}\\
\setlength{\baselineskip}{\XLingPaperabbrbaselineskip}\raisebox{\baselineskip}[0pt]{\protect\hypertarget{vUX}{}}{UX}&\hyperlink{gtUX}{{\textit{User Experience}}}\\
\end{longtable}
}\clearpage
\thispagestyle{frontmatterfirstpage}{\vspace*{.65in}\noindent
\raisebox{\baselineskip}[0pt]{\pdfbookmark[1]{Typographical Conventions}{rXLingPapPreface3}}\raisebox{\baselineskip}[0pt]{\protect\hypertarget{rXLingPapPreface3}{}}\XLingPaperneedspace{3\baselineskip}\noindent
{\MakeUppercase{{\protect\centering
Typographical Conventions\protect\\}}}\markboth{Typographical Conventions}{Typographical Conventions}
\XLingPaperaddtocontents{rXLingPapPreface3}}\penalty10000\par{}
\vspace{10.8pt}\indent The following characters are used to provide special meaning to text outside of tables:\par{}{\parskip .5pt plus 1pt minus 1pt

\vspace{\baselineskip}

{\setlength{\XLingPapertempdim}{\XLingPaperbulletlistitemwidth+\parindent{}}\leftskip\XLingPapertempdim\relax
\interlinepenalty10000
\XLingPaperlistitem{\parindent{}}{\XLingPaperbulletlistitemwidth}{•}{Content within square brackets denotes either phonetic representations (such as allophones) or ISO 639-3:\hyperlink{rISO639-3}{2007}\protect\footnote[1]{{\leftskip0pt\parindent1em\raisebox{\baselineskip}[0pt]{\protect\hypertarget{nISO639-3}{}} As amended up to \hyperlink{ISO639-3-2018}{2018} by the ISO 639-3 registrar, SIL International.}} codes {\textsquarebracketleft{} \textsquarebracketright{}}.}}
{\setlength{\XLingPapertempdim}{\XLingPaperbulletlistitemwidth+\parindent{}}\leftskip\XLingPapertempdim\relax
\interlinepenalty10000
\XLingPaperlistitem{\parindent{}}{\XLingPaperbulletlistitemwidth}{•}{Content within forward slashes denotes phonemic representations //.}}
{\setlength{\XLingPapertempdim}{\XLingPaperbulletlistitemwidth+\parindent{}}\leftskip\XLingPapertempdim\relax
\interlinepenalty10000
\XLingPaperlistitem{\parindent{}}{\XLingPaperbulletlistitemwidth}{•}{Content within angle brackets denotes orthographic or graphemic representations\\ {\XLingPaperCambriaZMathFontFamily{\textup{\textmd{⟨   ⟩}}}}}}
{\setlength{\XLingPapertempdim}{\XLingPaperbulletlistitemwidth+\parindent{}}\leftskip\XLingPapertempdim\relax
\interlinepenalty10000
\XLingPaperlistitem{\parindent{}}{\XLingPaperbulletlistitemwidth}{•}{Content within double-slashes or pipes denotes morphophonemic representations\\ // // or \textbar{} \textbar{}.}}
{\setlength{\XLingPapertempdim}{\XLingPaperbulletlistitemwidth+\parindent{}}\leftskip\XLingPapertempdim\relax
\interlinepenalty10000
\XLingPaperlistitem{\parindent{}}{\XLingPaperbulletlistitemwidth}{•}{In discussions of syllable types C, V, and N represent Consonant, Vowel, and Nasal.}}
{\setlength{\XLingPapertempdim}{\XLingPaperbulletlistitemwidth+\parindent{}}\leftskip\XLingPapertempdim\relax
\interlinepenalty10000
\XLingPaperlistitem{\parindent{}}{\XLingPaperbulletlistitemwidth}{•}{Phonetic and phonemic representations follow the International Phonetic Alphabet unless otherwise indicated.}}
{\setlength{\XLingPapertempdim}{\XLingPaperbulletlistitemwidth+\parindent{}}\leftskip\XLingPapertempdim\relax
\interlinepenalty10000
\XLingPaperlistitem{\parindent{}}{\XLingPaperbulletlistitemwidth}{•}{In prose sections, Unicode characters appear in the following order upon first mention: {\XLingPaperCambriaZMathFontFamily{⟨ {\XLingPaperDejaVuZSerifFontFamily{‽}} ⟩}} U+203D 'INTERROBANG'. A more natural prose style is used for subsequent mentions (using any one of these three parts).}}
{\setlength{\XLingPapertempdim}{\XLingPaperbulletlistitemwidth+\parindent{}}\leftskip\XLingPapertempdim\relax
\interlinepenalty10000
\XLingPaperlistitem{\parindent{}}{\XLingPaperbulletlistitemwidth}{•}{Unicode characters which are combining marks will be shown relative to a dotted circle {\XLingPaperCambriaZMathFontFamily{⟨ {\XLingPaperDejaVuZSerifFontFamily{◌}} ⟩}}}}
{\setlength{\XLingPapertempdim}{\XLingPaperbulletlistitemwidth+\parindent{}}\leftskip\XLingPapertempdim\relax
\interlinepenalty10000
\XLingPaperlistitem{\parindent{}}{\XLingPaperbulletlistitemwidth}{•}{Content within a stylized square {\XLingPaperKeyboardZKeysExZExpandedFontFamily{\textup{\textmd{v}}}} then {\XLingPaperKeyboardZKeysExZExpandedFontFamily{\textup{\textmd{E}}}} or {\XLingPaperKeyboardZKeysExZExpandedFontFamily{\textup{\textmd{q}}}} + {\XLingPaperKeyboardZKeysExZExpandedFontFamily{\textup{\textmd{E}}}} indicate key press actions; the names of the keys are indicated within the stylized square.}}
{\setlength{\XLingPapertempdim}{\XLingPaperbulletlistitemwidth+\parindent{}}\leftskip\XLingPapertempdim\relax
\interlinepenalty10000
\XLingPaperlistitem{\parindent{}}{\XLingPaperbulletlistitemwidth}{•}{When referring to key positions the ISO 9995-1:\hyperlink{rISO9995}{2009} grid shall be the point of reference. See figure \hyperlink{ISO9995Grid}{11} for visualization.}}
{\setlength{\XLingPapertempdim}{\XLingPaperbulletlistitemwidth+\parindent{}}\leftskip\XLingPapertempdim\relax
\interlinepenalty10000
\XLingPaperlistitem{\parindent{}}{\XLingPaperbulletlistitemwidth}{•}{Commands to be typed in a computer terminal they will be styled to look like \\{\XLingPaperMonospacedFontFamily{\textup{\textmd{terminal input}}}}.}}
{\setlength{\XLingPapertempdim}{\XLingPaperbulletlistitemwidth+\parindent{}}\leftskip\XLingPapertempdim\relax
\interlinepenalty10000
\XLingPaperlistitem{\parindent{}}{\XLingPaperbulletlistitemwidth}{•}{Technical jargon will appear as italicised words which can be found in a glossary appearing in appendix \hyperlink{DefinedTerms}{B}.}}
{\setlength{\XLingPapertempdim}{\XLingPaperbulletlistitemwidth+\parindent{}}\leftskip\XLingPapertempdim\relax
\interlinepenalty10000
\XLingPaperlistitem{\parindent{}}{\XLingPaperbulletlistitemwidth}{•}{Proper names, especially of software tools, are sometimes italicized to contrast them with the surrounding text.}}
{\setlength{\XLingPapertempdim}{\XLingPaperbulletlistitemwidth+\parindent{}}\leftskip\XLingPapertempdim\relax
\interlinepenalty10000
\XLingPaperlistitem{\parindent{}}{\XLingPaperbulletlistitemwidth}{•}{Cited works in other languages present unique challenges. Proper names and the titles of cited works in foreign languages, especially when occurring in a non-Latin script are presented in both their original script and a Latin transliteration. When cited works are of works in other languages written in a Latin script, an effort is made to present a translation of the title. Translations and transliterations are presented in square brackets {\textsquarebracketleft{}\textsquarebracketright{}}. Parenthesis ( ) in citations are either original to the work's title, or in the case of works published with titles in two languages are used around the title of the second language.}}
\vspace{\baselineskip}
}\clearpage
\thispagestyle{frontmatterfirstpage}{\vspace*{.65in}\noindent
\raisebox{\baselineskip}[0pt]{\pdfbookmark[1]{Acknowledgements}{rXLingPapAcknowledgements}}\raisebox{\baselineskip}[0pt]{\protect\hypertarget{rXLingPapAcknowledgements}{}}\XLingPaperneedspace{3\baselineskip}\noindent
{\MakeUppercase{{\protect\centering
Acknowledgements\protect\\}}}\markboth{Acknowledgements}{Acknowledgements}
\XLingPaperaddtocontents{rXLingPapAcknowledgements}}\penalty10000\par{}
\vspace{10.8pt}\indent Several people have supported me through this research and writing process, most of them, typing via email, often in a language other than their mother tongue. {Margrit Bolli}, has been very kind to answer questions about Eastern Dan. {Valentin Vydrin} has answered many more questions, and provided me data to my heart's content. Dave Roberts has been wealth of knowledge about tone orthographies. {Andy Black} has tirelessly supported my use of {\XLingPaperCharisZSILFontFamily{\textit{XLingPaper}}} to typeset this thesis.\par{}{\parskip .5pt plus 1pt minus 1pt

\vspace{\baselineskip}

{\setlength{\XLingPapertempdim}{\XLingPaperbulletlistitemwidth+\parindent{}}\leftskip\XLingPapertempdim\relax
\interlinepenalty10000
\XLingPaperlistitem{\parindent{}}{\XLingPaperbulletlistitemwidth}{•}{Lenovo T530, iMac, MacBookPro, MacBookPro Becky, Mac Pro.}}
{\setlength{\XLingPapertempdim}{\XLingPaperbulletlistitemwidth+\parindent{}}\leftskip\XLingPapertempdim\relax
\interlinepenalty10000
\XLingPaperlistitem{\parindent{}}{\XLingPaperbulletlistitemwidth}{•}{MSWord, Typing, XLingPaper, Chary, UnicodeCCount}}
{\setlength{\XLingPapertempdim}{\XLingPaperbulletlistitemwidth+\parindent{}}\leftskip\XLingPapertempdim\relax
\interlinepenalty10000
\XLingPaperlistitem{\parindent{}}{\XLingPaperbulletlistitemwidth}{•}{Becky, Nate, Bob}}
\vspace{\baselineskip}
}\clearpage
\thispagestyle{frontmatterfirstpage}{\vspace*{.65in}\noindent
\raisebox{\baselineskip}[0pt]{\pdfbookmark[1]{Abstract}{rXLingPapAbstract0}}\raisebox{\baselineskip}[0pt]{\protect\hypertarget{rXLingPapAbstract0}{}}\XLingPaperneedspace{3\baselineskip}\noindent
{\MakeUppercase{{\protect\centering
Abstract\protect\\}}}\markboth{Abstract}{Abstract}
\XLingPaperaddtocontents{rXLingPapAbstract0}}\penalty10000\par{}
\vspace{10.8pt}\indent The current discussion on the General Keyboard Arrangement Problem ({\hyperlink{vGKAP}{{GKAP}}}) seeks to discover methods and present results for optimal keyboard layouts. These layouts are often done on a per language basis. Rarely are optimized keyboard layouts presented for the world's low-resourced languages. This thesis contributes to the GKAP discussion by proposing the addition of two metrics to commonly agreed upon metrics in the {\hyperlink{vGKAP}{{GKAP}}} literature. These two metrics take the form of an algorithm that can be applied to any keyboard layout for any language using a Latin based script. The first metric accounts for disruption in the phoneme (or functional unit) stream while engaged in typing, and the second accounts for non-character-producing keystrokes required while typing. These metrics are then combined with a fitness score (generally derived from haptic related heuristics) to produce a fitness score and haptic function. The use of this algorithm propels to the language development dialogue further by providing a quality assessment metric for the evaluation of the essential language-based digital product: the keyboarding method.\par{}\clearpage
\pagestyle{body}\pagenumbering{arabic}\thispagestyle{bodyfirstpage}\markboth{Introduction}{Introduction}
\XLingPaperaddtocontents{c1-Introduction}{\vspace*{.65in}\XLingPaperneedspace{3\baselineskip}\noindent
\fontsize{14}{16.8}\selectfont \textbf{{\centering
CHAPTER \raisebox{\baselineskip}[0pt]{\protect\hypertarget{c1-Introduction}{}}\raisebox{\baselineskip}[0pt]{\pdfbookmark[1]{1 Introduction}{c1-Introduction}}1\protect\\}}}\par{}
{\XLingPaperneedspace{3\baselineskip}\noindent
\fontsize{14}{16.8}\selectfont \textbf{{\centering
Introduction\protect\\}}}\par{}
\vspace{16pt}\indent Since at least the early 1990's scholars have acknowledged the benefits of a textual modality for minority-language communities \hyperlink{rSchneider2010Langu}{(Schneider  2010:15}; \hyperlink{Bernard}{Bernard  1992)}. EGIDS \hyperlink{rLewis2010Asses}{(Lewis \& Simons  2010)} suggests that a language community is less in danger of losing their language if they have a stable written literature. A stable written literature assumes that new literature is being produced and consumed by the language-using community. A stable literature requires tools for producing texts in the concerned language. \hyperlink{Thoms}{Thomas \& Simons (2017)} indicate that over 1,100 languages have text input solutions via SIL’s {\XLingPaperCharisZSILFontFamily{\textit{Keyman}}} software, while the Ethnologue \hyperlink{rSimonsAndFennig2018}{(Simons \& Fennig  2018)} suggests that some 3,909 languages have orthographies. Even with these impressive numbers indicating a great range of diversity for written modes of language use, there are evidently still barriers to the use of these tools. We still see many language communities which have the capacity to write their language, but don't write it — even with digital tools.\par{}\indent Many try to address the gap in transferable literacy or stable literature production via orthography reform, some point to the need for the teaching of composition skills \hyperlink{rWeberetal}{(Weber et al.  2007)}. However, few if any at all have written about text input method reform, and the discussion of improved \hyperlink{gtKeyboardLayout}{{\textit{keyboard layouts}}} has been limited to majority languages and cast as an issue within computer science.\par{}\indent The Keyboard Arrangement Problem ({\hyperlink{vKAP}{{KAP}}}) when approached through computer science is often used as a test case for algorithm implementations; applying specific \hyperlink{gtMetaheuristic}{{\textit{metaheuristic}}} methods to solve for optimal keyboard layouts. This has the unfortunate result that the body of academic literature about text input and the body of literature about language development do not often intersect. However, for speakers of minority languages using computers, what they need is a keyboard layout which supports their language completely and “well”. Many would benefit from an optimized text input solution for their language. Most algorithmic models take multiple ergonomic conditions into account when applying heuristics. One problem with status quo heuristics models for text input is that the standard for evaluation and design is often defined by English or by simple character combination frequencies, and English based typography. The English \hyperlink{gtWritingSystem}{{\textit{writing system}}} and \hyperlink{gtOrthography}{{\textit{orthography}}} are not as “exotic” as many languages with \hyperlink{gtDiacritic}{{\textit{diacritic marks}}} and the myriad of things that these marks can mean.\par{}\vspace{11pt plus 2pt minus 1pt}\setbox0=\vbox{\protect\centering \leavevmode
\vspace*{0pt}{\XeTeXpdffile "../Resources/diacritics.pdf" scaled 700}\\[0pt]\protect\hypertarget{fDiacritics}{}\XLingPaperaddtocontents{fDiacritics}{\singlespacing
{Figure }{1.}{ Diacritics are commonly understood to be additional marks above or below a letter.\\}}}\box0\par{}\vspace{11pt plus 2pt minus 1pt}\indent Existing text input solutions which reach common market availability are often physically designed around how many buttons are needed for inputting English text. Solving the \hyperlink{gtMinorityLanguage}{{\textit{minority language}}} text input problem is what is needed to enable minority language writers to fully enjoy, and empower their communities with, their written language.\par{}\indent As previously alluded, academic and industry approaches to the minority language text input problem have had a wide degree of approaches including: “orthography reform”, “multi-lingual keyboard layouts”, “full optimizations for a given language”, and “partial refinement of existing keyboard layouts”. As a case study, this thesis takes a look at what it would mean to ‘implement well’ a text input solution for the Eastern Dan writing system.\par{}\indent Eastern Dan is one of three separate writing traditions that fall under the ISO 639-3 code for Dan \hyperlink{dnj}{\textsquarebracketleft{}dnj\textsquarebracketright{}}. To separate these three writing traditions one would need to use {\hyperlink{vBCP}{{BCP}}}47\protect\footnote[1]{{\leftskip0pt\parindent1em\raisebox{\baselineskip}[0pt]{\protect\hypertarget{nBCP47}{}} \href{https://tools.ietf.org/html/bcp47}{\textcolor[rgb]{0,0,0}{https://tools.ietf.org/html/bcp47}}}}. {\hyperlink{vBCP}{{BCP}}}47 \hyperlink{rPhilips2009Tagsf}{(Philips \& Davis  2009)} is an Internet Engineering Task Force best practice document which outlines how to use standards like {\hyperlink{vISO}{{ISO}}} 639-3 to identify recorded expressions of language. {\hyperlink{vBCP}{{BCP}}}47 allows for the tagging of language, script, and country in which language expression takes place to better serve the language needs of digital device users. The case of language tagging in Dan is complex. Eastern and Western Dan would both qualify for inclusion under the {\hyperlink{vBCP}{{BCP}}}47 code {\XLingPaperCharisZSILFontFamily{\textit{dnj-ci}}}, while Gio could be distinguished with {\XLingPaperCharisZSILFontFamily{\textit{dnj-lr}}}\protect\footnote[2]{{\leftskip0pt\parindent1em\raisebox{\baselineskip}[0pt]{\protect\hypertarget{nLatinOnly}{}} Since Latin is the only known script used to write Dan, it can be assumed by default. {\hyperlink{vBCP}{{BCP}}}47 does not distinguish between upper and lower case, but lower case is preferred.}}. Further encoding is needed to distinguish Eastern from Western Dan, and also for each orthography version they have had. Crúbadán language data \hyperlink{rScannell2009Dan}{(Scannell  2009)} for Eastern Dan uses: {\XLingPaperCharisZSILFontFamily{\textit{dnj-x-east}}} but it is unclear if his corpus is based on the same orthography as the orthography described in this thesis as orthography version 3 – even if the language content is from the same language variety. Writing System evolution can nullify the significance of broad, but shallow computational projects for under-resourced languages. When the corpora, such as those generated by the Crúbadán project are not created in conjunction with those who are knowledgeable of the evolutionary processes of the evolutionary stages of a language\protect\footnote[3]{{\leftskip0pt\parindent1em\raisebox{\baselineskip}[0pt]{\protect\hypertarget{nElkeKaran}{}} Some such as \hyperlink{rKaranE2014}{Elke (2014)} argue that a long evolutionary process of "orthography development" is beneficial. While this makes some sense in non-digital contexts (such as pre-1980), the sociological impact of today's ubiquitous text messaging technology is that a "long development process" thwarts open source efforts to create custom orthography and language specific digital support for under-resourced languages.}}.\par{}\indent Dan is considered by some to be a macro-language comprised of a dialect chain of over 40 dialects \hyperlink{rInternational2018Dan}{(SIL International  2018b)} and \hyperlink{rRobertssubmittedChapt}{Roberts \& Vydrin  (submitted)}. As recently as 2012 the ISO 639-3 registrar approved request 2012-083 \hyperlink{rVydrin2012ISO63}{(Vydrin  2012)} to split one of these dialects off into its own language retiring the previous ISO 639‑3 code {\textsquarebracketleft{}daf\textsquarebracketright{}} and creating two new codes; one for Eastern and Western Dan {\textsquarebracketleft{}dnj\textsquarebracketright{}} and one for Kla {\textsquarebracketleft{}lda\textsquarebracketright{}}. Eastern and Western Dan have had their own separate writing traditions for over 40 years. The third writing tradition for Dan exists in Liberia where the language is known as {\textit{{\XLingPaperCharisZSILFontFamily{\textit{Gio}}}}}. Gio and Western Dan are not covered in this thesis.\par{}\vspace{11pt plus 2pt minus 1pt}\setbox0=\vbox{\protect\centering \leavevmode
\vspace*{0pt}{\XeTeXpicfile "../Resources/Dan-Mano-Tura.png" scaled 200}\\[0pt]\protect\hypertarget{fMapOfDan}{}\XLingPaperaddtocontents{fMapOfDan}{\singlespacing
{Figure }{2.}{ \setcounter{footnote}{3}A map of the Dan speaking area \footnotemark{}\\}}\protect\footnotetext[4]{{\leftskip0pt\parindent1em\raisebox{\baselineskip}[0pt]{\protect\hypertarget{nMapCitation}{}} Originally published in \hyperlink{rVydrine2003Map}{Vydrine et al. (2003)}}}}\box0\protect\footnotetext[4]{{\leftskip0pt\parindent1em\raisebox{\baselineskip}[0pt]{\protect\hypertarget{nMapCitation}{}} Originally published in \hyperlink{rVydrine2003Map}{Vydrine et al. (2003)}}}\par{}\vspace{11pt plus 2pt minus 1pt}\indent There are significant segmental and suprasegmental differences between Eastern and Western Dan. The language is used in the Ivory Coast ({\textit{Côte d'Ivoire}}) and Liberia by approximately 1.65 million people. \hyperlink{rVydrinValentin200842F41742B41A414}{Vydrin \textsquarebracketleft{}Выдрин\textsquarebracketright{} (2008)} indicates that the multilingual situation is complex. There are L2 speakers of Dan, and most L1 speakers of Dan are also multilingual. Of the total population of language users approximately 650,000 are users of Eastern Dan\protect\footnote[5]{{\leftskip0pt\parindent1em\raisebox{\baselineskip}[0pt]{\protect\hypertarget{nRadioUse}{}} \hyperlink{rVydrinValentin200842F41742B41A414}{Vydrin \textsquarebracketleft{}Выдрин\textsquarebracketright{} (2008)} also indicates that there were regularly held radio broadcasts in Dan prior to the civil war in 2002; adding to the rich tradition of language use in multiple mediums.}}. Eastern Dan has been written since at least 1978. The typographic development of Eastern Dan has been heavily influenced by French, as written in {\textit{Côte d'Ivoire}}. The orthography has had several significant evolutionary stages which are presented in table \hyperlink{ntHistoryOfDanOrthography}{11} in chapter \hyperlink{sEDWritingSystem}{3}. This thesis works with the orthography as it was defined in 1994, or as I call it - version 3. Chapter \hyperlink{sEDWritingSystem}{3} describes the orthography in detail.\par{}\indent Eastern Dan is describable as a low resource language\protect\footnote[6]{{\leftskip0pt\parindent1em\raisebox{\baselineskip}[0pt]{\protect\hypertarget{nLowResource}{}} For a discussion of the terms and concepts around "low resource" and "under-resourced" languages see \hyperlink{rCieri2016}{Cieri et al. (2016)}.}}. This means that digital infrastructure for Dan at any level is effectively none. There is some data in the Unicode Common Locale Data Repository\protect\footnote[7]{{\leftskip0pt\parindent1em\raisebox{\baselineskip}[0pt]{\protect\hypertarget{nCLDR}{}} The Unicode {\hyperlink{vCLDR}{{CLDR}}} provides key building blocks for software to support the world's languages. This data is used by a wide spectrum of companies for their software internationalization and localization, adapting software to the conventions of different languages for such common software tasks. At the time of writing only population, literacy, and population percentage data for Dan exists in {\hyperlink{vCLDR}{{CLDR}}}. {\hyperlink{vCLDR}{{CLDR}}} also suggests that the likely {\hyperlink{vCLDR}{{CLDR}}} sub-tag would be: dnj\_Latn\_CI. \href{http://www.unicode.org/cldr/charts/latest/supplemental/likely\_subtags.html\#dnj}{\textcolor[rgb]{0,0,0}{http://www.unicode.org/cldr/charts/latest/supplemental/likely\_subtags.html\#dnj}}.\par{}\indent There is a difference between BCP47 codes and {\hyperlink{vCLDR}{{CLDR}}} \hyperlink{gtLocale}{{\textit{Locale}}} Codes. {\hyperlink{vBCP}{{BCP}}}47 requires {\XLingPaperCambriaZMathFontFamily{⟨ - ⟩}} while {\hyperlink{vCLDR}{{CLDR}}} permits {\XLingPaperCambriaZMathFontFamily{⟨ \_ ⟩}} or {\XLingPaperCambriaZMathFontFamily{⟨ - ⟩}} (for additional details see \hyperlink{rDavisetal2018LDML}{Davis et al.  2018} §3.2, 3.3). Crúbadán language data \hyperlink{rScannell2009Dan}{(Scannell  2009)}, is correct to use {\XLingPaperCharisZSILFontFamily{\textit{dnj-x-east}}} because that is following best practices as outlined in the {\hyperlink{vBCP}{{BCP}}}47 documentation, where {\XLingPaperCharisZSILFontFamily{\textit{-x-}}} denotes that what ever follows is for private use. This makes sense in the context of the Crúbadán language data project where the contrast is only between {\XLingPaperCharisZSILFontFamily{\textit{-x-east}}} and {\XLingPaperCharisZSILFontFamily{\textit{-x-west}}} and is scoped to only have relevance within that project. The only thing the Crúbadán might have done to improve their tagging was to scope their private tag to their project. Something like: {\XLingPaperCharisZSILFontFamily{\textit{-x-crub-west}}}.}} for Dan, but none specifically for Eastern Dan as it is spoken or written in the Ivory Coast\protect\footnote[8]{{\leftskip0pt\parindent1em\raisebox{\baselineskip}[0pt]{\protect\hypertarget{nLiberiaKeyboard}{}} There is a multi-language keyboard designed for the orthographies of Liberia which supports the Gio orthography. \href{https://keyman.com/keyboards/libtralo}{\textcolor[rgb]{0,0,0}{https://keyman.com/keyboards/libtralo}}}}. To the best of my knowledge, no keyboard layout, spell check, text-to-voice, voice-to-text, part-of-speech tagging, or language related digital infrastructure exists specifically for users of Eastern (or Western) Dan. Like many minority language users, Eastern Dan users are using their language on their computers, albeit in ways which are not technologically supported with language aware features. Usage of Eastern Dan on {\hyperlink{vSMS}{{SMS}}} and mobile digital devices, is unattested. However, the International Telecommunications Union reports that mobile subscriptions in the Ivory Coast are 130\% of the national population \hyperlink{Statista}{(Statista  2018)}, so use of Eastern Dan on mobile devices should be assumed as is reported for other languages in West Africa ({\hyperlink{vCf}{{cf}}}. \hyperlink{rVoldLexander2011Texti}{Vold Lexander  2011}, \hyperlink{rDeumert2013}{Deumert \& Vold Lexander  2013}, \hyperlink{rDeumert2017}{Deumert  2017)}\protect\footnote[9]{{\leftskip0pt\parindent1em\raisebox{\baselineskip}[0pt]{\protect\hypertarget{nJohnLennon}{}} When considering figure \hyperlink{CI;MobileSubscriptions}{3} the following ancillary facts may be helpful. Acceleration in the Ivory Coast's mobile device market has had multiple sources. In 2007, there was a major reduction in the cost of telephony services; mobile became more affordable. In 2007, the iPhone debuted in the USA. Apple stores limited the purchase of iPhones per customer to five per transaction. People from outside the USA when traveling to the USA would take the iPhones home and sell them at a premium, more than paying for the cost of their trip. According to \hyperlink{rGSMA2018}{GSMA (2018)}, in 2008 {\XLingPaperCharisZSILFontFamily{\textit{Orange}}} a local service provider launched the first mobile banking service in the Ivory Coast, bringing banking services to many who did not previously have banking services. In 2009 Orange starts to sell service for the iPhone in the Ivory Coast with an official license. The iPhone uses CLDR settings for the Ivory Coast and is localized in French. \hyperlink{rJumia2015}{Jumia (2015)} reports that 76\% new sales are in their Abidjan market region. However, local merchants might buy phones and transport them north to sell. \hyperlink{rJumia2015}{Jumia (2015)}, while only one retailer, reported that 74\% of their sales are {\XLingPaperCharisZSILFontFamily{\textit{Android}}} (smart phones), and that 30\% of these are {\XLingPaperCharisZSILFontFamily{\textit{Samsung}}} phones. Mobile users in the Ivory Coast are also reported to have a high rate of multiple phone service providers. Personal experience in Nigeria suggests that this is due to either variations of service consistency in certain geographical regions or significant price wars amongst providers. It is estimated that about 25-35\% of mobile subscriptions in 2017 use mobile Internet and that 27\% of all mobile devices are smart phones.}}.\par{}\vspace{11pt plus 2pt minus 1pt}\setbox0=\vbox{\protect\centering \leavevmode
\vspace*{0pt}{\XeTeXpdffile "../Resources/Mobile_Subscriptions_CI.pdf" scaled 1500}\\[0pt]\protect\hypertarget{CI;MobileSubscriptions}{}\XLingPaperaddtocontents{CI;MobileSubscriptions}{\singlespacing
{Figure }{3.}{ Mobile subscriptions by year in Ivory Coast\\}}}\box0\par{}\vspace{11pt plus 2pt minus 1pt}\indent It is common practice across the world's minority language using communities to exercise all of their linguistic competencies in digital social media. This means that it is reasonable to expect multilingual Eastern Dan using communities to also approach their digitally mediated social interactions with language choices which afford them the most opportunity in the interaction, and that the choice of language used should follow the same norms non-digitally mediated interactions would incur.\par{}\indent The presence of social media is prevalent all through West Africa, for instance \hyperlink{rIWS}{Internet World Stats  2018} claims that Facebook has a 15.3\% penetration rate in the Ivory Coast as of December 2017. I take this to mean that social media still has 2-4 years before penetration levels are near 100\%, and I take this to mean that most digitally mediated written communications right now in the Ivory Coast are \hyperlink{gtSMS}{{\textit{SMS}}} messages (though on smart phones I expect {\XLingPaperCharisZSILFontFamily{\textit{WhatsApp}}} to be supplanting the default {\hyperlink{vSMS}{{SMS}}} service).\par{}\indent Linguists (and select local publishers) have used virtual keyboard layouts optimized for the \hyperlink{gtTypesetting}{{\textit{typesetting}}}\protect\footnote[10]{{\leftskip0pt\parindent1em\raisebox{\baselineskip}[0pt]{\protect\hypertarget{n-typing-def}{}} I present \hyperlink{gtTypesetting}{{\textit{typesetting}}} as a distinct activity from \hyperlink{gtTyping}{{\textit{typing}}}. Typesetting in general involves many more characters, often with unique page formatting requirements. \hyperlink{gtTyping}{{\textit{Typing}}} is broader in its text input application and would cover general document creation, blogging, email, and digital communication. Both are text input activities.}} of linguistic text to write Dan. Two keyboard layouts are presented in chapter \hyperlink{cMethods}{4}: {\XLingPaperCharisZSILFontFamily{\textit{AFU}}} and {\XLingPaperCharisZSILFontFamily{\textit{Trans-Mande}}}. SIL in the Ivory Coast was consulted but no keyboard could be found which was designated to serve Eastern Dan. {Margrit Bolli} was also consulted as she was influential in the Bible translation which was digitally typeset but again no keyboard layout for Eastern Dan was found. With this as background we are now free to explore a means to optimize a keyboard layout solution for Eastern Dan typists.\par{}{\vspace{15pt}\XLingPaperneedspace{3\baselineskip}\noindent
\fontsize{13}{15.6}\selectfont \textbf{{\noindent
\raisebox{\baselineskip}[0pt]{\pdfbookmark[2]{{1.1 } The keyboard layout in language vitality and the under-resourced language}{sKBD-underresourced-languages}}\raisebox{\baselineskip}[0pt]{\protect\hypertarget{sKBD-underresourced-languages}{}}{1.1 }The keyboard layout in language vitality and the under-resourced language}}\markboth{The keyboard layout in language vitality and the under-resourced language}{Introduction}\XLingPaperaddtocontents{sKBD-underresourced-languages}}\par{}
\penalty10000\vspace{10pt}\penalty10000\indent Digital language death is a real threat for many small and \hyperlink{gtUnderResourcedLanguage}{{\textit{under-resourced languages}}}. To use Kornai’s \hyperlink{Kornai}{(2013)} terms, the “digital moribund state” of languages is accelerated by the high social pressure to communicate in written digital form without the support of intuitive and useful text input methods. I contend that even if text input methods currently exist for a language, it is often the case that they are disruptive to the composition task, by inducing \hyperlink{gtCognitiveFriction}{{\textit{cognitive friction}}} to the digital text production process due to inadequately optimized keyboards. This disruption then serves as a barrier to creating vibrant communities using digitally mitigated communication. What is observed and reported in the literature can be cast into a two part typology. Both parts represent user based responses to cognitive friction around digital language use. Both options force writers to change their linguistic habit. When people encounter digital technology\protect\footnote[11]{{\leftskip0pt\parindent1em\raisebox{\baselineskip}[0pt]{\protect\hypertarget{nL1L2}{}} The assumption here is that the writer has the competency to communicate with their interlocutor in a variety of languages. We might assume that the writer has the option to write and speak in their L1 or L2 and that the recipient/interlocutor is equally skilled in both languages and has the same L1 and L2 ordering of languages.}}: (1) Some choose to write another language; this is influenced by their perception of language space of which \hyperlink{gtLocale}{{\textit{Locale}}} implementation is one major contributor. (2) Some choose to change the written representation of their language. In (1) the individual has switched linguistic contexts to overcome cognitive friction, whereas in (2) the user has chosen to adapt their linguistic performance to the imperialist demands of the technology. The results of either choice work to suppress the use of the under-resourced language in computer mitigated communication. ({\hyperlink{vCMC}{{CMC}}})\par{}\indent \hyperlink{Paterson}{Paterson (2014)} lays a foundation for the kinds of orthographical issues (language specific knowledge) encountered by minority-languages typists as they use keyboard layouts to create characters. My appeal in \hyperlink{Paterson}{(2014)} is to creators of resources for under-resourced languages to use industrial design principles \hyperlink{Vits153}{(Vitsœ  2012}, \hyperlink{rUSDHHS-UX}{U.S. Department of Health \& Human Services  2018)} in the creation of their language resource products\protect\footnote[12]{{\leftskip0pt\parindent1em\raisebox{\baselineskip}[0pt]{\protect\hypertarget{nMateriality}{}} Foundational to this appeal is the perspective that keyboard layouts are “things” even if they are virtual. Subsequently as “things” they have \hyperlink{gtMateriality}{{\textit{materiality}}} attributes and people relate to these layouts as they would other kinds of objects.}}. There, I argue that better design of text input solutions, which takes into account linguistic knowledge about a language's orthography, will increase the opportunities for digital vitality in an under-resourced language. \hyperlink{Paterson}{Paterson (2014)} does not, however, present a methodology for assessing the disruptive nature of a text input solution, so we still lack a metric for comparing the relative ease of typing on any given keyboard layout for a given language\protect\footnote[13]{{\leftskip0pt\parindent1em\raisebox{\baselineskip}[0pt]{\protect\hypertarget{nRelativeToWhat}{}} The relative scale is explained in chapter \hyperlink{cMethods}{4}.}}.\par{}\indent Language development practitioners and linguists working in many different languages have created many keyboard layouts \hyperlink{Thoms}{(Thomas \& Simons  2017)}. Each keyboard layout comes with a design history which embodies the design goals embraced by the designers. I find that there are three design patterns which are used to create keyboard layouts: (1) copy \& modify, (2) optimize for writing, based on use, (3) organize the layout based on some ordering\protect\footnote[14]{{\leftskip0pt\parindent1em\raisebox{\baselineskip}[0pt]{\protect\hypertarget{nSILMethod}{}} Within SIL International, my experience is the keyboard layouts are created using the copy \& modify strategy. This observation is supported with two bits of evidence: (1) SIL has historically had a significant corporate interest in the typesetting of works of "high moral significance". The translation and typesetting of these large works have historically taken a great many keystrokes across a variety of scripts serving a greater number of languages. SIL’s operational practice has been to “partner” with community members who are native users of the under-resourced languages. These individuals are either hired as employees or have a high degree of morally motivated affinity with the proposed outcomes of the translation and publication task. To these individuals, the “possible to type language” has been enough. For the purposes of SIL’s NGO funded goals, their corporate practice has been "whatever is needed for sacred text production", rather than to step back and design products which elevate the status, utility, or presence of these under-resourced languages in computer mediated communication via a wide social embrace of text input methods. (2) SIL keyboard layout designers acknowledge that they use the copy \& modify strategy (Video discussion in \hyperlink{rKeyman2018}{SIL International  2018a} {\hyperlink{vHMS}{{H:M:S}}} 3:12:00-3:15:00).}}. Each design pattern allows for areas of cognitive friction. However I find that many keyboard creators, designing keyboards layouts for languages which use the Latin script have not adequately accounted for the linguistic dissonance (a sub-type of cognitive friction) between the psycholinguistic and phonemic realities of a spoken language and a speaker's tactile experience in typing that language. This has created a gap between the “possible to write/type” \hyperlink{gtOrthography}{{\textit{orthography}}} and the “easy to write/type” \hyperlink{gtOrthography}{{\textit{orthography}}}\protect\footnote[15]{{\leftskip0pt\parindent1em\raisebox{\baselineskip}[0pt]{\protect\hypertarget{nTCFcomparison}{}} A presentation of the {\textit{Malinaltepec Meꞌphaa}} \textsquarebracketleft{}tcf\textsquarebracketright{} keyboard in section \hyperlink{sMappingLtoT}{2.5} demonstrates the finger work load distribution for their keyboard layout. The keyboard layout is one component in the hard-to-type vs. easy-to-type equation.}}.\par{}{\vspace{15pt}\XLingPaperneedspace{3\baselineskip}\noindent
\fontsize{13}{15.6}\selectfont \textbf{{\noindent
\raisebox{\baselineskip}[0pt]{\pdfbookmark[2]{{1.2 } Cognitive Friction and the Linguistic Enterprise}{sCongnitiveFriction}}\raisebox{\baselineskip}[0pt]{\protect\hypertarget{sCongnitiveFriction}{}}{1.2 }Cognitive Friction and the Linguistic Enterprise}}\markboth{Cognitive Friction and the Linguistic Enterprise}{Introduction}\XLingPaperaddtocontents{sCongnitiveFriction}}\par{}
\penalty10000\vspace{10pt}\penalty10000\indent A core tenant of this thesis is that typists encounter friction when they type. In one sense I mean friction just as physicists do – resistance. In a physical work there is physical friction. Typing is certainly physical. However, typing is also a mental activity and a linguistic activity (see detailed discussion in section \hyperlink{sLingusiticActivity}{2.4}). Therefore, I suggest that physical friction is not the only type of resistance that typists encounter. Typists encounter social friction, interaction friction, cognitive friction and emotional friction as they encounter the products which enable them to type. These various kinds of frictions culminate into very real perceptions about tasks and their difficulty and requirements. These perceptions about tasks lead to strategies to accomplish or circumnavigate the task, and these choices in turn drive communicative decisions which surface as linguistic expression.\par{}\indent \hyperlink{gtCognitiveFriction}{{\textit{Cognitive friction}}} is a term initially coined by Alan \hyperlink{rCooper1998}{Cooper (1998)} to describe the confusion a software user encounters when the state of the device they are interacting with behaves in an unexpected manner. This term has seen limited use outside of software design circles, but typing today, is for all practical purposes implemented by software. But more important than the software implementation of typing, the term has merit to human interactions outside of human-software interactions. For instance \hyperlink{rEhrensbergerDowMaureenSharonOBrien2015Ergon}{Ehrensberger-Dow \& O'Brien (2015)} has applied it to the professional task of translation and the mental workload involved in professional translation, also noting that software now plays a large part in the professional translation task. \hyperlink{rEhrensbergerDowMaureenSharonOBrien2015Ergon}{Ehrensberger-Dow \& O'Brien (2015)} frame their discussion as part of an overarching ergonomics discussion, where they situate mental workload and mental tasks in ergonomic frameworks. Ergonomics, when viewed as the stresses encountered by humans in interactions is much broader than just the optimization or accounting of tactile interactions. Ergonomics then can be applied to sociological human organization as a stress inducing factor as well as the emotional reaction that is produced by interactions (with devices or organizations). \hyperlink{rSzameitat2009}{Szameitat, Rummel, Szameitat \& Sterr (2009)} report on how small time delays in the interactions of computer devices cause emotional responses in users. To some it might be obvious that if something is not going as expected, humans have and emotional reactions. However, in the context of typing and in the context of product design, we should understand that time (both time-to-task-completion: duration; and time-as-continuitiy-of-activity; rhythm) evoke emotional responses which are part of the decision making process and impact computer – and language – users.\par{}\indent \hyperlink{rRehki2017}{Rehki (2017)}, in a blog post written for software product entrepreneurs, situates cognitive friction into a useful hierarchy (see figure \hyperlink{fHiarchyoffriction}{4}). He suggests that there are three types of friction: interaction friction, cognitive friction, and emotional friction. To summarize his points, the relationship between these three kinds of friction, we might say that there is friction at the level of creating a product user (i.e. getting a user to install software which enables the keyboard to operate on their computing system), there is friction at the level of cognition and mental work load (i.e. The effort that the user must go through to match their understanding of the sound system and orthography of their language to their understanding of the concepts presented in the software and to the functions the software performs). Finally there is the emotional friction which is encountered by users (i.e. The emotional responses to the success or failure of the connections between concepts and performance).\par{}\vspace{11pt plus 2pt minus 1pt}\setbox0=\vbox{\protect\centering \leavevmode
\vspace*{0pt}{\XeTeXpdffile "../Resources/Interaction_Friction3.pdf" scaled 700}\\[0pt]\protect\hypertarget{fHiarchyoffriction}{}\XLingPaperaddtocontents{fHiarchyoffriction}{\singlespacing
{Figure }{4.}{ Hierarchy of user friction\\}}}\box0\par{}\vspace{11pt plus 2pt minus 1pt}\indent Digital product entrepreneurs understand that they can loose users at any level of the hierarchy. The location within the hierarchy in which a user encounters friction is important to keep in mind as we make any language product. But perhaps more relevant to the topic of language development, we should ask the question, “how does the hierarchy of user friction and especially \hyperlink{gtCognitiveFriction}{{\textit{cognitive friction}}} map to linguistics and the theories of language use in the discipline of linguistics?” Both \hyperlink{rEhrensbergerDowMaureenSharonOBrien2015Ergon}{Ehrensberger-Dow \& O'Brien (2015)} and \hyperlink{rRehki2017}{Rehki (2017)} map \hyperlink{gtCognitiveFriction}{{\textit{cognitive friction}}} maps to cognitive load. Cognitive friction is the sudden increase in \hyperlink{gtCognitiveLoad}{{\textit{cognitive load}}}, or brain processing activity in working memory. \hyperlink{gtCognitiveLoad}{{\textit{Cognitive load}}} is a term in frequent use in the sub-field of psycholinguistics. I discuss the concept further in section \hyperlink{sCognitiveLoad}{1.2.2}. The other elements in hierarchy of user friction can be seen as mapping to language use behaviors in the following ways: Interaction friction can be seen as discourse interaction and challenges at this level as mis-matches in a discourse frame or the socio-cultural expected interaction. Emotional friction (particularly the avoidance thereof) seems to best be described as the kinds of accommodations interlocutors are described to make in a conversation as would be acknowledged by frameworks like communication accommodation theory.\par{}\indent \hyperlink{Thoms}{Thomas \& Simons (2017)} reference the thousands of keyboards available for minority languages. These are available via {\XLingPaperCharisZSILFontFamily{\textit{Keyman}}}\protect\footnote[16]{{\leftskip0pt\parindent1em\raisebox{\baselineskip}[0pt]{\protect\hypertarget{nAboutKeyman}{}} {\XLingPaperCharisZSILFontFamily{\textit{Keyman}}} was previously a licensed commercial product. In 2018 SIL International acquired and open sourced the application.}}. \hyperlink{rRaafi2011}{Raafi \& Nasir (2011)} classify Keyman as a transliteration tool because it captures characters generated by the OS's conversion of \hyperlink{gtKeyCodes}{{\textit{key codes}}}, captures the OS's interpreted characters, transliterates the characters according to a defined set of rules, and then passes the transliterations on to some other application. Keyman has been influential to minority language typist around the world. Marc Durdin, Keyman's creator presents a beautiful case for the ability of Keyman to reduce some areas of cognitive friction during the text input of some \hyperlink{gtComplexScript}{{\textit{complex scripts}}} and particularly addresses issues found with Khmer {\textsquarebracketleft{}khm\textsquarebracketright{}} typing \hyperlink{rKeyman2018}{(SIL International  2018a}:{\hyperlink{vHMS}{{H:M:S}}} 0:40:18-0:46:03; \hyperlink{rDurn2018}{Durdin  2018)}. I restate his major points here to set them in the cognitive friction framework. Khmer, according to \hyperlink{rSoketal2017}{Horton et al. (2017)}, has eight cases where the ordering of Unicode characters might be different but the visually expected output is the same. Marc along with a team at Keyman has created the Khmer Angkor "keyboard"\protect\footnote[17]{{\leftskip0pt\parindent1em\raisebox{\baselineskip}[0pt]{\protect\hypertarget{nKeyboardLinks}{}} Overview: \href{https://help.keyman.com/keyboard/khmer\_angkor/1.0.5/khmer\_angkor.php}{\textcolor[rgb]{0,0,0}{https://help.keyman.com/keyboard/khmer\_angkor/1.0.5/khmer\_angkor.php}}\par{}\indent Documentation: \href{https://help.keyman.com/keyboard/khmer\_angkor/1.0.5/KAK\_Documentation\_EN.pdf}{\textcolor[rgb]{0,0,0}{https://help.keyman.com/keyboard/khmer\_angkor/1.0.5/KAK\_Documentation\_EN.pdf}}}} which, when active, controls the output of the text and only allows the well formed version to be sent to the receiving application\protect\footnote[18]{{\leftskip0pt\parindent1em\raisebox{\baselineskip}[0pt]{\protect\hypertarget{nSmartUsers}{}} Or put another way, removes the penalty to users for bad keypress order, and simultaneously allows alternative preferences for keypress order. This is analogous in Latin scripts to spell checking.}}. This solves a real input challenge which is a source of cognitive friction, before the problem has a chance to become a point of emotional friction. This is visualized in the diagrams in table \hyperlink{ntKhmer}{2}.\par{}\vspace{11pt plus 2pt minus 1pt}\XLingPaperneedspace{3\baselineskip}\protect\hypertarget{ntKhmer}{}\XLingPaperaddtocontents{ntKhmer}{\protect\raggedright{\singlespacing
{Table }{2.}{  Construction of the Khmer autonym\\}}}\vspace{0pt}{\singlespacing
\hspace*{.25in}{\setcounter{footnote}{18}
\XLingPaperminmaxcellincolumn{direction}{\XLingPapermincola}{\hyperlink{ntKhmer}{2} (a) Normal Khmer reading direction is left to right. Well formed Unicode storage is left to right. Well formed character input would have the following memory order.}{\XLingPapermaxcola}{+0\tabcolsep}
\XLingPaperminmaxcellincolumn{}{\XLingPapermincola}{\vspace*{0pt}{\XeTeXpdffile "../Resources/Khmer1.pdf" scaled 600}}{\XLingPapermaxcola}{+0\tabcolsep}
\XLingPaperminmaxcellincolumn{rendering}{\XLingPapermincola}{\hyperlink{ntKhmer}{2} (b) Khmer is a complex script. At text input the glyphs change based on context. Some are even placed to the left of previously entered characters. Which is opposite the normal text rendering direction. Well formed characters would change visually as the user progressively types the characters.}{\XLingPapermaxcola}{+0\tabcolsep}
\XLingPaperminmaxcellincolumn{}{\XLingPapermincola}{\vspace*{0pt}{\XeTeXpdffile "../Resources/Khmer2.pdf" scaled 600}}{\XLingPapermaxcola}{+0\tabcolsep}
\XLingPaperminmaxcellincolumn{orderings}{\XLingPapermincola}{\hyperlink{ntKhmer}{2} (c) Alternative text entry orders exist. These orderings do cause the glyphs to appear correctly in when rendered in applications. However, these orders do not lead to well formed text strings in memory (or storage in documents).}{\XLingPapermaxcola}{+0\tabcolsep}
\XLingPaperminmaxcellincolumn{}{\XLingPapermincola}{\vspace*{0pt}{\XeTeXpdffile "../Resources/Khmer5.pdf" scaled 600}}{\XLingPapermaxcola}{+0\tabcolsep}
\XLingPaperminmaxcellincolumn{searches.}{\XLingPapermincola}{\hyperlink{ntKhmer}{2} (d) Visually the same but not equal. Varying the sequence of the Unicode characters produces unequal results when doing searches. In March 2018 Marc reported that the correct sequence generated 18,200,000 Google hits while the incorrect sequence generated 389,000 hits. The friction point to users is that both strings look identical, while providing the user different results in non-print media uses.}{\XLingPapermaxcola}{+0\tabcolsep}
\XLingPaperminmaxcellincolumn{}{\XLingPapermincola}{\vspace*{0pt}{\XeTeXpdffile "../Resources/Khmer3.pdf" scaled 600}}{\XLingPapermaxcola}{+0\tabcolsep}
\setlength{\XLingPaperavailabletablewidth}{433.62pt}
\setlength{\XLingPapertableminwidth}{\XLingPapermincola}
\setlength{\XLingPapertablemaxwidth}{\XLingPapermaxcola}
\XLingPapercalculatetablewidthratio{}
\XLingPapersetcolumnwidth{\XLingPapercolawidth}{\XLingPapermincola}{\XLingPapermaxcola}{-0\tabcolsep}\setcounter{footnote}{18}\singlespacing\vspace*{-3\baselineskip}
\begin{longtable}
[l]{@{}p{\XLingPapercolawidth}@{}}\toprule\multicolumn{1}{@{}p{\XLingPapercolawidth}@{}}{\hyperlink{ntKhmer}{2} (a) Normal Khmer reading direction is left to right. Well formed Unicode storage is left to right. Well formed character input would have the following memory order.}\\%
\multicolumn{1}{@{}>{\centering}p{\XLingPapercolawidth}@{}}{\vspace*{0pt}{\XeTeXpdffile "../Resources/Khmer1.pdf" scaled 600}}\\%
\multicolumn{1}{@{}p{\XLingPapercolawidth}@{}}{\hyperlink{ntKhmer}{2} (b) Khmer is a complex script. At text input the glyphs change based on context. Some are even placed to the left of previously entered characters. Which is opposite the normal text rendering direction. Well formed characters would change visually as the user progressively types the characters.}\\%
\multicolumn{1}{@{}>{\centering}p{\XLingPapercolawidth}@{}}{\vspace*{0pt}{\XeTeXpdffile "../Resources/Khmer2.pdf" scaled 600}}\\%
\multicolumn{1}{@{}p{\XLingPapercolawidth}@{}}{\hyperlink{ntKhmer}{2} (c) Alternative text entry orders exist. These orderings do cause the glyphs to appear correctly in when rendered in applications. However, these orders do not lead to well formed text strings in memory (or storage in documents).}\\%
\multicolumn{1}{@{}>{\centering}p{\XLingPapercolawidth}@{}}{\vspace*{0pt}{\XeTeXpdffile "../Resources/Khmer5.pdf" scaled 600}}\\%
\multicolumn{1}{@{}p{\XLingPapercolawidth}@{}}{\hyperlink{ntKhmer}{2} (d) Visually the same but not equal. Varying the sequence of the Unicode characters produces unequal results when doing searches. In March 2018 Marc reported that the correct sequence generated 18,200,000 Google hits while the incorrect sequence generated 389,000 hits\protect\footnote{{\leftskip0pt\parindent1em\raisebox{\baselineskip}[0pt]{\protect\hypertarget{nReplicateKhemrResults}{}} I replicated these results from the USA on 25. November 2018 using Wasta Linux (Ubuntu 16.04) and Opera 54.0.2952.64. I got 133,000,000 hits for the well formed string, and 398,000 hits for the incorrectly formed string.}}. The friction point to users is that both strings look identical, while providing the user different results in non-print media uses.}\\%
\multicolumn{1}{@{}>{\centering}p{\XLingPapercolawidth}@{}}{\vspace*{0pt}{\XeTeXpdffile "../Resources/Khmer3.pdf" scaled 600}}\\\bottomrule%
\end{longtable}
}
}\indent However other kinds of cognitive friction exist even with tools like Keyman. Performance in a task such as the duration of a task and rhythm a user encounters while working with a tool impact cognitive friction. Keyman does not attempt to solve problems which relate to rhythm or to the duration that a task takes. Reducing user friction at the task rhythm and task duration levels takes language specific knowledge.\par{}{\vspace{10pt}\XLingPaperneedspace{3\baselineskip}\noindent
\fontsize{13}{15.6}\selectfont \textit{{\noindent
\raisebox{\baselineskip}[0pt]{\pdfbookmark[3]{{1.2.1 } The place of digital communication}{sPlaceOfDigitalCommunication}}\raisebox{\baselineskip}[0pt]{\protect\hypertarget{sPlaceOfDigitalCommunication}{}}{1.2.1 }The {\XLingPaperCharisZSILFontFamily{\textit{place}}} of digital communication}}\markboth{The {\XLingPaperCharisZSILFontFamily{\textit{place}}} of digital communication}{Introduction}\XLingPaperaddtocontents{sPlaceOfDigitalCommunication}}\par{}
\penalty10000\vspace{10pt}\penalty10000\indent In {\XLingPaperCharisZSILFontFamily{\textit{Mental Spaces}}} \hyperlink{rFauconnierGilles.1985Menta}{(1985},\hyperlink{rFauconnierGilles1994Menta}{1994)} Fauconnier lays out a framework arguing for frames and how people interact with frames when making language choices. An analogy to frames might be a scene in a play or a script which is repeated with which a person is familiar. The person knows where the props are in the scene and how the characters are supposed to interact. In the set of all the scenes with which a person might be familiar, each one is encapsulated by a mental place or frame. As humans we can blend these frames or scripts, we can mix and match the elements to create new ones. But the analogy is that we have a "place" in our mental faculties for each frame. There are at least two ways in which Fauconnier's work applies to the discussion of keyboard layouts.\par{}\indent The first is through \hyperlink{gtSkeuomorphism}{{\textit{skeuomorphism}}}. The smart phone looks like it has a keyboard, even though it could have any visual design imagined. Designers of smart phones source design heritage from the physical keyboard. The mental model (or frame) of typing on a physical device maps to the mental model of typing on a smart phone. This connection is important because as a user's understanding of the text in put process changes it allows the user to input different text. The skeuomorphic connection is also important because as we make design choices in either smart phone typing or in physical keyboard layouts designers need to be aware of which capabilities map to the other technology and which capabilities become part of a users default frame. One example of this is how the \hyperlink{gtPressAndHoldKey}{{\textit{press-and-hold}}} key press action technique came into popular design usage. It was first brought to the masses via iOS and then later ported to MacOS as is illustrated in \hyperlink{ntPressAndHold}{3} .\par{}\vspace{11pt plus 2pt minus 1pt}\XLingPaperneedspace{3\baselineskip}\protect\hypertarget{ntPressAndHold}{}\XLingPaperaddtocontents{ntPressAndHold}{\protect\centering {\singlespacing
{Table }{3.}{  Press and hold on Apple products\\}}}\vspace{0pt}{\singlespacing
\hspace*{.25in}{
\XLingPaperminmaxcellincolumn{11.4}{\XLingPapermincola}{\textbf{iOS 11.4}}{\XLingPapermaxcola}{+0\tabcolsep}
\XLingPaperminmaxcellincolumn{MacOS}{\XLingPapermincolb}{\textbf{MacOS 10.13}}{\XLingPapermaxcolb}{+0\tabcolsep}
\XLingPaperminmaxcellincolumn{}{\XLingPapermincola}{\vspace*{0pt}{\XeTeXpicfile "../Resources/iOSLongPress2-300.png" scaled 700}}{\XLingPapermaxcola}{+0\tabcolsep}
\XLingPaperminmaxcellincolumn{}{\XLingPapermincolb}{\vspace*{0pt}{\XeTeXpicfile "../Resources/MacOSLongPress2-3001.png" scaled 700}}{\XLingPapermaxcolb}{+0\tabcolsep}
\setlength{\XLingPaperavailabletablewidth}{433.62pt}
\setlength{\XLingPapertableminwidth}{\XLingPapermincola+\XLingPapermincolb}
\setlength{\XLingPapertablemaxwidth}{\XLingPapermaxcola+\XLingPapermaxcolb}
\XLingPapercalculatetablewidthratio{}
\XLingPapersetcolumnwidth{\XLingPapercolawidth}{\XLingPapermincola}{\XLingPapermaxcola}{-0\tabcolsep}
\XLingPapersetcolumnwidth{\XLingPapercolbwidth}{\XLingPapermincolb}{\XLingPapermaxcolb}{-2\tabcolsep}\singlespacing\vspace*{-3\baselineskip}
\begin{longtable}
[c]{@{}>{\centering}p{\XLingPapercolawidth}>{\centering}p{\XLingPapercolbwidth}@{}}\toprule\multicolumn{1}{@{}>{\centering}p{\XLingPapercolawidth}}{\textbf{iOS 11.4}}&\multicolumn{1}{>{\centering}p{\XLingPapercolbwidth}@{}}{\textbf{MacOS 10.13}}\\%
\midrule\endhead \multicolumn{1}{@{}>{\centering}p{\XLingPapercolawidth}}{\vspace*{0pt}{\XeTeXpicfile "../Resources/iOSLongPress2-300.png" scaled 700}}&\multicolumn{1}{>{\centering}p{\XLingPapercolbwidth}@{}}{\vspace*{0pt}{\XeTeXpicfile "../Resources/MacOSLongPress2-3001.png" scaled 700}}\\\bottomrule%
\end{longtable}
}
}\noindent When presented with text input and developing a frame unique to their mother tongue language, users will encounter various kinds of friction. When the performance of their text input mechanism does not perform within the context of a frame with with they are familiar (perhaps the text input process of the {\hyperlink{vLWC}{{LWC}}}), then the preferred experience rapidly becomes the default. Many people who are now presented the opportunity to write via smart phones would never had the opportunity to type on a computer. It means that their frame is being defined by their smart phone interactions.\par{}\indent The second way in which Fauconnier's work is important is the mental place of computers and the kinds of spaces in which {\hyperlink{vCMC}{{CMC}}} exists. There is then a second set of frames which includes the places where language is used and the kinds of spaces where language exists. These language frames in a multilingual context might have one or more languages attached to them, but based on my observations in Northwest Nigeria, frames have one language attached, but the defining nature of the frame is not the language. Academics have been reporting on {\hyperlink{vCMC}{{CMC}}} in Africa since the late 1990s . Earlier in the narrative of Africa's Internet connectivity we see reports like \hyperlink{rAdomi2007Overn}{(Adomi  2007}, \hyperlink{rSairosse2004Useof}{Sairosse et al.  2004)} where the computer exists in a very public space. The computer is rare and is symbolic of both new, prestige, and communication with foreigners and extra-national sources of knowledge. It is the coalescence of three mental spaces. Spaces which have corollaries with existing frames which have designated language choices already assigned. So, in part the computer does not need to be multi-lingual be cause the computer needs to be in the language that exists in those frames, those mental spaces.\par{}\indent With the rise of the mobile device, and the smart-phone digital devices and {\hyperlink{vCMC}{{CMC}}} became nearly ubiquitous \hyperlink{rKaigwaMark2017FromC}{(Kaigwa  2017)}. Now the {\hyperlink{vCMC}{{CMC}}} and communication device has entered new frames; frames which traditionally had different languages associated with them. Now even more than ever the device becomes an imperialistic dictator of conformity to its immutable design. In a way, those who choose option one from the typology presented in section \hyperlink{sKBD-underresourced-languages}{1.1} (choosing to write another language) can are choosing to be accommodating in a similar way as to what would be described with Communication Accommodation Theory \hyperlink{r}{ ()}. In accommodation theory, people adjust their language choices based on a mental space. Sometimes this mental space is a physical place, such as the market, or the home. Sometimes the mental space determined by meeting someone from a particular mental space. Such as meeting someone from my home, or from the market. In this way accommodation takes place with respect to the frame of reference and the known relationship between the individuals. In this way, the digital device itself defines a space and dictates the language used.\par{}{\vspace{10pt}\XLingPaperneedspace{3\baselineskip}\noindent
\fontsize{13}{15.6}\selectfont \textit{{\noindent
\raisebox{\baselineskip}[0pt]{\pdfbookmark[3]{{1.2.2 } Cognitive Load}{sCognitiveLoad}}\raisebox{\baselineskip}[0pt]{\protect\hypertarget{sCognitiveLoad}{}}{1.2.2 }Cognitive Load}}\markboth{Cognitive Load}{Introduction}\XLingPaperaddtocontents{sCognitiveLoad}}\par{}
\penalty10000\vspace{10pt}\penalty10000\indent Cognitive load important in four ways\par{}{\parskip .5pt plus 1pt minus 1pt
                    
\vspace{\baselineskip}

{\setlength{\XLingPapertempdim}{\XLingPapersingledigitlistitemwidth+\parindent{}}\leftskip\XLingPapertempdim\relax
\interlinepenalty10000
\XLingPaperlistitem{\parindent{}}{\XLingPapersingledigitlistitemwidth}{1.}{The phyio-motor capacity of spelling and typing}}
{\setlength{\XLingPapertempdim}{\XLingPapersingledigitlistitemwidth+\parindent{}}\leftskip\XLingPapertempdim\relax
\interlinepenalty10000
\XLingPaperlistitem{\parindent{}}{\XLingPapersingledigitlistitemwidth}{2.}{The operational context of the device}}
{\setlength{\XLingPapertempdim}{\XLingPapersingledigitlistitemwidth+\parindent{}}\leftskip\XLingPapertempdim\relax
\interlinepenalty10000
\XLingPaperlistitem{\parindent{}}{\XLingPapersingledigitlistitemwidth}{3.}{The choice of language and negotiation of meaning in the digital device.}}
{\setlength{\XLingPapertempdim}{\XLingPapersingledigitlistitemwidth+\parindent{}}\leftskip\XLingPapertempdim\relax
\interlinepenalty10000
\XLingPaperlistitem{\parindent{}}{\XLingPapersingledigitlistitemwidth}{4.}{The interpretation of symbols and their relation to sounds.}}
{\setlength{\XLingPapertempdim}{\XLingPapersingledigitlistitemwidth+\parindent{}}\leftskip\XLingPapertempdim\relax
\interlinepenalty10000
\XLingPaperlistitem{\parindent{}}{\XLingPapersingledigitlistitemwidth}{5.}{The mental connection of the storage of words and their expression in speech, and typing, perception in hearing, and reading.}}
\vspace{\baselineskip}
}\indent Now enters the scene Cognitive load. Functional load might be characterized as looking at the relative utility of one component in a system, whereas cognitive load would look a the stress put on the whole system. How do we compare functional load to cognitive load? If we are looking at the whole system then what is the throughput of the whole system and what is the throughput of each individual part? Within this area is then the information through-put rate. This also controls motor movement nerves. This is the foundation for Fitts law and Shannon's Theorem.\par{}\indent MacKenzie, I. Scott. 2013. A note on the validity of the Shannon formulation for Fitts’ index of difficulty. Open Journal of Applied Sciences 3.6: 360-8.\par{}\indent Fitts, Paul M. 1954. The information capacity of the human motor system in controlling the amplitude of movement. Journal of Experimental Psychology 47.6: 381-91.\par{}\indent Fitts, Paul M. 1992. The Information Capacity of the Human Motor System in Controlling the Amplitude of Movement. Journal of Experimental Psychology: General 121.3: 262-9.\par{}\indent Schiffer, Marcelo. 1991. Shannon's information is not entropy. Physics Letters A 154.7–8: 361-5.\par{}\indent Stam, A. J. 1959. Some inequalities satisfied by the quantities of information of Fisher and Shannon. Information and Control 2.2: 101-12.\par{}\indent Levitin, Lev B. and Zeev Reingold. 1994. Entropy of natural languages: Theory and experiment. Chaos, Solitons \& Fractals 4.5: 709-43.\par{}\indent MacKenzie, I. Scott. 1989. A note on the information-theoretic basis for Fitts' law. Journal of Motor Behavior 21.3: 323-30.\par{}\indent Seow, Steven C. 2005. Information Theoretic Models of HCI: A Comparison of the Hick-Hyman Law and Fitts’ Law. Human–Computer Interaction 20315-52.\par{}\indent Soukoreff, R. William and I. Scott MacKenzie. 2009. An Informatic Rationale for the Speed-Accuracy Trade-Off. Proceedings of the 2009 IEEE International Conference on Systems, Man, and Cybernetics (San Antonio, TX, USA - October 2009), 2890-6. Piscataway, NJ, USA: IEEE Press.\par{}{\vspace{15pt}\XLingPaperneedspace{3\baselineskip}\noindent
\fontsize{13}{15.6}\selectfont \textbf{{\noindent
\raisebox{\baselineskip}[0pt]{\pdfbookmark[2]{{1.3 } Linguistic orientation}{sDefinitions}}\raisebox{\baselineskip}[0pt]{\protect\hypertarget{sDefinitions}{}}{1.3 }Linguistic orientation}}\markboth{Linguistic orientation}{Introduction}\XLingPaperaddtocontents{sDefinitions}}\par{}
\penalty10000\vspace{10pt}\penalty10000{\vspace{10pt}\XLingPaperneedspace{3\baselineskip}\noindent
\fontsize{13}{15.6}\selectfont \textit{{\noindent
\raisebox{\baselineskip}[0pt]{\pdfbookmark[3]{{1.3.1 } Tell of characters}{sTerms}}\raisebox{\baselineskip}[0pt]{\protect\hypertarget{sTerms}{}}{1.3.1 }Tell of characters}}\markboth{Tell of characters}{Introduction}\XLingPaperaddtocontents{sTerms}}\par{}
\penalty10000\vspace{10pt}\penalty10000\indent In this section I tell the story of characters. Understanding characters is foundational to being able to count them. I give a overview of some of the technical terminology I use. Many terms have subtitle, but significant differences and therefore may appear to a casual reader to be ambiguous. Several terms have different meanings, due to their author's academic field of specialty. Other times terms have not been judiciously chosen within the cross-disciplinary nature of text input, text production, or orthography research. I have found the glossary from \hyperlink{NRSIGlossary}{Lyons (2001)} influential\protect\footnote[20]{{\leftskip0pt\parindent1em\raisebox{\baselineskip}[0pt]{\protect\hypertarget{nNRSIGlossary}{}} Some definitions from the glossary in \hyperlink{NRSIGlossary}{Lyons (2001)} are replicated in this work. I provide a general acknowledgement here instead of a citation to each location in their web based glossary or page number in Lyons (2001). Similarly, I have found the online Glossary of Unicode Terms very helpful. It can be found at: \href{http://www.unicode.org/glossary/}{\textcolor[rgb]{0,0,0}{http://www.unicode.org/glossary}}.}}. The terms discussed here, along with other technical terms introduced throughout this work, are replicated in a glossary which appears as appendix \hyperlink{DefinedTerms}{B}.\par{}\indent Three disciplines donate ideas and terminology to describe characters: Typography (drawing letters), Computer Science (programming), and Linguistics (orthography, graphemics, and graphology). One of the most confusing of terms in the present area of research is the term {\XLingPaperCharisZSILFontFamily{\textit{character}}}. The confusion is obvious when we understand that the term {\XLingPaperCharisZSILFontFamily{\textit{character}}} is used both for an idea and an instantiation of that idea, or as the case is with written languages, many instantiations of an idea. Written language users from a \hyperlink{gtLatinScript}{{\textit{Latin script}}} background often think of words in terms of the ʻlettersʼ or graphical units from which they are formed. I follow \hyperlink{rConstableonCharacters}{Constable (2001:10)} and call these \hyperlink{gtOrthographicCharacter}{{\textit{orthographical characters}}}\protect\footnote[21]{{\leftskip0pt\parindent1em\raisebox{\baselineskip}[0pt]{\protect\hypertarget{nExtendedGraphemeClusters}{}} However, as discussed later, if these orthographical characters are made up of more than one graphical character then Unicode calls these \hyperlink{gtGraphemeCluster}{{\textit{Grapheme Clusters}}}. Unicode even has a special algorithm for allowing text software to interact with these clusters as single units called the Extended Grapheme Cluster algorithm. For details consult \hyperlink{rDavisetal2018}{Davis et al. (2018)}.}}. They are orthographical characters because \hyperlink{gtLanguage}{{\textit{language}}}\protect\footnote[22]{{\leftskip0pt\parindent1em\raisebox{\baselineskip}[0pt]{\protect\hypertarget{nSub-language}{}} To follow \hyperlink{Constable}{Constable (2002)}, as is illustrated in figure \hyperlink{fWritingSystemsNoCommnet}{9}, one must include the terminology “sub-language variant”. This seems reasonable if we can find a reasonable way to standardize what the sub-language variants are. As far as I know there is no standard with a wide degree of acceptance which has published a list of the sub-language variants. I suppose that the model presented in \hyperlink{Constable}{Constable (2002)} was designed to take into account standard practices in the technology industry which include separating languages like English as spoke in Great Britain (en-gb) and English as spoken in the USA (en-us).}} specific information or perceptions are applied to the character by readers. Readers of a language have a socially learned notion about what a character is – based on their understanding of their language and writing system. For instance, many English speakers will think that {\XLingPaperCambriaZMathFontFamily{\textup{\textmd{⟨ {\XLingPaperCharisZSILFontFamily{\textup{\textup{\textmd{c, q, r, x}}}}} ⟩}}}} and {\XLingPaperCambriaZMathFontFamily{\textup{\textmd{⟨ {\XLingPaperCharisZSILFontFamily{\textup{\textup{\textmd{z}}}}} ⟩}}}} are consonants. However, for Natügu {\textsquarebracketleft{}ntu\textsquarebracketright{}} readers these are vowels \hyperlink{rBoerger1996}{(Boerger  1996)}. Therefore a more general, cross-language (and cross-script) model of characters is required if we are going to refer to characters by their visual attributes or their computational properties.\par{}\indent Unicode provides a model where we can talk about the visual attributes and the computational properties of characters without the orthographical assumptions\protect\footnote[23]{{\leftskip0pt\parindent1em\raisebox{\baselineskip}[0pt]{\protect\hypertarget{nUnicodeModel}{}} For more details on the Unicode character property model see Unicode Technical Report \#23 \hyperlink{rWhistler}{(Whistler \& Freytag  2015)}.\par{}\indent Writing Systems Specialists also need a framework where we can talk about orthographic specific assumptions. As far as I know this does not exist. {\hyperlink{vBCP}{{BCP}}}47 does allow one to indicate which orthography is being used in the encoding of a text, but it does not define or identify where to define the identified orthography.}}. Within Unicode, all the components needed to create the orthographical characters from the worldʼs writing systems are arranged in a giant table\protect\footnote[24]{{\leftskip0pt\parindent1em\raisebox{\baselineskip}[0pt]{\protect\hypertarget{nUnicodeCharacter}{}} Unicode did not start out with all the world's characters, but rather as a framework to bring them into Unicode. So, there does exist a divide between characters in Unicode and characters which might or do exist and have not yet been added to Unicode. Therefore one could say there is such a thing as a non-Unicode character, and a Unicode character, where \hyperlink{gtUnicodeCharacter}{{\textit{Unicode characters}}} are those characters which have been added to the Unicode standard. One could also say that a character encoded in some other standard is not a Unicode character.}}. Each character is given an identifying reference place in that table called a \hyperlink{gtCodePoint}{{\textit{code point}}}. Conceptually, \hyperlink{gtGraphicalCharacters}{{\textit{graphical characters}}} in Unicode are a sub-set of characters and have three components. They contain a graphical component called a \hyperlink{gtGlyph}{{\textit{glyph}}} and a database \hyperlink{gtCodePoint}{{\textit{code point}}}. The code point is assigned several \hyperlink{gtCharacterProperties}{{\textit{properties}}} which are then recorded in the \hyperlink{gtUCD}{{\textit{Unicode Character Database}}} ({\hyperlink{vUCD}{{UCD}}}). I illustrate this relationship in figure \hyperlink{fConceptualCharacter}{5}.\par{}\vspace{11pt plus 2pt minus 1pt}\setbox0=\vbox{\protect\centering \leavevmode
\vspace*{0pt}{\XeTeXpdffile "../Resources/Unicode-Character-Composition.pdf" scaled 750}\\[0pt]\protect\hypertarget{fConceptualCharacter}{}\XLingPaperaddtocontents{fConceptualCharacter}{\singlespacing
{Figure }{5.}{ Conceptual and graphical components of a Unicode character\\}}}\box0\par{}\vspace{11pt plus 2pt minus 1pt}\indent Graphical characters have an abstract notion which may hold several graphical components together. This abstract notion is called an \hyperlink{gtAbstractUnicodeCharacter}{{\textit{abstract character}}}. An abstract character may have zero or more glyphs associated with them. For instance the character for “space” often used to form word breaks is not normally assigned a glyph.\par{}\indent There are several ways that an abstract character can have multiple glyphs associated with it. Design variation is the first way. Strictly speaking the lower case letter {\XLingPaperCambriaZMathFontFamily{\textup{\textmd{⟨ {\XLingPaperCharisZSILFontFamily{\textup{\textup{\textmd{a}}}}} ⟩}}}} is a different glyph than the italic \hyperlink{gtLowerCase}{{\textit{lower case}}} letter {\XLingPaperCambriaZMathFontFamily{\textup{\textmd{⟨ {\XLingPaperCharisZSILFontFamily{\textit{a}}} ⟩}}}}. Both the italic version and the non-italic version are connected to the same code point in Unicode but are different \hyperlink{gtGlyph}{{\textit{glyphs}}}. Fonts allow us to play with the perceptual properties of an abstract character. Both versions of the letter {\XLingPaperCambriaZMathFontFamily{\textup{\textmd{⟨ {\XLingPaperCharisZSILFontFamily{\textup{\textup{\textmd{a}}}}} ⟩}}}} are different instances of the same \hyperlink{gtAbstractUnicodeCharacter}{{\textit{abstract character}}}. Formally, the term \hyperlink{gtGlyph}{{\textit{glyph}}} is defined in ISO 15924:\hyperlink{rISO15924}{2004} as: “recognizable abstract graphic symbol which is independent of any specific design \textsquarebracketleft{}SOURCE: ISO/IEC 9541-1:\hyperlink{rISO9541-1}{1991}\textsquarebracketright{}”\protect\footnote[25]{{\leftskip0pt\parindent1em\raisebox{\baselineskip}[0pt]{\protect\hypertarget{nSpecificDesign}{}} \hyperlink{gtSpecificDesign}{{\textit{Specific design}}} in this case is understood to be font implemented features which might be things like stroke width, ascender and descender ratios, spacing, serif implementation, font-face (italic, or oblique), font weight, etc.}}. The second way multiple glyphs can be associated is through \hyperlink{gtMultiGraph}{{\textit{multi-graphs}}}. Individual sounds of a language may be represented in an orthography by a single visual element or by a set of visual elements. Sets of visual elements are called multi-graphs. There are three major types of \hyperlink{gtMultiGraph}{{\textit{multi-graphs}}} in non-\hyperlink{gtLogographicScripts}{{\textit{Logographic scripts}}}\protect\footnote[26]{{\leftskip0pt\parindent1em\raisebox{\baselineskip}[0pt]{\protect\hypertarget{nLogographicScriptExamples}{}} Two examples of Logographic scripts include Chinese and Japanese uses of Hanzi and Egyptian Hieroglyphs.}}: (1) \hyperlink{gtString}{{\textit{strings}}} of base characters, (2) \hyperlink{gtDiacritic}{{\textit{diacritic}}} modification of base characters and (3) combinations of (1) and (2). The German {\textsquarebracketleft{}deu\textsquarebracketright{}} \hyperlink{gtTriGraph}{{\textit{tri-graph}}} {\XLingPaperCambriaZMathFontFamily{\textup{\textmd{⟨ {\XLingPaperCharisZSILFontFamily{\textup{\textup{\textmd{sch}}}}} ⟩}}}} is an example of a string. The German orthography uses {\XLingPaperCambriaZMathFontFamily{\textup{\textmd{⟨ {\XLingPaperCharisZSILFontFamily{\textup{\textup{\textmd{sch}}}}} ⟩}}}} to represent a voiceless postalveolar fricative {\XLingPaperCharisZSILFontFamily{/ ʃ /}} . Another case is the English \hyperlink{gtMultiGraph}{{\textit{di-graph}}} {\XLingPaperCambriaZMathFontFamily{\textup{\textmd{⟨ {\XLingPaperCharisZSILFontFamily{\textup{\textup{\textmd{ch}}}}} ⟩}}}} which is used to represent the voiceless palato-alveolar affricate {\XLingPaperCharisZSILFontFamily{/ t͡ʃ /}}. Figure \hyperlink{fMultiGraphs}{6} demonstrates three types of tri-graphs, the first is a triple width while the second is a double width trigraph and the third is a single width. Some tone languages, such as Moore {\textsquarebracketleft{}mos\textsquarebracketright{}} with long vowels use the the double width trigraph.\par{}\vspace{11pt plus 2pt minus 1pt}\setbox0=\vbox{\protect\centering \leavevmode
\vspace*{0pt}{\XeTeXpdffile "../Resources/3trigraphs-var-width-centered.pdf" scaled 750}\\[0pt]\protect\hypertarget{fMultiGraphs}{}\XLingPaperaddtocontents{fMultiGraphs}{\singlespacing
{Figure }{6.}{ Visual demonstration of Multi-graphs – three tri-graphs\\}}}\box0\par{}\vspace{11pt plus 2pt minus 1pt}\noindent In order to talk more succinctly about multi-graphs with diacritics we need to talk about the parts of the orthographic characters which comprise the multi-graph. Figure \hyperlink{fComplexCharacter}{7} shows a single width di-graph which has as its base what might be called lower case {\XLingPaperCambriaZMathFontFamily{\textup{\textmd{⟨ {\XLingPaperCharisZSILFontFamily{\textup{\textup{\textmd{a}}}}} ⟩}}}}. This base is a \hyperlink{gtGlyph}{{\textit{glyph}}} and its diacritic is also a glyph. Together they form a \hyperlink{gtComplexCharacter}{{\textit{complex character}}} which if the orthography description indicates, is considered a single orthographic character, consisting of two glyphs.\par{}\vspace{11pt plus 2pt minus 1pt}\setbox0=\vbox{\protect\centering \leavevmode
\vspace*{0pt}{\XeTeXpdffile "../Resources/ComplexCharacter.pdf" scaled 750}\\[0pt]\protect\hypertarget{fComplexCharacter}{}\XLingPaperaddtocontents{fComplexCharacter}{\singlespacing
{Figure }{7.}{ Conceptual breakdown of complex characters\\}}}\box0\par{}\vspace{11pt plus 2pt minus 1pt}\noindent For legacy backwards compatibility, Unicode allows us to represent some complex characters using single code points using an encoding called Normialization Form C ({\hyperlink{vNFC}{{NFC}}})\protect\footnote[27]{{\leftskip0pt\parindent1em\raisebox{\baselineskip}[0pt]{\protect\hypertarget{nNFC-NFD}{}} This presentation is an over simplification of {\hyperlink{vNFC}{{NFC}}} and {\hyperlink{vNFD}{{NFD}}}, for a fuller understanding please constult the Unicode documentation. {\hyperlink{vNFC}{{NFC}}} is an encoding format that "builds" up into as few code points as possible all characters, whereas {\hyperlink{vNFD}{{NFD}}} breaks down all graphical characters into their most basic constituent parts.}}. Sometimes two or more codepoints are required or opted for to represent these complex characters, as no \hyperlink{gtCompositeCharacter}{{\textit{composite character}}} exists\protect\footnote[28]{{\leftskip0pt\parindent1em\raisebox{\baselineskip}[0pt]{\protect\hypertarget{nCanonically}{}} Unicode dictates that these sequences be canonically equivalent. That is, applications must treat both the NFC and the NFD forms as equivalent.}}. When this is done, it may appear that a document has a mix of \hyperlink{gtCompositeCharacter}{{\textit{pre-composed}}} and composed characters, but this is not the case. It is just that by design there are no composite forms for some sequences.\par{}\vspace{11pt plus 2pt minus 1pt}\setbox0=\vbox{\protect\centering \leavevmode
\vspace*{0pt}{\XeTeXpdffile "../Resources/GentiumAlt3.pdf" scaled 700}\\[0pt]\protect\hypertarget{fHelvetica-CSIL}{}\XLingPaperaddtocontents{fHelvetica-CSIL}{\singlespacing
{Figure }{8.}{ Demonstrating a vareity of font, and normalization varieties\\}}}\box0\par{}\vspace{11pt plus 2pt minus 1pt}\noindent Eastern Dan's orthography employs diacritics. Some must be \hyperlink{gtComposedCharacter}{{\textit{composed characters}}}, some may be composed or composite characters. Table \hyperlink{tEasternDanCharactersdiacritics}{19} shows which orthographic characters can be represented as composed or pre-composed, and which ones must be composed.\par{}\noindent Within the context of computing {\hyperlink{vNFC}{{NFC}}} and {\hyperlink{vNFD}{{NFD}}} are important to note because they can impact \hyperlink{gtString}{{\textit{string}}} length. A string with a composite character will be shorter than a string with a composed character. And if the individual counting the string length is looking to verify their string length by counting orthographical characters then they are very likely to be confused if they count a string with composed characters.\par{}\noindent String length is important in the discussion of text input for at least two reasons: (1) some keyboard layout editing tools have limits on the number of bytes they can input at the same time.\protect\footnote[29]{{\leftskip0pt\parindent1em\raisebox{\baselineskip}[0pt]{\protect\hypertarget{nMSKLC}{}} One such tool is {\hyperlink{vMSKLC}{{MSKLC}}}, which has a limit on how many bytes may be indicated to be sent to the operating system based on a single key press. Because the bytes are limited, the result is that the number of Unicode code points are limited. MSKLC can be downloaded here: \href{https://www.microsoft.com/en-us/download/details.aspx?id=22339}{\textcolor[rgb]{0,0,0}{https://www.microsoft.com/en-us/download/details.aspx?id=22339}}.}} (2) It is important to make sure that the correct composition for a character, based on an orthography, is returned when a key press occurs.\par{}\noindent One of the pressing issues in manual text input is {\XLingPaperCharisZSILFontFamily{\textit{how can languages which use diacritics efficiently input those diacritics?}}} Before we can start to look at that question we need to grapple with four more concepts: (1) What is a diacritic? (2) What is a grapheme? (3) What is the relationship between a \hyperlink{gtWritingSystem}{{\textit{writing system}}} and an \hyperlink{gtOrthography}{{\textit{orthography}}}? (4) What is good text input?\par{}\noindent A diacritic is surprisingly a hard thing to define. Some authors such as \hyperlink{rCrystal2008}{Crystal (2008:142)} and \hyperlink{rApple}{Apple (2018)} prefer to define \hyperlink{gtDiacritic}{{\textit{diacritic}}} according to its function. For example, \hyperlink{rApple}{Apple (2018)} in its TrueType documentation says the following:\par{}\XLingPaperblockquote{.25in}{{\singlespacing
\vspace{-1.3\baselineskip}Diacritical marks are any mark added to a glyph or a combination of glyphs in a particular script that creates a new phonetic value that is different from the unmarked glyph or glyphs. Diacritical marks include accents, umlauts, tildes, cedillas, or dots that signal the presence of vowels (such as in Hebrew or Arabic).\\Languages such as Arabic need this feature because there are two forms of the written language: with vowels, for children's books; and without vowels, because adults have learned to read the language without the vowels.\par{}}}{\baselineskip}{\baselineskip}\noindent The Apple definition leaves out some important functions such as the Thai script's tone marks which are shown in table \hyperlink{ntThaiTone}{4}. But as \hyperlink{rVitrano-Wilson2018}{Vitrano-Wilson et al. (2018)} point out, diacritics can move between scripts and their functions can change in their moves. Therefore while,\par{}\vspace{11pt plus 2pt minus 1pt}\XLingPaperneedspace{3\baselineskip}\protect\hypertarget{ntThaiTone}{}\XLingPaperaddtocontents{ntThaiTone}{\protect\raggedright{\singlespacing
{Table }{4.}{  Thai Tone Marks\\}}}\vspace{0pt}{\singlespacing
\hspace*{.25in}{
\XLingPaperminmaxcellincolumn{Character}{\XLingPapermincola}{\textbf{Character Name}}{\XLingPapermaxcola}{+0\tabcolsep}
\XLingPaperminmaxcellincolumn{Glyph}{\XLingPapermincolb}{\textbf{Glyph}}{\XLingPapermaxcolb}{+0\tabcolsep}
\XLingPaperminmaxcellincolumn{Point}{\XLingPapermincolc}{\textbf{Code Point}}{\XLingPapermaxcolc}{+0\tabcolsep}
\XLingPaperminmaxcellincolumn{CHARACTER}{\XLingPapermincola}{THAI CHARACTER MAI EK (THAI TONE MAI EK)}{\XLingPapermaxcola}{+0\tabcolsep}
\XLingPaperminmaxcellincolumn{◌่}{\XLingPapermincolb}{\XLingPaperCharisZSILFontFamily{\fontsize{24}{28.799999999999997}\selectfont ◌{\XLingPaperTavirajFontFamily{\fontsize{24}{28.799999999999997}\selectfont ่}}}}{\XLingPapermaxcolb}{+0\tabcolsep}
\XLingPaperminmaxcellincolumn{U+0E48}{\XLingPapermincolc}{U+0E48}{\XLingPapermaxcolc}{+0\tabcolsep}
\XLingPaperminmaxcellincolumn{CHARACTER}{\XLingPapermincola}{THAI CHARACTER MAI THO (THAI TONE MAI THO)}{\XLingPapermaxcola}{+0\tabcolsep}
\XLingPaperminmaxcellincolumn{◌้}{\XLingPapermincolb}{\XLingPaperCharisZSILFontFamily{\fontsize{24}{28.799999999999997}\selectfont ◌{\XLingPaperTavirajFontFamily{\fontsize{24}{28.799999999999997}\selectfont ้}}}}{\XLingPapermaxcolb}{+0\tabcolsep}
\XLingPaperminmaxcellincolumn{U+0E49}{\XLingPapermincolc}{U+0E49}{\XLingPapermaxcolc}{+0\tabcolsep}
\XLingPaperminmaxcellincolumn{CHARACTER}{\XLingPapermincola}{THAI CHARACTER MAI TRI (THAI TONE MAI TRI)}{\XLingPapermaxcola}{+0\tabcolsep}
\XLingPaperminmaxcellincolumn{◌๊}{\XLingPapermincolb}{\XLingPaperCharisZSILFontFamily{\fontsize{24}{28.799999999999997}\selectfont ◌{\XLingPaperTavirajFontFamily{\fontsize{24}{28.799999999999997}\selectfont ๊}}}}{\XLingPapermaxcolb}{+0\tabcolsep}
\XLingPaperminmaxcellincolumn{U+0E4A}{\XLingPapermincolc}{U+0E4A}{\XLingPapermaxcolc}{+0\tabcolsep}
\XLingPaperminmaxcellincolumn{CHARACTER}{\XLingPapermincola}{THAI CHARACTER MAI CHATTAWA (THAI TONE MAI CHATTAWA)}{\XLingPapermaxcola}{+0\tabcolsep}
\XLingPaperminmaxcellincolumn{◌๋}{\XLingPapermincolb}{\XLingPaperCharisZSILFontFamily{\fontsize{24}{28.799999999999997}\selectfont ◌{\XLingPaperTavirajFontFamily{\fontsize{24}{28.799999999999997}\selectfont ๋}}}}{\XLingPapermaxcolb}{+0\tabcolsep}
\XLingPaperminmaxcellincolumn{U+0E4B}{\XLingPapermincolc}{U+0E4B}{\XLingPapermaxcolc}{+0\tabcolsep}
\setlength{\XLingPaperavailabletablewidth}{433.62pt}
\setlength{\XLingPapertableminwidth}{\XLingPapermincola+\XLingPapermincolb+\XLingPapermincolc}
\setlength{\XLingPapertablemaxwidth}{\XLingPapermaxcola+\XLingPapermaxcolb+\XLingPapermaxcolc}
\XLingPapercalculatetablewidthratio{}
\XLingPapersetcolumnwidth{\XLingPapercolawidth}{\XLingPapermincola}{\XLingPapermaxcola}{-0\tabcolsep}
\XLingPapersetcolumnwidth{\XLingPapercolbwidth}{\XLingPapermincolb}{\XLingPapermaxcolb}{-2\tabcolsep}
\XLingPapersetcolumnwidth{\XLingPapercolcwidth}{\XLingPapermincolc}{\XLingPapermaxcolc}{-2\tabcolsep}\singlespacing\vspace*{-3\baselineskip}
\begin{longtable}
[l]{@{}p{\XLingPapercolawidth}p{\XLingPapercolbwidth}p{\XLingPapercolcwidth}@{}}\toprule\multicolumn{1}{@{}p{\XLingPapercolawidth}}{\textbf{Character Name}}&\multicolumn{1}{p{\XLingPapercolbwidth}}{\textbf{Glyph}}&\multicolumn{1}{p{\XLingPapercolcwidth}@{}}{\textbf{Code Point}}\\%
\midrule\endhead \multicolumn{1}{@{}p{\XLingPapercolawidth}}{THAI CHARACTER MAI EK (THAI TONE MAI EK)}&\multicolumn{1}{>{\centering}p{\XLingPapercolbwidth}}{\XLingPaperCharisZSILFontFamily{\fontsize{24}{28.799999999999997}\selectfont ◌{\XLingPaperTavirajFontFamily{\fontsize{24}{28.799999999999997}\selectfont ่}}}}&\multicolumn{1}{p{\XLingPapercolcwidth}@{}}{U+0E48}\\%
\multicolumn{1}{@{}p{\XLingPapercolawidth}}{THAI CHARACTER MAI THO (THAI TONE MAI THO)}&\multicolumn{1}{>{\centering}p{\XLingPapercolbwidth}}{\XLingPaperCharisZSILFontFamily{\fontsize{24}{28.799999999999997}\selectfont ◌{\XLingPaperTavirajFontFamily{\fontsize{24}{28.799999999999997}\selectfont ้}}}}&\multicolumn{1}{p{\XLingPapercolcwidth}@{}}{U+0E49}\\%
\multicolumn{1}{@{}p{\XLingPapercolawidth}}{THAI CHARACTER MAI TRI (THAI TONE MAI TRI)}&\multicolumn{1}{>{\centering}p{\XLingPapercolbwidth}}{\XLingPaperCharisZSILFontFamily{\fontsize{24}{28.799999999999997}\selectfont ◌{\XLingPaperTavirajFontFamily{\fontsize{24}{28.799999999999997}\selectfont ๊}}}}&\multicolumn{1}{p{\XLingPapercolcwidth}@{}}{U+0E4A}\\%
\multicolumn{1}{@{}p{\XLingPapercolawidth}}{THAI CHARACTER MAI CHATTAWA (THAI TONE MAI CHATTAWA)}&\multicolumn{1}{>{\centering}p{\XLingPapercolbwidth}}{\XLingPaperCharisZSILFontFamily{\fontsize{24}{28.799999999999997}\selectfont ◌{\XLingPaperTavirajFontFamily{\fontsize{24}{28.799999999999997}\selectfont ๋}}}}&\multicolumn{1}{p{\XLingPapercolcwidth}@{}}{U+0E4B}\\\bottomrule%
\end{longtable}
}
}\noindent Apple has stated it well, that a diacrictic is a glyph or a set of glyphs. And that might fit well within the common understand of many who use use almost exclusively the Latin Script. For instance \hyperlink{fDiacritics}{1}\par{}\noindent However, a function based definition breaks down if we consider evidence presented by \hyperlink{r}{ ()} some glyphs like the Japanese wakiten {\XLingPaperSimSunFontFamily{\fontsize{18}{21.599999999999998}\selectfont \textit{}}} and Tibetan {\XLingPaperJomolhariFontFamily{\textit{{\XLingPaperJomolhariFontFamily{\fontsize{28}{33.6}\selectfont ༵}}}}} {⟨\XLingPaperJomolhariFontFamily{\textit{༵}}⟩} {\XLingPaperJomolhariFontFamily{\fontsize{10}{12}\selectfont ༵}} scripts as presented in examples \hyperlink{xJapanese1}{(1)} and \hyperlink{xTibet1}{(2)} respectively. which are used for emphasis not phonetic value.\par{}{\vspace{11pt plus 2pt minus 1pt}\raggedright{}{\singlespacing
\XLingPaperexample{0in}{0in}{2.75em}{\raisebox{\baselineskip}[0pt]{\protect\hypertarget{xJapanese1}{}}(1)}{{\parbox[t]{\textwidth - 0in - 0in}{\vspace*{-\baselineskip}{\XeTeXpicfile "../Resources/wakiten.png" scaled 750}}}\ }
}}{\vspace{11pt plus 2pt minus 1pt}\raggedright{}{\singlespacing
\XLingPaperexample{0in}{0in}{2.75em}{\raisebox{\baselineskip}[0pt]{\protect\hypertarget{xTibet1}{}}(2)}{{{\XLingPaperJomolhariFontFamily{\fontsize{28}{33.6}\selectfont སྐལ་ལྡན་གདུལ་བྱར་སྣང་བའི་བསོ༵ད་ནམ༵ས་གཟུགས།}}}\ }
}\vspace{11pt plus 2pt minus 1pt}}\noindent The next logical proposition is to try and define a diacritic based on form. That is what constitutes a diacritic and what does not? Authors and designers Adam Twardoch \hyperlink{rTwardoch}{(2009}; writing about Polish) and Cristian Kit Paul \hyperlink{rPaul2008}{(2008}; writing about Romanian) refere to the base glyph along with their secondary glyph as “diacritics”. And in both cases these authors indicate that these second, derived complex typographical units are considered separate letters from the letters without modifying glyphs\protect\footnote[30]{{\leftskip0pt\parindent1em\raisebox{\baselineskip}[0pt]{\protect\hypertarget{nEverson}{}} \hyperlink{rEvertype1994}{Everson \& Sigurðsson (1994:§ 1.3)} discuss this difference between a glyph and its derivative saying: “basic constituents (“basic letters of the Latin alphabet”) and in terms of its derived constituents (“letters derived from basic letters of the Latin alphabet”). A derived letter is one which is, in terms of its historical development, based on one or more other Latin characters, whether by some deformation of the character itself, or by the addition of one or more modifying marks or signs.”}}. I feel this is too broad for our purposes\par{}\noindent However, I prefer to define it based on its form, even so a definition which spans across scripts is hard. I propose the following:\par{}\noindent Kill graphemes?\par{}\XLingPaperblockquote{.25in}{{\singlespacing
\vspace{-1.3\baselineskip}A diacritic is a glyph that is positioned relative to a base glyph. Usually the diacritic takes no additional space on the \hyperlink{gtBaseline}{{\textit{baseline}}}. Generally they change the phonetic value, add phonetic detail, or bring additional semantic information to an orthographic character. Overall they have a clarifying role in the communicative value of a base glyph.}}{\baselineskip}{\baselineskip}\noindent page 142\par{}\noindent Florian Coulmas writes on page 153 " Vowel indication in Ethopic may be described as a system of diacritical marks what have become integral parts of the basic consonant signs."\par{}\noindent What is a grapheme?\par{}\vspace{11pt plus 2pt minus 1pt}\XLingPaperneedspace{3\baselineskip}\protect\hypertarget{ntEthopic}{}\XLingPaperaddtocontents{ntEthopic}{\protect\raggedright{\singlespacing
{Table }{5.}{  Diacritics in the Ge'ez script\\}}}\vspace{0pt}{\singlespacing
\hspace*{.25in}{\XLingPaperxkcdZScriptFontFamily{
\XLingPaperminmaxcellincolumn{Glyph}{\XLingPapermincola}{\textbf{Base Glyph}}{\XLingPapermaxcola}{+0\tabcolsep}
\XLingPaperminmaxcellincolumn{diacritic}{\XLingPapermincolb}{\textbf{Mid-height diacritic placement}}{\XLingPapermaxcolb}{+0\tabcolsep}
\XLingPaperminmaxcellincolumn{diacritic}{\XLingPapermincolc}{\textbf{Low-height diacritic placement}}{\XLingPapermaxcolc}{+0\tabcolsep}
\XLingPaperminmaxcellincolumn{}{\XLingPapermincola}{\vspace*{0pt}{\XeTeXpdffile "../Resources/Ethopic-null-plain.pdf" scaled 315}}{\XLingPapermaxcola}{+0\tabcolsep}
\XLingPaperminmaxcellincolumn{}{\XLingPapermincolb}{\vspace*{0pt}{\XeTeXpdffile "../Resources/Ethopic-mid-plain.pdf" scaled 300}}{\XLingPapermaxcolb}{+0\tabcolsep}
\XLingPaperminmaxcellincolumn{}{\XLingPapermincolc}{\vspace*{0pt}{\XeTeXpdffile "../Resources/Ethopic-low-plain.pdf" scaled 300}}{\XLingPapermaxcolc}{+0\tabcolsep}
\setlength{\XLingPaperavailabletablewidth}{433.62pt}
\setlength{\XLingPapertableminwidth}{\XLingPapermincola+\XLingPapermincolb+\XLingPapermincolc}
\setlength{\XLingPapertablemaxwidth}{\XLingPapermaxcola+\XLingPapermaxcolb+\XLingPapermaxcolc}
\XLingPapercalculatetablewidthratio{}
\XLingPapersetcolumnwidth{\XLingPapercolawidth}{\XLingPapermincola}{\XLingPapermaxcola}{-0\tabcolsep}
\XLingPapersetcolumnwidth{\XLingPapercolbwidth}{\XLingPapermincolb}{\XLingPapermaxcolb}{-2\tabcolsep}
\XLingPapersetcolumnwidth{\XLingPapercolcwidth}{\XLingPapermincolc}{\XLingPapermaxcolc}{-2\tabcolsep}\singlespacing\vspace*{-3\baselineskip}
\begin{longtable}
[l]{@{}>{\centering}p{\XLingPapercolawidth}>{\centering}p{\XLingPapercolbwidth}>{\centering}p{\XLingPapercolcwidth}@{}}\toprule\multicolumn{1}{@{}>{\centering}p{\XLingPapercolawidth}}{\textbf{Base Glyph}}&\multicolumn{1}{>{\centering}p{\XLingPapercolbwidth}}{\textbf{Mid-height diacritic placement}}&\multicolumn{1}{>{\centering}p{\XLingPapercolcwidth}@{}}{\textbf{Low-height diacritic placement}}\\%
\midrule\endhead \multicolumn{1}{@{}p{\XLingPapercolawidth}}{\raggedright \vspace*{0pt}{\XeTeXpdffile "../Resources/Ethopic-null-plain.pdf" scaled 315}}&\multicolumn{1}{p{\XLingPapercolbwidth}}{\raggedright \vspace*{0pt}{\XeTeXpdffile "../Resources/Ethopic-mid-plain.pdf" scaled 300}}&\multicolumn{1}{p{\XLingPapercolcwidth}@{}}{\raggedright \vspace*{0pt}{\XeTeXpdffile "../Resources/Ethopic-low-plain.pdf" scaled 300}}\\\bottomrule%
\end{longtable}
}}
}\indent In terms of code points and orthographical characters, these types of multi-graphs are strings of characters which have a one-to-one character to orthographic-character relationship. Readers of the writing system must infer that serieses of orthographic characters indicate a single sound in the language being read. (2) As a methodology, diacritic modification of orthographic characters, is the practice of modifying an existing orthographical character, usually with small marks above, below or to the side of the original character. To the reader of an orthography, multi-graphs may represent one unit of information or they may represent multiple units of information. Each graphical unit which encodes a single unit of information is a grapheme. That is, the visual components of an orthographical character may also have sub-units which individually relay information. Following (2003: 28) graphemes are anything that functions as a distinct unit within an orthography. This is an important notion because the orthography is the domain of control for the meaning of a graphical element. The same visual shape may appear in several orthographies but (1) have different meanings or (2) not have independent meaning apart in an independent form. For instance, the commonly known umlaut marks which in Unicode are called 'COMBINING DIAERESIS' U+0308 {\XLingPaperCambriaZMathFontFamily{⟨ {\XLingPaperCharisZSILFontFamily{\textup{\textup{\textmd{◌̈}}}}} ⟩}} serve several functions across the worldʼs writing systems. In English they are diaeresis and serve to tell the reader that two vowels are not a \hyperlink{gtDiGraph}{{\textit{di-graph}}} and should be read as separate vowels. We can see its use in the following words {\XLingPaperCambriaZMathFontFamily{\textup{\textmd{⟨ {\XLingPaperCharisZSILFontFamily{\textup{\textup{\textmd{naïve, Noël, coöperation}}}}} ⟩}}}}. The wide spread use of diaeresis in English has for the most part been abandoned or replaced by using the dash such that we might see either of the following {\XLingPaperCambriaZMathFontFamily{\textup{\textmd{⟨ {\XLingPaperCharisZSILFontFamily{\textup{\textup{\textmd{cooperation, co-operation}}}}} ⟩}}}}. The same mark (diaeresis) in the Swedish \textsquarebracketleft{}swe\textsquarebracketright{} orthography, is part of the Swedish letter {\XLingPaperCambriaZMathFontFamily{\textup{\textmd{⟨ {\XLingPaperCharisZSILFontFamily{\textup{\textup{\textmd{Ä}}}}} ⟩}}}} \hyperlink{rGranberry1991}{(Granberry  1991:7}, \hyperlink{rSvenskaAkademien2006Svens}{Svenska Akademien  2006)}. The way that the Swedish letter is conceptualized, by users of Swedish, is such that the mark is not removable from the letter. There is not just an addition to an existing letter but a whole new letter of the alphabet. In a typographical sense the diaeresis is an independent, removable, and alternating component in the Swedish orthography. However, in the minds of the reader, the grapheme is the whole letter {\XLingPaperCambriaZMathFontFamily{\textup{\textmd{⟨ {\XLingPaperCharisZSILFontFamily{\textup{\textup{\textmd{Ä}}}}} ⟩}}}} not a combination of two graphemes {\XLingPaperCambriaZMathFontFamily{\textup{\textmd{⟨ {\XLingPaperCharisZSILFontFamily{\textup{\textup{\textmd{A {\XLingPaperCambriaFontFamily{\textit{+ }}}◌̈}}}}} ⟩}}}}. This means that the way that readers and the way that typographers conceive of the character are different. In other languages this distinction between the way readers and typographers conceive of the character may not obtain. For instance, in German the case is not always clear. Some may claim that {\XLingPaperCambriaZMathFontFamily{\textup{\textmd{⟨ {\XLingPaperCharisZSILFontFamily{\textup{\textup{\textmd{ä}}}}} ⟩}}}} and {\XLingPaperCambriaZMathFontFamily{\textup{\textmd{⟨ {\XLingPaperCharisZSILFontFamily{\textup{\textup{\textmd{a}}}}} ⟩}}}} are not separate letters of the German alphabet for two reasons: (1) because of legislation matching the German “orthography” 6 to the ISO Latin Alphabet\_\_(need to cite this legislation see discussion in: (Johnson 2005))\_\_ which contains only 26 letters, and (2) because rules for alphabetical ordering of {\XLingPaperCambriaZMathFontFamily{\textup{\textmd{⟨ {\XLingPaperCharisZSILFontFamily{\textup{\textup{\textmd{ä}}}}} ⟩}}}} and {\XLingPaperCambriaZMathFontFamily{\textup{\textmd{⟨ {\XLingPaperCharisZSILFontFamily{\textup{\textup{\textmd{a}}}}} ⟩}}}} call for words in which {\XLingPaperCambriaZMathFontFamily{\textup{\textmd{⟨ {\XLingPaperCharisZSILFontFamily{\textup{\textup{\textmd{ä}}}}} ⟩}}}} occurs to collate along with {\XLingPaperCambriaZMathFontFamily{\textup{\textmd{⟨ {\XLingPaperCharisZSILFontFamily{\textup{\textup{\textmd{a}}}}} ⟩}}}} as if it were {\XLingPaperCambriaZMathFontFamily{\textup{\textmd{⟨ {\XLingPaperCharisZSILFontFamily{\textup{\textup{\textmd{ae}}}}} ⟩}}}} . This second reason matches the historical development of the marking in German orthography. However, Germans have a name for each letter and often think of them as independent “Buchstaben”, a term which dates back to the manual printing process of books and imprinting (stabbing) them with type, but functions as the German term used for “letter” as in the “letter of the alphabet”. In German language instructional curricula, both in materials for German for German speakers, and in materials for German as a foreign language \hyperlink{r}{ ()} (Sevin \& Sevin 2000: 4) {\XLingPaperCambriaZMathFontFamily{\textup{\textmd{⟨ {\XLingPaperCharisZSILFontFamily{\textup{\textup{\textmd{ä}}}}} ⟩}}}} and {\XLingPaperCambriaZMathFontFamily{\textup{\textmd{⟨ {\XLingPaperCharisZSILFontFamily{\textup{\textup{\textmd{a}}}}} ⟩}}}} are often presented as separate letters in the alphabet. In a phonological sense, in German the {\XLingPaperCambriaZMathFontFamily{\textup{\textmd{⟨ {\XLingPaperCharisZSILFontFamily{\textup{\textup{\textmd{◌ }}}}} ⟩}}}}̈ marks a fronting of the base vowel, and in that sense the {\XLingPaperCambriaZMathFontFamily{\textup{\textmd{⟨ {\XLingPaperCharisZSILFontFamily{\textup{\textup{\textmd{◌̈}}}}} ⟩}}}} does represent meaning lending itself to the interpretation of a grapheme. However, most Germans will not be able to tell you that there is a “fronting process” and would just tell you that it is a separate letter/“Buchstabe” representing a separate sound. German publishers may also use {\XLingPaperCambriaZMathFontFamily{⟨  ̈ ⟩}} as diaeresis. However, in these uses it is often the case that the word in which it is used is a loan word or that the base character is not {\XLingPaperCambriaZMathFontFamily{\textup{\textmd{⟨ {\XLingPaperCharisZSILFontFamily{\textup{\textup{\textmd{a, o, u}}}}} ⟩}}}}. For instance a German reader would know that the dots above an {\XLingPaperCambriaZMathFontFamily{\textup{\textmd{⟨ {\XLingPaperCharisZSILFontFamily{\textup{\textup{\textmd{ë}}}}} ⟩}}}} would be an instance of diaeresis. If we take a similar case from Malinaltepec Meꞌphaa \textsquarebracketleft{}tcf\textsquarebracketright{} , an indigenous language spoken in Mexico, and look at the use of the macron which is used to indicate tone ⟨ ā, a, a̱ ⟩. In this case the macron is indicating tone and functions as a separate grapheme. It is positioned in various locations around the base character depending on the pitch of the tone. In the Meꞌphaa case it is clear that the macron is a distinct grapheme independent of the base {\XLingPaperCambriaZMathFontFamily{\textup{\textmd{⟨ {\XLingPaperCharisZSILFontFamily{\textup{\textup{\textmd{a}}}}} ⟩}}}}.\par{}\indent Blockquote from http://www.unicode.org/versions/Unicode11.0.0/ch06.pdf\#G7382 page 256 section 61. Writing Systems. The Unicode Standard / the Unicode Consortium; edited by the Unicode Consortium. — Version 11.0. Published in Mountain View, CA June 2018 \par{}\XLingPaperblockquote{.25in}{{\singlespacing
\vspace{-1.3\baselineskip}Alphabets. A writing system that consists of letters for the writing of both consonants and vowels is called an alphabet. The term “alphabet” is derived from the first two letters of the Greek script: alpha, beta. Consonants and vowels have equal status as letters in such a system. The Latin alphabet is the most widespread and well-known example of an alphabet, having been adapted for use in writing thousands of languages. The correspondence between letters and sounds may be either more or less exact. Many alphabets do not exhibit a one-to-one correspondence between distinct sounds and letters or groups of letters used to represent them; often this is an indication of original spellings that were not changed as the language changed. Not only are many sounds represented by letter combinations, such as “th” in English, but the language may have evolved since the writing conventions were settled. Examples range from cases such as Italian or Finnish, where the match between letter and sound is rather close, to English, which has notoriously complex and arbitrary spelling.\par{}}}{\baselineskip}{\baselineskip}\vspace{11pt plus 2pt minus 1pt}\setbox0=\vbox{\protect\centering \leavevmode
\vspace*{0pt}{\XeTeXpdffile "../Resources/Scriptnocomment.pdf" scaled 400}\\[0pt]\protect\hypertarget{fWritingSystemsNoCommnet}{}\XLingPaperaddtocontents{fWritingSystemsNoCommnet}{\singlespacing
{Figure }{9.}{ \setcounter{footnote}{30}Relationship between languages, Writing Systems and Orthographies\footnotemark{}\\}}\protect\footnotetext[31]{{\leftskip0pt\parindent1em\raisebox{\baselineskip}[0pt]{\protect\hypertarget{nWritingSystemSource}{}} From \hyperlink{Constable}{Constable (2002)}.}}}\box0\protect\footnotetext[31]{{\leftskip0pt\parindent1em\raisebox{\baselineskip}[0pt]{\protect\hypertarget{nWritingSystemSource}{}} From \hyperlink{Constable}{Constable (2002)}.}}\par{}\vspace{11pt plus 2pt minus 1pt}\noindent An alphabetic order is important but isn't part of a writing system per se. It is part of a locale. An alphabetic order should have a defence and and should consider the ordering of non-alphabetical characters. https://www.niso.org/sites/default/files/2017-08/tr03.pdf NISO Press, Bethesda, Maryland, U.S.A. Published by the National Information Standards Organization Bethesda, Maryland NISO Technical Report 03 TR03-1999 Hans H. Wellisch Guidelines for Alphabetical Arrangement of Letters and Sorting of Numerals and Other Symbols presents a good exemplar with rationale\par{}{\vspace{10pt}\XLingPaperneedspace{3\baselineskip}\noindent
\fontsize{13}{15.6}\selectfont \textit{{\noindent
\raisebox{\baselineskip}[0pt]{\pdfbookmark[3]{{1.3.2 } To Delete}{sTemp}}\raisebox{\baselineskip}[0pt]{\protect\hypertarget{sTemp}{}}{1.3.2 }To Delete}}\markboth{To Delete}{Introduction}\XLingPaperaddtocontents{sTemp}}\par{}
\penalty10000\vspace{10pt}\penalty10000\indent {\XLingPaperneedspace{5\baselineskip}

\penalty-3000
\begin{description}
\setlength{\topsep}{0pt}\setlength{\partopsep}{0pt}\setlength{\itemsep}{0pt}\setlength{\parsep}{0pt}\setlength{\parskip}{0pt}\setlength{\leftmargini}{1em}\setlength{\leftmarginii}{1em}\setlength{\leftmarginiii}{1em}\setlength{\leftmarginiv}{1em}\penalty10000\item[Characters{:}]{are single Unicode code points}
\end{description}
\XLingPaperneedspace{5\baselineskip}

\penalty-3000
\begin{description}
\setlength{\topsep}{0pt}\setlength{\partopsep}{0pt}\setlength{\itemsep}{0pt}\setlength{\parsep}{0pt}\setlength{\parskip}{0pt}\setlength{\leftmargini}{1em}\setlength{\leftmarginii}{1em}\setlength{\leftmarginiii}{1em}\setlength{\leftmarginiv}{1em}\penalty10000\item[Letters{:}]{are typographical units for the purposes of pedagogy.}
\penalty10000\item[Linguistic Description{:}]{A linguistic description for the purposes of a writing system implementation is the necessary association of writing system elements to linguistic features. For instance, this would include phonetic or phonological details for the characters used in the encoding of the text, or grammatical places to use commas and periods.}
\penalty10000\item[Phoneme{:}]{A {\textbf{phoneme}} is a unique and contrastive sound unit in a language. Many times an alphabet is based on a list of phonemes. But to the extent that two typographical characters are used together as a digraphs, an alphabet might have fewer {\textbf{letters}}/components than a list of phonemes in the same language.}
\penalty10000\item[Writing System{:}]{As laid out by Peter Constable \hyperlink{Constable}{Constable (2002)}, a Writing System is a superordinate category of a collection of technologies and/or metadata on how an orthography is to be implemented. The following image situates the terms and relationships around orthographies and languages.\\}
\end{description}
}\par{}{\vspace{10pt}\XLingPaperneedspace{3\baselineskip}\noindent
\fontsize{13}{15.6}\selectfont \textit{{\noindent
\raisebox{\baselineskip}[0pt]{\pdfbookmark[3]{{1.3.3 } Phonology}{sPhonology}}\raisebox{\baselineskip}[0pt]{\protect\hypertarget{sPhonology}{}}{1.3.3 }Phonology}}\markboth{Phonology}{Introduction}\XLingPaperaddtocontents{sPhonology}}\par{}
\penalty10000\vspace{10pt}\penalty10000\indent In this study I do not make any phonological claims of my own. Rather I work with the claims others have made about the languages I discuss. I use a basic phonemic inventory, and look for a phonological description of the tone system. Tonal descriptions are often problematic due to a wide degree of variance in the interpretation of tone and tonal patterns. I find that the most helpful tonal descriptions are ones which also have looked for the occurrence of tonal patterns (sometime referred to as ‘melodies’ in tonal literature) following the methods and discussion put forward by \hyperlink{rSnider2014}{Snider (2014)} and \hyperlink{rHymanLary2014}{Hyman (2014)}. One way to think about tonal ‘melodies’ is as non-sequential morphology within a word. Section \hyperlink{sTonalMelodies}{2.5.2} discusses the relevance of tonal melodies to the writing process with more detail. Where Snider and Hyman differ from many other tonologists, is that many have assumed that tone is a sequential sort of phenomena, rather than a serial-segmental pitch event. A melody based approach to tone analyses provides researchers or analysts with insights into the complexities that readers and writers encounter with regards to the mental processing of tone, regardless of how it is transcribed.\par{}\indent However, this does not mean that I do not espouse a theory of tone.\par{}{\vspace{10pt}\XLingPaperneedspace{3\baselineskip}\noindent
\fontsize{13}{15.6}\selectfont \textit{{\noindent
\raisebox{\baselineskip}[0pt]{\pdfbookmark[3]{{1.3.4 } Tone}{sTone}}\raisebox{\baselineskip}[0pt]{\protect\hypertarget{sTone}{}}{1.3.4 }Tone}}\markboth{Tone}{Introduction}\XLingPaperaddtocontents{sTone}}\par{}
\penalty10000\vspace{10pt}\penalty10000{\vspace{10pt}\XLingPaperneedspace{3\baselineskip}\noindent
\fontsize{13}{15.6}\selectfont \textit{{\noindent
\raisebox{\baselineskip}[0pt]{\pdfbookmark[3]{{1.3.5 } Functional Load}{sFunctionalLoad}}\raisebox{\baselineskip}[0pt]{\protect\hypertarget{sFunctionalLoad}{}}{1.3.5 }Functional Load}}\markboth{Functional Load}{Introduction}\XLingPaperaddtocontents{sFunctionalLoad}}\par{}
\penalty10000\vspace{10pt}\penalty10000\indent Functional load has been claimed (too often in my opinion) without grounds for poor orthography design. One assumption is that a one-to-one (character to keystroke) implementation of a typing system for an orthography with a high functional load would test out as a “poor implementation of a writing system”. It might be possible to create a many-to-one typing efficiency and still have a high functional load orthography, and then people still won't type their language, or won't use it in non-oral communication. So: What is functional load and how do we measure it? If we imagine information transmission as water flowing, then we might imagine the neuro-transmission lines as pipes. In the process of typing there are several pipes that must be used by the typist: The information pathway from the brain to the fingers, telling the fingers which buttons to press. The information pathway from the eye to the brain telling the brain which buttons were pressed. It transcribing, then there is the activity of listening and the capacity to reproduce what was heard. There is the brain process of converting ideas to words in a language. There is the brain process of converting those words to an orthography - an orthography which visually displays contrasts which in some way match to contrasts utilized in another modality. It might be assumed that orthographies can vary in their efficiencies in encoding relevant kinds of contrasts. The question is often, which contrasts are kept and then how often do non-matching sequences occur in a text? And then how cognitively laborious is it for readers, and writers to perform these non-matching tasks?\par{}\indent Reading Hockett, Charles F. 1966. The quantification of functional load - a linguistic problem. Stanta Monica, California: The Rand Corporation.\par{}\indent King, Robert D. 1967. A Measure For Functional Load. Studia Linguistica 21.1: 1-14.\par{}\indent Wang, W. S. Y. 1967. The Measurement of Functional Load. Phonetica 16.1: 36-54.\par{}\indent Hockett, Charles F. 1969. The Quantification of Functional Load. Word 23.1: 300-20.\par{}\indent Terzuolo, C. A. and P. Viviani. 1980. Determinants and characteristics of motor patterns used for typing. Neuroscience 5.6: 1085-103.\par{}\indent McCalla, Kim. 1985. Entropy In Natural Languages. Folia Linguistica 19.3-4: 343-186.\par{}\indent Surendran, Dinoj and Partha Niyogi. 2003. Measuring the Usefulness (Functional Load) of Phonological Contrasts.\par{}\indent Stokes, Stephanie F. and Dinoj Surendran. 2005. Articulatory Complexity, Ambient Frequency, and Functional Load as Predictors of Consonant Development in Children. Journal of Speech, Language, and Hearing Research 48.3: 577-91.\par{}\indent Oh, Yoon Mi, François Pellegrino, Christophe Coupé and Egidio Marsico. 2013. Cross-language Comparison of Functional Load for Vowels, Consonants, and Tones. In Bimbot, Frédéric (ed.), interspeech 2013.\par{}\indent Wedel, Andrew, Abby Kaplan and Scott Jackson. 2013. High functional load inhibits phonological contrast loss: A corpus study. Cognition 128.2: 179–86.\par{}{\vspace{10pt}\XLingPaperneedspace{3\baselineskip}\noindent
\fontsize{13}{15.6}\selectfont \textit{{\noindent
\raisebox{\baselineskip}[0pt]{\pdfbookmark[3]{{1.3.6 } Orthography}{sOrthography}}\raisebox{\baselineskip}[0pt]{\protect\hypertarget{sOrthography}{}}{1.3.6 }Orthography}}\markboth{Orthography}{Introduction}\XLingPaperaddtocontents{sOrthography}}\par{}
\penalty10000\vspace{10pt}\penalty10000\indent To the best of my knowledge, no measure-of-fit metric exists for orthographies. A measure-of-fit for orthographies might:\par{}{\parskip .5pt plus 1pt minus 1pt

\vspace{\baselineskip}

{\setlength{\XLingPapertempdim}{\XLingPaperbulletlistitemwidth+\parindent{}}\leftskip\XLingPapertempdim\relax
\interlinepenalty10000
\XLingPaperlistitem{\parindent{}}{\XLingPaperbulletlistitemwidth}{•}{include variation from the phonetic statistical norm for a given character. That is, does the symbol represent the same sound as the local or global norm?}}
{\setlength{\XLingPapertempdim}{\XLingPaperbulletlistitemwidth+\parindent{}}\leftskip\XLingPapertempdim\relax
\interlinepenalty10000
\XLingPaperlistitem{\parindent{}}{\XLingPaperbulletlistitemwidth}{•}{include how close to a one-to-one ratio exists in the system when comparing phoneme-to-symbol relationships,}}
{\setlength{\XLingPapertempdim}{\XLingPaperbulletlistitemwidth+\parindent{}}\leftskip\XLingPapertempdim\relax
\interlinepenalty10000
\XLingPaperlistitem{\parindent{}}{\XLingPaperbulletlistitemwidth}{•}{report on the adequacy of punctuation characters defined for the writing system for the grammatical constructs which are required.}}
\vspace{\baselineskip}
}\noindent Such a metric would be ideal in assessing the quantity of difficulty which an orthography brings to the task of keyboard based text input for languages. In this study I do not attempt to make claims about the efficiency of orthographies as they encode words, thoughts, or phonemes. Rather I measure all the difficulty at the layer of the keyboard layout arrangement. To make these assessments I work with the published orthographic descriptions and compare those with the published phonological descriptions. Since the discussion of orthography design is perhaps more popular in linguistics than car design is in the automotive industry I'll leave the orthography assessment to the orthographists. By separating the assessment and design of text input task from the assessment and design process of the orthography, I intentionally avoid making claims about orthography design which support technological imperialism. For example, some such as \hyperlink{rBoerger2007Natqg}{Boerger (2007}; {\textsquarebracketleft{}ntu\textsquarebracketright{}}), \hyperlink{Cooper}{Cooper (2005:160}; {\textsquarebracketleft{}kls\textsquarebracketright{}}), \hyperlink{GuE9rin}{Guérin (2008}; {\textsquarebracketleft{}mkv\textsquarebracketright{}}), and \hyperlink{Jany}{Jany (2010}; {\textsquarebracketleft{}pxm\textsquarebracketright{}}), acknowledge the difficulty in minority language literature production but turn to orthography redesign as a way to solve the text input problem perceived by language community members\protect\footnote[32]{{\leftskip0pt\parindent1em\raisebox{\baselineskip}[0pt]{\protect\hypertarget{nRomanArabic}{}} This is not to say that language using communities do not feel pressure based on their perceived technological choices. In fact many do, for instance some Arabic using communities in Egypt \hyperlink{rWarschauer2002Langu}{(Warschauer et al.  2002)} and Jordan \hyperlink{rAlKhatib2008Langu}{(Al-Khatib \& Sabbah  2008}; \hyperlink{rMustafa2011SMSCo}{Mustafa  2011)} are reported to switch to a Latin script version of Arabic. This is in contrast with some communities in Syria who use Arabic Script in digital interactions (personal interviews of iOS users). It is also interesting to note that in some Europoean contexts such as Denmark, when smart phones were just comming to the market one could tell if the sender of a text message was using a BlackBerry or an iOS device. Blackberries did not support {\XLingPaperCambriaZMathFontFamily{\textup{\textmd{⟨ ø ⟩}}}} so those users opted for {\XLingPaperCambriaZMathFontFamily{\textup{\textmd{⟨ oe ⟩}}}} instead (personal interviews).}}. Guérin \hyperlink{GuE9rin}{(2008:57)} raises the question of text input difficulty to the level of primary consideration in orthography redesign: “technological usability: Are the symbols/graphemes readily available on a standard local keyboard/typewriter?” Guérin is suggesting that orthographies be restructured so that they can be matched to the existing keys on the keyboard. Meaning that the writing system is subject to the technology, and those that push and distribute technology have an unduly persuasive role in shaping the orthographies of minority language communities. \hyperlink{rLFCpke}{Lüpke (2011:333-4)} acknowledges that not only is the presence of characters on keys important, but also the position of those keys should be based on the ergonomics of their usage. But the suggestion here seems to be that if a character is in an inconvenient location on the keyboard then perhaps a different one should be chosen. In opposition to the orthography redesign strategy for solving the text input challenge, \hyperlink{rBailey2007}{Bailey (2007)} and \hyperlink{Paterson}{Paterson (2014)} hold that orthographies should not be forced to succumb to technological imperialism – a sentiment perhaps independently arrived at but never the less an echo of \hyperlink{rSmalley1959}{Smalley (1959}, \hyperlink{rSmalley1963}{1963:125)} wherein he argues against the notion that the orthography of Moore {\textsquarebracketleft{}mos\textsquarebracketright{}} (and West African languages in general) should succumb to the “smaks of “mechanical imperialism””. That is, technology should be made to support the design decisions at the level of the orthography, the orthography should not be designed to fit within the limits of some technology.\par{}{\vspace{10pt}\XLingPaperneedspace{3\baselineskip}\noindent
\fontsize{13}{15.6}\selectfont \textit{{\noindent
\raisebox{\baselineskip}[0pt]{\pdfbookmark[3]{{1.3.7 } Typing as a linguistic activity}{sTypingIsLingusitics}}\raisebox{\baselineskip}[0pt]{\protect\hypertarget{sTypingIsLingusitics}{}}{1.3.7 }Typing as a linguistic activity}}\markboth{Typing as a linguistic activity}{Introduction}\XLingPaperaddtocontents{sTypingIsLingusitics}}\par{}
\penalty10000\vspace{10pt}\penalty10000\indent Make congruent or obsolete section \hyperlink{sLingusiticActivity}{2.4}. Texts to cite: \hyperlink{rPinetSvetlanaJohannesC.ZieglerF.XavierAlario122016Typin}{Pinet, Ziegler \& Alario (2016)}, \hyperlink{rME9aryDavidCatherineCharyRichardPalluelGermainJeanPierreOrliaguet2005Visua}{Méary et al. (2005)}\par{}\clearpage
\thispagestyle{bodyfirstpage}\markboth{Multiple perspectives on the keyboard layout problem}{Multiple perspectives on the keyboard layout problem}
\XLingPaperaddtocontents{c2CS}{\vspace*{.65in}\XLingPaperneedspace{3\baselineskip}\noindent
\fontsize{14}{16.8}\selectfont \textbf{{\centering
CHAPTER \raisebox{\baselineskip}[0pt]{\protect\hypertarget{c2CS}{}}\raisebox{\baselineskip}[0pt]{\pdfbookmark[1]{2 Multiple perspectives on the keyboard layout problem}{c2CS}}2\protect\\}}}\par{}
{\XLingPaperneedspace{3\baselineskip}\noindent
\fontsize{14}{16.8}\selectfont \textbf{{\centering
Multiple perspectives on the keyboard layout problem\protect\\}}}\par{}
\vspace{16pt}\indent Despite being language-related resources, keyboard layouts and text input methods are rarely mentioned in the academic language development literature. Two standout exemptions are \hyperlink{rBailey2007}{Bailey (2007)} who describes the keyboard layout production process for creating a single keyboard to serve several of the minority languages in South Africa. And \hyperlink{Anderson}{Anderson (2012)} who describes the production of a keyboard layout in the Latin Script for Tifinagh {\textsquarebracketleft{}tzm\textsquarebracketright{}}\protect\footnote[1]{{\leftskip0pt\parindent1em\raisebox{\baselineskip}[0pt]{\protect\hypertarget{nTwoMorecitations}{}} There are several mentions of keyobard layouts for minority languages in the computer science literature. \hyperlink{rBodomoetal2006}{Bodomo et al. (2006)} discuss a keyboard layout for Twi {\textsquarebracketleft{}twi\textsquarebracketright{}} and Southern Dagaare {\textsquarebracketleft{}dga\textsquarebracketright{}}, unfortuantely, even though the keyboard was encoded with Unicode, the article descriging it was not – so some of the detail of their publication is lost in the annals of digital publishing. \hyperlink{rBodomoetal2006}{Bodomo et al. (2006)} do not overtly mention which variety of Dagaare. Southern Dagaare is assumed as it is spoken in Ghana the same country as Twi.}}. This is not to say that there aren't other minority language keyboards, \hyperlink{Thoms}{Thomas \& Simons (2017)} suggests that there are around 1,100 keyboard layouts available through SIL's Keyman software, only that their existence and creation process doesn't percolate to the level of an academic publication\protect\footnote[2]{{\leftskip0pt\parindent1em\raisebox{\baselineskip}[0pt]{\protect\hypertarget{nScopeMobile}{}} This doesn't mean that there aren't academic publications which do address keyboard layouts for other applications. In deed there are. For instance, \hyperlink{Bakkali}{Bakkali et al. (2015)} write about an Android keyboard for Tifinagh {\textsquarebracketleft{}tzm\textsquarebracketright{}}, and there are many applications of virtual keyboards in the internent-of-things and embedded operating systems i.e., car stereos, and small electrical devices like GPS units, but this thesis limits its discussion to full size mechanical keyboards.}}. In contrast to the lack of academic description of keyboard layouts, their design criteria, testing or development process, the need for minority language text input solutions is frequently discussed by language planers \hyperlink{Diki-Kidiri}{(Diki-Kidiri  2011}; \hyperlink{Zhozhikov-et-al}{Zhozhikov et al.  ‎2011)}, educators \hyperlink{Galla}{(Galla  2009}; \hyperlink{Silva-Donaghy}{Silva \& Donaghy  2004)}, speakers \hyperlink{Bernard}{(Bernard  1992)}, and academics \hyperlink{Trosterud}{(Trosterud  2012}; \hyperlink{rScannell2011}{Scannell  2011)}.\par{}\indent This chapter reviews optimized keyboard layouts and presents several ways keyboards are optimized.\par{}\indent \hyperlink{rBailey2007}{Bailey (2007)} and many of the SIL Keyman keyboards, such as the one for SIL Cameroon,\protect\footnote[3]{{\leftskip0pt\parindent1em\raisebox{\baselineskip}[0pt]{\protect\hypertarget{nSILCameroon}{}} \href{http://cameroon.keymankeyboards.com}{\textcolor[rgb]{0,0,0}{http://cameroon.keymankeyboards.com}}}} and the LIBTRALO keyboard mentioned previously\footnote[4]{{See footnote }\hyperlink{nLiberiaKeyboard}{8} in chapter 1.} are represent an optimization at the institutional support level. That is, an institution assumes or is tasked with supporting text input in several languages and optimizes a single keyboard layout on the bases of the kinds of needed glyphs across all the languages it is supporting. In a way this keyboard is a compromise in that it creates a "possible to type" language. Text input solutions which have auto-suggest features are optimized on the bases of a corpus. These corpora are generally set for a single language meaning that the text input solution is optimized for a single language, although \hyperlink{rBi2012Multi}{Bi et al. (2012)} present a mobile keyboard which is optimized on the bases of a corpus which contains four languages.\par{}\indent A second method used to optimize keyboard layouts is to match them phonetically to the phonetic values of the glyphs on the visual components of the keyboard. This is often done for people working with two scripts or with transliteration tasks.\par{}\indent A third way to optimize keyboard layouts is arrangement by some order which is familiar to the user, such as alphabetic order.\par{}\indent Finally, a forth way to optimize keyboards is to arrange the characters in an order which creates less work for the user in terms of hand and finger movement while also stabilizing (regularizing) the rhythm of keypresses. This final method is the process which is commonly pursued in applied mathematics and computer science. These fields, in contrast to language development\protect\footnote[5]{{\leftskip0pt\parindent1em\raisebox{\baselineskip}[0pt]{\protect\hypertarget{nAppliedLingusitics}{}} Sometimes lumped with Applied linguistics.}} and linguistics, focus on making keyboards optimized, independent of the language being typed. In later chapters I return to the language specific factors related to keyboard layouts.\par{}\indent Here I survey the assessment of keyboards from the Computer Science perspective including optimization approaches, metrics, algorithms, user experience approaches, and the shortfalls of these approaches when considering minority language issues.\par{}\indent Then I demonstrate using the Meꞌphaa {\textsquarebracketleft{}tcf\textsquarebracketright{}} (Otomanguean, Mexico) keyboard layout and current keyboard assessment methods how minority languages users have to conduct more work to type their language than relevant languages of wider communication. I attribute some of the lack of minority language use in digital mediums to the relative ease of using a different language to communicate the same message.\par{}\indent This chapter closes with a discussion of some of the interruptive processes that typing causes for minority language typists.\par{}{\vspace{15pt}\XLingPaperneedspace{3\baselineskip}\noindent
\fontsize{13}{15.6}\selectfont \textbf{{\noindent
\raisebox{\baselineskip}[0pt]{\pdfbookmark[2]{{2.1 } Keyboards}{sKeyboardsGenreal}}\raisebox{\baselineskip}[0pt]{\protect\hypertarget{sKeyboardsGenreal}{}}{2.1 }Keyboards}}\markboth{Keyboards}{Multiple perspectives on the keyboard layout problem}\XLingPaperaddtocontents{sKeyboardsGenreal}}\par{}
\penalty10000\vspace{10pt}\penalty10000\indent There are several typologies for categorizing physical keyboards. It is good to be aware of them to understand how they do or do not impact the contribution that this thesis makes. Many of these typologies come from design arguments and testing in the field of ergonomics.\par{}\indent Split or non-split keyboards: Split keyboards also known as bifurcated keyboards divide the typing area into two separate spaces as is demonstrated by the layout in figure \hyperlink{fAtreus}{10}. These are often marketed as being more ergonomic. The most famous models/manufacturers are: {\XLingPaperCharisZSILFontFamily{\textit{Kinesis}}}, {\XLingPaperCharisZSILFontFamily{\textit{Ergo}}}, and {\XLingPaperCharisZSILFontFamily{\textit{Maltron}}}\protect\footnote[6]{{\leftskip0pt\parindent1em\raisebox{\baselineskip}[0pt]{\protect\hypertarget{nSplitManufacturers}{}} As manufacturers these companies have produced more than one model of keyboard, but they are generally best known for their market entry keyboard.}}. Other manufacturers also produce split keyboards. In contrast to split keyboards non-split keyboards such as is demonstrated in figure \hyperlink{ISO9995Grid}{11} are almost ubiquitous and are the default on laptops.\par{}\vspace{11pt plus 2pt minus 1pt}\setbox0=\vbox{\protect\centering \leavevmode
\vspace*{0pt}{\XeTeXpdffile "../Resources/split-vertical-align-qwerty.pdf" scaled 500}\\[0pt]\protect\hypertarget{fAtreus}{}\XLingPaperaddtocontents{fAtreus}{\singlespacing
{Figure }{10.}{ \setcounter{footnote}{6}{\XLingPaperCharisZSILFontFamily{\textit{Atreus}}} an example of a vertically aligned split keyboard.\footnotemark{}\\}}\protect\footnotetext[7]{{\leftskip0pt\parindent1em\raisebox{\baselineskip}[0pt]{\protect\hypertarget{nAtreusSource}{}} Image soured from Phil Hagelberg, and the Atreus website at \href{https://atreus.technomancy.us}{\textcolor[rgb]{0,0,0}{https://atreus.technomancy.us}}.}}}\box0\protect\footnotetext[7]{{\leftskip0pt\parindent1em\raisebox{\baselineskip}[0pt]{\protect\hypertarget{nAtreusSource}{}} Image soured from Phil Hagelberg, and the Atreus website at \href{https://atreus.technomancy.us}{\textcolor[rgb]{0,0,0}{https://atreus.technomancy.us}}.}}\par{}\vspace{11pt plus 2pt minus 1pt}\indent Vertical key arrangement vs slanted key arrangement: Keyboards may have two types of key arrangements. The most popular is the slanted or “off set” arrangement. However there are keyboards with a strict row and column arrangement or “vertical” arrangement. The off set arrangement can be described by the grid from ISO 9995 as illustrated in figure \hyperlink{ISO9995Grid}{11}.\par{}\vspace{11pt plus 2pt minus 1pt}\setbox0=\vbox{\protect\centering \leavevmode
\vspace*{0pt}{\XeTeXpicfile "../Resources/images/ISO9995Grid.png" scaled 500}\\[0pt]\protect\hypertarget{ISO9995Grid}{}\XLingPaperaddtocontents{ISO9995Grid}{\singlespacing
{Figure }{11.}{ \setcounter{footnote}{7}Sections, zones, and reference grid of a keyboard according to ISO/IEC 9995-1:2009\footnotemark{}.\\}}\protect\footnotetext[8]{{\leftskip0pt\parindent1em\raisebox{\baselineskip}[0pt]{\protect\hypertarget{nISO9995note}{}} Image soured from \hyperlink{rWikimediaISO9995}{Wikimedia (2012)} and licensed under the Creative Commons Attribution-Share Alike 3.0 Unported license.}}}\box0\protect\footnotetext[8]{{\leftskip0pt\parindent1em\raisebox{\baselineskip}[0pt]{\protect\hypertarget{nISO9995note}{}} Image soured from \hyperlink{rWikimediaISO9995}{Wikimedia (2012)} and licensed under the Creative Commons Attribution-Share Alike 3.0 Unported license.}}\par{}\vspace{11pt plus 2pt minus 1pt}\indent Key profile has several contributing factors including: vertical key arrangement vs slanted key arrangement, keyboard layout slope, key height, and actuator technologies.\par{}\vspace{11pt plus 2pt minus 1pt}\setbox0=\vbox{\protect\centering \leavevmode
\vspace*{0pt}{\XeTeXpicfile "../Resources/images/KeyProfile.png" scaled 450}\\[0pt]\protect\hypertarget{Keyheight}{}\XLingPaperaddtocontents{Keyheight}{\singlespacing
{Figure }{12.}{ \setcounter{footnote}{8}Illustration of different key profiles.\footnotemark{}.\\}}\protect\footnotetext[9]{{\leftskip0pt\parindent1em\raisebox{\baselineskip}[0pt]{\protect\hypertarget{nSource}{}} Image soured from \hyperlink{rDeskthority-keyprofile}{Deskthority (2012)}.}}}\box0\protect\footnotetext[9]{{\leftskip0pt\parindent1em\raisebox{\baselineskip}[0pt]{\protect\hypertarget{nSource}{}} Image soured from \hyperlink{rDeskthority-keyprofile}{Deskthority (2012)}.}}\par{}\vspace{11pt plus 2pt minus 1pt}\indent Keyboard layout slope: There are several slope types with keyboards as is illustrated in figure \hyperlink{Keyheight}{12}. Laptops keyboards are generally flat and parallel to the orientation of the computer, less expensive keyboards some times have legs to create a wedge or angle for what would otherwise be a flat surface. Other keyboards have a built in slope, usually where keys in row {\textbf{A}} would have a lower key height than keys in row {\textbf{E}}. Still other keyboards like {\XLingPaperCharisZSILFontFamily{\textit{MALTRON L89}}}, have “pockets” where there is a significant concave arrangement of the keys, allowing fingers to reach down to conduct key presses.\par{}\indent Key height: while the keyboard layout slope can impact the user perceived row height, the actual height of each key impacts its mass and therefore the haptic interaction around that key.\par{}\indent Actuators: Keyboards have a variety of actuator systems which can impact the noise (audio feed back) and tactile pressure required to press a key.\par{}\indent Perhaps the most important metric to be aware of, and certainly a challenge in this study is the variation of the physical number of keys on a keyboard. Various countries have established national standards which sometimes dictate the number of required physical keys. The number of physical keys then allow for variation in the assignment of key functions to physical keys. These variations impact keyboard layout designers and web designers who are looking to create keyboard based actions for their websites. Bringing consistency to keyboard initiated actions has the motivation for the efforts behind \hyperlink{rKacmarcik2017}{Kacmarcik \& Leithead (2017)}. Their work presents eight different physical keyboard layouts used around the world. Of those this study engaged with two ISO and ANSI. Table \hyperlink{ntISOJISANSI}{6} presents these two layouts and a third known as JIS which is discussed in chapter \hyperlink{cDiscussion}{7}.\par{}\vspace{11pt plus 2pt minus 1pt}\pagebreak\XLingPaperneedspace{3\baselineskip}\protect\hypertarget{ntISOJISANSI}{}\XLingPaperaddtocontents{ntISOJISANSI}{\protect\raggedright{\singlespacing
{Table }{6.}{  \setcounter{footnote}{9}Different physical keybard layouts\footnotemark{}\\}}}\vspace{0pt}{\singlespacing
\hspace*{.25in}{\setcounter{footnote}{10}\singlespacing\vspace*{-3\baselineskip}
\begin{longtable}
[l]{@{}c@{}}\toprule\multicolumn{1}{@{}c@{}}{\textbf{Physical differences highlited in blue}}\\%
\midrule\endhead \multicolumn{1}{@{}c@{}}{\protect\footnotetext[10]{{\leftskip0pt\parindent1em\raisebox{\baselineskip}[0pt]{\protect\hypertarget{nKyeboardImageSorce}{}} Images are from \hyperlink{rKacmarcik2017}{Kacmarcik \& Leithead (2017)}.}}{\hyperlink{vJIS}{{JIS}}}}\\%
\multicolumn{1}{@{}c@{}}{\vspace*{0pt}{\XeTeXpdffile "../Resources/keyboard-106-japanese.pdf" scaled 400}}\\%
\multicolumn{1}{@{}c@{}}{{\hyperlink{vISO}{{ISO}}}}\\%
\multicolumn{1}{@{}c@{}}{\vspace*{0pt}{\XeTeXpdffile "../Resources/keyboard-102-uk.pdf" scaled 400}}\\%
\multicolumn{1}{@{}c@{}}{{\hyperlink{vANSI}{{ANSI}}}}\\%
\multicolumn{1}{@{}c@{}}{\vspace*{0pt}{\XeTeXpdffile "../Resources/keyboard-101-us.pdf" scaled 400}}\\\bottomrule%
\end{longtable}
}
}\indent Demonstrating Variation of physical keyboard arrangements\par{}\indent For ergonomic comments cite: \hyperlink{rZecevic2000Aneva}{Zecevic et al. (2000)}\hyperlink{rBaker2007Digit}{Baker et al. (2007a)}\hyperlink{rBaker2007Kinamatics}{Baker et al. (2007b)}\par{}\vspace{11pt plus 2pt minus 1pt}\XLingPaperneedspace{3\baselineskip}\protect\hypertarget{ntHandPositions}{}\XLingPaperaddtocontents{ntHandPositions}{\protect\centering {\singlespacing
{Table }{7.}{  Bad hand positions\\}}}\vspace{0pt}{\singlespacing
\hspace*{.25in}{
\XLingPaperminmaxcellincolumn{positions}{\XLingPapermincola}{\textbf{Wrist positions}}{\XLingPapermaxcola}{+0\tabcolsep}
\XLingPaperminmaxcellincolumn{reaches}{\XLingPapermincolb}{\textbf{Finger and thumb reaches}}{\XLingPapermaxcolb}{+0\tabcolsep}
\XLingPaperminmaxcellincolumn{}{\XLingPapermincola}{\vspace*{0pt}{\XeTeXpicfile "../Resources/1-s2.0-S0268003306001665-gr1.jpg" scaled 1800}}{\XLingPapermaxcola}{+0\tabcolsep}
\XLingPaperminmaxcellincolumn{}{\XLingPapermincolb}{\vspace*{0pt}{\XeTeXpicfile "../Resources/Finger-Reach.png" scaled 350}}{\XLingPapermaxcolb}{+0\tabcolsep}
\setlength{\XLingPaperavailabletablewidth}{433.62pt}
\setlength{\XLingPapertableminwidth}{\XLingPapermincola+\XLingPapermincolb}
\setlength{\XLingPapertablemaxwidth}{\XLingPapermaxcola+\XLingPapermaxcolb}
\XLingPapercalculatetablewidthratio{}
\XLingPapersetcolumnwidth{\XLingPapercolawidth}{\XLingPapermincola}{\XLingPapermaxcola}{-0\tabcolsep}
\XLingPapersetcolumnwidth{\XLingPapercolbwidth}{\XLingPapermincolb}{\XLingPapermaxcolb}{-2\tabcolsep}\singlespacing\vspace*{-3\baselineskip}
\begin{longtable}
[c]{@{}>{\centering}p{\XLingPapercolawidth}>{\centering}p{\XLingPapercolbwidth}@{}}\toprule\multicolumn{1}{@{}>{\centering}p{\XLingPapercolawidth}}{\textbf{Wrist positions}}&\multicolumn{1}{>{\centering}p{\XLingPapercolbwidth}@{}}{\textbf{Finger and thumb reaches}}\\%
\midrule\endhead \multicolumn{1}{@{}>{\centering}p{\XLingPapercolawidth}}{\vspace*{0pt}{\XeTeXpicfile "../Resources/1-s2.0-S0268003306001665-gr1.jpg" scaled 1800}}&\multicolumn{1}{>{\centering}p{\XLingPapercolbwidth}@{}}{\vspace*{0pt}{\XeTeXpicfile "../Resources/Finger-Reach.png" scaled 350}}\\\bottomrule%
\end{longtable}
}
}{\vspace{15pt}\XLingPaperneedspace{3\baselineskip}\noindent
\fontsize{13}{15.6}\selectfont \textbf{{\noindent
\raisebox{\baselineskip}[0pt]{\pdfbookmark[2]{{2.2 } Keyboards in Computer Science}{sCS}}\raisebox{\baselineskip}[0pt]{\protect\hypertarget{sCS}{}}{2.2 }Keyboards in Computer Science}}\markboth{Keyboards in Computer Science}{Multiple perspectives on the keyboard layout problem}\XLingPaperaddtocontents{sCS}}\par{}
\penalty10000\vspace{10pt}\penalty10000\indent Not only is the difficulty in minority language text production interesting technological and sociological problems, it is also an interesting mathematical problem. In \hyperlink{rEggersetalPreprint2003}{Eggers et al. (2003a:9)} the mathematical problem of optimizing a keyboard layout is described as follows:\par{}\XLingPaperblockquote{.25in}{{\singlespacing
\vspace{-1.3\baselineskip}The {\hyperlink{vKAP}{{KAP}}} is a discrete, combinatorial optimization problem. The evaluation function is neither linear nor convex. As such, the natural solution strategy may be found in metaheuristic optimization algorithms such as taboo search, genetic algorithms, simulated annealing or ant colony systems. Another important characteristic of the evaluation function should be stated: Compared to more classical problems as the traveling salesman problem or the quadratic assignment problem, the time necessary to calculate the global grade is excessive. An efficient algorithm should therefore not demand the evaluation of many solutions.\par{}}}{\baselineskip}{\baselineskip}\par\indent The above quote describes the problem as being descrite. That is, that the set of possible solutions is finite. The search space is large but finite. The kind of problem we are dealing with is one where we are looking for the best combination among a finite set of combinations. Best in this case is determined by some sort of ranking method which must also be defined. But, with regard to the computational methods which can be used to find the best solution layout, the methods which are usable to solve this kind of problem can not include linear math which would allow us to categorically rule out certain regions of the search space\protect\footnote[11]{{\leftskip0pt\parindent1em\raisebox{\baselineskip}[0pt]{\protect\hypertarget{nLinearProblem}{}} It is possible that when dealing with such a large search space that all system interaction keyboard short cuts could be reserved. This would narrow down the search space but perhaps not significantly. An agressive form of this reservation system would be to reserve all system short cuts across all operating systems (MacOS, Windows, Linux) regardless of the system assigned function.}}. Metaheuristic algorithms find approximations towards a better solution based on evaluation criteria. These kinds of methods allow us to find an approximate solution rather that a definitive solution. The hope is that by testing random samples and then making strategic modifications to those samples and retesting them a suitable solution can be found quicker than testing every possible permutation in the search space.\par{}\indent As background information Eggers et al. references the quadratic assignment problem ({\hyperlink{vQAP}{{QAP}}}). The {\hyperlink{vQAP}{{QAP}}} is a square problem, that is both sides of the optimization are equal. The problem is situated like this: If I have 10 pegs how can I put those in 10 holes using the shortest amount of time or traveling the shortest distance. The traveling salesman problem is similar, in that the question has a constraint on the distances between the destinations. The traveling salesman problem is like this: If someone wanted to visit 10 cities on a limited budget given a known cost to visit each city what path does the salesman need to take so that they visit each city only once while incuring the smallest cost?\par{}\indent The {\hyperlink{vKAP}{{KAP}}} is different from the {\hyperlink{vQAP}{{QAP}}} in that the sides of the optimization are not symmetrical. For instance, if we take all the characters of a writing system\protect\footnote[12]{{\leftskip0pt\parindent1em\raisebox{\baselineskip}[0pt]{\protect\hypertarget{nWhatToOptimize}{}} What should be optimized? The glyphs of a writing system? The orthographical characters of a writing system? The factional units of a writing system? The graphpemes? The Phonemes?}} and assign them to the available keys on a keyboard the count of the constituents in each set might be different. To appreciate the complexity of the problem we should define the search space. That is, {\XLingPaperCharisZSILFontFamily{\textit{how many possible solutions would we need to test if we were going to test each possible option?}}} As discussed in section \hyperlink{sKeyboardsGenreal}{2.1} and shown in table \hyperlink{ntISOJISANSI}{6} keyboards come in various physical sizes. Table \hyperlink{ntKeyTypes}{8} shows how many character producing keys and modifier keys exist for each of the keyboards presented in table \hyperlink{ntISOJISANSI}{6}.\par{}\vspace{11pt plus 2pt minus 1pt}\XLingPaperneedspace{3\baselineskip}\protect\hypertarget{ntKeyTypes}{}\XLingPaperaddtocontents{ntKeyTypes}{\protect\raggedright{\singlespacing
{Table }{8.}{  Key types on Physical Keyboards\\}}}\vspace{0pt}{\singlespacing
\hspace*{.25in}{
\XLingPaperminmaxcellincolumn{Physical}{\XLingPapermincola}{\textbf{Physical Keyboard Layout Type}}{\XLingPapermaxcola}{+0\tabcolsep}
\XLingPaperminmaxcellincolumn{Character}{\XLingPapermincolb}{\textbf{Character Producing Keys}}{\XLingPapermaxcolb}{+0\tabcolsep}
\XLingPaperminmaxcellincolumn{Modifier}{\XLingPapermincolc}{\textbf{Modifier Keys}}{\XLingPapermaxcolc}{+0\tabcolsep}
\XLingPaperminmaxcellincolumn{}{\XLingPapermincola}{{\hyperlink{vJIS}{{JIS}}}}{\XLingPapermaxcola}{+0\tabcolsep}
\XLingPaperminmaxcellincolumn{49}{\XLingPapermincolb}{49}{\XLingPapermaxcolb}{+0\tabcolsep}
\XLingPaperminmaxcellincolumn{8}{\XLingPapermincolc}{8}{\XLingPapermaxcolc}{+0\tabcolsep}
\XLingPaperminmaxcellincolumn{}{\XLingPapermincola}{{\hyperlink{vISO}{{ISO}}}}{\XLingPapermaxcola}{+0\tabcolsep}
\XLingPaperminmaxcellincolumn{48}{\XLingPapermincolb}{48}{\XLingPapermaxcolb}{+0\tabcolsep}
\XLingPaperminmaxcellincolumn{5}{\XLingPapermincolc}{5}{\XLingPapermaxcolc}{+0\tabcolsep}
\XLingPaperminmaxcellincolumn{}{\XLingPapermincola}{{\hyperlink{vANSI}{{ANSI}}}}{\XLingPapermaxcola}{+0\tabcolsep}
\XLingPaperminmaxcellincolumn{47}{\XLingPapermincolb}{47}{\XLingPapermaxcolb}{+0\tabcolsep}
\XLingPaperminmaxcellincolumn{4}{\XLingPapermincolc}{4}{\XLingPapermaxcolc}{+0\tabcolsep}
\setlength{\XLingPaperavailabletablewidth}{433.62pt}
\setlength{\XLingPapertableminwidth}{\XLingPapermincola+\XLingPapermincolb+\XLingPapermincolc}
\setlength{\XLingPapertablemaxwidth}{\XLingPapermaxcola+\XLingPapermaxcolb+\XLingPapermaxcolc}
\XLingPapercalculatetablewidthratio{}
\XLingPapersetcolumnwidth{\XLingPapercolawidth}{\XLingPapermincola}{\XLingPapermaxcola}{-0\tabcolsep}
\XLingPapersetcolumnwidth{\XLingPapercolbwidth}{\XLingPapermincolb}{\XLingPapermaxcolb}{-2\tabcolsep}
\XLingPapersetcolumnwidth{\XLingPapercolcwidth}{\XLingPapermincolc}{\XLingPapermaxcolc}{-2\tabcolsep}\singlespacing\vspace*{-3\baselineskip}
\begin{longtable}
[l]{@{}p{\XLingPapercolawidth}p{\XLingPapercolbwidth}p{\XLingPapercolcwidth}@{}}\toprule\multicolumn{1}{@{}p{\XLingPapercolawidth}}{\textbf{Physical Keyboard Layout Type}}&\multicolumn{1}{p{\XLingPapercolbwidth}}{\textbf{Character Producing Keys}}&\multicolumn{1}{p{\XLingPapercolcwidth}@{}}{\textbf{Modifier Keys}}\\%
\midrule\endhead \multicolumn{1}{@{}p{\XLingPapercolawidth}}{{\hyperlink{vJIS}{{JIS}}}}&\multicolumn{1}{p{\XLingPapercolbwidth}}{49}&\multicolumn{1}{p{\XLingPapercolcwidth}@{}}{8}\\%
\multicolumn{1}{@{}p{\XLingPapercolawidth}}{{\hyperlink{vISO}{{ISO}}}}&\multicolumn{1}{p{\XLingPapercolbwidth}}{48}&\multicolumn{1}{p{\XLingPapercolcwidth}@{}}{5}\\%
\multicolumn{1}{@{}p{\XLingPapercolawidth}}{{\hyperlink{vANSI}{{ANSI}}}}&\multicolumn{1}{p{\XLingPapercolbwidth}}{47}&\multicolumn{1}{p{\XLingPapercolcwidth}@{}}{4}\\\bottomrule%
\end{longtable}
}
}\indent In terms of combinations the would be 49! x 8!. But this isn't the entire search space either. A user can press a key in multiple ways with respect to modifier keys as is shown in\par{}\vspace{11pt plus 2pt minus 1pt}\XLingPaperneedspace{3\baselineskip}\protect\hypertarget{ntKeypressTypes}{}\XLingPaperaddtocontents{ntKeypressTypes}{\protect\raggedright{\singlespacing
{Table }{9.}{  \\}}}\vspace{0pt}{\singlespacing
\hspace*{.25in}{
\XLingPaperminmaxcellincolumn{Modifier}{\XLingPapermincola}{\textbf{Modifier Keypress Type}}{\XLingPapermaxcola}{+0\tabcolsep}
\XLingPaperminmaxcellincolumn{Style}{\XLingPapermincolb}{\textbf{Style}}{\XLingPapermaxcolb}{+0\tabcolsep}
\XLingPaperminmaxcellincolumn{Name}{\XLingPapermincolc}{\textbf{Name}}{\XLingPapermaxcolc}{+0\tabcolsep}
\XLingPaperminmaxcellincolumn{character}{\XLingPapermincola}{Press before character key}{\XLingPapermaxcola}{+0\tabcolsep}
\XLingPaperminmaxcellincolumn{Sequence}{\XLingPapermincolb}{Sequence (broken chord)}{\XLingPapermaxcolb}{+0\tabcolsep}
\XLingPaperminmaxcellincolumn{Deadkey}{\XLingPapermincolc}{Deadkey}{\XLingPapermaxcolc}{+0\tabcolsep}
\XLingPaperminmaxcellincolumn{character}{\XLingPapermincola}{Press after character key}{\XLingPapermaxcola}{+0\tabcolsep}
\XLingPaperminmaxcellincolumn{Sequence}{\XLingPapermincolb}{Sequence (broken chord)}{\XLingPapermaxcolb}{+0\tabcolsep}
\XLingPaperminmaxcellincolumn{Modifier}{\XLingPapermincolc}{Modifier}{\XLingPapermaxcolc}{+0\tabcolsep}
\XLingPaperminmaxcellincolumn{Press}{\XLingPapermincola}{Long Press}{\XLingPapermaxcola}{+0\tabcolsep}
\XLingPaperminmaxcellincolumn{depressed}{\XLingPapermincolb}{Press and keep depressed}{\XLingPapermaxcolb}{+0\tabcolsep}
\XLingPaperminmaxcellincolumn{}{\XLingPapermincolc}{}{\XLingPapermaxcolc}{+0\tabcolsep}
\XLingPaperminmaxcellincolumn{Multi-tap}{\XLingPapermincola}{Multi-tap}{\XLingPapermaxcola}{+0\tabcolsep}
\XLingPaperminmaxcellincolumn{Rota}{\XLingPapermincolb}{Rota}{\XLingPapermaxcolb}{+0\tabcolsep}
\XLingPaperminmaxcellincolumn{}{\XLingPapermincolc}{}{\XLingPapermaxcolc}{+0\tabcolsep}
\XLingPaperminmaxcellincolumn{}{\XLingPapermincola}{}{\XLingPapermaxcola}{+0\tabcolsep}
\XLingPaperminmaxcellincolumn{}{\XLingPapermincolb}{}{\XLingPapermaxcolb}{+0\tabcolsep}
\XLingPaperminmaxcellincolumn{}{\XLingPapermincolc}{}{\XLingPapermaxcolc}{+0\tabcolsep}
\setlength{\XLingPaperavailabletablewidth}{433.62pt}
\setlength{\XLingPapertableminwidth}{\XLingPapermincola+\XLingPapermincolb+\XLingPapermincolc}
\setlength{\XLingPapertablemaxwidth}{\XLingPapermaxcola+\XLingPapermaxcolb+\XLingPapermaxcolc}
\XLingPapercalculatetablewidthratio{}
\XLingPapersetcolumnwidth{\XLingPapercolawidth}{\XLingPapermincola}{\XLingPapermaxcola}{-0\tabcolsep}
\XLingPapersetcolumnwidth{\XLingPapercolbwidth}{\XLingPapermincolb}{\XLingPapermaxcolb}{-2\tabcolsep}
\XLingPapersetcolumnwidth{\XLingPapercolcwidth}{\XLingPapermincolc}{\XLingPapermaxcolc}{-2\tabcolsep}\singlespacing\vspace*{-3\baselineskip}
\begin{longtable}
[l]{@{}p{\XLingPapercolawidth}p{\XLingPapercolbwidth}p{\XLingPapercolcwidth}@{}}\toprule\multicolumn{1}{@{}p{\XLingPapercolawidth}}{\textbf{Modifier Keypress Type}}&\multicolumn{1}{p{\XLingPapercolbwidth}}{\textbf{Style}}&\multicolumn{1}{p{\XLingPapercolcwidth}@{}}{\textbf{Name}}\\%
\midrule\endhead \multicolumn{1}{@{}p{\XLingPapercolawidth}}{Press before character key}&\multicolumn{1}{p{\XLingPapercolbwidth}}{Sequence (broken chord)}&\multicolumn{1}{p{\XLingPapercolcwidth}@{}}{Deadkey}\\%
\multicolumn{1}{@{}p{\XLingPapercolawidth}}{Press after character key}&\multicolumn{1}{p{\XLingPapercolbwidth}}{Sequence (broken chord)}&\multicolumn{1}{p{\XLingPapercolcwidth}@{}}{Modifier}\\%
\multicolumn{1}{@{}p{\XLingPapercolawidth}}{Long Press}&\multicolumn{1}{p{\XLingPapercolbwidth}}{Press and keep depressed}&\multicolumn{1}{p{\XLingPapercolcwidth}@{}}{}\\%
\multicolumn{1}{@{}p{\XLingPapercolawidth}}{Multi-tap}&\multicolumn{1}{p{\XLingPapercolbwidth}}{Rota}&\multicolumn{1}{p{\XLingPapercolcwidth}@{}}{}\\%
\multicolumn{1}{@{}p{\XLingPapercolawidth}}{}&\multicolumn{1}{p{\XLingPapercolbwidth}}{}&\multicolumn{1}{p{\XLingPapercolcwidth}@{}}{}\\\bottomrule%
\end{longtable}
}
}\indent : (1) press before, (2) press after, (2) long press, hold-and-press (chorded), multi-press, and simultaneous press.\par{}\indent x λ x Ω! where λ is the number of interactions such as press before, press after, long press, hold-and-press, multi-press, and simultaneous press, and Ω is the number of characters in the writing system.\par{}\indent Many languages have need of more typogrpahical characters than keys which fit on a keyboard.\par{}\indent Difficulty in (digital) text production is directly related to the complexity of the typing experience. One way to simplify the typing experience is to optimize the keyboard layout for text input in the target language\protect\footnote[13]{{\leftskip0pt\parindent1em\raisebox{\baselineskip}[0pt]{\protect\hypertarget{nTwoQuestions}{}} There are two separate but related questions: (1) what should a text input layout look like on a mobile touch screen device, and (2) what should a physical layout for a device like a laptop be. This paper only seeks to look at the later question.}}. Solving for an optimized keyboard layout is a type of quadratic assignment problem. It is classed as NP-Hard and in some cases also NP-Complete – depending on the parameters used for optimization, the orthography, and the physical arrangement in question. Within the keyboard optimization literature there are several heuristic methods which have been used to propose optimized keyboard layouts for full size keyboards, as summarized in table \hyperlink{ntKeyboardOptimization}{10}\protect\footnote[14]{{\leftskip0pt\parindent1em\raisebox{\baselineskip}[0pt]{\protect\hypertarget{nScope}{}} Others have used similar methods to optimize for mobile text entry, which is outside of the scope of this study.}}.\par{}\indent applying specific methods such as: \hyperlink{gtSimulatedAnnealing}{{\textit{simulated annealing}}}, \hyperlink{gtTabooSearch}{{\textit{taboo search}}}, \hyperlink{gtGeneticAlgorithms}{{\textit{genetic algorithms}}}, \hyperlink{gtevolutionaryalgorithms}{{\textit{evolutionary algorithms}}}, \hyperlink{gtAntColonyOptimization}{{\textit{ant colony optimization}}}, \hyperlink{gtcyberswarm}{{\textit{cyber swarm}}}, or \hyperlink{gtparticleswarmoptimization}{{\textit{particle swarm optimization}}}.\par{}\indent R. E. Burkard and J. Offermann, Entwurf von Schreibmaschinentastaturen mittels quadratischer Zuordnungsprobleme, Zeitschrift f¨ur Operations Research 21, 1977, B121–B132, (in German).\par{}\vspace{11pt plus 2pt minus 1pt}\XLingPaperneedspace{3\baselineskip}\protect\hypertarget{ntKeyboardOptimization}{}\XLingPaperaddtocontents{ntKeyboardOptimization}{\protect\raggedright{\singlespacing
{Table }{10.}{  Keyboard Optimization Methods\\}}}\vspace{0pt}{\singlespacing
\hspace*{.25in}{\singlespacing\vspace*{-3\baselineskip}
\begin{longtable}
[l]{@{}lll@{}}\toprule\multicolumn{1}{@{}l}{\textbf{Approach}}&\multicolumn{1}{l}{\textbf{Citation}}&\multicolumn{1}{l@{}}{\textbf{Language}}\\%
\midrule\endhead \multicolumn{1}{@{}l}{\multirow{2}{*}[1.25ex]{Evolutionary Algorithms}}&\multicolumn{1}{l}{\hyperlink{rWalker2003Evolv}{Walker (2003)}}&\multicolumn{1}{l@{}}{eng}\\%
&\multicolumn{1}{l}{{\XLingPaperLateefFontFamily{\hyperlink{r62A62763164A62E62F63164A62764162An.d.64A64368664A646}{پذيرش, تاريخ دريافت \& تاريخ  (1394 (2015))}}}}&\multicolumn{1}{l@{}}{eng}\\%
\multicolumn{1}{@{}l}{\multirow{3}{*}[2.5ex]{Genetic Algorithm}}&\multicolumn{1}{l}{\hyperlink{rDeshwal2006Ergon}{Deshwal et al. (2006)}}&\multicolumn{1}{l@{}}{hin}\\%
&\multicolumn{1}{l}{\hyperlink{rMalas2008Towar}{Malas et al. (2008)}}&\multicolumn{1}{l@{}}{ara}\\%
&\multicolumn{1}{l}{\hyperlink{rLiao2013Chine}{Liao \& Choe (2013)}}&\multicolumn{1}{l@{}}{cmn}\\%
\multicolumn{1}{@{}l}{Chromosome Evaluation}&\multicolumn{1}{l}{\hyperlink{rWandT2016}{Wołosik \& Tabędzki (2016)}}&\multicolumn{1}{l@{}}{pol}\\%
\multicolumn{1}{@{}l}{Cyber Swarm}&\multicolumn{1}{l}{\hyperlink{rYinAndSu}{Yin \& Su (2011)}}&\multicolumn{1}{l@{}}{eng}\\%
\multicolumn{1}{@{}l}{\multirow{3}{*}[2.5ex]{Ant Colony Algorithm}}&\multicolumn{1}{l}{\hyperlink{rEggersetalPreprint2003}{Eggers et al. (2003a)}}&\multicolumn{1}{l@{}}{fra}\\%
&\multicolumn{1}{l}{\hyperlink{rEggers2003Optim}{Eggers et al. (2003b)}}&\multicolumn{1}{l@{}}{fra}\\%
&\multicolumn{1}{l}{\hyperlink{rWagner2003}{Wagner et al. (2003)}}&\multicolumn{1}{l@{}}{fra}\\%
\multicolumn{1}{@{}l}{\multirow{4}{*}[3.75ex]{Simulated Annealing}}&\multicolumn{1}{l}{\hyperlink{Light-Anderson}{Light \& Anderson (1993)}}&\multicolumn{1}{l@{}}{eng}\\%
&\multicolumn{1}{l}{\hyperlink{rBehbahan2011Optim}{Behbahan (2011)}}&\multicolumn{1}{l@{}}{fsa}\\%
&\multicolumn{1}{l}{\hyperlink{rV12Btoli1461612011}{Vītoliņš (2011)}}&\multicolumn{1}{l@{}}{lav}\\%
&\multicolumn{1}{l}{\hyperlink{rSalvo2016}{Salvo et al. (2016)}}&\multicolumn{1}{l@{}}{fil}\\%
\multicolumn{1}{@{}l}{Stastical Frequency}&\multicolumn{1}{l}{\hyperlink{rMarinaras1993Desig}{Marinaras \& Lyritzis (1993)}}&\multicolumn{1}{l@{}}{ell}\\%
\multicolumn{1}{@{}l}{Integer Programming}&\multicolumn{1}{l}{\hyperlink{rKarrenbauerAndreasAnttiOulasvirta2014Impro}{Karrenbauer \& Oulasvirta (2014)}}&\multicolumn{1}{l@{}}{eng}\\%
\multicolumn{1}{@{}l}{Cuckoo Search}&\multicolumn{1}{l}{\hyperlink{rSotoetal2016}{Soto et al. (2016)}}&\multicolumn{1}{l@{}}{n/a}\\%
\multicolumn{1}{@{}l}{}&\multicolumn{1}{l}{\hyperlink{rKhorshid2010}{Khorshid et al. (2010)}}&\multicolumn{1}{l@{}}{ara}\\%
\multicolumn{1}{@{}l}{}&\multicolumn{1}{l}{}&\multicolumn{1}{l@{}}{ara}\\%
\multicolumn{1}{@{}l}{}&\multicolumn{1}{l}{}&\multicolumn{1}{l@{}}{}\\%
\multicolumn{1}{@{}l}{}&\multicolumn{1}{l}{}&\multicolumn{1}{l@{}}{}\\%
\multicolumn{1}{@{}l}{}&\multicolumn{1}{l}{}&\multicolumn{1}{l@{}}{}\\\bottomrule%
\end{longtable}
}
}\indent \hyperlink{rEggers2003Optim}{Eggers et al. (2003b:672)} highlight the lack of consistency across the literature on how to measure the ergonomic interactions related to typing namely: (1) what counts as a penalty and (2) where and how is a penalty weighted in an algorithm. Despite these variations in measurement, most researchers agree on the broad types of actions which should be measured (Shieh and Lin, 1999:114; Wagner, 2001; Eggers et al., 2003; Yin and Su, 2011:44). These broad actions include: key tapping load distribution, the total number of keystrokes, hand alternation, finger alternation, finger posture, and hit direction (little finger to thumb). Even more challenging than Eggers’ point about ergonomic penalty systems, is that the results of various inquiries are not comparable because the performance ratings (often cast as a fitness score) for a given keyboard layout are software and/or corpus dependent. Therefore, results are not cross-experiment comparable.\par{}{\vspace{15pt}\XLingPaperneedspace{3\baselineskip}\noindent
\fontsize{13}{15.6}\selectfont \textbf{{\noindent
\raisebox{\baselineskip}[0pt]{\pdfbookmark[2]{{2.3 } Current state of assessing comparing keyboard efficiency}{sAsseingKeyboards}}\raisebox{\baselineskip}[0pt]{\protect\hypertarget{sAsseingKeyboards}{}}{2.3 }Current state of assessing comparing keyboard efficiency}}\markboth{Current state of assessing comparing keyboard efficiency}{Multiple perspectives on the keyboard layout problem}\XLingPaperaddtocontents{sAsseingKeyboards}}\par{}
\penalty10000\vspace{10pt}\penalty10000\indent To date, researchers in the field of language development are effectively blind when it comes to our ability to compare efficiency of two proposed keyboard layouts serving the same language. We can, along with the computer science researchers, look to ergonomic models to optimize keyboards, but the sorts of questions being asked and their answers do not give us any indication on how well a keyboard fits a given language based on the language's linguistic properties. A good framework does not exist for allowing researchers to ask the question: does QWERTY serve English users equally as well as AZERTY does to French users? - or any number of keyboards and their corresponding languages. There are two basic ways which typing interactions are conceptualized. The first is as a linear distance between fixed targets. When this is the case then researchers have attempted to measure the combined linear distance. Variation in linear distance measures do exist in the literature. Some researchers have measured from key to key directly, others assume that the finger comes to a rest in a neutral position. When researchers remove characters from the scope of their study they also then remove the associated distance to those keys. The second method of conceptualizing the typing task is as a series of pointing gestures. When this is the basis of measurement then Fitts’s law is usually used to measure the movement needed to complete the typing task. By focusing on the movement needed, researchers can make an assessment on the total effort needed. The general keyboard arrangement problem (GKAP) has been situated as a time over distance problem, i.e., speed. The assumption has been that if fingers travel a smaller distance, then faster typing should be the result. A secondary argument to increase speed has been to reduce effort. The assessment of effort has been tied to Fitts’ law \hyperlink{rMacKenzie1992}{(MacKenzie  1992)} for measurements of physical effort. However, mental effort or user-perceived effort remains unassessed. It is in this component of a fitness score that the proposed metrics fit best.\par{}\indent By visualizing keystrokes, we can identify the locus of impact in any given language. In language development work it is often the case that low-usage keys in an English-QWERTY layout are replaced with “new characters” to form keyboard layouts. The consequence has been the placement of frequently used keys under weak fingers, as demonstrated by the {\textit{Me'phaa}} keyboard layout presented in figure \hyperlink{ftcfHeatMap}{13}. Notice the high frequency red(dish) areas and the lower frequency blue areas. These red areas would normally be struck with the pinky fingers.\par{}\vspace{11pt plus 2pt minus 1pt}\setbox0=\vbox{\protect\centering \leavevmode
\vspace*{0pt}{\XeTeXpicfile "../Resources/images/[tcf] heatmap with full text.jpg" scaled 500}\\[0pt]\protect\hypertarget{ftcfHeatMap}{}\XLingPaperaddtocontents{ftcfHeatMap}{\singlespacing
{Figure }{13.}{ Heatmap of {\textit{Me'phaa}} Keyboard while typing James\\}}}\box0\par{}\vspace{11pt plus 2pt minus 1pt}\indent Spanish, distributes the same amount of information across multiple fingers as shown in figure \hyperlink{fspaHeatMap}{14} below. According to current keyboard layout theory even the Spanish keyboard is heavy on the weaker fingers of the left hand, but at least is better than the Me’phaa layout.\par{}\vspace{11pt plus 2pt minus 1pt}\setbox0=\vbox{\protect\centering \leavevmode
\vspace*{0pt}{\XeTeXpicfile "../Resources/images/[spa] heatmap with full text.jpg" scaled 500}\\[0pt]\protect\hypertarget{fspaHeatMap}{}\XLingPaperaddtocontents{fspaHeatMap}{\singlespacing
{Figure }{14.}{ Heatmap of Spanish Keyboard while typing James\\}}}\box0\par{}\vspace{11pt plus 2pt minus 1pt}\indent Contrasting these two languages allows us to visualize how they are different with respect to the actions of the pinky finger. It also demonstrates how the attempt to ‘conserve QWERTY’ impacts typing in minority languages. Notice how the {\textit{Me'phaa}} keyboard layout replaces the Spanish {\XLingPaperCambriaZMathFontFamily{\textup{\textmd{⟨ ç ⟩}}}} character with a tone diacritic key. Also note that these tone diacritics are very frequent in Meꞌphaa, and create a typing imbalance. Additionally, the {\textit{Me'phaa}} Saltillo {\XLingPaperCambriaZMathFontFamily{\textup{\textmd{⟨ ' ⟩}}}} key is also on the right side with heavy use, further contributing to the typing imbalance. Heatmaps tell the story of where the target is for typist fingers. However, a chart tells us how busy each finger is. Chart 1 compares the same content (a translation of the Epistle of James from the Christian New Testament in each language) on the standard keyboard for each language.\par{}{\vspace{15pt}\XLingPaperneedspace{3\baselineskip}\noindent
\fontsize{13}{15.6}\selectfont \textbf{{\noindent
\raisebox{\baselineskip}[0pt]{\pdfbookmark[2]{{2.4 } Typing as a linguistic activity}{sLingusiticActivity}}\raisebox{\baselineskip}[0pt]{\protect\hypertarget{sLingusiticActivity}{}}{2.4 }Typing as a linguistic activity}}\markboth{Typing as a linguistic activity}{Multiple perspectives on the keyboard layout problem}\XLingPaperaddtocontents{sLingusiticActivity}}\par{}
\penalty10000\vspace{10pt}\penalty10000\indent Is writing a linguistic activity?\par{}\indent Is typing a linguistic activity?\par{}\indent How does typing integrate with brain processes?\par{}\indent What does this mean for typing orthographires and tone?\par{}{\vspace{15pt}\XLingPaperneedspace{3\baselineskip}\noindent
\fontsize{13}{15.6}\selectfont \textbf{{\noindent
\raisebox{\baselineskip}[0pt]{\pdfbookmark[2]{{2.5 } Mapping linguistics to technology: orthographies in keyboards}{sMappingLtoT}}\raisebox{\baselineskip}[0pt]{\protect\hypertarget{sMappingLtoT}{}}{2.5 }Mapping linguistics to technology: orthographies in keyboards}}\markboth{Mapping linguistics to technology: orthographies in keyboards}{Multiple perspectives on the keyboard layout problem}\XLingPaperaddtocontents{sMappingLtoT}}\par{}
\penalty10000\vspace{10pt}\penalty10000{\vspace{10pt}\XLingPaperneedspace{3\baselineskip}\noindent
\fontsize{13}{15.6}\selectfont \textit{{\noindent
\raisebox{\baselineskip}[0pt]{\pdfbookmark[3]{{2.5.1 } Digraphs}{sDigraphs}}\raisebox{\baselineskip}[0pt]{\protect\hypertarget{sDigraphs}{}}{2.5.1 }Digraphs}}\markboth{Digraphs}{Multiple perspectives on the keyboard layout problem}\XLingPaperaddtocontents{sDigraphs}}\par{}
\penalty10000\vspace{10pt}\penalty10000\indent There are two Unicode models which are relevant to this study: The writing systems model (Constable 2002) and the character model (Unicode Technical Report \#17, Unicode Technical Report \#23). Unicode also hosts keyboard locale data in the Unicode Common Locale Data Repository (CLDR v33 at the time of writing). Only the first two Unicode items are discussed in this section, but the person interested in this study should also be aware of the locale data for keyboards.\par{}\indent Unicode has a model for scripts, writing systems, and orthographies. These terms all carry specific meaning as is outlined by Constable (2002). Following Constable (2002) there is a rather narrow definition of orthography. This more technical sense of the term presumes a writing system has been identified and adds the following features to that writing system: an orthography specifies specific spelling conventions, when uppercase letters should be used, conventions for hyphenation, abbreviations, contractions. Constable is unclear about word break conventions and conventions for Bantu like languages which use hyphenation not as word dividers but as morphology connectors. While a writing system selects the characters which are used in expressing certain grammatical features (commas and full stops, etc.), it is the orthography which says how they are applied.\par{}\indent Unicode has a character model. For some writing systems, such as those with diacritics and especially those with stacking diacritics, the encoding can become more complex than other writing systems. This is in part because the notion of ‘a character’ is context dependent. Three disciplines donate ideas and terms to describe characters: Typography (drawing letters), Computer Science (programing), and Linguistics (orthography). Each context or discipline defines a ‘character’ in a slightly different manner. Written language users often think of words in terms of the ‘letters’ or graphical units from which they are formed. Constable (2001: 10) calls these ‘orthographical characters’, others such as Holm (1971:5) call these units ‘functional units’. A functional unit might consist of one or more distinct graphical elements known as graphemes. When more than one graphical element is used in combination, then the combination can be called a multigraph.\par{}\indent To the reader of an orthography, multi-graphs may represent one unit of information or they may represent multiple units of information. The visual components of an orthographical character may also have sub-units which individually relay information. Following Constable (2003: 28) graphemes are anything that functions as a distinct unit within an orthography. This is an important notion because the orthography is the domain of control for the meaning of a graphical element. The same visual shape may appear in several orthographies but (1) have different meanings or (2) not have independent meaning apart in an independent form. \par{}\indent For example, diaeresis marks (sometimes known as umlauts or trema) which in Unicode are called 'COMBINING DIAERESIS' U+0308 ⟨ ¨ ⟩ serve several functions across the world’s writing systems. In English they tell the reader that two vowels are not a di-graph and should be read as separate vowels, as in: naïve, Noël, coöperation. The same mark in the Swedish \textsquarebracketleft{}swe\textsquarebracketright{} orthography, is part of the Swedish letter ⟨ Ä ⟩ (Granberry 1991: 7, Svenska Akademien 2006). The way that the Swedish letter is conceptualized, by users of Swedish, is such that the mark is not removable from the letter. There is not just an addition to an existing letter but a whole new letter of the alphabet. In a typographical sense the diaeresis in an independent, removable, and alternating component in the Swedish orthography. However, in the minds of the reader, the grapheme is the whole letter ⟨ Ä ⟩ not a combination of two graphemes ⟨ A + ¨ ⟩.\par{}\indent This distinction between typographical separation, technical separation and orthographic separation is relevant to the development of keyboard layouts for two reasons. The first is that the keyboard is the first line of encoding for the technical capacity. As such the keyboard needs to input the correct characters in the correct encoding. But secondly, previous studies (which have mostly all been on English texts) have developed the idea of computing text input efficiency in terms of ‘keystrokes per character’ (MacKenzie 2002, Soukoreff \& MacKenzie 2003). However, these studies have left the definition of ‘character’ ambiguous. That is, does ‘character’ mean the ‘functional unit’ or ‘orthographical character’, or does it mean the single Unicode code point?\par{}\indent Functional units are language dependent and are not immediately inferable via the Unicode method of encoding. This means that while previous studies may be well founded for English, ASCII characters, and some languages which use precomposed characters, additional computation is needed to be able to apply previous metrics to many of the world’s writing systems which use combining diacritics.\par{}{\vspace{10pt}\XLingPaperneedspace{3\baselineskip}\noindent
\fontsize{13}{15.6}\selectfont \textit{{\noindent
\raisebox{\baselineskip}[0pt]{\pdfbookmark[3]{{2.5.2 } Tonal Melodies}{sTonalMelodies}}\raisebox{\baselineskip}[0pt]{\protect\hypertarget{sTonalMelodies}{}}{2.5.2 }Tonal Melodies}}\markboth{Tonal Melodies}{Multiple perspectives on the keyboard layout problem}\XLingPaperaddtocontents{sTonalMelodies}}\par{}
\penalty10000\vspace{10pt}\penalty10000\indent Orthographies are not composed of Unicode characters, but of linguistic functional units. The encoding of a functional unit can change from orthography to orthography. This is missed by many GKAP researchers. One needs to look beyond Unicode characters and toward ratios of keystrokes to functional units.\par{}\indent One way that linguists look at functional units differently than “pure technologists” is in the area of tone. \hyperlink{rSnider2014}{Snider (2014)} and \hyperlink{rHymanLary2014}{Hyman (2014)} both argue for an analysis of tone that looks at patterns across the domain of tonal attachment (usually the word or morpheme) rather than the more structuralist and segmental view of “letter plus pitch” which is the encoding method that Unicode follows. The difference in behavior becomes clearer when we contrast reading diacritic tone marks with a language which does not use diacritical marking. Imagine the two sentences in figure \hyperlink{fWith-without-toneMarks}{15} represents the same sentence in two different languages, one with tone and one without tone.\par{}\vspace{11pt plus 2pt minus 1pt}\setbox0=\vbox{\protect\centering \leavevmode
\vspace*{0pt}{\XeTeXpdffile "../Resources/SVG-PDFs/Sentencewith-without.pdf" scaled 600}\\[0pt]\protect\hypertarget{fWith-without-toneMarks}{}\XLingPaperaddtocontents{fWith-without-toneMarks}{\singlespacing
{Figure }{15.}{ The Same sentence – with and without tone marks.\\}}}\box0\par{}\vspace{11pt plus 2pt minus 1pt}\indent A tone pattern analysis (following Snider and Hyman) would suggest that we should understand the tonally marked sentence to represent the segmental and the suprasegmental tiers separately, something like what is indicated in figure \hyperlink{fReadingtone}{16}.\par{}\vspace{11pt plus 2pt minus 1pt}\setbox0=\vbox{\protect\centering \leavevmode
\vspace*{0pt}{\XeTeXpdffile "../Resources/SVG-PDFs/Tone-Explination.pdf" scaled 600}\\[0pt]\protect\hypertarget{fReadingtone}{}\XLingPaperaddtocontents{fReadingtone}{\singlespacing
{Figure }{16.}{ Tonal Melodies\\}}}\box0\par{}\vspace{11pt plus 2pt minus 1pt}\indent It is observed, anecdotally, that many writers of tone languages parse the writing task into two activities, one for each tier of information. Usually the segmental tier is completed first and then the suprasegmental tier. Figure \hyperlink{fReadingtwoAccross}{17} illustrates this as two passes; the orange line first then the green line (lower, then upper lines).\par{}\vspace{11pt plus 2pt minus 1pt}\setbox0=\vbox{\protect\centering \leavevmode
\vspace*{0pt}{\XeTeXpdffile "../Resources/SVG-PDFs/AssumeDiacriticReadtwoPass.pdf" scaled 600}\\[0pt]\protect\hypertarget{fReadingtwoAccross}{}\XLingPaperaddtocontents{fReadingtwoAccross}{\singlespacing
{Figure }{17.}{ Reading each line twice\\}}}\box0\par{}\vspace{11pt plus 2pt minus 1pt}\indent This makes the writing activity much like handwriting in the United States where we say: “don’t forget to dot your 'i's and cross your 't's”. The writer first completes the segmental tier and then comes back across with a second pass and completes dots the ‘i’s and crosses the ‘t’s.\par{}\indent However, the nature of typing, and Unicode’s structural nature, requires typists to complete the text production task linearly by codepoint. This creates dissonance from what many language users find “natural”, requiring them to frequently alternate their attention between the two tiers, as shown by the up and down movement in figure \hyperlink{fReadingUpDown}{18}.\par{}\vspace{11pt plus 2pt minus 1pt}\setbox0=\vbox{\protect\centering \leavevmode
\vspace*{0pt}{\XeTeXpdffile "../Resources/SVG-PDFs/ActualDiacriticRead.pdf" scaled 600}\\[0pt]\protect\hypertarget{fReadingUpDown}{}\XLingPaperaddtocontents{fReadingUpDown}{\singlespacing
{Figure }{18.}{ General logical flow for the typing process for typing tonal languages\\}}}\box0\par{}\vspace{11pt plus 2pt minus 1pt}\clearpage
\thispagestyle{bodyfirstpage}\markboth{Eastern Dan writing system}{Eastern Dan writing system}
\XLingPaperaddtocontents{sEDWritingSystem}{\vspace*{.65in}\XLingPaperneedspace{3\baselineskip}\noindent
\fontsize{14}{16.8}\selectfont \textbf{{\centering
CHAPTER \raisebox{\baselineskip}[0pt]{\protect\hypertarget{sEDWritingSystem}{}}\raisebox{\baselineskip}[0pt]{\pdfbookmark[1]{3 Eastern Dan writing system}{sEDWritingSystem}}3\protect\\}}}\par{}
{\XLingPaperneedspace{3\baselineskip}\noindent
\fontsize{14}{16.8}\selectfont \textbf{{\centering
Eastern Dan writing system\protect\\}}}\par{}
\vspace{16pt}\indent This chapter lays out the Eastern Dan writing system as best as is currently deducible from the academic record, and texts produced by Eastern Dan first language users. This analysis was conducted in the process of compiling and editing the corpus of Eastern Dan texts which is used for the experiments presented in chapter \hyperlink{cExperiments}{5}\protect\footnote[1]{{\leftskip0pt\parindent1em\raisebox{\baselineskip}[0pt]{\protect\hypertarget{nCorpus-location}{}} This corpus is unpublished, but temporarily accessible at: \href{https://github.com/HughP/dnj-corpus}{\textcolor[rgb]{0,0,0}{https://github.com/HughP/dnj-corpus}} }}. The corpus was provided by {Valentin Vydrin} who has worked extensively in Dan. I consulted many of Vydrin's published works and unpublished works. The corpus content includes a small collection issues from an newspaper, {\textit{˗Pamɛbhamɛ}}, published and circulated in the Ivory Coast\protect\footnote[2]{{\leftskip0pt\parindent1em\raisebox{\baselineskip}[0pt]{\protect\hypertarget{nDownloadNewsPaper}{}} Some issues of the newspaper are downloadable from http://mandelang.kunstkamera.ru/files/mandelang/*. There does not seem to be a list of PDFs. But by altering the file name between {\XLingPaperCharisZSILFontFamily{\textit{gweta10.pdf}}} and {\XLingPaperCharisZSILFontFamily{\textit{gweta39.pdf}}} one can find many of the issues used to create the corpus. eg. \href{http://mandelang.kunstkamera.ru/files/mandelang/gweta10.pdf}{\textcolor[rgb]{0,0,0}{http://mandelang.kunstkamera.ru/files/mandelang/gweta10.pdf}}}}, and medical counsels (chapters) from {\XLingPaperCharisZSILFontFamily{\textit{While waiting for a medical doctor}}} translated into Eastern Dan \hyperlink{rKessE9gbeu2007}{(Kességbeu  2007)}. \par{}\indent This chapter attempts a modest application of the principles set out by {Martin Hosken} \hyperlink{Hosken}{(2003)} for \hyperlink{gtWSD}{{\textit{Orthography descriptions}}}. Writing to a mostly SIL audience, \hyperlink{Hosken}{Hosken (2003)} distinguishes between an orthography description and an orthography statement, often an internally required document, which:\par{}\XLingPaperblockquote{.25in}{{\singlespacing
\vspace{-1.3\baselineskip}justif\textsquarebracketleft{}ies\textsquarebracketright{} the decisions that went into creating a particular orthography. As such it is written in terms of the orthography's linguistic and sociolinguistic basis and contains the details required to justify the decisions made.\par{}}}{\baselineskip}{\baselineskip}\noindent Often these documents look like a phoneme list with the glyphs chosen to represent the phonemes. Depending on the complexity of the language issues of allophony might be addressed. In contrast to an orthography statement, an orthography description, according to Hosken is much more technical and is aimed at implementing the writing system in digital contexts. Work following Hosken's outline of issues focuses on the writing system, rather than the orthography, therefore it makes sense to me to call it a writing system description instead of an orthography description. Much of the content in an orthography statement would be included in a writing system description, but the \hyperlink{gtWSD}{{\textit{writing system description}}} will include more clarity for computer technicians who build software to support languages\protect\footnote[3]{{\leftskip0pt\parindent1em\raisebox{\baselineskip}[0pt]{\protect\hypertarget{nGeezWritingSystemDescription}{}} \hyperlink{rYacob2016}{Yacob \& Ishida (2016)} present an excellent, albeit work-in-progress, exemplar of a writing system description for orthographies which use the Ethiopic script.}}.\par{}\indent Under-described writing systems are a major roadblock to the development of full locale support for languages. It is often the same set of details that are used to describe writing systems which contribute to the development of digital resources which specifically target language users; such language resources as spell checkers, grammar checkers, speech to text engines, appropriate font support, and default keyboard layouts.\par{}\indent Every orthography and writing system has an origin story. Some origin stories are more well-documented than others. Just as keyboard layouts are objects of design, orthographies and writing systems are objects of design. These design decisions, and their underlying rationels should be documented just as much as the system as a whole should be described following proposals like Hosken's. As an orthography evolves, the documentation should evolve with it.\par{}\indent The narrative for the evolution of the Eastern Dan orthography suggests that language development workers {Margrit Bolli} and {Eva Flik} generally focused on Western Dan, first and then soon after or simultaneously adapted orthography changes to Eastern Dan. A distinct narrative for Eastern Dan, independent from Western Dan does not appear until 1982. However, some literacy was happening in Eastern Dan under their mentorship as early as 1972. The closest thing to a formal \hyperlink{gtWSD}{{\textit{writing system description}}} for Eastern Dan is a community oriented reader \hyperlink{rBolli1994Cours}{(Bolli \& Flik  1994)} which covers: vowels, consonants, numbers, and basic punctuation. The 1994 reader improves upon a community oriented reader \hyperlink{rBF1982}{(Bolli \& Flik  1982)} by offering sections on numbers and punctuation. However, neither book presents an alphabetic order, or an alphabet in whole (all at one time)\protect\footnote[4]{{\leftskip0pt\parindent1em\raisebox{\baselineskip}[0pt]{\protect\hypertarget{nDictionary}{}} An Eastern Dan dictionary \hyperlink{rVydrin2008EDDictionary}{(Vydrine \& Kességbeu   2008:10, 366-368)} does have a sort order based on a phoneme list and the transcriptions of those phonemes. This is addressed in more detail in section \hyperlink{sAlphabet}{3.4.1}.}}. The readers are designed for people who have already learned to read French, and are using transferable skills to learn to read Eastern Dan. The comparisons to French writing, and pedagogical assumptions about what Dan readers/writers already know about French are so strong. It begs one to ask: {\XLingPaperCharisZSILFontFamily{\textit{is the presentation of writing in Dan 'French orthography adapted for Dan', or is it a 'unique writing system for Dan' ready to stand on its own and greet a world of writing systems?}}} Given the socio-political environment Bolli and Flik were working in, it makes sense that they cast their works as an adaption of French to an indigenous language of Côte d'Ivore \hyperlink{rBurmeister1980Toneo}{(Burmeister  1980)}. Certainly today, the global understanding of and desire for equality, for all languages and their speakers, has changed.\par{}\indent Several forthcoming works do offer linguistic descriptions of the orthography \hyperlink{rRobertssubmittedChapt}{(Roberts \& Vydrin  (submitted))}, orthography testing \hyperlink{rVandRChapter}{(Vydrin \& Roberts  (submitted)}; \hyperlink{rRobertsn.d.Marki}{Roberts, Basnight-Brown \& Vydrin  (forthcoming))}, and a newly proposed writing system (version 4, discussed in table \hyperlink{ntHistoryOfDanOrthography}{11}), but these works focus on the correspondences between linguistic units and typographical units, leaving room for other academic works to provide details at the technical and writing system levels; focusing on the current writing system and orthography.\par{}\indent Although Dan (both Eastern and Western) had a vigorous literacy program\protect\footnote[5]{{\leftskip0pt\parindent1em\raisebox{\baselineskip}[0pt]{\protect\hypertarget{nLiteracy-ProgramDescription}{}} The Literacy program is described in detail in \hyperlink{rBolli1980Progr}{Bolli (1980a}, \hyperlink{rBolli1980Yacou}{1980b}, \hyperlink{rBolliMargrit1983jofr}{1983b}, \hyperlink{rBolli1991Ortho}{1991)} and \hyperlink{rLauber1982Thein}{Lauber (1982}, \hyperlink{rLauber1983Thein}{1983)}.}} during the 1970's and the early 1980's, Eastern Dan does not have a \hyperlink{gtWSD}{{\textit{writing system description}}} beyond the 1994 reader; it fails to give full technical details such as Unicode values for characters in the orthography. The following subsections provide descriptive detail for the orthography version 3, based on analysis of the texts (corpus) made available. Because no official writing system description has been published the exact Unicode encoding of characters has not been ever officially established. Formally published scripture texts give us an insight into what professional typesetters (who are targeting readers in the Eastern Dan community) suggest for this orthography, but this still leave us guessing what, if anything is official recognized by the national or provincial governments. As demonstrated in the corpus and in available scripture texts, Eastern Dan’s orthography has some interesting applications of punctuation and symbols. That is, these punctuation marks represent tone patterns. These peculiarities are not unique in the world's writing systems, though they may be considered a regional phenomenon. That is, other languages of the Ivory Coast have or have had similar usages for punctuation and symbols \hyperlink{rBurmeister1980Toneo}{(Burmeister  1980}, \hyperlink{rHartell1993}{Hartell  1993)}\protect\footnote[6]{{\leftskip0pt\parindent1em\raisebox{\baselineskip}[0pt]{\protect\hypertarget{nChartofLanguages}{}} For a chart of languages which at one time or another have been reported to use punctuation marks to indicate tone see table \hyperlink{ntLanguagesWithTonePunctuation}{17} in section \hyperlink{sEasternDanOrthography}{3.3}.}}. Practically speaking though when writing practice does occur in Eastern Dan, it is often done without regard for the appropriate Unicode characters. Hypotheses are for this are explored in footnote \footnote[7]{{See footnote }\hyperlink{nApostrophyindan}{10} in chapter 3.} and in more detail in chapter \hyperlink{cDiscussion}{7}. Table \hyperlink{ntHistoryOfDanOrthography}{11} presents a chronological display of the evolutionary stages in the development of Eastern Dan's writing system.\par{}\vspace{11pt plus 2pt minus 1pt}\XLingPaperneedspace{3\baselineskip}\protect\hypertarget{ntHistoryOfDanOrthography}{}\XLingPaperaddtocontents{ntHistoryOfDanOrthography}{\protect\centering {\singlespacing
{Table }{11.}{  Evolutionary stages of the Eastern Dan writing system\\}}}\vspace{0pt}{\singlespacing
\hspace*{.25in}{\setcounter{footnote}{7}\singlespacing\vspace*{-3\baselineskip}
\begin{longtable}
[c]{@{}p{.5in}p{.8in}p{2in}p{.75in}p{1.25in}@{}}\toprule\multicolumn{1}{@{}p{.5in}}{\raggedright\textbf{Version}}&\multicolumn{1}{p{.8in}}{\raggedright\textbf{Date}}&\multicolumn{1}{p{2in}}{\raggedright\textbf{Evolutionary steps}}&\multicolumn{1}{p{.75in}}{\raggedright\textbf{Mentor/​Artist}}&\multicolumn{1}{p{1.25in}@{}}{\raggedright\textbf{Reference}}\\%
\midrule\endhead \multicolumn{1}{@{}p{.5in}}{\raggedright 0.0}&\multicolumn{1}{p{.8in}}{\raggedright 1943}&\multicolumn{1}{p{2in}}{\raggedright Unknown}&\multicolumn{1}{p{.75in}}{\raggedright Unknown}&\multicolumn{1}{p{1.25in}@{}}{\raggedright \hyperlink{rInternational1996Dan}{SIL International  1996}}\\%
\multicolumn{1}{@{}p{.5in}}{\raggedright }&\multicolumn{1}{p{.8in}}{\raggedright }&\multicolumn{1}{p{2in}}{\raggedright }&\multicolumn{1}{p{.75in}}{\raggedright }&\multicolumn{1}{p{1.25in}@{}}{\raggedright }\\%
\multicolumn{1}{@{}p{.5in}}{\raggedright 0.1}&\multicolumn{1}{p{.8in}}{\raggedright 1960's protestant}&\multicolumn{1}{p{2in}}{\raggedright Unknown}&\multicolumn{1}{p{.75in}}{\raggedright Unknown}&\multicolumn{1}{p{1.25in}@{}}{\raggedright \hyperlink{rBolli1980Progr}{Bolli  1980a}}\\%
\multicolumn{1}{@{}p{.5in}}{\raggedright }&\multicolumn{1}{p{.8in}}{\raggedright }&\multicolumn{1}{p{2in}}{\raggedright }&\multicolumn{1}{p{.75in}}{\raggedright }&\multicolumn{1}{p{1.25in}@{}}{\raggedright }\\%
\multicolumn{1}{@{}p{.5in}}{\raggedright \XLingPaperCharisZSILFontFamily{\fontsize{8}{9.6}\selectfont \textup{\textup{0.2a}}}}&\multicolumn{1}{p{.8in}}{\raggedright \XLingPaperCharisZSILFontFamily{\fontsize{8}{9.6}\selectfont \textup{\textup{pre-1970 protestant}}}}&\multicolumn{1}{p{2in}}{\raggedright \XLingPaperCharisZSILFontFamily{\fontsize{8}{9.6}\selectfont \textup{\textup{Imported from Liberia}}}}&\multicolumn{1}{p{.75in}}{\raggedright \XLingPaperCharisZSILFontFamily{\fontsize{8}{9.6}\selectfont \textup{\textup{Mission Biblique}}}}&\multicolumn{1}{p{1.25in}@{}}{\raggedright \XLingPaperCharisZSILFontFamily{\fontsize{8}{9.6}\selectfont \textup{\textup{\hyperlink{rBolli1983TheVi}{Bolli  1983a}}}}}\\%
\multicolumn{1}{@{}p{.5in}}{\raggedright }&\multicolumn{1}{p{.8in}}{\raggedright }&\multicolumn{1}{p{2in}}{\raggedright }&\multicolumn{1}{p{.75in}}{\raggedright }&\multicolumn{1}{p{1.25in}@{}}{\raggedright }\\%
\multicolumn{1}{@{}p{.5in}}{\raggedright \XLingPaperCharisZSILFontFamily{\fontsize{8}{9.6}\selectfont \textup{\textup{0.2b}}}}&\multicolumn{1}{p{.8in}}{\raggedright \XLingPaperCharisZSILFontFamily{\fontsize{8}{9.6}\selectfont \textup{\textup{pre-1970 catholic}}}}&\multicolumn{1}{p{2in}}{\raggedright \XLingPaperCharisZSILFontFamily{\fontsize{8}{9.6}\selectfont \textup{\textup{concurrent with but separate from version 0.2a}}}}&\multicolumn{1}{p{.75in}}{\raggedright \XLingPaperCharisZSILFontFamily{\fontsize{8}{9.6}\selectfont \textup{\textup{Roman Catholic Church}}}}&\multicolumn{1}{p{1.25in}@{}}{\raggedright \XLingPaperCharisZSILFontFamily{\fontsize{8}{9.6}\selectfont \textup{\textup{\hyperlink{rBolli1983TheVi}{Bolli  1983a}}}}}\\%
\multicolumn{1}{@{}p{.5in}}{\raggedright }&\multicolumn{1}{p{.8in}}{\raggedright }&\multicolumn{1}{p{2in}}{\raggedright }&\multicolumn{1}{p{.75in}}{\raggedright }&\multicolumn{1}{p{1.25in}@{}}{\raggedright }\\%
\multicolumn{1}{@{}p{.5in}}{\raggedright \XLingPaperCharisZSILFontFamily{\fontsize{8}{9.6}\selectfont \textup{\textup{0.3}}}}&\multicolumn{1}{p{.8in}}{\raggedright \XLingPaperCharisZSILFontFamily{\fontsize{8}{9.6}\selectfont \textup{\textup{pre-1972}}}}&\multicolumn{1}{p{2in}}{\raggedright \XLingPaperCharisZSILFontFamily{\fontsize{8}{9.6}\selectfont \textup{\textup{high tone is marked at the beginning of the word with an apostrophe}}}}&\multicolumn{1}{p{.75in}}{\raggedright \XLingPaperCharisZSILFontFamily{\fontsize{8}{9.6}\selectfont \textup{\textup{Margrit Bolli / {Eva Flik}}}}}&\multicolumn{1}{p{1.25in}@{}}{\raggedright \XLingPaperCharisZSILFontFamily{\fontsize{8}{9.6}\selectfont \textup{\textup{\hyperlink{rBolli1978Writi}{Bolli (1978)}}}}}\\%
\multicolumn{1}{@{}p{.5in}}{\raggedright }&\multicolumn{1}{p{.8in}}{\raggedright }&\multicolumn{1}{p{2in}}{\raggedright }&\multicolumn{1}{p{.75in}}{\raggedright }&\multicolumn{1}{p{1.25in}@{}}{\raggedright }\\%
\multicolumn{1}{@{}p{.5in}}{\raggedright \XLingPaperCharisZSILFontFamily{\fontsize{8}{9.6}\selectfont \textup{\textup{0.4}}}}&\multicolumn{1}{p{.8in}}{\raggedright \XLingPaperCharisZSILFontFamily{\fontsize{8}{9.6}\selectfont \textup{\textup{1974}}}}&\multicolumn{1}{p{2in}}{\raggedright \XLingPaperCharisZSILFontFamily{\fontsize{8}{9.6}\selectfont \textup{\textup{There seems to be some change; not quite documented\protect\footnote{{\leftskip0pt\parindent1em\raisebox{\baselineskip}[0pt]{\protect\hypertarget{n1974Orthgraphy}{}} This evidence is a reader which contains no orthography statement, so it is difficult to tell if there might have been printing discrepancies or if writing system experimentation was ongoing.}}.}}}}&\multicolumn{1}{p{.75in}}{\raggedright \XLingPaperCharisZSILFontFamily{\fontsize{8}{9.6}\selectfont \textup{\textup{Margrit Bolli / {Eva Flik}}}}}&\multicolumn{1}{p{1.25in}@{}}{\raggedright \XLingPaperCharisZSILFontFamily{\fontsize{8}{9.6}\selectfont \textup{\textup{\hyperlink{rBaba1978Yaobh}{Baba (1978)}\footnote{{See footnote }\hyperlink{n1974Orthgraphy}{8} in chapter 3.}; \hyperlink{rRobertssubmittedChapt}{Roberts \& Vydrin  (submitted)}}}}}\\%
\multicolumn{1}{@{}p{.5in}}{\raggedright }&\multicolumn{1}{p{.8in}}{\raggedright }&\multicolumn{1}{p{2in}}{\raggedright ​}&\multicolumn{1}{p{.75in}}{\raggedright }&\multicolumn{1}{p{1.25in}@{}}{\raggedright }\\%
\multicolumn{1}{@{}p{.5in}}{\raggedright \XLingPaperCharisZSILFontFamily{\fontsize{8}{9.6}\selectfont \textup{\textup{0.5}}}}&\multicolumn{1}{p{.8in}}{\raggedright \XLingPaperCharisZSILFontFamily{\fontsize{8}{9.6}\selectfont \textup{\textup{1978}}}}&\multicolumn{1}{p{2in}}{\raggedright \XLingPaperCharisZSILFontFamily{\fontsize{8}{9.6}\selectfont \textup{\textup{full stop {\XLingPaperCambriaZMathFontFamily{\textup{\textmd{⟨ {\XLingPaperCharisZSILFontFamily{\textup{\textup{\textmd{.}}}}} ⟩}}}} is at the beginning of words to indicate low tone, {\XLingPaperCambriaZMathFontFamily{\textup{\textmd{⟨ {\XLingPaperCharisZSILFontFamily{\textup{\textup{\textmd{ô}}}}} ⟩}}}} is used, {\XLingPaperCambriaZMathFontFamily{\textup{\textmd{⟨ {\XLingPaperCharisZSILFontFamily{\textup{\textup{\textmd{.CVV'-}}}}} ⟩}}}} is a tone pattern used to indicate low-mid-fall}}}}&\multicolumn{1}{p{.75in}}{\raggedright \XLingPaperCharisZSILFontFamily{\fontsize{8}{9.6}\selectfont \textup{\textup{Margrit Bolli / {Eva Flik}}}}}&\multicolumn{1}{p{1.25in}@{}}{\raggedright \XLingPaperCharisZSILFontFamily{\fontsize{8}{9.6}\selectfont \textup{\textup{\hyperlink{rBolli1978Writi}{Bolli (1978)} In this resource the author does not indicate if they are discussing Eastern Dan, Western Dan, or both. In the 1982 version of the Western Dan reading primer the word final apostrophe hyphen sequence is present.}}}}\\%
\multicolumn{1}{@{}p{.5in}}{\raggedright }&\multicolumn{1}{p{.8in}}{\raggedright }&\multicolumn{1}{p{2in}}{\raggedright }&\multicolumn{1}{p{.75in}}{\raggedright }&\multicolumn{1}{p{1.25in}@{}}{\raggedright }\\%
\multicolumn{1}{@{}p{.5in}}{\raggedright \XLingPaperCharisZSILFontFamily{\fontsize{8}{9.6}\selectfont \textup{\textup{1.0}}}}&\multicolumn{1}{p{.8in}}{\raggedright \XLingPaperCharisZSILFontFamily{\fontsize{8}{9.6}\selectfont \textup{\textup{1982-1990}}}}&\multicolumn{1}{p{2in}}{\raggedright \XLingPaperCharisZSILFontFamily{\fontsize{8}{9.6}\selectfont \textup{\textup{No indication of full stop {\XLingPaperCambriaZMathFontFamily{\textup{\textmd{⟨ {\XLingPaperCharisZSILFontFamily{\textup{\textup{\textmd{.}}}}} ⟩}}}} usage at the beginning of words. No indication of word final apostrophe hyphen sequences {\XLingPaperCambriaZMathFontFamily{\textup{\textmd{⟨ {\XLingPaperCharisZSILFontFamily{\textup{\textup{\textmd{CVV'-}}}}} ⟩}}}}.}}}}&\multicolumn{1}{p{.75in}}{\raggedright \XLingPaperCharisZSILFontFamily{\fontsize{8}{9.6}\selectfont \textup{\textup{Margrit Bolli / {Eva Flik}}}}}&\multicolumn{1}{p{1.25in}@{}}{\raggedright \XLingPaperCharisZSILFontFamily{\fontsize{8}{9.6}\selectfont \textup{\textup{\hyperlink{rBF1982}{Bolli \& Flik (1982)} – Transitional Primer}}}}\\%
\multicolumn{1}{@{}p{.5in}}{\raggedright }&\multicolumn{1}{p{.8in}}{\raggedright }&\multicolumn{1}{p{2in}}{\raggedright }&\multicolumn{1}{p{.75in}}{\raggedright }&\multicolumn{1}{p{1.25in}@{}}{\raggedright }\\%
\multicolumn{1}{@{}p{.5in}}{\raggedright \XLingPaperCharisZSILFontFamily{\fontsize{8}{9.6}\selectfont \textup{\textup{2.0}}}}&\multicolumn{1}{p{.8in}}{\raggedright \XLingPaperCharisZSILFontFamily{\fontsize{8}{9.6}\selectfont \textup{\textup{1994}}}}&\multicolumn{1}{p{2in}}{\raggedright \XLingPaperCharisZSILFontFamily{\fontsize{8}{9.6}\selectfont \textup{\textup{The start of using double U+0022 at the end of words appears in a course book for learning to read. The letters {\XLingPaperCambriaZMathFontFamily{\textup{\textmd{⟨ ɩ ⟩}}}}, {\XLingPaperCambriaZMathFontFamily{\textup{\textmd{⟨ ʋ̈ ⟩}}}}, {\XLingPaperCambriaZMathFontFamily{\textup{\textmd{⟨ ʋ ⟩}}}} appear, which did not appear in orthography version 1.}}}}&\multicolumn{1}{p{.75in}}{\raggedright \XLingPaperCharisZSILFontFamily{\fontsize{8}{9.6}\selectfont \textup{\textup{Margrit Bolli / {Eva Flik}}}}}&\multicolumn{1}{p{1.25in}@{}}{\raggedright \XLingPaperCharisZSILFontFamily{\fontsize{8}{9.6}\selectfont \textup{\textup{\hyperlink{rBolli1994Cours}{Bolli \& Flik (1994)} – Transitional Primer}}}}\\%
\multicolumn{1}{@{}p{.5in}}{\raggedright }&\multicolumn{1}{p{.8in}}{\raggedright }&\multicolumn{1}{p{2in}}{\raggedright }&\multicolumn{1}{p{.75in}}{\raggedright }&\multicolumn{1}{p{1.25in}@{}}{\raggedright }\\%
\multicolumn{1}{@{}p{.5in}}{\raggedright \XLingPaperCharisZSILFontFamily{\fontsize{8}{9.6}\selectfont \textup{\textup{Western Dan}}}}&\multicolumn{1}{p{.8in}}{\raggedright \XLingPaperCharisZSILFontFamily{\fontsize{8}{9.6}\selectfont \textup{\textup{2000}}}}&\multicolumn{1}{p{2in}}{\raggedright \XLingPaperCharisZSILFontFamily{\fontsize{8}{9.6}\selectfont \textup{\textup{In Western Dan Biblical text preprints (for community circulation) use U+2013 instead of U+002D to indicate tone. (Forever muddling which character is correct in all future writing.)}}}}&\multicolumn{1}{p{.75in}}{\raggedright \XLingPaperCharisZSILFontFamily{\fontsize{8}{9.6}\selectfont \textup{\textup{Margrit Bolli / {Eva Flik}}}}}&\multicolumn{1}{p{1.25in}@{}}{\raggedright \XLingPaperCharisZSILFontFamily{\fontsize{8}{9.6}\selectfont \textup{\textup{Ruth \hyperlink{rBolli2000RutF6}{Bolli \& Flik (2000a)} and Jonah \hyperlink{rBolli2000Jonah}{Bolli \& Flik (2000b)}}}}}\\%
\multicolumn{1}{@{}p{.5in}}{\raggedright }&\multicolumn{1}{p{.8in}}{\raggedright }&\multicolumn{1}{p{2in}}{\raggedright }&\multicolumn{1}{p{.75in}}{\raggedright }&\multicolumn{1}{p{1.25in}@{}}{\raggedright }\\%
\multicolumn{1}{@{}p{.5in}}{\raggedright \XLingPaperCharisZSILFontFamily{\fontsize{8}{9.6}\selectfont \textup{\textup{3.0}}}}&\multicolumn{1}{p{.8in}}{\raggedright \XLingPaperCharisZSILFontFamily{\fontsize{8}{9.6}\selectfont \textup{\textup{c. 2005-2014}}}}&\multicolumn{1}{p{2in}}{\raggedright \XLingPaperCharisZSILFontFamily{\fontsize{8}{9.6}\selectfont \textup{\textup{These texts contain U+201C, U+201D, and U+0022 as tone markers before and after words.\protect\footnote{{\leftskip0pt\parindent1em\raisebox{\baselineskip}[0pt]{\protect\hypertarget{nApostrophyindan}{}} It might have been the idea that only U+0027 would be used twice and that human input habits chose to input U+0022 as a quicker step, and then word processing software auto-corrected some of these to U+201C, and U+201D.}}}}}}&\multicolumn{1}{p{.75in}}{\raggedright \XLingPaperCharisZSILFontFamily{\fontsize{8}{9.6}\selectfont \textup{\textup{Margrit Bolli/{Valintin Vydrin}}}}}&\multicolumn{1}{p{1.25in}@{}}{\raggedright \XLingPaperCharisZSILFontFamily{\fontsize{8}{9.6}\selectfont \textup{\textup{The corpus used for experiments presented and discussed in sections \hyperlink{cExperiments}{5} and \hyperlink{cResults}{6} is representative of this stage in the orthography.}}}}\\%
\multicolumn{1}{@{}p{.5in}}{\raggedright }&\multicolumn{1}{p{.8in}}{\raggedright }&\multicolumn{1}{p{2in}}{\raggedright }&\multicolumn{1}{p{.75in}}{\raggedright }&\multicolumn{1}{p{1.25in}@{}}{\raggedright }\\%
\multicolumn{1}{@{}p{.5in}}{\raggedright \XLingPaperCharisZSILFontFamily{\fontsize{8}{9.6}\selectfont \textup{\textup{4.0}}}}&\multicolumn{1}{p{.8in}}{\raggedright \XLingPaperCharisZSILFontFamily{\fontsize{8}{9.6}\selectfont \textup{\textup{2014+}}}}&\multicolumn{1}{p{2in}}{\raggedright \XLingPaperCharisZSILFontFamily{\fontsize{8}{9.6}\selectfont \textup{\textup{There are significant changes to vowel and tone markers. In general away from digraphs towards single width characters, and away from pre and post stem tone indication via punctuation towards diacritic indication of tone over the the stem\protect\footnote{{\leftskip0pt\parindent1em\raisebox{\baselineskip}[0pt]{\protect\hypertarget{nLinkToOrthographyComparison}{}} For a complete comparison consult the description at the online Eastern Dan corpus: \href{http://cormand.huma-num.fr/dan/orthographe.html}{\textcolor[rgb]{0,0,0}{http://cormand.huma-num.fr/dan/orthographe.html}}}}.}}}}&\multicolumn{1}{p{.75in}}{\raggedright \XLingPaperCharisZSILFontFamily{\fontsize{8}{9.6}\selectfont \textup{\textup{{Valintin Vydrin}}}}}&\multicolumn{1}{p{1.25in}@{}}{\raggedright \XLingPaperCharisZSILFontFamily{\fontsize{8}{9.6}\selectfont \textup{\textup{\hyperlink{rRobertsn.d.Marki}{Roberts, Basnight-Brown \& Vydrin  (forthcoming)}, \hyperlink{rRobertssubmittedChapt}{Roberts \& Vydrin  (submitted)}, \hyperlink{rVandRChapter}{Vydrin \& Roberts  (submitted)}}}}}\\\bottomrule%
\end{longtable}
}
}{\vspace{15pt}\XLingPaperneedspace{3\baselineskip}\noindent
\fontsize{13}{15.6}\selectfont \textbf{{\noindent
\raisebox{\baselineskip}[0pt]{\pdfbookmark[2]{{3.1 } Previous work on Eastern Dan}{sEasternDanBib}}\raisebox{\baselineskip}[0pt]{\protect\hypertarget{sEasternDanBib}{}}{3.1 }Previous work on Eastern Dan}}\markboth{Previous work on Eastern Dan}{Eastern Dan writing system}\XLingPaperaddtocontents{sEasternDanBib}}\par{}
\penalty10000\vspace{10pt}\penalty10000\indent The literature describing Dan is produced by no less than a dozen authors. Prolific authors who make the distinction between eastern Dan and Western Dan include {Margrit Bolli}, {Eva Flik}, and {Valintin Vydrin}\protect\footnote[12]{{\leftskip0pt\parindent1em\raisebox{\baselineskip}[0pt]{\protect\hypertarget{n-NeedsALabel-.xlingpaper.1..styledPaper.1..lingPaper.1..chapter.3..section1.1..p.1..endnote.1.}{}} {Dr. Vydrin} has published extensively in several languages. In English works, his last name appears as {Vydrin}, in French works as {\textit{Vydrine}}, and in Russian works as {\textit{Выдрин}}. In this work, I have regularized the in line citation form of his name. However, for citations the name appears as published.}}. A bibliography of works on Dan is presented as appendix \hyperlink{EasternDanBibliography}{C}.\par{}{\vspace{15pt}\XLingPaperneedspace{3\baselineskip}\noindent
\fontsize{13}{15.6}\selectfont \textbf{{\noindent
\raisebox{\baselineskip}[0pt]{\pdfbookmark[2]{{3.2 } Phonology}{sEasternDanPhonology}}\raisebox{\baselineskip}[0pt]{\protect\hypertarget{sEasternDanPhonology}{}}{3.2 }Phonology}}\markboth{Phonology}{Eastern Dan writing system}\XLingPaperaddtocontents{sEasternDanPhonology}}\par{}
\penalty10000\vspace{10pt}\penalty10000\indent I present the following cursory phonemic inventory and orthographical associations after consulting several resources including: \hyperlink{rSILInternationalBolli1982Guide}{SIL International (Boll \& Flik) (1982)}, \hyperlink{rNazam1983}{Halaoui et al. (1983)}, \hyperlink{rBolli1994Cours}{Bolli \& Flik (1994)}, \hyperlink{rVydrin2008EDDictionary}{Vydrine \& Kességbeu  (2008)}, \hyperlink{rRobertssubmittedChapt}{Roberts \& Vydrin  (submitted)} and others as noted. I present just enough detail about the phonology to provide orientation to the writing system's application in Eastern Dan.\par{}{\vspace{10pt}\XLingPaperneedspace{3\baselineskip}\noindent
\fontsize{13}{15.6}\selectfont \textit{{\noindent
\raisebox{\baselineskip}[0pt]{\pdfbookmark[3]{{3.2.1 } Syllabics}{sSylabis}}\raisebox{\baselineskip}[0pt]{\protect\hypertarget{sSylabis}{}}{3.2.1 }Syllabics}}\markboth{Syllabics}{Eastern Dan writing system}\XLingPaperaddtocontents{sSylabis}}\par{}
\penalty10000\vspace{10pt}\penalty10000\indent Eastern Dan has a strong CV and CVV structure. For further details the reader is referred to \hyperlink{rVydrin2010Lepie}{Vydrin (2010)}. Metrical structure and the presence or absence of nasality play a role in the allophonic distribution of sounds. Two interesting patterns which showed up in the corpus (as prepared for this thesis) are the word initial orthographic consonant clusters {\XLingPaperCambriaZMathFontFamily{\textup{\textmd{⟨ {\XLingPaperCharisZSILFontFamily{\textup{\textup{\textmd{sl}}}}} ⟩}}}}, {\XLingPaperCambriaZMathFontFamily{\textup{\textmd{⟨ {\XLingPaperCharisZSILFontFamily{\textup{\textup{\textmd{zl}}}}} ⟩}}}}. I take these to be consonant clusters. Vydrin ({\hyperlink{vpc}{{p.c.}}}), suggests that these are phonemically separate but phonetically reduced as described below.\par{}\XLingPaperblockquote{.25in}{{\singlespacing
\vspace{-1.3\baselineskip}In a nasal feet, any consonant is nasalized; phonemes {\XLingPaperCharisZSILFontFamily{/ {\XLingPaperCharisZSILFontFamily{\textup{\textup{\textmd{ɓ, ɗ, y, w}}}}} /}} are represented by their allophones {\XLingPaperDoulosZSILFontFamily{\textsquarebracketleft{} {\XLingPaperCharisZSILFontFamily{\textup{\textup{\textmd{m, n, ɲ, w̃w}}}}} \textsquarebracketright{}}}. Foot-internal {\XLingPaperCharisZSILFontFamily{/ {\XLingPaperCharisZSILFontFamily{\textup{\textup{\textmd{l}}}}} /}} is realized as {\XLingPaperDoulosZSILFontFamily{\textsquarebracketleft{} {\XLingPaperCharisZSILFontFamily{\textup{\textup{\textmd{ɾ}}}}} \textsquarebracketright{}}} when preceded by dental or palatal consonants, and as {\XLingPaperDoulosZSILFontFamily{\textsquarebracketleft{} {\XLingPaperCharisZSILFontFamily{\textup{\textup{\textmd{l}}}}} \textsquarebracketright{}}} after labial and velar consonants. Combinations {\XLingPaperCambriaZMathFontFamily{\textup{\textmd{⟨ {\XLingPaperCharisZSILFontFamily{\textup{\textup{\textmd{sl-}}}}} ⟩}}}}, {\XLingPaperCambriaZMathFontFamily{\textup{\textmd{⟨ {\XLingPaperCharisZSILFontFamily{\textup{\textup{\textmd{zl-}}}}} ⟩}}}} are realized respectively as {\XLingPaperDoulosZSILFontFamily{\textsquarebracketleft{} {\XLingPaperCharisZSILFontFamily{\textup{\textup{\textmd{ɬ}}}}} \textsquarebracketright{}}}, {\XLingPaperDoulosZSILFontFamily{\textsquarebracketleft{} {\XLingPaperCharisZSILFontFamily{\textup{\textup{\textmd{ɮ}}}}} \textsquarebracketright{}}}: {\XLingPaperCambriaZMathFontFamily{\textup{\textmd{⟨ {\XLingPaperCharisZSILFontFamily{\textup{\textup{\textmd{slʌ̄ʌ̄}}}}} ⟩}}}} slëë \textsquarebracketleft{}ɬʌ̄ʌ̄\textsquarebracketright{} ꞊slɔɔ slaaˮ ꞊slɔɔ˗ ‘turn’, zláȁ ʼzlaa˗ {\XLingPaperDoulosZSILFontFamily{\textsquarebracketleft{} {\XLingPaperCharisZSILFontFamily{\textup{\textup{\textmd{ɮáȁ}}}}} \textsquarebracketright{}}} ‘younger sibling’.\par{}}}{\baselineskip}{\baselineskip}\par\indent The status of these consonant clusters is relevant to the discussion of functional units in section \hyperlink{sFunctional-Units}{3.4.2} and will be further discussed in that section.\par{}{\vspace{10pt}\XLingPaperneedspace{3\baselineskip}\noindent
\fontsize{13}{15.6}\selectfont \textit{{\noindent
\raisebox{\baselineskip}[0pt]{\pdfbookmark[3]{{3.2.2 } Consonants}{sConsonants}}\raisebox{\baselineskip}[0pt]{\protect\hypertarget{sConsonants}{}}{3.2.2 }Consonants}}\markboth{Consonants}{Eastern Dan writing system}\XLingPaperaddtocontents{sConsonants}}\par{}
\penalty10000\vspace{10pt}\penalty10000\indent The basic phonemic consonants are presented in table \hyperlink{Consonants}{12}.\par{}\vspace{11pt plus 2pt minus 1pt}\XLingPaperneedspace{3\baselineskip}\protect\hypertarget{Consonants}{}\XLingPaperaddtocontents{Consonants}{\protect\raggedright{\singlespacing
{Table }{12.}{  List of consonants in Eastern Dan\\}}}\vspace{0pt}{\singlespacing
\hspace*{.25in}{
\XLingPaperminmaxcellincolumn{}{\XLingPapermincola}{\textbf{}}{\XLingPapermaxcola}{+0\tabcolsep}
\XLingPaperminmaxcellincolumn{Labial}{\XLingPapermincolb}{\textbf{Labial}}{\XLingPapermaxcolb}{+0\tabcolsep}
\XLingPaperminmaxcellincolumn{Dental}{\XLingPapermincolc}{\textbf{Dental}}{\XLingPapermaxcolc}{+0\tabcolsep}
\XLingPaperminmaxcellincolumn{Palatal}{\XLingPapermincold}{\textbf{Palatal}}{\XLingPapermaxcold}{+0\tabcolsep}
\XLingPaperminmaxcellincolumn{Velar}{\XLingPapermincole}{\textbf{Velar}}{\XLingPapermaxcole}{+0\tabcolsep}
\XLingPaperminmaxcellincolumn{Labio-velar}{\XLingPapermincolf}{\textbf{Labio-velar}}{\XLingPapermaxcolf}{+0\tabcolsep}
\XLingPaperminmaxcellincolumn{Voiceless}{\XLingPapermincola}{Voiceless Stops}{\XLingPapermaxcola}{+0\tabcolsep}
\XLingPaperminmaxcellincolumn{p}{\XLingPapermincolb}{p}{\XLingPapermaxcolb}{+0\tabcolsep}
\XLingPaperminmaxcellincolumn{t}{\XLingPapermincolc}{t}{\XLingPapermaxcolc}{+0\tabcolsep}
\XLingPaperminmaxcellincolumn{}{\XLingPapermincold}{}{\XLingPapermaxcold}{+0\tabcolsep}
\XLingPaperminmaxcellincolumn{k}{\XLingPapermincole}{k}{\XLingPapermaxcole}{+0\tabcolsep}
\XLingPaperminmaxcellincolumn{k͡p,}{\XLingPapermincolf}{k͡p, k͡w}{\XLingPapermaxcolf}{+0\tabcolsep}
\XLingPaperminmaxcellincolumn{Voiced}{\XLingPapermincola}{Voiced Stops}{\XLingPapermaxcola}{+0\tabcolsep}
\XLingPaperminmaxcellincolumn{b}{\XLingPapermincolb}{b}{\XLingPapermaxcolb}{+0\tabcolsep}
\XLingPaperminmaxcellincolumn{d}{\XLingPapermincolc}{d}{\XLingPapermaxcolc}{+0\tabcolsep}
\XLingPaperminmaxcellincolumn{}{\XLingPapermincold}{}{\XLingPapermaxcold}{+0\tabcolsep}
\XLingPaperminmaxcellincolumn{g}{\XLingPapermincole}{g}{\XLingPapermaxcole}{+0\tabcolsep}
\XLingPaperminmaxcellincolumn{ɡ͡b,}{\XLingPapermincolf}{ɡ͡b, g͡w}{\XLingPapermaxcolf}{+0\tabcolsep}
\XLingPaperminmaxcellincolumn{Voiceless}{\XLingPapermincola}{Voiceless fricatives}{\XLingPapermaxcola}{+0\tabcolsep}
\XLingPaperminmaxcellincolumn{f}{\XLingPapermincolb}{f}{\XLingPapermaxcolb}{+0\tabcolsep}
\XLingPaperminmaxcellincolumn{s}{\XLingPapermincolc}{s}{\XLingPapermaxcolc}{+0\tabcolsep}
\XLingPaperminmaxcellincolumn{}{\XLingPapermincold}{}{\XLingPapermaxcold}{+0\tabcolsep}
\XLingPaperminmaxcellincolumn{}{\XLingPapermincole}{}{\XLingPapermaxcole}{+0\tabcolsep}
\XLingPaperminmaxcellincolumn{}{\XLingPapermincolf}{}{\XLingPapermaxcolf}{+0\tabcolsep}
\XLingPaperminmaxcellincolumn{Voiced}{\XLingPapermincola}{Voiced Fricatives}{\XLingPapermaxcola}{+0\tabcolsep}
\XLingPaperminmaxcellincolumn{v}{\XLingPapermincolb}{v}{\XLingPapermaxcolb}{+0\tabcolsep}
\XLingPaperminmaxcellincolumn{z}{\XLingPapermincolc}{z}{\XLingPapermaxcolc}{+0\tabcolsep}
\XLingPaperminmaxcellincolumn{}{\XLingPapermincold}{}{\XLingPapermaxcold}{+0\tabcolsep}
\XLingPaperminmaxcellincolumn{}{\XLingPapermincole}{}{\XLingPapermaxcole}{+0\tabcolsep}
\XLingPaperminmaxcellincolumn{}{\XLingPapermincolf}{}{\XLingPapermaxcolf}{+0\tabcolsep}
\XLingPaperminmaxcellincolumn{Implosives}{\XLingPapermincola}{Implosives}{\XLingPapermaxcola}{+0\tabcolsep}
\XLingPaperminmaxcellincolumn{ɓ}{\XLingPapermincolb}{ɓ}{\XLingPapermaxcolb}{+0\tabcolsep}
\XLingPaperminmaxcellincolumn{ɗ}{\XLingPapermincolc}{ɗ}{\XLingPapermaxcolc}{+0\tabcolsep}
\XLingPaperminmaxcellincolumn{}{\XLingPapermincold}{}{\XLingPapermaxcold}{+0\tabcolsep}
\XLingPaperminmaxcellincolumn{}{\XLingPapermincole}{}{\XLingPapermaxcole}{+0\tabcolsep}
\XLingPaperminmaxcellincolumn{}{\XLingPapermincolf}{}{\XLingPapermaxcolf}{+0\tabcolsep}
\XLingPaperminmaxcellincolumn{Continuants}{\XLingPapermincola}{Continuants}{\XLingPapermaxcola}{+0\tabcolsep}
\XLingPaperminmaxcellincolumn{}{\XLingPapermincolb}{}{\XLingPapermaxcolb}{+0\tabcolsep}
\XLingPaperminmaxcellincolumn{l}{\XLingPapermincolc}{l}{\XLingPapermaxcolc}{+0\tabcolsep}
\XLingPaperminmaxcellincolumn{y}{\XLingPapermincold}{y}{\XLingPapermaxcold}{+0\tabcolsep}
\XLingPaperminmaxcellincolumn{}{\XLingPapermincole}{}{\XLingPapermaxcole}{+0\tabcolsep}
\XLingPaperminmaxcellincolumn{w}{\XLingPapermincolf}{w}{\XLingPapermaxcolf}{+0\tabcolsep}
\setlength{\XLingPaperavailabletablewidth}{433.62pt}
\setlength{\XLingPapertableminwidth}{\XLingPapermincola+\XLingPapermincolb+\XLingPapermincolc+\XLingPapermincold+\XLingPapermincole+\XLingPapermincolf}
\setlength{\XLingPapertablemaxwidth}{\XLingPapermaxcola+\XLingPapermaxcolb+\XLingPapermaxcolc+\XLingPapermaxcold+\XLingPapermaxcole+\XLingPapermaxcolf}
\XLingPapercalculatetablewidthratio{}
\XLingPapersetcolumnwidth{\XLingPapercolawidth}{\XLingPapermincola}{\XLingPapermaxcola}{-0\tabcolsep}
\XLingPapersetcolumnwidth{\XLingPapercolbwidth}{\XLingPapermincolb}{\XLingPapermaxcolb}{-2\tabcolsep}
\XLingPapersetcolumnwidth{\XLingPapercolcwidth}{\XLingPapermincolc}{\XLingPapermaxcolc}{-2\tabcolsep}
\XLingPapersetcolumnwidth{\XLingPapercoldwidth}{\XLingPapermincold}{\XLingPapermaxcold}{-2\tabcolsep}
\XLingPapersetcolumnwidth{\XLingPapercolewidth}{\XLingPapermincole}{\XLingPapermaxcole}{-2\tabcolsep}
\XLingPapersetcolumnwidth{\XLingPapercolfwidth}{\XLingPapermincolf}{\XLingPapermaxcolf}{-2\tabcolsep}\singlespacing\vspace*{-3\baselineskip}
\begin{longtable}
[l]{@{}p{\XLingPapercolawidth}p{\XLingPapercolbwidth}p{\XLingPapercolcwidth}p{\XLingPapercoldwidth}p{\XLingPapercolewidth}p{\XLingPapercolfwidth}@{}}\toprule\multicolumn{1}{@{}p{\XLingPapercolawidth}}{\textbf{}}&\multicolumn{1}{p{\XLingPapercolbwidth}}{\textbf{Labial}}&\multicolumn{1}{p{\XLingPapercolcwidth}}{\textbf{Dental}}&\multicolumn{1}{p{\XLingPapercoldwidth}}{\textbf{Palatal}}&\multicolumn{1}{p{\XLingPapercolewidth}}{\textbf{Velar}}&\multicolumn{1}{p{\XLingPapercolfwidth}@{}}{\textbf{Labio-velar}}\\%
\midrule\endhead \multicolumn{1}{@{}p{\XLingPapercolawidth}}{Voiceless Stops}&\multicolumn{1}{p{\XLingPapercolbwidth}}{p}&\multicolumn{1}{p{\XLingPapercolcwidth}}{t}&\multicolumn{1}{p{\XLingPapercoldwidth}}{}&\multicolumn{1}{p{\XLingPapercolewidth}}{k}&\multicolumn{1}{p{\XLingPapercolfwidth}@{}}{k͡p, k͡w}\\%
\multicolumn{1}{@{}p{\XLingPapercolawidth}}{Voiced Stops}&\multicolumn{1}{p{\XLingPapercolbwidth}}{b}&\multicolumn{1}{p{\XLingPapercolcwidth}}{d}&\multicolumn{1}{p{\XLingPapercoldwidth}}{}&\multicolumn{1}{p{\XLingPapercolewidth}}{g}&\multicolumn{1}{p{\XLingPapercolfwidth}@{}}{ɡ͡b, g͡w}\\%
\multicolumn{1}{@{}p{\XLingPapercolawidth}}{Voiceless fricatives}&\multicolumn{1}{p{\XLingPapercolbwidth}}{f}&\multicolumn{1}{p{\XLingPapercolcwidth}}{s}&\multicolumn{1}{p{\XLingPapercoldwidth}}{}&\multicolumn{1}{p{\XLingPapercolewidth}}{}&\multicolumn{1}{p{\XLingPapercolfwidth}@{}}{}\\%
\multicolumn{1}{@{}p{\XLingPapercolawidth}}{Voiced Fricatives}&\multicolumn{1}{p{\XLingPapercolbwidth}}{v}&\multicolumn{1}{p{\XLingPapercolcwidth}}{z}&\multicolumn{1}{p{\XLingPapercoldwidth}}{}&\multicolumn{1}{p{\XLingPapercolewidth}}{}&\multicolumn{1}{p{\XLingPapercolfwidth}@{}}{}\\%
\multicolumn{1}{@{}p{\XLingPapercolawidth}}{Implosives}&\multicolumn{1}{p{\XLingPapercolbwidth}}{ɓ}&\multicolumn{1}{p{\XLingPapercolcwidth}}{ɗ}&\multicolumn{1}{p{\XLingPapercoldwidth}}{}&\multicolumn{1}{p{\XLingPapercolewidth}}{}&\multicolumn{1}{p{\XLingPapercolfwidth}@{}}{}\\%
\multicolumn{1}{@{}p{\XLingPapercolawidth}}{Continuants}&\multicolumn{1}{p{\XLingPapercolbwidth}}{}&\multicolumn{1}{p{\XLingPapercolcwidth}}{l}&\multicolumn{1}{p{\XLingPapercoldwidth}}{y}&\multicolumn{1}{p{\XLingPapercolewidth}}{}&\multicolumn{1}{p{\XLingPapercolfwidth}@{}}{w}\\\bottomrule%
\end{longtable}
}
}{\vspace{10pt}\XLingPaperneedspace{3\baselineskip}\noindent
\fontsize{13}{15.6}\selectfont \textit{{\noindent
\raisebox{\baselineskip}[0pt]{\pdfbookmark[3]{{3.2.3 } Vowels}{sVowels}}\raisebox{\baselineskip}[0pt]{\protect\hypertarget{sVowels}{}}{3.2.3 }Vowels}}\markboth{Vowels}{Eastern Dan writing system}\XLingPaperaddtocontents{sVowels}}\par{}
\penalty10000\vspace{10pt}\penalty10000\indent Linguistically, Eastern Dan is claimed to have a 12 point vowel system as indicated in table \hyperlink{ntVowelsOral}{13}. Vowels are claimed to have length, pitch, and nasalization distinctions. Length has been analyzed as two sequential vowels. Pitch patterns are covered section \hyperlink{sTone-Marking}{3.2.4}.\par{}\vspace{11pt plus 2pt minus 1pt}\XLingPaperneedspace{3\baselineskip}\protect\hypertarget{ntVowelsOral}{}\XLingPaperaddtocontents{ntVowelsOral}{\protect\raggedright{\singlespacing
{Table }{13.}{  List of oral vowels in Eastern Dan\\}}}\vspace{0pt}{\singlespacing
\hspace*{.25in}{
\XLingPaperminmaxcellincolumn{}{\XLingPapermincola}{\textbf{}}{\XLingPapermaxcola}{+0\tabcolsep}
\XLingPaperminmaxcellincolumn{Unrounded}{\XLingPapermincolb}{\textbf{Front Unrounded}}{\XLingPapermaxcolb}{+0\tabcolsep}
\XLingPaperminmaxcellincolumn{Unrounded}{\XLingPapermincolc}{\textbf{Back Unrounded}}{\XLingPapermaxcolc}{+0\tabcolsep}
\XLingPaperminmaxcellincolumn{Rounded}{\XLingPapermincold}{\textbf{Back Rounded}}{\XLingPapermaxcold}{+0\tabcolsep}
\XLingPaperminmaxcellincolumn{Close}{\XLingPapermincola}{Close}{\XLingPapermaxcola}{+0\tabcolsep}
\XLingPaperminmaxcellincolumn{i}{\XLingPapermincolb}{i}{\XLingPapermaxcolb}{+0\tabcolsep}
\XLingPaperminmaxcellincolumn{ɯ}{\XLingPapermincolc}{ɯ}{\XLingPapermaxcolc}{+0\tabcolsep}
\XLingPaperminmaxcellincolumn{u}{\XLingPapermincold}{u}{\XLingPapermaxcold}{+0\tabcolsep}
\XLingPaperminmaxcellincolumn{Near-close}{\XLingPapermincola}{Near-close}{\XLingPapermaxcola}{+0\tabcolsep}
\XLingPaperminmaxcellincolumn{}{\XLingPapermincolb}{}{\XLingPapermaxcolb}{+0\tabcolsep}
\XLingPaperminmaxcellincolumn{}{\XLingPapermincolc}{}{\XLingPapermaxcolc}{+0\tabcolsep}
\XLingPaperminmaxcellincolumn{}{\XLingPapermincold}{}{\XLingPapermaxcold}{+0\tabcolsep}
\XLingPaperminmaxcellincolumn{Mid}{\XLingPapermincola}{Mid}{\XLingPapermaxcola}{+0\tabcolsep}
\XLingPaperminmaxcellincolumn{e}{\XLingPapermincolb}{e}{\XLingPapermaxcolb}{+0\tabcolsep}
\XLingPaperminmaxcellincolumn{ɤ}{\XLingPapermincolc}{ɤ}{\XLingPapermaxcolc}{+0\tabcolsep}
\XLingPaperminmaxcellincolumn{o}{\XLingPapermincold}{o}{\XLingPapermaxcold}{+0\tabcolsep}
\XLingPaperminmaxcellincolumn{Open-mid}{\XLingPapermincola}{Open-mid}{\XLingPapermaxcola}{+0\tabcolsep}
\XLingPaperminmaxcellincolumn{ɛ}{\XLingPapermincolb}{ɛ}{\XLingPapermaxcolb}{+0\tabcolsep}
\XLingPaperminmaxcellincolumn{ʌ}{\XLingPapermincolc}{ʌ}{\XLingPapermaxcolc}{+0\tabcolsep}
\XLingPaperminmaxcellincolumn{ɔ}{\XLingPapermincold}{ɔ}{\XLingPapermaxcold}{+0\tabcolsep}
\XLingPaperminmaxcellincolumn{Near-open}{\XLingPapermincola}{Near-open}{\XLingPapermaxcola}{+0\tabcolsep}
\XLingPaperminmaxcellincolumn{æ}{\XLingPapermincolb}{æ}{\XLingPapermaxcolb}{+0\tabcolsep}
\XLingPaperminmaxcellincolumn{}{\XLingPapermincolc}{}{\XLingPapermaxcolc}{+0\tabcolsep}
\XLingPaperminmaxcellincolumn{}{\XLingPapermincold}{}{\XLingPapermaxcold}{+0\tabcolsep}
\XLingPaperminmaxcellincolumn{Open}{\XLingPapermincola}{Open}{\XLingPapermaxcola}{+0\tabcolsep}
\XLingPaperminmaxcellincolumn{}{\XLingPapermincolb}{}{\XLingPapermaxcolb}{+0\tabcolsep}
\XLingPaperminmaxcellincolumn{a}{\XLingPapermincolc}{a}{\XLingPapermaxcolc}{+0\tabcolsep}
\XLingPaperminmaxcellincolumn{ɒ}{\XLingPapermincold}{ɒ}{\XLingPapermaxcold}{+0\tabcolsep}
\setlength{\XLingPaperavailabletablewidth}{433.62pt}
\setlength{\XLingPapertableminwidth}{\XLingPapermincola+\XLingPapermincolb+\XLingPapermincolc+\XLingPapermincold}
\setlength{\XLingPapertablemaxwidth}{\XLingPapermaxcola+\XLingPapermaxcolb+\XLingPapermaxcolc+\XLingPapermaxcold}
\XLingPapercalculatetablewidthratio{}
\XLingPapersetcolumnwidth{\XLingPapercolawidth}{\XLingPapermincola}{\XLingPapermaxcola}{-0\tabcolsep}
\XLingPapersetcolumnwidth{\XLingPapercolbwidth}{\XLingPapermincolb}{\XLingPapermaxcolb}{-2\tabcolsep}
\XLingPapersetcolumnwidth{\XLingPapercolcwidth}{\XLingPapermincolc}{\XLingPapermaxcolc}{-2\tabcolsep}
\XLingPapersetcolumnwidth{\XLingPapercoldwidth}{\XLingPapermincold}{\XLingPapermaxcold}{-2\tabcolsep}\singlespacing\vspace*{-3\baselineskip}
\begin{longtable}
[l]{@{}p{\XLingPapercolawidth}p{\XLingPapercolbwidth}p{\XLingPapercolcwidth}p{\XLingPapercoldwidth}@{}}\toprule\multicolumn{1}{@{}p{\XLingPapercolawidth}}{\textbf{}}&\multicolumn{1}{p{\XLingPapercolbwidth}}{\textbf{Front Unrounded}}&\multicolumn{1}{p{\XLingPapercolcwidth}}{\textbf{Back Unrounded}}&\multicolumn{1}{p{\XLingPapercoldwidth}@{}}{\textbf{Back Rounded}}\\%
\midrule\endhead \multicolumn{1}{@{}p{\XLingPapercolawidth}}{Close}&\multicolumn{1}{p{\XLingPapercolbwidth}}{i}&\multicolumn{1}{p{\XLingPapercolcwidth}}{ɯ}&\multicolumn{1}{p{\XLingPapercoldwidth}@{}}{u}\\%
\multicolumn{1}{@{}p{\XLingPapercolawidth}}{Near-close}&\multicolumn{1}{p{\XLingPapercolbwidth}}{}&\multicolumn{1}{p{\XLingPapercolcwidth}}{}&\multicolumn{1}{p{\XLingPapercoldwidth}@{}}{}\\%
\multicolumn{1}{@{}p{\XLingPapercolawidth}}{Mid}&\multicolumn{1}{p{\XLingPapercolbwidth}}{e}&\multicolumn{1}{p{\XLingPapercolcwidth}}{ɤ}&\multicolumn{1}{p{\XLingPapercoldwidth}@{}}{o}\\%
\multicolumn{1}{@{}p{\XLingPapercolawidth}}{Open-mid}&\multicolumn{1}{p{\XLingPapercolbwidth}}{ɛ}&\multicolumn{1}{p{\XLingPapercolcwidth}}{ʌ}&\multicolumn{1}{p{\XLingPapercoldwidth}@{}}{ɔ}\\%
\multicolumn{1}{@{}p{\XLingPapercolawidth}}{Near-open}&\multicolumn{1}{p{\XLingPapercolbwidth}}{æ}&\multicolumn{1}{p{\XLingPapercolcwidth}}{}&\multicolumn{1}{p{\XLingPapercoldwidth}@{}}{}\\%
\multicolumn{1}{@{}p{\XLingPapercolawidth}}{Open}&\multicolumn{1}{p{\XLingPapercolbwidth}}{}&\multicolumn{1}{p{\XLingPapercolcwidth}}{a}&\multicolumn{1}{p{\XLingPapercoldwidth}@{}}{ɒ}\\\bottomrule%
\end{longtable}
}
}\indent Nasalization occurs phonemically only on 9 vowels as is show in table \hyperlink{ntVowelsNasal}{14}. The velar nasal {\XLingPaperCharisZSILFontFamily{/ ŋ /}}, orthographically indicated as {\XLingPaperCambriaZMathFontFamily{\textup{\textmd{⟨ {\XLingPaperCharisZSILFontFamily{\textup{\textup{\textmd{ng}}}}} ⟩}}}}, is linguistically considered a vowel in Eastern Dan (\hyperlink{rSILInternationalBolli1982Guide}{SIL International (Boll \& Flik)  1982}, \hyperlink{rVydrin2008EDDictionary}{Vydrine \& Kességbeu   2008)}\protect\footnote[13]{{\leftskip0pt\parindent1em\raisebox{\baselineskip}[0pt]{\protect\hypertarget{nEngUsage}{}} This is in contrast to the typologically normal analysis and IPA symbol {\XLingPaperCharisZSILFontFamily{/ ŋ /}} usage as a consonant.}}. This brings the total number of vowels to 22.\par{}\vspace{11pt plus 2pt minus 1pt}\XLingPaperneedspace{3\baselineskip}\protect\hypertarget{ntVowelsNasal}{}\XLingPaperaddtocontents{ntVowelsNasal}{\protect\raggedright{\singlespacing
{Table }{14.}{  List of nasal vowels in Eastern Dan\\}}}\vspace{0pt}{\singlespacing
\hspace*{.25in}{
\XLingPaperminmaxcellincolumn{}{\XLingPapermincola}{\textbf{}}{\XLingPapermaxcola}{+0\tabcolsep}
\XLingPaperminmaxcellincolumn{Unrounded}{\XLingPapermincolb}{\textbf{Front Unrounded}}{\XLingPapermaxcolb}{+0\tabcolsep}
\XLingPaperminmaxcellincolumn{Unrounded}{\XLingPapermincolc}{\textbf{Back Unrounded}}{\XLingPapermaxcolc}{+0\tabcolsep}
\XLingPaperminmaxcellincolumn{Rounded}{\XLingPapermincold}{\textbf{Back Rounded}}{\XLingPapermaxcold}{+0\tabcolsep}
\XLingPaperminmaxcellincolumn{Close}{\XLingPapermincola}{Close}{\XLingPapermaxcola}{+0\tabcolsep}
\XLingPaperminmaxcellincolumn{ĩ}{\XLingPapermincolb}{ĩ}{\XLingPapermaxcolb}{+0\tabcolsep}
\XLingPaperminmaxcellincolumn{ɯ̃}{\XLingPapermincolc}{ɯ̃}{\XLingPapermaxcolc}{+0\tabcolsep}
\XLingPaperminmaxcellincolumn{ũ}{\XLingPapermincold}{ũ}{\XLingPapermaxcold}{+0\tabcolsep}
\XLingPaperminmaxcellincolumn{Near-close}{\XLingPapermincola}{Near-close}{\XLingPapermaxcola}{+0\tabcolsep}
\XLingPaperminmaxcellincolumn{}{\XLingPapermincolb}{}{\XLingPapermaxcolb}{+0\tabcolsep}
\XLingPaperminmaxcellincolumn{}{\XLingPapermincolc}{}{\XLingPapermaxcolc}{+0\tabcolsep}
\XLingPaperminmaxcellincolumn{}{\XLingPapermincold}{}{\XLingPapermaxcold}{+0\tabcolsep}
\XLingPaperminmaxcellincolumn{Mid}{\XLingPapermincola}{Mid}{\XLingPapermaxcola}{+0\tabcolsep}
\XLingPaperminmaxcellincolumn{}{\XLingPapermincolb}{}{\XLingPapermaxcolb}{+0\tabcolsep}
\XLingPaperminmaxcellincolumn{}{\XLingPapermincolc}{}{\XLingPapermaxcolc}{+0\tabcolsep}
\XLingPaperminmaxcellincolumn{}{\XLingPapermincold}{}{\XLingPapermaxcold}{+0\tabcolsep}
\XLingPaperminmaxcellincolumn{Open-mid}{\XLingPapermincola}{Open-mid}{\XLingPapermaxcola}{+0\tabcolsep}
\XLingPaperminmaxcellincolumn{ɛ̃}{\XLingPapermincolb}{ɛ̃}{\XLingPapermaxcolb}{+0\tabcolsep}
\XLingPaperminmaxcellincolumn{ʌ̃}{\XLingPapermincolc}{ʌ̃}{\XLingPapermaxcolc}{+0\tabcolsep}
\XLingPaperminmaxcellincolumn{ɔ̃}{\XLingPapermincold}{ɔ̃}{\XLingPapermaxcold}{+0\tabcolsep}
\XLingPaperminmaxcellincolumn{Near-open}{\XLingPapermincola}{Near-open}{\XLingPapermaxcola}{+0\tabcolsep}
\XLingPaperminmaxcellincolumn{æ̃}{\XLingPapermincolb}{æ̃}{\XLingPapermaxcolb}{+0\tabcolsep}
\XLingPaperminmaxcellincolumn{}{\XLingPapermincolc}{}{\XLingPapermaxcolc}{+0\tabcolsep}
\XLingPaperminmaxcellincolumn{}{\XLingPapermincold}{}{\XLingPapermaxcold}{+0\tabcolsep}
\XLingPaperminmaxcellincolumn{Open}{\XLingPapermincola}{Open}{\XLingPapermaxcola}{+0\tabcolsep}
\XLingPaperminmaxcellincolumn{}{\XLingPapermincolb}{}{\XLingPapermaxcolb}{+0\tabcolsep}
\XLingPaperminmaxcellincolumn{ã}{\XLingPapermincolc}{ã}{\XLingPapermaxcolc}{+0\tabcolsep}
\XLingPaperminmaxcellincolumn{ɒ̃}{\XLingPapermincold}{ɒ̃}{\XLingPapermaxcold}{+0\tabcolsep}
\setlength{\XLingPaperavailabletablewidth}{433.62pt}
\setlength{\XLingPapertableminwidth}{\XLingPapermincola+\XLingPapermincolb+\XLingPapermincolc+\XLingPapermincold}
\setlength{\XLingPapertablemaxwidth}{\XLingPapermaxcola+\XLingPapermaxcolb+\XLingPapermaxcolc+\XLingPapermaxcold}
\XLingPapercalculatetablewidthratio{}
\XLingPapersetcolumnwidth{\XLingPapercolawidth}{\XLingPapermincola}{\XLingPapermaxcola}{-0\tabcolsep}
\XLingPapersetcolumnwidth{\XLingPapercolbwidth}{\XLingPapermincolb}{\XLingPapermaxcolb}{-2\tabcolsep}
\XLingPapersetcolumnwidth{\XLingPapercolcwidth}{\XLingPapermincolc}{\XLingPapermaxcolc}{-2\tabcolsep}
\XLingPapersetcolumnwidth{\XLingPapercoldwidth}{\XLingPapermincold}{\XLingPapermaxcold}{-2\tabcolsep}\singlespacing\vspace*{-3\baselineskip}
\begin{longtable}
[l]{@{}p{\XLingPapercolawidth}p{\XLingPapercolbwidth}p{\XLingPapercolcwidth}p{\XLingPapercoldwidth}@{}}\toprule\multicolumn{1}{@{}p{\XLingPapercolawidth}}{\textbf{}}&\multicolumn{1}{p{\XLingPapercolbwidth}}{\textbf{Front Unrounded}}&\multicolumn{1}{p{\XLingPapercolcwidth}}{\textbf{Back Unrounded}}&\multicolumn{1}{p{\XLingPapercoldwidth}@{}}{\textbf{Back Rounded}}\\%
\midrule\endhead \multicolumn{1}{@{}p{\XLingPapercolawidth}}{Close}&\multicolumn{1}{p{\XLingPapercolbwidth}}{ĩ}&\multicolumn{1}{p{\XLingPapercolcwidth}}{ɯ̃}&\multicolumn{1}{p{\XLingPapercoldwidth}@{}}{ũ}\\%
\multicolumn{1}{@{}p{\XLingPapercolawidth}}{Near-close}&\multicolumn{1}{p{\XLingPapercolbwidth}}{}&\multicolumn{1}{p{\XLingPapercolcwidth}}{}&\multicolumn{1}{p{\XLingPapercoldwidth}@{}}{}\\%
\multicolumn{1}{@{}p{\XLingPapercolawidth}}{Mid}&\multicolumn{1}{p{\XLingPapercolbwidth}}{}&\multicolumn{1}{p{\XLingPapercolcwidth}}{}&\multicolumn{1}{p{\XLingPapercoldwidth}@{}}{}\\%
\multicolumn{1}{@{}p{\XLingPapercolawidth}}{Open-mid}&\multicolumn{1}{p{\XLingPapercolbwidth}}{ɛ̃}&\multicolumn{1}{p{\XLingPapercolcwidth}}{ʌ̃}&\multicolumn{1}{p{\XLingPapercoldwidth}@{}}{ɔ̃}\\%
\multicolumn{1}{@{}p{\XLingPapercolawidth}}{Near-open}&\multicolumn{1}{p{\XLingPapercolbwidth}}{æ̃}&\multicolumn{1}{p{\XLingPapercolcwidth}}{}&\multicolumn{1}{p{\XLingPapercoldwidth}@{}}{}\\%
\multicolumn{1}{@{}p{\XLingPapercolawidth}}{Open}&\multicolumn{1}{p{\XLingPapercolbwidth}}{}&\multicolumn{1}{p{\XLingPapercolcwidth}}{ã}&\multicolumn{1}{p{\XLingPapercoldwidth}@{}}{ɒ̃}\\\bottomrule%
\end{longtable}
}
}\indent Allophonic variation of vowels does occur. Some varieties of Eastern Dan contain three more phonemic vowels than other varieties. Even though the phonemic status is not attested ubiquitously across Eastern Dan the orthography attempts to be pan-lectical. This accounts for the addition of three letters {\XLingPaperCambriaZMathFontFamily{\textup{\textmd{⟨ {\XLingPaperCharisZSILFontFamily{\textup{\textup{\textmd{ɩ}}}}} ⟩}}}}, {\XLingPaperCambriaZMathFontFamily{\textup{\textmd{⟨ {\XLingPaperCharisZSILFontFamily{\textup{\textup{\textmd{ʋ}}}}} ⟩}}}}, and {\XLingPaperCambriaZMathFontFamily{\textup{\textmd{⟨ {\XLingPaperCharisZSILFontFamily{\textup{\textup{\textmd{ʋ̈}}}}} ⟩}}}} between the 1982 and the 1994 versions of the reading primers. Linguistically, these allophones are attributed to Extra High Tone interaction with the phonemes: {\XLingPaperCharisZSILFontFamily{/ e /}}, {\XLingPaperCharisZSILFontFamily{/ o /}}, and {\XLingPaperCharisZSILFontFamily{/ ɤ /}} respectively.\par{}\indent Vowel length has been linguistically analyzed as two separate vowels and is indicated orthographically by sequential characters i.e. {\XLingPaperCambriaZMathFontFamily{\textup{\textmd{⟨ {\XLingPaperCharisZSILFontFamily{\textup{\textup{\textmd{aa}}}}} ⟩}}}}. Eastern Dan has diphthongs as indicated in table \hyperlink{ntVowelsDipthongs}{15}. Diphthongs (vowels that start at one phonetic value and finish at another value) are orthographically indicated with sequential characters. Some "short" vowels are also orthographically indicated by a digraphs {\XLingPaperCambriaZMathFontFamily{\textup{\textmd{⟨ {\XLingPaperCharisZSILFontFamily{\textup{\textup{\textmd{ɛa, aɔ}}}}} ⟩}}}}; these are not diphthongs.\par{}\indent In Eastern Dan, dieresis are not graphemic. It is used to form unique characters. Vowels with dieresis are thought of as a single character or letter of the alphabet\protect\footnote[14]{{\leftskip0pt\parindent1em\raisebox{\baselineskip}[0pt]{\protect\hypertarget{nDieresisUsage}{}} Dieresis is not an orthographically separable unit (even though at the character encoding level in UTF-8 it is separable).}}. Orthographically nasality on vowels is indicated by an {\XLingPaperCambriaZMathFontFamily{\textup{\textmd{⟨ {\XLingPaperCharisZSILFontFamily{\textup{\textup{\textmd{n}}}}} ⟩}}}} following the vowel.\par{}\vspace{11pt plus 2pt minus 1pt}\XLingPaperneedspace{3\baselineskip}\protect\hypertarget{ntVowelsDipthongs}{}\XLingPaperaddtocontents{ntVowelsDipthongs}{\protect\raggedright{\singlespacing
{Table }{15.}{  List of diphthongs vowels in Eastern Dan\\}}}\vspace{0pt}{\singlespacing
\hspace*{.25in}{
\XLingPaperminmaxcellincolumn{Orthography}{\XLingPapermincola}{\textbf{Orthography}}{\XLingPapermaxcola}{+0\tabcolsep}
\XLingPaperminmaxcellincolumn{IPA}{\XLingPapermincolb}{\textbf{IPA Transcription}}{\XLingPapermaxcolb}{+0\tabcolsep}
\XLingPaperminmaxcellincolumn{Description}{\XLingPapermincolc}{\textbf{Description}}{\XLingPapermaxcolc}{+0\tabcolsep}
\XLingPaperminmaxcellincolumn{iʋ̈}{\XLingPapermincola}{iʋ̈}{\XLingPapermaxcola}{+0\tabcolsep}
\XLingPaperminmaxcellincolumn{iɯ̞̈}{\XLingPapermincolb}{iɯ̞̈}{\XLingPapermaxcolb}{+0\tabcolsep}
\XLingPaperminmaxcellincolumn{}{\XLingPapermincolc}{}{\XLingPapermaxcolc}{+0\tabcolsep}
\XLingPaperminmaxcellincolumn{iö}{\XLingPapermincola}{iö}{\XLingPapermaxcola}{+0\tabcolsep}
\XLingPaperminmaxcellincolumn{iɤ}{\XLingPapermincolb}{iɤ}{\XLingPapermaxcolb}{+0\tabcolsep}
\XLingPaperminmaxcellincolumn{}{\XLingPapermincolc}{}{\XLingPapermaxcolc}{+0\tabcolsep}
\XLingPaperminmaxcellincolumn{ië}{\XLingPapermincola}{ië }{\XLingPapermaxcola}{+0\tabcolsep}
\XLingPaperminmaxcellincolumn{iʌ}{\XLingPapermincolb}{iʌ}{\XLingPapermaxcolb}{+0\tabcolsep}
\XLingPaperminmaxcellincolumn{}{\XLingPapermincolc}{}{\XLingPapermaxcolc}{+0\tabcolsep}
\XLingPaperminmaxcellincolumn{uë}{\XLingPapermincola}{uë}{\XLingPapermaxcola}{+0\tabcolsep}
\XLingPaperminmaxcellincolumn{uʌ}{\XLingPapermincolb}{uʌ}{\XLingPapermaxcolb}{+0\tabcolsep}
\XLingPaperminmaxcellincolumn{}{\XLingPapermincolc}{}{\XLingPapermaxcolc}{+0\tabcolsep}
\XLingPaperminmaxcellincolumn{ʋë}{\XLingPapermincola}{ʋë}{\XLingPapermaxcola}{+0\tabcolsep}
\XLingPaperminmaxcellincolumn{ʊʌ}{\XLingPapermincolb}{ʊʌ}{\XLingPapermaxcolb}{+0\tabcolsep}
\XLingPaperminmaxcellincolumn{}{\XLingPapermincolc}{}{\XLingPapermaxcolc}{+0\tabcolsep}
\XLingPaperminmaxcellincolumn{ʋ̈ü}{\XLingPapermincola}{ʋ̈ü}{\XLingPapermaxcola}{+0\tabcolsep}
\XLingPaperminmaxcellincolumn{ʊɯ}{\XLingPapermincolb}{ʊɯ}{\XLingPapermaxcolb}{+0\tabcolsep}
\XLingPaperminmaxcellincolumn{}{\XLingPapermincolc}{}{\XLingPapermaxcolc}{+0\tabcolsep}
\XLingPaperminmaxcellincolumn{ia}{\XLingPapermincola}{ia}{\XLingPapermaxcola}{+0\tabcolsep}
\XLingPaperminmaxcellincolumn{ia}{\XLingPapermincolb}{ia}{\XLingPapermaxcolb}{+0\tabcolsep}
\XLingPaperminmaxcellincolumn{}{\XLingPapermincolc}{}{\XLingPapermaxcolc}{+0\tabcolsep}
\XLingPaperminmaxcellincolumn{ian}{\XLingPapermincola}{ian}{\XLingPapermaxcola}{+0\tabcolsep}
\XLingPaperminmaxcellincolumn{ĩã}{\XLingPapermincolb}{ĩã}{\XLingPapermaxcolb}{+0\tabcolsep}
\XLingPaperminmaxcellincolumn{}{\XLingPapermincolc}{}{\XLingPapermaxcolc}{+0\tabcolsep}
\XLingPaperminmaxcellincolumn{ɩa}{\XLingPapermincola}{ɩa}{\XLingPapermaxcola}{+0\tabcolsep}
\XLingPaperminmaxcellincolumn{/ea/,}{\XLingPapermincolb}{/ea/, \textsquarebracketleft{}ɪa\textsquarebracketright{}}{\XLingPapermaxcolb}{+0\tabcolsep}
\XLingPaperminmaxcellincolumn{}{\XLingPapermincolc}{}{\XLingPapermaxcolc}{+0\tabcolsep}
\setlength{\XLingPaperavailabletablewidth}{433.62pt}
\setlength{\XLingPapertableminwidth}{\XLingPapermincola+\XLingPapermincolb+\XLingPapermincolc}
\setlength{\XLingPapertablemaxwidth}{\XLingPapermaxcola+\XLingPapermaxcolb+\XLingPapermaxcolc}
\XLingPapercalculatetablewidthratio{}
\XLingPapersetcolumnwidth{\XLingPapercolawidth}{\XLingPapermincola}{\XLingPapermaxcola}{-0\tabcolsep}
\XLingPapersetcolumnwidth{\XLingPapercolbwidth}{\XLingPapermincolb}{\XLingPapermaxcolb}{-2\tabcolsep}
\XLingPapersetcolumnwidth{\XLingPapercolcwidth}{\XLingPapermincolc}{\XLingPapermaxcolc}{-2\tabcolsep}\singlespacing\vspace*{-3\baselineskip}
\begin{longtable}
[l]{@{}p{\XLingPapercolawidth}p{\XLingPapercolbwidth}p{\XLingPapercolcwidth}@{}}\toprule\multicolumn{1}{@{}p{\XLingPapercolawidth}}{\textbf{Orthography}}&\multicolumn{1}{p{\XLingPapercolbwidth}}{\textbf{IPA Transcription}}&\multicolumn{1}{p{\XLingPapercolcwidth}@{}}{\textbf{Description}}\\%
\midrule\endhead \multicolumn{1}{@{}p{\XLingPapercolawidth}}{iʋ̈}&\multicolumn{1}{p{\XLingPapercolbwidth}}{iɯ̞̈}&\multicolumn{1}{p{\XLingPapercolcwidth}@{}}{}\\%
\multicolumn{1}{@{}p{\XLingPapercolawidth}}{iö}&\multicolumn{1}{p{\XLingPapercolbwidth}}{iɤ}&\multicolumn{1}{p{\XLingPapercolcwidth}@{}}{}\\%
\multicolumn{1}{@{}p{\XLingPapercolawidth}}{ië }&\multicolumn{1}{p{\XLingPapercolbwidth}}{iʌ}&\multicolumn{1}{p{\XLingPapercolcwidth}@{}}{}\\%
\multicolumn{1}{@{}p{\XLingPapercolawidth}}{uë}&\multicolumn{1}{p{\XLingPapercolbwidth}}{uʌ}&\multicolumn{1}{p{\XLingPapercolcwidth}@{}}{}\\%
\multicolumn{1}{@{}p{\XLingPapercolawidth}}{ʋë}&\multicolumn{1}{p{\XLingPapercolbwidth}}{ʊʌ}&\multicolumn{1}{p{\XLingPapercolcwidth}@{}}{}\\%
\multicolumn{1}{@{}p{\XLingPapercolawidth}}{ʋ̈ü}&\multicolumn{1}{p{\XLingPapercolbwidth}}{ʊɯ}&\multicolumn{1}{p{\XLingPapercolcwidth}@{}}{}\\%
\multicolumn{1}{@{}p{\XLingPapercolawidth}}{ia}&\multicolumn{1}{p{\XLingPapercolbwidth}}{ia}&\multicolumn{1}{p{\XLingPapercolcwidth}@{}}{}\\%
\multicolumn{1}{@{}p{\XLingPapercolawidth}}{ian}&\multicolumn{1}{p{\XLingPapercolbwidth}}{ĩã}&\multicolumn{1}{p{\XLingPapercolcwidth}@{}}{}\\%
\multicolumn{1}{@{}p{\XLingPapercolawidth}}{ɩa}&\multicolumn{1}{p{\XLingPapercolbwidth}}{/ea/, \textsquarebracketleft{}ɪa\textsquarebracketright{}}&\multicolumn{1}{p{\XLingPapercolcwidth}@{}}{}\\\bottomrule%
\end{longtable}
}
}{\vspace{10pt}\XLingPaperneedspace{3\baselineskip}\noindent
\fontsize{13}{15.6}\selectfont \textit{{\noindent
\raisebox{\baselineskip}[0pt]{\pdfbookmark[3]{{3.2.4 } Patterns of pitch}{sTone-Marking}}\raisebox{\baselineskip}[0pt]{\protect\hypertarget{sTone-Marking}{}}{3.2.4 }Patterns of pitch}}\markboth{Patterns of pitch}{Eastern Dan writing system}\XLingPaperaddtocontents{sTone-Marking}}\par{}
\penalty10000\vspace{10pt}\penalty10000\indent As indicated in section \hyperlink{sPhonology}{1.3.3} pitch can be analysed in a variety of ways. Here I present some of the things which have been said about pitch in Eastern Dan.\par{}\indent There are four characters which are used to indicate one of ten possible tone patterns for a given word. Not that there are ten possible patterns per word, but rather there are ten patterns in the language. The characters used in the language have no specified Unicode encoding per any known statement. However based on the behavior of various Unicode characters the following are the obvious correct choice – they are the only look a like characters with letter attributes: {\XLingPaperCambriaZMathFontFamily{⟨ {\XLingPaperCharisZSILFontFamily{\textup{\textup{\textmd{˗}}}}} ⟩}} U+02D7 'MODIFIER LETTER MINUS SIGN', {\XLingPaperCambriaZMathFontFamily{⟨ {\XLingPaperCharisZSILFontFamily{\textup{\textup{\textmd{ʼ}}}}} ⟩}} U+02BC 'MODIFIER LETTER APOSTROPHE', {\XLingPaperCambriaZMathFontFamily{⟨ {\XLingPaperCharisZSILFontFamily{\textup{\textup{\textmd{ˮ}}}}} ⟩}} U+02EE MODIFIER LETTER DOUBLE APOSTROPHE, {\XLingPaperCambriaZMathFontFamily{⟨ {\XLingPaperCharisZSILFontFamily{\textup{\textup{\textmd{꞊}}}}} ⟩}} U+A78A 'MODIFIER LETTER SHORT EQUALS SIGN'. \hyperlink{rVydrin2016Tonal}{Vydrin (2016)}\par{}{\vspace{15pt}\XLingPaperneedspace{3\baselineskip}\noindent
\fontsize{13}{15.6}\selectfont \textbf{{\noindent
\raisebox{\baselineskip}[0pt]{\pdfbookmark[2]{{3.3 } Sound visualizations}{sEasternDanOrthography}}\raisebox{\baselineskip}[0pt]{\protect\hypertarget{sEasternDanOrthography}{}}{3.3 }Sound visualizations}}\markboth{Sound visualizations}{Eastern Dan writing system}\XLingPaperaddtocontents{sEasternDanOrthography}}\par{}
\penalty10000\vspace{10pt}\penalty10000\indent The matter of sound visualization is important from several perspectives. There is the association of sounds to symbols, the patterns and rhythms of symbols creating context to the "words", there are the visual rhythms created by the chosen glyphs to represent sounds, and then finally as discussed in section \hyperlink{sKeyboards}{4.2}, there are the patterns which fingers use to create text. I present the sounds and the glyphs used to represent those sounds as presented in \hyperlink{rBolli1994Cours}{Bolli \& Flik (1994)}. In sections \hyperlink{sUnitsAndOrders}{3.4} the groupings which are relevant for pedagogy and literacy are discussed.\par{}{\vspace{10pt}\XLingPaperneedspace{3\baselineskip}\noindent
\fontsize{13}{15.6}\selectfont \textit{{\noindent
\raisebox{\baselineskip}[0pt]{\pdfbookmark[3]{{3.3.1 } Segmental}{sSegmental}}\raisebox{\baselineskip}[0pt]{\protect\hypertarget{sSegmental}{}}{3.3.1 }Segmental}}\markboth{Segmental}{Eastern Dan writing system}\XLingPaperaddtocontents{sSegmental}}\par{}
\penalty10000\vspace{10pt}\penalty10000\vspace{11pt plus 2pt minus 1pt}\XLingPaperneedspace{3\baselineskip}\protect\hypertarget{ntOrthographyVowels}{}\XLingPaperaddtocontents{ntOrthographyVowels}{\protect\raggedright{\singlespacing
{Table }{16.}{  Functional units organized by phoneme\\}}}\vspace{0pt}{\singlespacing
\hspace*{.25in}{\singlespacing\vspace*{-3\baselineskip}
\begin{longtable}
[l]{@{}p{1.8in}p{.8in}p{.78in}p{2.3in}@{}}\toprule\multicolumn{1}{@{}p{1.8in}}{\raggedright\textbf{Codepoint (NFC)}}&\multicolumn{1}{p{.8in}}{\raggedright\textbf{Functional Unit}}&\multicolumn{1}{p{.78in}}{\raggedright\textbf{IPA equivalent}}&\multicolumn{1}{p{2.3in}@{}}{\raggedright\textbf{Phonetic description}}\\%
\midrule\endhead \multicolumn{1}{@{}l}{\XLingPaperCharisZSILFontFamily{\fontsize{8}{9.6}\selectfont \textup{\textup{\vbox{\hbox{\strut{}Uppercase, }\hbox{\strut{}lowercase}}}}}}&\multicolumn{1}{l}{\XLingPaperCharisZSILFontFamily{\fontsize{8}{9.6}\selectfont \textup{\textup{\vbox{\hbox{\strut{}Uppercase, }\hbox{\strut{}lowercase}}}}}}&\multicolumn{1}{p{.78in}}{\raggedright \XLingPaperCharisZSILFontFamily{\fontsize{8}{9.6}\selectfont \textup{\textup{}}}}&\multicolumn{1}{l@{}}{\XLingPaperCharisZSILFontFamily{\fontsize{8}{9.6}\selectfont \textup{\textup{}}}}\\[1.5pt]%
\midrule\multicolumn{1}{@{}l}{\XLingPaperCharisZSILFontFamily{\fontsize{8}{9.6}\selectfont \textup{\textup{\vbox{\hbox{\strut{}U+004E U+0067}\hbox{\strut{}U+006E U+0067}}}}}}&\multicolumn{1}{l}{\XLingPaperCharisZSILFontFamily{\fontsize{8}{9.6}\selectfont \textup{\textup{\vbox{\hbox{\strut{}Ng}\hbox{\strut{}ng}}}}}}&\multicolumn{1}{c}{\XLingPaperCharisZSILFontFamily{\fontsize{8}{9.6}\selectfont \textup{\textup{\vbox{\hbox{\strut{}ŋ}\hbox{\strut{}}}}}}}&\multicolumn{1}{l@{}}{\XLingPaperCharisZSILFontFamily{\fontsize{8}{9.6}\selectfont \textup{\textup{\vbox{\hbox{\strut{}Velar Nasal}\hbox{\strut{}}}}}}}\\[1.5pt]%
\multicolumn{1}{@{}l}{\XLingPaperCharisZSILFontFamily{\fontsize{8}{9.6}\selectfont \textup{\textup{\vbox{\hbox{\strut{}U+0041 U+0061 U+006E}\hbox{\strut{}U+0061 U+0061 U+006E}}}}}}&\multicolumn{1}{l}{\XLingPaperCharisZSILFontFamily{\fontsize{8}{9.6}\selectfont \textup{\textup{\vbox{\hbox{\strut{}Aan}\hbox{\strut{}aan}}}}}}&\multicolumn{1}{p{.78in}}{\vspace{-1.7\baselineskip}\center \XLingPaperCharisZSILFontFamily{\fontsize{8}{9.6}\selectfont \textup{\textup{\vbox{\hbox{\strut{}ãã​}\hbox{\strut{}}}}}}}&\multicolumn{1}{l@{}}{\XLingPaperCharisZSILFontFamily{\fontsize{8}{9.6}\selectfont \textup{\textup{\vbox{\hbox{\strut{}Long nasalized front open unrounded vowel}\hbox{\strut{}}}}}}}\\[1.5pt]%
\multicolumn{1}{@{}l}{\XLingPaperCharisZSILFontFamily{\fontsize{8}{9.6}\selectfont \textup{\textup{\vbox{\hbox{\strut{}U+0041 U+0061}\hbox{\strut{}U+0061 U+0061}}}}}}&\multicolumn{1}{l}{\XLingPaperCharisZSILFontFamily{\fontsize{8}{9.6}\selectfont \textup{\textup{\vbox{\hbox{\strut{}Aa}\hbox{\strut{}aa}}}}}}&\multicolumn{1}{c}{\XLingPaperCharisZSILFontFamily{\fontsize{8}{9.6}\selectfont \textup{\textup{\vbox{\hbox{\strut{}aa}\hbox{\strut{}}}}}}}&\multicolumn{1}{l@{}}{\XLingPaperCharisZSILFontFamily{\fontsize{8}{9.6}\selectfont \textup{\textup{\vbox{\hbox{\strut{}Long front open unrounded vowel}\hbox{\strut{}}}}}}}\\[1.5pt]%
\multicolumn{1}{@{}l}{\XLingPaperCharisZSILFontFamily{\fontsize{8}{9.6}\selectfont \textup{\textup{\vbox{\hbox{\strut{}U+0190 U+0061 U+006E}\hbox{\strut{}U+025B U+0061 U+006E}}}}}}&\multicolumn{1}{l}{\XLingPaperCharisZSILFontFamily{\fontsize{8}{9.6}\selectfont \textup{\textup{\vbox{\hbox{\strut{}Ɛan}\hbox{\strut{}ɛan}}}}}}&\multicolumn{1}{c}{\XLingPaperCharisZSILFontFamily{\fontsize{8}{9.6}\selectfont \textup{\textup{\vbox{\hbox{\strut{}æ̃}\hbox{\strut{}}}}}}}&\multicolumn{1}{l@{}}{\XLingPaperCharisZSILFontFamily{\fontsize{8}{9.6}\selectfont \textup{\textup{\vbox{\hbox{\strut{}Short nasalized near-open front unrounded vowel}\hbox{\strut{}}}}}}}\\[1.5pt]%
\multicolumn{1}{@{}l}{\XLingPaperCharisZSILFontFamily{\fontsize{8}{9.6}\selectfont \textup{\textup{\vbox{\hbox{\strut{}U+0190 U+0061}\hbox{\strut{}U+025B U+0061}}}}}}&\multicolumn{1}{l}{\XLingPaperCharisZSILFontFamily{\fontsize{8}{9.6}\selectfont \textup{\textup{\vbox{\hbox{\strut{}Ɛa}\hbox{\strut{}ɛa}}}}}}&\multicolumn{1}{c}{\XLingPaperCharisZSILFontFamily{\fontsize{8}{9.6}\selectfont \textup{\textup{\vbox{\hbox{\strut{}æ}\hbox{\strut{}}}}}}}&\multicolumn{1}{l@{}}{\XLingPaperCharisZSILFontFamily{\fontsize{8}{9.6}\selectfont \textup{\textup{\vbox{\hbox{\strut{}Short near-open front unrounded vowel}\hbox{\strut{}}}}}}}\\[1.5pt]%
\multicolumn{1}{@{}l}{\XLingPaperCharisZSILFontFamily{\fontsize{8}{9.6}\selectfont \textup{\textup{\vbox{\hbox{\strut{}U+0041 U+0254}\hbox{\strut{}U+0061 U+0254}}}}}}&\multicolumn{1}{l}{\XLingPaperCharisZSILFontFamily{\fontsize{8}{9.6}\selectfont \textup{\textup{\vbox{\hbox{\strut{}Aɔn}\hbox{\strut{}aɔn}}}}}}&\multicolumn{1}{c}{\XLingPaperCharisZSILFontFamily{\fontsize{8}{9.6}\selectfont \textup{\textup{\vbox{\hbox{\strut{}ɒ̃}\hbox{\strut{}}}}}}}&\multicolumn{1}{l@{}}{\XLingPaperCharisZSILFontFamily{\fontsize{8}{9.6}\selectfont \textup{\textup{\vbox{\hbox{\strut{}Short nasalized back rounded vowel}\hbox{\strut{}}}}}}}\\[1.5pt]%
\multicolumn{1}{@{}l}{\XLingPaperCharisZSILFontFamily{\fontsize{8}{9.6}\selectfont \textup{\textup{\vbox{\hbox{\strut{}U+0041 U+0254}\hbox{\strut{}U+0061 U+0254}}}}}}&\multicolumn{1}{l}{\XLingPaperCharisZSILFontFamily{\fontsize{8}{9.6}\selectfont \textup{\textup{\vbox{\hbox{\strut{}Aɔ}\hbox{\strut{}aɔ}}}}}}&\multicolumn{1}{c}{\XLingPaperCharisZSILFontFamily{\fontsize{8}{9.6}\selectfont \textup{\textup{\vbox{\hbox{\strut{}ɒ}\hbox{\strut{}}}}}}}&\multicolumn{1}{l@{}}{\XLingPaperCharisZSILFontFamily{\fontsize{8}{9.6}\selectfont \textup{\textup{\vbox{\hbox{\strut{}Short back rounded vowel}\hbox{\strut{}}}}}}}\\[1.5pt]%
\multicolumn{1}{@{}l}{\XLingPaperCharisZSILFontFamily{\fontsize{8}{9.6}\selectfont \textup{\textup{\vbox{\hbox{\strut{}U+0041 U+006E}\hbox{\strut{}U+0061 U+006E}}}}}}&\multicolumn{1}{l}{\XLingPaperCharisZSILFontFamily{\fontsize{8}{9.6}\selectfont \textup{\textup{\vbox{\hbox{\strut{}An}\hbox{\strut{}an}}}}}}&\multicolumn{1}{c}{\XLingPaperCharisZSILFontFamily{\fontsize{8}{9.6}\selectfont \textup{\textup{\vbox{\hbox{\strut{}ã}\hbox{\strut{}}}}}}}&\multicolumn{1}{l@{}}{\XLingPaperCharisZSILFontFamily{\fontsize{8}{9.6}\selectfont \textup{\textup{\vbox{\hbox{\strut{}Short nasalized front open unrounded vowel}\hbox{\strut{}}}}}}}\\[1.5pt]%
\multicolumn{1}{@{}l}{\XLingPaperCharisZSILFontFamily{\fontsize{8}{9.6}\selectfont \textup{\textup{\vbox{\hbox{\strut{}U+0190}\hbox{\strut{}U+025B}}}}}}&\multicolumn{1}{l}{\XLingPaperCharisZSILFontFamily{\fontsize{8}{9.6}\selectfont \textup{\textup{\vbox{\hbox{\strut{}Ɛ}\hbox{\strut{}ɛ}}}}}}&\multicolumn{1}{c}{\XLingPaperCharisZSILFontFamily{\fontsize{8}{9.6}\selectfont \textup{\textup{\vbox{\hbox{\strut{}ɛ}\hbox{\strut{}}}}}}}&\multicolumn{1}{l@{}}{\XLingPaperCharisZSILFontFamily{\fontsize{8}{9.6}\selectfont \textup{\textup{\vbox{\hbox{\strut{}Short open-mid front unrounded vowel}\hbox{\strut{}}}}}}}\\[1.5pt]%
\multicolumn{1}{@{}l}{\XLingPaperCharisZSILFontFamily{\fontsize{8}{9.6}\selectfont \textup{\textup{\vbox{\hbox{\strut{}U+0190 U+025B}\hbox{\strut{}U+025B U+025B}}}}}}&\multicolumn{1}{l}{\XLingPaperCharisZSILFontFamily{\fontsize{8}{9.6}\selectfont \textup{\textup{\vbox{\hbox{\strut{}Ɛɛ}\hbox{\strut{}ɛɛ}}}}}}&\multicolumn{1}{c}{\XLingPaperCharisZSILFontFamily{\fontsize{8}{9.6}\selectfont \textup{\textup{\vbox{\hbox{\strut{}ɛɛ}\hbox{\strut{}}}}}}}&\multicolumn{1}{l@{}}{\XLingPaperCharisZSILFontFamily{\fontsize{8}{9.6}\selectfont \textup{\textup{\vbox{\hbox{\strut{}Long open-mid front unrounded vowel}\hbox{\strut{}}}}}}}\\[1.5pt]%
\multicolumn{1}{@{}l}{\XLingPaperCharisZSILFontFamily{\fontsize{8}{9.6}\selectfont \textup{\textup{\vbox{\hbox{\strut{}U+0190 U+025B U+006E}\hbox{\strut{}U+025B U+025B U+006E}}}}}}&\multicolumn{1}{l}{\XLingPaperCharisZSILFontFamily{\fontsize{8}{9.6}\selectfont \textup{\textup{\vbox{\hbox{\strut{}Ɛɛn}\hbox{\strut{}ɛɛn}}}}}}&\multicolumn{1}{c}{\XLingPaperCharisZSILFontFamily{\fontsize{8}{9.6}\selectfont \textup{\textup{\vbox{\hbox{\strut{}ɛ̃ɛ̃}\hbox{\strut{}}}}}}}&\multicolumn{1}{l@{}}{\XLingPaperCharisZSILFontFamily{\fontsize{8}{9.6}\selectfont \textup{\textup{\vbox{\hbox{\strut{}Long nasalized open-mid front unrounded vowel}\hbox{\strut{}}}}}}}\\[1.5pt]%
\multicolumn{1}{@{}l}{\XLingPaperCharisZSILFontFamily{\fontsize{8}{9.6}\selectfont \textup{\textup{\vbox{\hbox{\strut{}U+0190 U+006E}\hbox{\strut{}U+025B U+006E}}}}}}&\multicolumn{1}{l}{\XLingPaperCharisZSILFontFamily{\fontsize{8}{9.6}\selectfont \textup{\textup{\vbox{\hbox{\strut{}Ɛn}\hbox{\strut{}ɛn}}}}}}&\multicolumn{1}{c}{\XLingPaperCharisZSILFontFamily{\fontsize{8}{9.6}\selectfont \textup{\textup{\vbox{\hbox{\strut{}ɛ̃}\hbox{\strut{}}}}}}}&\multicolumn{1}{l@{}}{\XLingPaperCharisZSILFontFamily{\fontsize{8}{9.6}\selectfont \textup{\textup{\vbox{\hbox{\strut{}Short nasalized open-mid front unrounded vowel}\hbox{\strut{}}}}}}}\\[1.5pt]%
\multicolumn{1}{@{}l}{\XLingPaperCharisZSILFontFamily{\fontsize{8}{9.6}\selectfont \textup{\textup{\vbox{\hbox{\strut{}U+0186}\hbox{\strut{}U+0254}}}}}}&\multicolumn{1}{l}{\XLingPaperCharisZSILFontFamily{\fontsize{8}{9.6}\selectfont \textup{\textup{\vbox{\hbox{\strut{}Ɔ}\hbox{\strut{}ɔ}}}}}}&\multicolumn{1}{c}{\XLingPaperCharisZSILFontFamily{\fontsize{8}{9.6}\selectfont \textup{\textup{\vbox{\hbox{\strut{}ɔ}\hbox{\strut{}}}}}}}&\multicolumn{1}{l@{}}{\XLingPaperCharisZSILFontFamily{\fontsize{8}{9.6}\selectfont \textup{\textup{\vbox{\hbox{\strut{}Short open-mid back rounded vowel}\hbox{\strut{}}}}}}}\\[1.5pt]%
\multicolumn{1}{@{}l}{\XLingPaperCharisZSILFontFamily{\fontsize{8}{9.6}\selectfont \textup{\textup{\vbox{\hbox{\strut{}U+0186 U+0254}\hbox{\strut{}U+0254 U+0254}}}}}}&\multicolumn{1}{l}{\XLingPaperCharisZSILFontFamily{\fontsize{8}{9.6}\selectfont \textup{\textup{\vbox{\hbox{\strut{}Ɔɔ}\hbox{\strut{}ɔɔ}}}}}}&\multicolumn{1}{c}{\XLingPaperCharisZSILFontFamily{\fontsize{8}{9.6}\selectfont \textup{\textup{\vbox{\hbox{\strut{}ɔɔ}\hbox{\strut{}}}}}}}&\multicolumn{1}{l@{}}{\XLingPaperCharisZSILFontFamily{\fontsize{8}{9.6}\selectfont \textup{\textup{\vbox{\hbox{\strut{}Long open-mid back rounded vowel}\hbox{\strut{}}}}}}}\\[1.5pt]%
\multicolumn{1}{@{}l}{\XLingPaperCharisZSILFontFamily{\fontsize{8}{9.6}\selectfont \textup{\textup{\vbox{\hbox{\strut{}U+0186 U+0254 U+006E}\hbox{\strut{}U+0254 U+0254 U+006E}}}}}}&\multicolumn{1}{l}{\XLingPaperCharisZSILFontFamily{\fontsize{8}{9.6}\selectfont \textup{\textup{\vbox{\hbox{\strut{}Ɔɔn}\hbox{\strut{}ɔɔn}}}}}}&\multicolumn{1}{c}{\XLingPaperCharisZSILFontFamily{\fontsize{8}{9.6}\selectfont \textup{\textup{\vbox{\hbox{\strut{}ɔ̃ɔ̃}\hbox{\strut{}}}}}}}&\multicolumn{1}{l@{}}{\XLingPaperCharisZSILFontFamily{\fontsize{8}{9.6}\selectfont \textup{\textup{\vbox{\hbox{\strut{}Long nasalized open-mid back rounded vowel}\hbox{\strut{}}}}}}}\\[1.5pt]%
\multicolumn{1}{@{}l}{\XLingPaperCharisZSILFontFamily{\fontsize{8}{9.6}\selectfont \textup{\textup{\vbox{\hbox{\strut{}U+0186 U+006E}\hbox{\strut{}U+0254 U+006E}}}}}}&\multicolumn{1}{l}{\XLingPaperCharisZSILFontFamily{\fontsize{8}{9.6}\selectfont \textup{\textup{\vbox{\hbox{\strut{}Ɔn}\hbox{\strut{}ɔn}}}}}}&\multicolumn{1}{c}{\XLingPaperCharisZSILFontFamily{\fontsize{8}{9.6}\selectfont \textup{\textup{\vbox{\hbox{\strut{}ɔ̃}\hbox{\strut{}}}}}}}&\multicolumn{1}{l@{}}{\XLingPaperCharisZSILFontFamily{\fontsize{8}{9.6}\selectfont \textup{\textup{\vbox{\hbox{\strut{}Short nasalized open-mid back rounded vowel}\hbox{\strut{}}}}}}}\\[1.5pt]%
\multicolumn{1}{@{}l}{\XLingPaperCharisZSILFontFamily{\fontsize{8}{9.6}\selectfont \textup{\textup{\vbox{\hbox{\strut{}U+00DC}\hbox{\strut{}U+00FC}}}}}}&\multicolumn{1}{l}{\XLingPaperCharisZSILFontFamily{\fontsize{8}{9.6}\selectfont \textup{\textup{\vbox{\hbox{\strut{}Ü}\hbox{\strut{}ü}}}}}}&\multicolumn{1}{c}{\XLingPaperCharisZSILFontFamily{\fontsize{8}{9.6}\selectfont \textup{\textup{\vbox{\hbox{\strut{}ɯ}\hbox{\strut{}}}}}}}&\multicolumn{1}{l@{}}{\XLingPaperCharisZSILFontFamily{\fontsize{8}{9.6}\selectfont \textup{\textup{\vbox{\hbox{\strut{}Short close back unrounded vowel}\hbox{\strut{}}}}}}}\\[1.5pt]%
\multicolumn{1}{@{}l}{\XLingPaperCharisZSILFontFamily{\fontsize{8}{9.6}\selectfont \textup{\textup{\vbox{\hbox{\strut{}U+00DC U+00FC}\hbox{\strut{}U+00FC U+00FC}}}}}}&\multicolumn{1}{l}{\XLingPaperCharisZSILFontFamily{\fontsize{8}{9.6}\selectfont \textup{\textup{\vbox{\hbox{\strut{}Üü}\hbox{\strut{}üü}}}}}}&\multicolumn{1}{c}{\XLingPaperCharisZSILFontFamily{\fontsize{8}{9.6}\selectfont \textup{\textup{\vbox{\hbox{\strut{}ɯɯ}\hbox{\strut{}}}}}}}&\multicolumn{1}{l@{}}{\XLingPaperCharisZSILFontFamily{\fontsize{8}{9.6}\selectfont \textup{\textup{\vbox{\hbox{\strut{}Long close back unrounded vowel}\hbox{\strut{}}}}}}}\\[1.5pt]%
\multicolumn{1}{@{}l}{\XLingPaperCharisZSILFontFamily{\fontsize{8}{9.6}\selectfont \textup{\textup{\vbox{\hbox{\strut{}U+00CB}\hbox{\strut{}U+00EB}}}}}}&\multicolumn{1}{l}{\XLingPaperCharisZSILFontFamily{\fontsize{8}{9.6}\selectfont \textup{\textup{\vbox{\hbox{\strut{}Ë}\hbox{\strut{}ë}}}}}}&\multicolumn{1}{c}{\XLingPaperCharisZSILFontFamily{\fontsize{8}{9.6}\selectfont \textup{\textup{\vbox{\hbox{\strut{}ʌ}\hbox{\strut{}}}}}}}&\multicolumn{1}{l@{}}{\XLingPaperCharisZSILFontFamily{\fontsize{8}{9.6}\selectfont \textup{\textup{\vbox{\hbox{\strut{}Short open-mid back unrounded vowel}\hbox{\strut{}}}}}}}\\[1.5pt]%
\multicolumn{1}{@{}l}{\XLingPaperCharisZSILFontFamily{\fontsize{8}{9.6}\selectfont \textup{\textup{\vbox{\hbox{\strut{}U+00D6}\hbox{\strut{}U+00F6}}}}}}&\multicolumn{1}{l}{\XLingPaperCharisZSILFontFamily{\fontsize{8}{9.6}\selectfont \textup{\textup{\vbox{\hbox{\strut{}Ö}\hbox{\strut{}ö}}}}}}&\multicolumn{1}{c}{\XLingPaperCharisZSILFontFamily{\fontsize{8}{9.6}\selectfont \textup{\textup{\vbox{\hbox{\strut{}ɤ}\hbox{\strut{}}}}}}}&\multicolumn{1}{l@{}}{\XLingPaperCharisZSILFontFamily{\fontsize{8}{9.6}\selectfont \textup{\textup{\vbox{\hbox{\strut{}Short close-mid back unrounded vowel}\hbox{\strut{}}}}}}}\\[1.5pt]%
\multicolumn{1}{@{}l}{\XLingPaperCharisZSILFontFamily{\fontsize{8}{9.6}\selectfont \textup{\textup{\vbox{\hbox{\strut{}U+00D6 U+00F6}\hbox{\strut{}U+00F6 U+00F6}}}}}}&\multicolumn{1}{l}{\XLingPaperCharisZSILFontFamily{\fontsize{8}{9.6}\selectfont \textup{\textup{\vbox{\hbox{\strut{}Öö}\hbox{\strut{}öö}}}}}}&\multicolumn{1}{c}{\XLingPaperCharisZSILFontFamily{\fontsize{8}{9.6}\selectfont \textup{\textup{\vbox{\hbox{\strut{}ɤɤ}\hbox{\strut{}}}}}}}&\multicolumn{1}{l@{}}{\XLingPaperCharisZSILFontFamily{\fontsize{8}{9.6}\selectfont \textup{\textup{\vbox{\hbox{\strut{}Long close-mid back unrounded vowel}\hbox{\strut{}}}}}}}\\[1.5pt]%
\multicolumn{1}{@{}l}{\XLingPaperCharisZSILFontFamily{\fontsize{8}{9.6}\selectfont \textup{\textup{\vbox{\hbox{\strut{}U+00CB U+00EB}\hbox{\strut{}U+00EB U+00EB}}}}}}&\multicolumn{1}{l}{\XLingPaperCharisZSILFontFamily{\fontsize{8}{9.6}\selectfont \textup{\textup{\vbox{\hbox{\strut{}Ëë}\hbox{\strut{}ëë}}}}}}&\multicolumn{1}{c}{\XLingPaperCharisZSILFontFamily{\fontsize{8}{9.6}\selectfont \textup{\textup{\vbox{\hbox{\strut{}ʌʌ}\hbox{\strut{}}}}}}}&\multicolumn{1}{l@{}}{\XLingPaperCharisZSILFontFamily{\fontsize{8}{9.6}\selectfont \textup{\textup{\vbox{\hbox{\strut{}Long open-mid back unrounded vowel}\hbox{\strut{}}}}}}}\\[1.5pt]%
\multicolumn{1}{@{}l}{\XLingPaperCharisZSILFontFamily{\fontsize{8}{9.6}\selectfont \textup{\textup{\vbox{\hbox{\strut{}U+00CB U+00EB U+006E}\hbox{\strut{}U+00EB U+00EB U+006E}}}}}}&\multicolumn{1}{l}{\XLingPaperCharisZSILFontFamily{\fontsize{8}{9.6}\selectfont \textup{\textup{\vbox{\hbox{\strut{}Ëën}\hbox{\strut{}ëën}}}}}}&\multicolumn{1}{c}{\XLingPaperCharisZSILFontFamily{\fontsize{8}{9.6}\selectfont \textup{\textup{\vbox{\hbox{\strut{}ʌ̃ʌ̃}\hbox{\strut{}}}}}}}&\multicolumn{1}{l@{}}{\XLingPaperCharisZSILFontFamily{\fontsize{8}{9.6}\selectfont \textup{\textup{\vbox{\hbox{\strut{}Long nasalized open-mid back unrounded vowel}\hbox{\strut{}}}}}}}\\[1.5pt]%
\multicolumn{1}{@{}l}{\XLingPaperCharisZSILFontFamily{\fontsize{8}{9.6}\selectfont \textup{\textup{\vbox{\hbox{\strut{}U+00CB U+006E}\hbox{\strut{}U+00EB U+006E}}}}}}&\multicolumn{1}{l}{\XLingPaperCharisZSILFontFamily{\fontsize{8}{9.6}\selectfont \textup{\textup{\vbox{\hbox{\strut{}Ën}\hbox{\strut{}ën}}}}}}&\multicolumn{1}{c}{\XLingPaperCharisZSILFontFamily{\fontsize{8}{9.6}\selectfont \textup{\textup{\vbox{\hbox{\strut{}ʌ̃}\hbox{\strut{}}}}}}}&\multicolumn{1}{l@{}}{\XLingPaperCharisZSILFontFamily{\fontsize{8}{9.6}\selectfont \textup{\textup{\vbox{\hbox{\strut{}Short nasalized open-mid back unrounded vowel}\hbox{\strut{}}}}}}}\\[1.5pt]%
\multicolumn{1}{@{}l}{\XLingPaperCharisZSILFontFamily{\fontsize{8}{9.6}\selectfont \textup{\textup{\vbox{\hbox{\strut{}U+0045}\hbox{\strut{}U+0065}}}}}}&\multicolumn{1}{l}{\XLingPaperCharisZSILFontFamily{\fontsize{8}{9.6}\selectfont \textup{\textup{\vbox{\hbox{\strut{}E}\hbox{\strut{}e}}}}}}&\multicolumn{1}{c}{\XLingPaperCharisZSILFontFamily{\fontsize{8}{9.6}\selectfont \textup{\textup{\vbox{\hbox{\strut{}e}\hbox{\strut{}}}}}}}&\multicolumn{1}{l@{}}{\XLingPaperCharisZSILFontFamily{\fontsize{8}{9.6}\selectfont \textup{\textup{\vbox{\hbox{\strut{}Short close-mid front unrounded vowel}\hbox{\strut{}}}}}}}\\[1.5pt]%
\multicolumn{1}{@{}l}{\XLingPaperCharisZSILFontFamily{\fontsize{8}{9.6}\selectfont \textup{\textup{\vbox{\hbox{\strut{}U+0045 U+0065}\hbox{\strut{}U+0065 U+0065}}}}}}&\multicolumn{1}{l}{\XLingPaperCharisZSILFontFamily{\fontsize{8}{9.6}\selectfont \textup{\textup{\vbox{\hbox{\strut{}Ee}\hbox{\strut{}ee}}}}}}&\multicolumn{1}{c}{\XLingPaperCharisZSILFontFamily{\fontsize{8}{9.6}\selectfont \textup{\textup{\vbox{\hbox{\strut{}ee}\hbox{\strut{}}}}}}}&\multicolumn{1}{l@{}}{\XLingPaperCharisZSILFontFamily{\fontsize{8}{9.6}\selectfont \textup{\textup{\vbox{\hbox{\strut{}Long close-mid front unrounded vowel}\hbox{\strut{}}}}}}}\\[1.5pt]%
\multicolumn{1}{@{}l}{\XLingPaperCharisZSILFontFamily{\fontsize{8}{9.6}\selectfont \textup{\textup{\vbox{\hbox{\strut{}U+0041}\hbox{\strut{}U+0061}}}}}}&\multicolumn{1}{l}{\XLingPaperCharisZSILFontFamily{\fontsize{8}{9.6}\selectfont \textup{\textup{\vbox{\hbox{\strut{}A}\hbox{\strut{}a}}}}}}&\multicolumn{1}{c}{\XLingPaperCharisZSILFontFamily{\fontsize{8}{9.6}\selectfont \textup{\textup{\vbox{\hbox{\strut{}a}\hbox{\strut{}}}}}}}&\multicolumn{1}{l@{}}{\XLingPaperCharisZSILFontFamily{\fontsize{8}{9.6}\selectfont \textup{\textup{\vbox{\hbox{\strut{}Short open front unrounded vowel}\hbox{\strut{}}}}}}}\\[1.5pt]%
\multicolumn{1}{@{}l}{\XLingPaperCharisZSILFontFamily{\fontsize{8}{9.6}\selectfont \textup{\textup{\vbox{\hbox{\strut{}U+00DC U+006E}\hbox{\strut{}U+00FC U+006E}}}}}}&\multicolumn{1}{l}{\XLingPaperCharisZSILFontFamily{\fontsize{8}{9.6}\selectfont \textup{\textup{\vbox{\hbox{\strut{}Ün}\hbox{\strut{}ün}}}}}}&\multicolumn{1}{c}{\XLingPaperCharisZSILFontFamily{\fontsize{8}{9.6}\selectfont \textup{\textup{\vbox{\hbox{\strut{}ɯ̃}\hbox{\strut{}}}}}}}&\multicolumn{1}{l@{}}{\XLingPaperCharisZSILFontFamily{\fontsize{8}{9.6}\selectfont \textup{\textup{\vbox{\hbox{\strut{}Short nasalized close back unrounded vowel}\hbox{\strut{}}}}}}}\\[1.5pt]%
\multicolumn{1}{@{}l}{\XLingPaperCharisZSILFontFamily{\fontsize{8}{9.6}\selectfont \textup{\textup{\vbox{\hbox{\strut{}U+00DC U+00FC U+006E}\hbox{\strut{}U+00FC U+00FC U+006E}}}}}}&\multicolumn{1}{l}{\XLingPaperCharisZSILFontFamily{\fontsize{8}{9.6}\selectfont \textup{\textup{\vbox{\hbox{\strut{}Üün}\hbox{\strut{}üün}}}}}}&\multicolumn{1}{c}{\XLingPaperCharisZSILFontFamily{\fontsize{8}{9.6}\selectfont \textup{\textup{\vbox{\hbox{\strut{}ɯ̃ɯ̃}\hbox{\strut{}}}}}}}&\multicolumn{1}{l@{}}{\XLingPaperCharisZSILFontFamily{\fontsize{8}{9.6}\selectfont \textup{\textup{\vbox{\hbox{\strut{}Long nasalized close back unrounded vowel}\hbox{\strut{}}}}}}}\\[1.5pt]%
\multicolumn{1}{@{}l}{\XLingPaperCharisZSILFontFamily{\fontsize{8}{9.6}\selectfont \textup{\textup{\vbox{\hbox{\strut{}U+0055}\hbox{\strut{}U+0075}}}}}}&\multicolumn{1}{l}{\XLingPaperCharisZSILFontFamily{\fontsize{8}{9.6}\selectfont \textup{\textup{\vbox{\hbox{\strut{}U}\hbox{\strut{}u}}}}}}&\multicolumn{1}{c}{\XLingPaperCharisZSILFontFamily{\fontsize{8}{9.6}\selectfont \textup{\textup{\vbox{\hbox{\strut{}u}\hbox{\strut{}}}}}}}&\multicolumn{1}{l@{}}{\XLingPaperCharisZSILFontFamily{\fontsize{8}{9.6}\selectfont \textup{\textup{\vbox{\hbox{\strut{}Short close back rounded vowel}\hbox{\strut{}}}}}}}\\[1.5pt]%
\multicolumn{1}{@{}l}{\XLingPaperCharisZSILFontFamily{\fontsize{8}{9.6}\selectfont \textup{\textup{\vbox{\hbox{\strut{}U+0055 U+0075}\hbox{\strut{}U+0075 U+0075}}}}}}&\multicolumn{1}{l}{\XLingPaperCharisZSILFontFamily{\fontsize{8}{9.6}\selectfont \textup{\textup{\vbox{\hbox{\strut{}Uu}\hbox{\strut{}uu}}}}}}&\multicolumn{1}{c}{\XLingPaperCharisZSILFontFamily{\fontsize{8}{9.6}\selectfont \textup{\textup{\vbox{\hbox{\strut{}uu}\hbox{\strut{}}}}}}}&\multicolumn{1}{l@{}}{\XLingPaperCharisZSILFontFamily{\fontsize{8}{9.6}\selectfont \textup{\textup{\vbox{\hbox{\strut{}Long close back rounded vowel}\hbox{\strut{}}}}}}}\\[1.5pt]%
\multicolumn{1}{@{}l}{\XLingPaperCharisZSILFontFamily{\fontsize{8}{9.6}\selectfont \textup{\textup{\vbox{\hbox{\strut{}U+0055 U+006E}\hbox{\strut{}U+0075 U+006E}}}}}}&\multicolumn{1}{l}{\XLingPaperCharisZSILFontFamily{\fontsize{8}{9.6}\selectfont \textup{\textup{\vbox{\hbox{\strut{}Un}\hbox{\strut{}un}}}}}}&\multicolumn{1}{c}{\XLingPaperCharisZSILFontFamily{\fontsize{8}{9.6}\selectfont \textup{\textup{\vbox{\hbox{\strut{}ũ}\hbox{\strut{}}}}}}}&\multicolumn{1}{l@{}}{\XLingPaperCharisZSILFontFamily{\fontsize{8}{9.6}\selectfont \textup{\textup{\vbox{\hbox{\strut{}Short nasalized close back rounded vowel}\hbox{\strut{}}}}}}}\\[1.5pt]%
\multicolumn{1}{@{}l}{\XLingPaperCharisZSILFontFamily{\fontsize{8}{9.6}\selectfont \textup{\textup{\vbox{\hbox{\strut{}U+0055 U+0075 U+006E}\hbox{\strut{}U+0075 U+0075 U+006E}}}}}}&\multicolumn{1}{l}{\XLingPaperCharisZSILFontFamily{\fontsize{8}{9.6}\selectfont \textup{\textup{\vbox{\hbox{\strut{}Uun}\hbox{\strut{}uun}}}}}}&\multicolumn{1}{c}{\XLingPaperCharisZSILFontFamily{\fontsize{8}{9.6}\selectfont \textup{\textup{\vbox{\hbox{\strut{}ũũ}\hbox{\strut{}}}}}}}&\multicolumn{1}{l@{}}{\XLingPaperCharisZSILFontFamily{\fontsize{8}{9.6}\selectfont \textup{\textup{\vbox{\hbox{\strut{}Long nasalized close back rounded vowel}\hbox{\strut{}}}}}}}\\[1.5pt]%
\multicolumn{1}{@{}l}{\XLingPaperCharisZSILFontFamily{\fontsize{8}{9.6}\selectfont \textup{\textup{\vbox{\hbox{\strut{}U+004F}\hbox{\strut{}U+006F}}}}}}&\multicolumn{1}{l}{\XLingPaperCharisZSILFontFamily{\fontsize{8}{9.6}\selectfont \textup{\textup{\vbox{\hbox{\strut{}O}\hbox{\strut{}o}}}}}}&\multicolumn{1}{c}{\XLingPaperCharisZSILFontFamily{\fontsize{8}{9.6}\selectfont \textup{\textup{\vbox{\hbox{\strut{}o}\hbox{\strut{}}}}}}}&\multicolumn{1}{l@{}}{\XLingPaperCharisZSILFontFamily{\fontsize{8}{9.6}\selectfont \textup{\textup{\vbox{\hbox{\strut{}Short close-mid back rounded vowel}\hbox{\strut{}}}}}}}\\[1.5pt]%
\multicolumn{1}{@{}l}{\XLingPaperCharisZSILFontFamily{\fontsize{8}{9.6}\selectfont \textup{\textup{\vbox{\hbox{\strut{}U+004F U+006F}\hbox{\strut{}U+006F U+006F}}}}}}&\multicolumn{1}{l}{\XLingPaperCharisZSILFontFamily{\fontsize{8}{9.6}\selectfont \textup{\textup{\vbox{\hbox{\strut{}Oo}\hbox{\strut{}oo}}}}}}&\multicolumn{1}{c}{\XLingPaperCharisZSILFontFamily{\fontsize{8}{9.6}\selectfont \textup{\textup{\vbox{\hbox{\strut{}oo}\hbox{\strut{}}}}}}}&\multicolumn{1}{l@{}}{\XLingPaperCharisZSILFontFamily{\fontsize{8}{9.6}\selectfont \textup{\textup{\vbox{\hbox{\strut{}Long close-mid back rounded vowel}\hbox{\strut{}}}}}}}\\[1.5pt]%
\multicolumn{1}{@{}l}{\XLingPaperCharisZSILFontFamily{\fontsize{8}{9.6}\selectfont \textup{\textup{\vbox{\hbox{\strut{}U+0049 U+0069 U+006E}\hbox{\strut{}U+0069 U+0069 U+006E}}}}}}&\multicolumn{1}{l}{\XLingPaperCharisZSILFontFamily{\fontsize{8}{9.6}\selectfont \textup{\textup{\vbox{\hbox{\strut{}Iin}\hbox{\strut{}iin}}}}}}&\multicolumn{1}{c}{\XLingPaperCharisZSILFontFamily{\fontsize{8}{9.6}\selectfont \textup{\textup{\vbox{\hbox{\strut{}ĩĩ}\hbox{\strut{}}}}}}}&\multicolumn{1}{l@{}}{\XLingPaperCharisZSILFontFamily{\fontsize{8}{9.6}\selectfont \textup{\textup{\vbox{\hbox{\strut{}Long nasalized close front unrounded vowel}\hbox{\strut{}}}}}}}\\[1.5pt]%
\multicolumn{1}{@{}l}{\XLingPaperCharisZSILFontFamily{\fontsize{8}{9.6}\selectfont \textup{\textup{\vbox{\hbox{\strut{}U+0049 U+0069}\hbox{\strut{}U+0069 U+0069}}}}}}&\multicolumn{1}{l}{\XLingPaperCharisZSILFontFamily{\fontsize{8}{9.6}\selectfont \textup{\textup{\vbox{\hbox{\strut{}Ii}\hbox{\strut{}ii}}}}}}&\multicolumn{1}{c}{\XLingPaperCharisZSILFontFamily{\fontsize{8}{9.6}\selectfont \textup{\textup{\vbox{\hbox{\strut{}ii}\hbox{\strut{}}}}}}}&\multicolumn{1}{l@{}}{\XLingPaperCharisZSILFontFamily{\fontsize{8}{9.6}\selectfont \textup{\textup{\vbox{\hbox{\strut{}Long close front unrounded vowel}\hbox{\strut{}}}}}}}\\[1.5pt]%
\multicolumn{1}{@{}l}{\XLingPaperCharisZSILFontFamily{\fontsize{8}{9.6}\selectfont \textup{\textup{\vbox{\hbox{\strut{}U+0049 U+006E}\hbox{\strut{}U+0069 U+006E}}}}}}&\multicolumn{1}{l}{\XLingPaperCharisZSILFontFamily{\fontsize{8}{9.6}\selectfont \textup{\textup{\vbox{\hbox{\strut{}In}\hbox{\strut{}in}}}}}}&\multicolumn{1}{c}{\XLingPaperCharisZSILFontFamily{\fontsize{8}{9.6}\selectfont \textup{\textup{\vbox{\hbox{\strut{}ĩ}\hbox{\strut{}}}}}}}&\multicolumn{1}{l@{}}{\XLingPaperCharisZSILFontFamily{\fontsize{8}{9.6}\selectfont \textup{\textup{\vbox{\hbox{\strut{}Short nasalized close front unrounded vowel}\hbox{\strut{}}}}}}}\\[1.5pt]%
\multicolumn{1}{@{}l}{\XLingPaperCharisZSILFontFamily{\fontsize{8}{9.6}\selectfont \textup{\textup{\vbox{\hbox{\strut{}U+0049}\hbox{\strut{}U+0069}}}}}}&\multicolumn{1}{l}{\XLingPaperCharisZSILFontFamily{\fontsize{8}{9.6}\selectfont \textup{\textup{\vbox{\hbox{\strut{}I}\hbox{\strut{}i}}}}}}&\multicolumn{1}{c}{\XLingPaperCharisZSILFontFamily{\fontsize{8}{9.6}\selectfont \textup{\textup{\vbox{\hbox{\strut{}i}\hbox{\strut{}}}}}}}&\multicolumn{1}{l@{}}{\XLingPaperCharisZSILFontFamily{\fontsize{8}{9.6}\selectfont \textup{\textup{\vbox{\hbox{\strut{}Short close front unrounded vowel}\hbox{\strut{}}}}}}}\\[1.5pt]%
\multicolumn{1}{@{}l}{\XLingPaperCharisZSILFontFamily{\fontsize{8}{9.6}\selectfont \textup{\textup{\vbox{\hbox{\strut{}U+0196 U+0269}\hbox{\strut{}U+0269 U+0269}}}}}}&\multicolumn{1}{l}{\XLingPaperCharisZSILFontFamily{\fontsize{8}{9.6}\selectfont \textup{\textup{\vbox{\hbox{\strut{}Ɩɩ}\hbox{\strut{}ɩɩ}}}}}}&\multicolumn{1}{c}{\XLingPaperCharisZSILFontFamily{\fontsize{8}{9.6}\selectfont \textup{\textup{\vbox{\hbox{\strut{}{\XLingPaperCharisZSILFontFamily{/ {\XLingPaperCharisZSILFontFamily{\textup{\textup{\textmd{ee}}}}} /}}}\hbox{\strut{}{\XLingPaperDoulosZSILFontFamily{\textsquarebracketleft{} {\XLingPaperCharisZSILFontFamily{\textup{\textup{\textmd{ɪɪ}}}}} \textsquarebracketright{}}}}}}}}}&\multicolumn{1}{l@{}}{\XLingPaperCharisZSILFontFamily{\fontsize{8}{9.6}\selectfont \textup{\textup{\vbox{\hbox{\strut{}Long near-close front unrounded vowel}\hbox{\strut{}}}}}}}\\[1.5pt]%
\multicolumn{1}{@{}l}{\XLingPaperCharisZSILFontFamily{\fontsize{8}{9.6}\selectfont \textup{\textup{\vbox{\hbox{\strut{}U+0196}\hbox{\strut{}U+0269}}}}}}&\multicolumn{1}{l}{\XLingPaperCharisZSILFontFamily{\fontsize{8}{9.6}\selectfont \textup{\textup{\vbox{\hbox{\strut{}Ɩ}\hbox{\strut{}ɩ}}}}}}&\multicolumn{1}{c}{\XLingPaperCharisZSILFontFamily{\fontsize{8}{9.6}\selectfont \textup{\textup{\vbox{\hbox{\strut{}/e/}\hbox{\strut{}\textsquarebracketleft{}ɪ\textsquarebracketright{}}}}}}}&\multicolumn{1}{l@{}}{\XLingPaperCharisZSILFontFamily{\fontsize{8}{9.6}\selectfont \textup{\textup{\vbox{\hbox{\strut{}Short near-close front unrounded vowel}\hbox{\strut{}}}}}}}\\[1.5pt]%
\multicolumn{1}{@{}l}{\XLingPaperCharisZSILFontFamily{\fontsize{8}{9.6}\selectfont \textup{\textup{\vbox{\hbox{\strut{}U+01B2}\hbox{\strut{}U+028B}}}}}}&\multicolumn{1}{l}{\XLingPaperCharisZSILFontFamily{\fontsize{8}{9.6}\selectfont \textup{\textup{\vbox{\hbox{\strut{}Ʋ}\hbox{\strut{}ʋ}}}}}}&\multicolumn{1}{c}{\XLingPaperCharisZSILFontFamily{\fontsize{8}{9.6}\selectfont \textup{\textup{\vbox{\hbox{\strut{}/o/}\hbox{\strut{}\textsquarebracketleft{}ʊ\textsquarebracketright{}}}}}}}&\multicolumn{1}{l@{}}{\XLingPaperCharisZSILFontFamily{\fontsize{8}{9.6}\selectfont \textup{\textup{\vbox{\hbox{\strut{}Short near-close near-back rounded vowel}\hbox{\strut{}}}}}}}\\[1.5pt]%
\multicolumn{1}{@{}l}{\XLingPaperCharisZSILFontFamily{\fontsize{8}{9.6}\selectfont \textup{\textup{\vbox{\hbox{\strut{}U+01B2 U+028B}\hbox{\strut{}U+028B U+028B}}}}}}&\multicolumn{1}{l}{\XLingPaperCharisZSILFontFamily{\fontsize{8}{9.6}\selectfont \textup{\textup{\vbox{\hbox{\strut{}Ʋʋ}\hbox{\strut{}ʋʋ}}}}}}&\multicolumn{1}{c}{\XLingPaperCharisZSILFontFamily{\fontsize{8}{9.6}\selectfont \textup{\textup{\vbox{\hbox{\strut{}/oo/}\hbox{\strut{}\textsquarebracketleft{}ʊʊ\textsquarebracketright{}}}}}}}&\multicolumn{1}{l@{}}{\XLingPaperCharisZSILFontFamily{\fontsize{8}{9.6}\selectfont \textup{\textup{\vbox{\hbox{\strut{}Long near-close near-back rounded vowel}\hbox{\strut{}}}}}}}\\[1.5pt]%
\multicolumn{1}{@{}l}{\XLingPaperCharisZSILFontFamily{\fontsize{8}{9.6}\selectfont \textup{\textup{\vbox{\hbox{\strut{}U+01B2 U+0308}\hbox{\strut{}U+028B U+0308}}}}}}&\multicolumn{1}{l}{\XLingPaperCharisZSILFontFamily{\fontsize{8}{9.6}\selectfont \textup{\textup{\vbox{\hbox{\strut{}Ʋ̈}\hbox{\strut{}ʋ̈}}}}}}&\multicolumn{1}{c}{\XLingPaperCharisZSILFontFamily{\fontsize{8}{9.6}\selectfont \textup{\textup{\vbox{\hbox{\strut{}/ɤ/}\hbox{\strut{}\textsquarebracketleft{}ʊ̜\textsquarebracketright{} or \textsquarebracketleft{}ɯ̞̈\textsquarebracketright{}}}}}}}&\multicolumn{1}{l@{}}{\XLingPaperCharisZSILFontFamily{\fontsize{8}{9.6}\selectfont \textup{\textup{\vbox{\hbox{\strut{}Short near-close (near) back unrounded vowel}\hbox{\strut{}}}}}}}\\[1.5pt]%
\multicolumn{1}{@{}l}{\XLingPaperCharisZSILFontFamily{\fontsize{8}{9.6}\selectfont \textup{\textup{\vbox{\hbox{\strut{}U+01B2 U+0308 U+028B U+0308}\hbox{\strut{}U+028B U+0308 U+028B U+0308}}}}}}&\multicolumn{1}{l}{\XLingPaperCharisZSILFontFamily{\fontsize{8}{9.6}\selectfont \textup{\textup{\vbox{\hbox{\strut{}Ʋ̈ʋ̈}\hbox{\strut{}ʋ̈ʋ̈}}}}}}&\multicolumn{1}{c}{\XLingPaperCharisZSILFontFamily{\fontsize{8}{9.6}\selectfont \textup{\textup{\vbox{\hbox{\strut{}/ɤ/}\hbox{\strut{}\textsquarebracketleft{}ʊ̜ʊ̜\textsquarebracketright{} or \textsquarebracketleft{}ɯ̞̈ɯ̞̈\textsquarebracketright{}}}}}}}&\multicolumn{1}{l@{}}{\XLingPaperCharisZSILFontFamily{\fontsize{8}{9.6}\selectfont \textup{\textup{\vbox{\hbox{\strut{}Long near-close (near) back unrounded vowel}\hbox{\strut{}}}}}}}\\[1.5pt]\bottomrule%
\end{longtable}
}
}{\vspace{10pt}\XLingPaperneedspace{3\baselineskip}\noindent
\fontsize{13}{15.6}\selectfont \textit{{\noindent
\raisebox{\baselineskip}[0pt]{\pdfbookmark[3]{{3.3.2 } Suprasegmental}{sSupraSegmental}}\raisebox{\baselineskip}[0pt]{\protect\hypertarget{sSupraSegmental}{}}{3.3.2 }Suprasegmental}}\markboth{Suprasegmental}{Eastern Dan writing system}\XLingPaperaddtocontents{sSupraSegmental}}\par{}
\penalty10000\vspace{10pt}\penalty10000\indent Table \hyperlink{ntLanguagesWithTonePunctuation}{17} presents a list of languages with orthographies which are also claimed to use punctuation – in a similar manner to Eastern and Western Dan – to show tone patterns.\par{}\vspace{11pt plus 2pt minus 1pt}\XLingPaperneedspace{3\baselineskip}\protect\hypertarget{ntLanguagesWithTonePunctuation}{}\XLingPaperaddtocontents{ntLanguagesWithTonePunctuation}{\protect\raggedright{\singlespacing
{Table }{17.}{  \setcounter{footnote}{14}List of languages reported to use punctuation marks according to the Ivorian tradition\footnotemark{}.\\}}}\vspace{0pt}{\singlespacing
\hspace*{.25in}{\setcounter{footnote}{15}
\XLingPaperminmaxcellincolumn{Language}{\XLingPapermincola}{\textbf{Language Name}}{\XLingPapermaxcola}{+0\tabcolsep}
\XLingPaperminmaxcellincolumn{639-3}{\XLingPapermincolb}{\textbf{ISO 639-3 code}}{\XLingPapermaxcolb}{+0\tabcolsep}
\XLingPaperminmaxcellincolumn{resources}{\XLingPapermincolc}{\textbf{Source}}{\XLingPapermaxcolc}{+0\tabcolsep}
\XLingPaperminmaxcellincolumn{Attié}{\XLingPapermincola}{Attié\protect\footnotetext[16]{{\leftskip0pt\parindent1em\raisebox{\baselineskip}[0pt]{\protect\hypertarget{nSILID}{}} In the source column, the Archive ID for SIL related resources as is customary for archived materials. Sometimes SIL has a record for the production of a resource but does not have an electronic copy. In such cases I have indicated that I have not seen the cited resource and the fact that SIL does not have a digital resource with "– No PDF".}}}{\XLingPapermaxcola}{+0\tabcolsep}
\XLingPaperminmaxcellincolumn{ati}{\XLingPapermincolb}{ati}{\XLingPapermaxcolb}{+0\tabcolsep}
\XLingPaperminmaxcellincolumn{34612}{\XLingPapermincolc}{\hyperlink{rKutschLojenga1993Thewr}{Kutsch Lojenga (1993:13}, \hyperlink{rKutschLojenga1986}{1986)}, SIL: 34612}{\XLingPapermaxcolc}{+0\tabcolsep}
\XLingPaperminmaxcellincolumn{Bakwé}{\XLingPapermincola}{Bakwé}{\XLingPapermaxcola}{+0\tabcolsep}
\XLingPaperminmaxcellincolumn{bjw}{\XLingPapermincolb}{bjw}{\XLingPapermaxcolb}{+0\tabcolsep}
\XLingPaperminmaxcellincolumn{SIL:34687}{\XLingPapermincolc}{SIL:34687 – No PDF}{\XLingPapermaxcolc}{+0\tabcolsep}
\XLingPaperminmaxcellincolumn{Guiberoua}{\XLingPapermincola}{Bété Guiberoua}{\XLingPapermaxcola}{+0\tabcolsep}
\XLingPaperminmaxcellincolumn{bet}{\XLingPapermincolb}{bet}{\XLingPapermaxcolb}{+0\tabcolsep}
\XLingPaperminmaxcellincolumn{34603}{\XLingPapermincolc}{\hyperlink{rHartell1993}{Hartell (1993:132)}, SIL: 34603}{\XLingPapermaxcolc}{+0\tabcolsep}
\XLingPaperminmaxcellincolumn{Yocoboué}{\XLingPapermincola}{Dida Yocoboué}{\XLingPapermaxcola}{+0\tabcolsep}
\XLingPaperminmaxcellincolumn{gud}{\XLingPapermincolb}{gud}{\XLingPapermaxcolb}{+0\tabcolsep}
\XLingPaperminmaxcellincolumn{SIL:34639}{\XLingPapermincolc}{SIL:34639}{\XLingPapermaxcolc}{+0\tabcolsep}
\XLingPaperminmaxcellincolumn{Godié}{\XLingPapermincola}{Godié}{\XLingPapermaxcola}{+0\tabcolsep}
\XLingPaperminmaxcellincolumn{god}{\XLingPapermincolb}{god}{\XLingPapermaxcolb}{+0\tabcolsep}
\XLingPaperminmaxcellincolumn{34551}{\XLingPapermincolc}{\hyperlink{rHartell1993}{Hartell (1993:137)}, SIL: 34551 – No PDF}{\XLingPapermaxcolc}{+0\tabcolsep}
\XLingPaperminmaxcellincolumn{Kouya}{\XLingPapermincola}{Kouya}{\XLingPapermaxcola}{+0\tabcolsep}
\XLingPaperminmaxcellincolumn{kyf}{\XLingPapermincolb}{kyf}{\XLingPapermaxcolb}{+0\tabcolsep}
\XLingPaperminmaxcellincolumn{34590}{\XLingPapermincolc}{SIL: 34590 – No PDF}{\XLingPapermaxcolc}{+0\tabcolsep}
\XLingPaperminmaxcellincolumn{Kroumen}{\XLingPapermincola}{Tépo Kroumen}{\XLingPapermaxcola}{+0\tabcolsep}
\XLingPaperminmaxcellincolumn{ted}{\XLingPapermincolb}{ted}{\XLingPapermaxcolb}{+0\tabcolsep}
\XLingPaperminmaxcellincolumn{57750}{\XLingPapermincolc}{\hyperlink{rHartell1993}{Hartell (1993:138)}, SIL: 57750}{\XLingPapermaxcolc}{+0\tabcolsep}
\XLingPaperminmaxcellincolumn{Mwan}{\XLingPapermincola}{Mwan}{\XLingPapermaxcola}{+0\tabcolsep}
\XLingPaperminmaxcellincolumn{moa}{\XLingPapermincolb}{moa}{\XLingPapermaxcolb}{+0\tabcolsep}
\XLingPaperminmaxcellincolumn{34606}{\XLingPapermincolc}{\hyperlink{rHartell1993}{Hartell (1993:139)}, SIL: 34606 – No PDF}{\XLingPapermaxcolc}{+0\tabcolsep}
\XLingPaperminmaxcellincolumn{Néyo}{\XLingPapermincola}{Néyo}{\XLingPapermaxcola}{+0\tabcolsep}
\XLingPaperminmaxcellincolumn{ney}{\XLingPapermincolb}{ney}{\XLingPapermaxcolb}{+0\tabcolsep}
\XLingPaperminmaxcellincolumn{SIL:34615}{\XLingPapermincolc}{SIL:34615}{\XLingPapermaxcolc}{+0\tabcolsep}
\XLingPaperminmaxcellincolumn{(Nyabwa)}{\XLingPapermincola}{Nyaboua (Nyabwa)}{\XLingPapermaxcola}{+0\tabcolsep}
\XLingPaperminmaxcellincolumn{nwb}{\XLingPapermincolb}{nwb}{\XLingPapermaxcolb}{+0\tabcolsep}
\XLingPaperminmaxcellincolumn{34540}{\XLingPapermincolc}{\hyperlink{rHartell1993}{Hartell (1993:140)}, SIL: 34540}{\XLingPapermaxcolc}{+0\tabcolsep}
\XLingPaperminmaxcellincolumn{Toura}{\XLingPapermincola}{Toura}{\XLingPapermaxcola}{+0\tabcolsep}
\XLingPaperminmaxcellincolumn{neb}{\XLingPapermincolb}{neb}{\XLingPapermaxcolb}{+0\tabcolsep}
\XLingPaperminmaxcellincolumn{revision}{\XLingPapermincolc}{\hyperlink{rHartell1993}{Hartell (1993:143)}}{\XLingPapermaxcolc}{+0\tabcolsep}
\XLingPaperminmaxcellincolumn{Wan}{\XLingPapermincola}{Wan}{\XLingPapermaxcola}{+0\tabcolsep}
\XLingPaperminmaxcellincolumn{wan}{\XLingPapermincolb}{wan}{\XLingPapermaxcolb}{+0\tabcolsep}
\XLingPaperminmaxcellincolumn{}{\XLingPapermincolc}{\hyperlink{rHartell1993}{Hartell (1993:144)}}{\XLingPapermaxcolc}{+0\tabcolsep}
\XLingPaperminmaxcellincolumn{Northern}{\XLingPapermincola}{Northern Wè}{\XLingPapermaxcola}{+0\tabcolsep}
\XLingPaperminmaxcellincolumn{wob}{\XLingPapermincolb}{wob}{\XLingPapermaxcolb}{+0\tabcolsep}
\XLingPaperminmaxcellincolumn{34533,}{\XLingPapermincolc}{\hyperlink{rHartell1993}{Hartell (1993:145)}, SIL: 34533, SIL: 34694}{\XLingPapermaxcolc}{+0\tabcolsep}
\XLingPaperminmaxcellincolumn{Northern}{\XLingPapermincola}{Northern Grebo}{\XLingPapermaxcola}{+0\tabcolsep}
\XLingPaperminmaxcellincolumn{gbo}{\XLingPapermincolb}{gbo}{\XLingPapermaxcolb}{+0\tabcolsep}
\XLingPaperminmaxcellincolumn{finish}{\XLingPapermincolc}{({\hyperlink{vpc}{{p.c.}}})}{\XLingPapermaxcolc}{+0\tabcolsep}
\XLingPaperminmaxcellincolumn{Yaouré}{\XLingPapermincola}{Yaouré}{\XLingPapermaxcola}{+0\tabcolsep}
\XLingPaperminmaxcellincolumn{yre}{\XLingPapermincolb}{yre}{\XLingPapermaxcolb}{+0\tabcolsep}
\XLingPaperminmaxcellincolumn{}{\XLingPapermincolc}{\hyperlink{rHartell1993}{Hartell (1993:146)}}{\XLingPapermaxcolc}{+0\tabcolsep}
\XLingPaperminmaxcellincolumn{Southern}{\XLingPapermincola}{Southern Wè}{\XLingPapermaxcola}{+0\tabcolsep}
\XLingPaperminmaxcellincolumn{gxx}{\XLingPapermincolb}{gxx}{\XLingPapermaxcolb}{+0\tabcolsep}
\XLingPaperminmaxcellincolumn{records.}{\XLingPapermincolc}{JW Bible, but no other records.}{\XLingPapermaxcolc}{+0\tabcolsep}
\setlength{\XLingPaperavailabletablewidth}{433.62pt}
\setlength{\XLingPapertableminwidth}{\XLingPapermincola+\XLingPapermincolb+\XLingPapermincolc}
\setlength{\XLingPapertablemaxwidth}{\XLingPapermaxcola+\XLingPapermaxcolb+\XLingPapermaxcolc}
\XLingPapercalculatetablewidthratio{}
\XLingPapersetcolumnwidth{\XLingPapercolawidth}{\XLingPapermincola}{\XLingPapermaxcola}{-0\tabcolsep}
\XLingPapersetcolumnwidth{\XLingPapercolbwidth}{\XLingPapermincolb}{\XLingPapermaxcolb}{-2\tabcolsep}
\XLingPapersetcolumnwidth{\XLingPapercolcwidth}{\XLingPapermincolc}{\XLingPapermaxcolc}{-2\tabcolsep}\setcounter{footnote}{15}\singlespacing\vspace*{-3\baselineskip}
\begin{longtable}
[l]{@{}p{\XLingPapercolawidth}p{\XLingPapercolbwidth}p{\XLingPapercolcwidth}@{}}\toprule\multicolumn{1}{@{}p{\XLingPapercolawidth}}{\textbf{Language Name}}&\multicolumn{1}{p{\XLingPapercolbwidth}}{\textbf{ISO 639-3 code}}&\multicolumn{1}{p{\XLingPapercolcwidth}@{}}{\textbf{Source\protect\footnote{{\leftskip0pt\parindent1em\raisebox{\baselineskip}[0pt]{\protect\hypertarget{nSILID}{}} In the source column, the Archive ID for SIL related resources as is customary for archived materials. Sometimes SIL has a record for the production of a resource but does not have an electronic copy. In such cases I have indicated that I have not seen the cited resource and the fact that SIL does not have a digital resource with "– No PDF".}}}}\\%
\midrule\endhead \multicolumn{1}{@{}p{\XLingPapercolawidth}}{\protect\footnotetext[15]{{\leftskip0pt\parindent1em\raisebox{\baselineskip}[0pt]{\protect\hypertarget{nIvorianTonePunctuation}{}} Other orthographies have chosen to use punctuation marks to indicate tonal phenomena, albeit with a different sort of pattern. For instance \hyperlink{rDuitsmanJohn1981APlus}{Duitsman (1981)} writes of using a {\XLingPaperCambriaZMathFontFamily{\textup{\textmd{⟨ {\XLingPaperCharisZSILFontFamily{\textup{\textup{\textmd{+}}}}} ⟩}}}} for plurals in Western Krahn {\textsquarebracketleft{}krw\textsquarebracketright{}}. Some Nigerian languages have also picked up this practice (p.c. Matthew Harley). Andreas Joswig (p.c. 3 August 2018) indicates that Tsamakko {\textsquarebracketleft{}tsb\textsquarebracketright{}} is experimenting with the use of tilde {\XLingPaperCambriaZMathFontFamily{\textup{\textmd{⟨ {\XLingPaperCharisZSILFontFamily{\textup{\textup{\textmd{\textasciitilde{}}}}}} ⟩}}}} to mark a {\hyperlink{vTAM}{{TAM}}} feature indicated via tone. While Lizzie Poole (p.c. 3 August 2018) indicates that Mbugwe {\textsquarebracketleft{}mgz\textsquarebracketright{}} a language of Tanzania, is experimenting with using a stand alone 'CIRCUMFLEX ACCENT' {\XLingPaperCambriaZMathFontFamily{\textup{\textmd{⟨ {\XLingPaperCharisZSILFontFamily{\textup{\textup{\textmd{\^{}}}}}} ⟩}}}} for verbal morpheme indicated by tone. Some languages use only a prefixing strategy, while others like Attié {\textsquarebracketleft{}ati\textsquarebracketright{}} use a prefixing and suffixing strategy. In general with the arrival of computers, punctuational indication of tone becomes more complex because Unicode has various characters which look alike but behave differently to accommodate these orthographies. \hyperlink{rLFCpke}{Lüpke (2011)} describes the and finally in 2018 SIL International issued some corporate recommendations concerning the encoding methods \hyperlink{rSIL2018non-alphabetic}{(SIL  2018)} used to represent tone.}}Attié\protect\footnotetext[16]{{\leftskip0pt\parindent1em\raisebox{\baselineskip}[0pt]{\protect\hypertarget{nSILID}{}} In the source column, the Archive ID for SIL related resources as is customary for archived materials. Sometimes SIL has a record for the production of a resource but does not have an electronic copy. In such cases I have indicated that I have not seen the cited resource and the fact that SIL does not have a digital resource with "– No PDF".}}}&\multicolumn{1}{p{\XLingPapercolbwidth}}{ati}&\multicolumn{1}{p{\XLingPapercolcwidth}@{}}{\hyperlink{rKutschLojenga1993Thewr}{Kutsch Lojenga (1993:13}, \hyperlink{rKutschLojenga1986}{1986)}, SIL: 34612}\\%
\multicolumn{1}{@{}p{\XLingPapercolawidth}}{Bakwé}&\multicolumn{1}{p{\XLingPapercolbwidth}}{bjw}&\multicolumn{1}{p{\XLingPapercolcwidth}@{}}{SIL:34687 – No PDF}\\%
\multicolumn{1}{@{}p{\XLingPapercolawidth}}{Bété Guiberoua}&\multicolumn{1}{p{\XLingPapercolbwidth}}{bet}&\multicolumn{1}{p{\XLingPapercolcwidth}@{}}{\hyperlink{rHartell1993}{Hartell (1993:132)}, SIL: 34603}\\%
\multicolumn{1}{@{}p{\XLingPapercolawidth}}{Dida Yocoboué}&\multicolumn{1}{p{\XLingPapercolbwidth}}{gud}&\multicolumn{1}{p{\XLingPapercolcwidth}@{}}{SIL:34639}\\%
\multicolumn{1}{@{}p{\XLingPapercolawidth}}{Godié}&\multicolumn{1}{p{\XLingPapercolbwidth}}{god}&\multicolumn{1}{p{\XLingPapercolcwidth}@{}}{\hyperlink{rHartell1993}{Hartell (1993:137)}, SIL: 34551 – No PDF}\\%
\multicolumn{1}{@{}p{\XLingPapercolawidth}}{Kouya}&\multicolumn{1}{p{\XLingPapercolbwidth}}{kyf}&\multicolumn{1}{p{\XLingPapercolcwidth}@{}}{SIL: 34590 – No PDF}\\%
\multicolumn{1}{@{}p{\XLingPapercolawidth}}{Tépo Kroumen}&\multicolumn{1}{p{\XLingPapercolbwidth}}{ted}&\multicolumn{1}{p{\XLingPapercolcwidth}@{}}{\hyperlink{rHartell1993}{Hartell (1993:138)}, SIL: 57750}\\%
\multicolumn{1}{@{}p{\XLingPapercolawidth}}{Mwan}&\multicolumn{1}{p{\XLingPapercolbwidth}}{moa}&\multicolumn{1}{p{\XLingPapercolcwidth}@{}}{\hyperlink{rHartell1993}{Hartell (1993:139)}, SIL: 34606 – No PDF}\\%
\multicolumn{1}{@{}p{\XLingPapercolawidth}}{Néyo}&\multicolumn{1}{p{\XLingPapercolbwidth}}{ney}&\multicolumn{1}{p{\XLingPapercolcwidth}@{}}{SIL:34615}\\%
\multicolumn{1}{@{}p{\XLingPapercolawidth}}{Nyaboua (Nyabwa)}&\multicolumn{1}{p{\XLingPapercolbwidth}}{nwb}&\multicolumn{1}{p{\XLingPapercolcwidth}@{}}{\hyperlink{rHartell1993}{Hartell (1993:140)}, SIL: 34540}\\%
\multicolumn{1}{@{}p{\XLingPapercolawidth}}{Toura}&\multicolumn{1}{p{\XLingPapercolbwidth}}{neb}&\multicolumn{1}{p{\XLingPapercolcwidth}@{}}{\hyperlink{rHartell1993}{Hartell (1993:143)}\protect\footnote{{\leftskip0pt\parindent1em\raisebox{\baselineskip}[0pt]{\protect\hypertarget{nToura}{}} It appears that a writing system revision around 1990 resulted in the removal of punctuation marks indicating tone in favor of marks over the vowels. See SIL: 34567}}}\\%
\multicolumn{1}{@{}p{\XLingPapercolawidth}}{Wan}&\multicolumn{1}{p{\XLingPapercolbwidth}}{wan}&\multicolumn{1}{p{\XLingPapercolcwidth}@{}}{\hyperlink{rHartell1993}{Hartell (1993:144)}}\\%
\multicolumn{1}{@{}p{\XLingPapercolawidth}}{Northern Wè}&\multicolumn{1}{p{\XLingPapercolbwidth}}{wob}&\multicolumn{1}{p{\XLingPapercolcwidth}@{}}{\hyperlink{rHartell1993}{Hartell (1993:145)}, SIL: 34533, SIL: 34694}\\%
\multicolumn{1}{@{}p{\XLingPapercolawidth}}{Northern Grebo}&\multicolumn{1}{p{\XLingPapercolbwidth}}{gbo}&\multicolumn{1}{p{\XLingPapercolcwidth}@{}}{({\hyperlink{vpc}{{p.c.}}})}\\%
\multicolumn{1}{@{}p{\XLingPapercolawidth}}{Yaouré}&\multicolumn{1}{p{\XLingPapercolbwidth}}{yre}&\multicolumn{1}{p{\XLingPapercolcwidth}@{}}{\hyperlink{rHartell1993}{Hartell (1993:146)}}\\%
\multicolumn{1}{@{}p{\XLingPapercolawidth}}{Southern Wè}&\multicolumn{1}{p{\XLingPapercolbwidth}}{gxx}&\multicolumn{1}{p{\XLingPapercolcwidth}@{}}{JW Bible, but no other records.}\\\bottomrule%
\end{longtable}
}
}\vspace{11pt plus 2pt minus 1pt}\setbox0=\vbox{\protect\centering \leavevmode
\vspace*{0pt}{\XeTeXpicfile "../Resources/Dan-Mano-Tura.png" scaled 200}\\[0pt]\protect\hypertarget{fOrthographyMap}{}\XLingPaperaddtocontents{fOrthographyMap}{\singlespacing
{Figure }{19.}{ \setcounter{footnote}{17}A map of the languages near Dan reported to be using similar orthographic devices for tone. \footnotemark{}\\}}\protect\footnotetext[18]{{\leftskip0pt\parindent1em\raisebox{\baselineskip}[0pt]{\protect\hypertarget{nOrthographyMapCitation}{}} Originally published in \hyperlink{rVydrine2003Map}{Vydrine et al. (2003)}}}}\box0\protect\footnotetext[18]{{\leftskip0pt\parindent1em\raisebox{\baselineskip}[0pt]{\protect\hypertarget{nOrthographyMapCitation}{}} Originally published in \hyperlink{rVydrine2003Map}{Vydrine et al. (2003)}}}\par{}\vspace{11pt plus 2pt minus 1pt}{\vspace{15pt}\XLingPaperneedspace{3\baselineskip}\noindent
\fontsize{13}{15.6}\selectfont \textbf{{\noindent
\raisebox{\baselineskip}[0pt]{\pdfbookmark[2]{{3.4 } Units and orders}{sUnitsAndOrders}}\raisebox{\baselineskip}[0pt]{\protect\hypertarget{sUnitsAndOrders}{}}{3.4 }Units and orders}}\markboth{Units and orders}{Eastern Dan writing system}\XLingPaperaddtocontents{sUnitsAndOrders}}\par{}
\penalty10000\vspace{10pt}\penalty10000{\vspace{10pt}\XLingPaperneedspace{3\baselineskip}\noindent
\fontsize{13}{15.6}\selectfont \textit{{\noindent
\raisebox{\baselineskip}[0pt]{\pdfbookmark[3]{{3.4.1 } Alphabet}{sAlphabet}}\raisebox{\baselineskip}[0pt]{\protect\hypertarget{sAlphabet}{}}{3.4.1 }Alphabet}}\markboth{Alphabet}{Eastern Dan writing system}\XLingPaperaddtocontents{sAlphabet}}\par{}
\penalty10000\vspace{10pt}\penalty10000\indent Neither the \hyperlink{rBolli1994Cours}{1994} reading primer nor the \hyperlink{rBF1982}{1982} reading primer present or address the issue of an \hyperlink{gtAlphabet}{{\textit{alphabet}}}, or alphabetical ordering\protect\footnote[19]{{\leftskip0pt\parindent1em\raisebox{\baselineskip}[0pt]{\protect\hypertarget{nAlphabet1982}{}} A Western Dan reading primer \hyperlink{rBF1982WDEDICEF}{(Bolli et al.  1982)} does present an alphabet which includes a separate element for each nasalized vowel. The presentation in \hyperlink{rBF1982WDEDICEF}{(Bolli et al.  1982)} looks much more like a list of functional units units, but is labeled as an "Alphabet".}}. Both resources present their audiences with a list of pedagogical learning units which match well with the phonemes of Eastern Dan (with a few exceptions). The lists in the primers present \hyperlink{gtFuntionalUnit}{{\textit{functional units}}} (a term I borrow from \hyperlink{Venezky1967}{Venezky  1967}, \hyperlink{Venezky1970}{1970}, and \hyperlink{Holm1971}{Holm  1971}) rather than \hyperlink{gtLetter}{{\textit{letters}}}, ordered and grouped by place of articulation (phonetic detail). The dictionary by \hyperlink{rVydrin2008EDDictionary}{Vydrine \& Kességbeu  (2008:10, 366)} makes a combined presentation of consonants and vowels in an ordering similar to many European language orderings. However, it too mixes letters and functional units, and leaves functional units for vowels with nasalization undressed. Therefore, according to the information which is available, it would appear that no alphabet statement has been made for Eastern Dan.\par{}\indent There are two possible assumptions about the lists presented in the dictionary: (1) they represent a reasonable working hypothesis and practical order upon which a language's first dictionary was based, (2) they represent a widely held community consensus. To be reasonably treated as an alphabet, I find that the order is overly complicated, with the use of digraphs and at the same time under representative in the cases of nasality and tonal patterns. As the list is presented, there is bias towards oral-segmental phonemes. If we were to assume that the "alphabet" presented in the dictionary is really just an ordered list of functional units then, we would find that the dictionary list does not include all of the functional units which the primers list. The ‘dictionnaire’ list is shorter. The challenge with accepting the declarations in the {\textit{dictionnaire}} as an alphabet is a theoretical one: they contain multiple base-character digraphs like {\XLingPaperCambriaZMathFontFamily{\textup{\textmd{⟨ {\XLingPaperCharisZSILFontFamily{\textup{\textup{\textmd{bh}}}}} ⟩}}}} and {\XLingPaperCambriaZMathFontFamily{\textup{\textmd{⟨ {\XLingPaperCharisZSILFontFamily{\textup{\textup{\textmd{gw}}}}} ⟩}}}}. While these sets of characters do represent a single phoneme linguistically, it is more challenging to see them as a single “letter”, especially when their components are also considered single “letters” in the same alphabet. That is, there is a logical contradiction between allowing components and their additive derivatives to both be present in a list of “letters”. That said, a letter list should be possible, and relevant to this section, though any ordering presented here would only be for practical reasons, and is not intended to be prescriptive. In table \hyperlink{tEasternDanCharacters}{18} I present diacritics as a component of the letters on which they occur. I do this because the diacritics themselves do not have a consistent meaning in the orthography. Table \hyperlink{tEasternDanCharacters}{18} does not include functional units, only the letters from which functional units can be created. This is true for vowels, tone patterns, and double articulated consonants. Based on the letters presented in the 1994 primer the following letters would need to be in an alphabet. This list includes 36 letters; 32 with casing pairs for a total of 68 alphabetic glyphs. A list of functional units is presented in table \hyperlink{ntFuntionalUnitsList}{20}.\par{}\vspace{11pt plus 2pt minus 1pt}\XLingPaperneedspace{3\baselineskip}\protect\hypertarget{tEasternDanCharacters}{}\XLingPaperaddtocontents{tEasternDanCharacters}{\protect\raggedright{\singlespacing
{Table }{18.}{  List of letters used in Eastern Dan\\}}}\vspace{0pt}{\singlespacing
\hspace*{.25in}{\setcounter{footnote}{19}\singlespacing\vspace*{-3\baselineskip}
\begin{longtable}
[l]{@{}ccccp{2.5in}@{}}\toprule\multicolumn{1}{@{}c}{\textbf{Uppercase}}&\multicolumn{1}{c}{\textbf{Lowercase}}&\multicolumn{2}{c}{\multirow{2}{*}[1.25ex]{\textbf{Glyph}}}&\multicolumn{1}{p{2.5in}@{}}{\raggedright\textbf{Approximate Unicode Name\protect\footnote{{\leftskip0pt\parindent1em\raisebox{\baselineskip}[0pt]{\protect\hypertarget{nUnicodeName}{}} Full Unicode names contain 'CAPITAL' or 'SMALL'.}}}}\\\multicolumn{1}{@{}c}{\textbf{NFD Encoding}}&\multicolumn{1}{c}{\textbf{NFD Encoding}}&&&\multicolumn{1}{p{2.5in}@{}}{\raggedright\textbf{}}\\\midrule\endhead \multicolumn{1}{@{}l}{U+0041}&\multicolumn{1}{l}{U+0061}&\multicolumn{1}{l}{A}&\multicolumn{1}{l}{a}&\multicolumn{1}{p{2in}@{}}{\raggedright LATIN LETTER A}\\%
\multicolumn{1}{@{}l}{U+0042}&\multicolumn{1}{l}{U+0062}&\multicolumn{1}{l}{B}&\multicolumn{1}{l}{b}&\multicolumn{1}{p{2in}@{}}{\raggedright LATIN LETTER B}\\%
\multicolumn{1}{@{}l}{U+0044}&\multicolumn{1}{l}{U+0064}&\multicolumn{1}{l}{D}&\multicolumn{1}{l}{d}&\multicolumn{1}{p{2in}@{}}{\raggedright LATIN LETTER D}\\%
\multicolumn{1}{@{}l}{U+0045}&\multicolumn{1}{l}{U+0065}&\multicolumn{1}{l}{E}&\multicolumn{1}{l}{e}&\multicolumn{1}{p{2in}@{}}{\raggedright LATIN LETTER E}\\%
\multicolumn{1}{@{}l}{U+0045 U+0308}&\multicolumn{1}{l}{U+0065 U+0308}&\multicolumn{1}{l}{Ë}&\multicolumn{1}{l}{ë}&\multicolumn{1}{p{2in}@{}}{\raggedright LATIN LETTER E with COMBINING DIAERESIS}\\%
\multicolumn{1}{@{}l}{U+0046}&\multicolumn{1}{l}{U+0066}&\multicolumn{1}{l}{F}&\multicolumn{1}{l}{f}&\multicolumn{1}{l@{}}{LATIN LETTER F}\\%
\multicolumn{1}{@{}l}{U+0047}&\multicolumn{1}{l}{U+0067}&\multicolumn{1}{l}{G}&\multicolumn{1}{l}{g}&\multicolumn{1}{l@{}}{LATIN LETTER G}\\%
\multicolumn{1}{@{}l}{U+0048}&\multicolumn{1}{l}{U+0068}&\multicolumn{1}{l}{H}&\multicolumn{1}{l}{h}&\multicolumn{1}{l@{}}{LATIN LETTER H}\\%
\multicolumn{1}{@{}l}{U+0049}&\multicolumn{1}{l}{U+0069}&\multicolumn{1}{l}{I}&\multicolumn{1}{l}{i}&\multicolumn{1}{l@{}}{LATIN LETTER I}\\%
\multicolumn{1}{@{}l}{U+004B}&\multicolumn{1}{l}{U+006B}&\multicolumn{1}{l}{K}&\multicolumn{1}{l}{k}&\multicolumn{1}{l@{}}{LATIN LETTER K}\\%
\multicolumn{1}{@{}l}{U+004C}&\multicolumn{1}{l}{U+006C}&\multicolumn{1}{l}{L}&\multicolumn{1}{l}{l}&\multicolumn{1}{l@{}}{LATIN LETTER L}\\%
\multicolumn{1}{@{}l}{U+004D}&\multicolumn{1}{l}{U+006D}&\multicolumn{1}{l}{M}&\multicolumn{1}{l}{m}&\multicolumn{1}{l@{}}{LATIN LETTER M}\\%
\multicolumn{1}{@{}l}{U+004E}&\multicolumn{1}{l}{U+006E}&\multicolumn{1}{l}{N}&\multicolumn{1}{l}{n}&\multicolumn{1}{l@{}}{LATIN LETTER N}\\%
\multicolumn{1}{@{}l}{U+004F}&\multicolumn{1}{l}{U+006F}&\multicolumn{1}{l}{O}&\multicolumn{1}{l}{o}&\multicolumn{1}{l@{}}{LATIN LETTER O}\\%
\multicolumn{1}{@{}l}{U+004F U+0308}&\multicolumn{1}{l}{U+006F U+0308}&\multicolumn{1}{l}{Ö}&\multicolumn{1}{l}{ö}&\multicolumn{1}{p{2in}@{}}{\raggedright LATIN LETTER O with COMBINING DIAERESIS}\\%
\multicolumn{1}{@{}l}{U+0050}&\multicolumn{1}{l}{U+0070}&\multicolumn{1}{l}{P}&\multicolumn{1}{l}{p}&\multicolumn{1}{l@{}}{LATIN LETTER P}\\%
\multicolumn{1}{@{}l}{U+0052}&\multicolumn{1}{l}{U+0072}&\multicolumn{1}{l}{R}&\multicolumn{1}{l}{r}&\multicolumn{1}{l@{}}{LATIN LETTER R}\\%
\multicolumn{1}{@{}l}{U+0053}&\multicolumn{1}{l}{U+0073}&\multicolumn{1}{l}{S}&\multicolumn{1}{l}{s}&\multicolumn{1}{l@{}}{LATIN LETTER S}\\%
\multicolumn{1}{@{}l}{U+0054}&\multicolumn{1}{l}{U+0074}&\multicolumn{1}{l}{T}&\multicolumn{1}{l}{t}&\multicolumn{1}{l@{}}{LATIN LETTER T}\\%
\multicolumn{1}{@{}l}{U+0055}&\multicolumn{1}{l}{U+0075}&\multicolumn{1}{l}{U}&\multicolumn{1}{l}{u}&\multicolumn{1}{l@{}}{LATIN LETTER U}\\%
\multicolumn{1}{@{}l}{U+0055 U+0308}&\multicolumn{1}{l}{U+0075 U+0308}&\multicolumn{1}{l}{Ü}&\multicolumn{1}{l}{ü}&\multicolumn{1}{p{2in}@{}}{\raggedright LATIN LETTER U with COMBINING DIAERESIS}\\%
\multicolumn{1}{@{}l}{U+0056}&\multicolumn{1}{l}{U+0076}&\multicolumn{1}{l}{V}&\multicolumn{1}{l}{v}&\multicolumn{1}{l@{}}{LATIN LETTER V}\\%
\multicolumn{1}{@{}l}{U+0057}&\multicolumn{1}{l}{U+0077}&\multicolumn{1}{l}{W}&\multicolumn{1}{l}{w}&\multicolumn{1}{l@{}}{LATIN LETTER W}\\%
\multicolumn{1}{@{}l}{U+0059}&\multicolumn{1}{l}{U+0079}&\multicolumn{1}{l}{Y}&\multicolumn{1}{l}{y}&\multicolumn{1}{l@{}}{LATIN LETTER Y}\\%
\multicolumn{1}{@{}l}{U+005A}&\multicolumn{1}{l}{U+007A}&\multicolumn{1}{l}{Z}&\multicolumn{1}{l}{z}&\multicolumn{1}{l@{}}{LATIN LETTER Z}\\%
\multicolumn{1}{@{}l}{U+0186}&\multicolumn{1}{l}{U+0254}&\multicolumn{1}{l}{Ɔ}&\multicolumn{1}{l}{ɔ}&\multicolumn{1}{l@{}}{LATIN LETTER OPEN O}\\%
\multicolumn{1}{@{}l}{U+0190}&\multicolumn{1}{l}{U+025B}&\multicolumn{1}{l}{Ɛ}&\multicolumn{1}{l}{ɛ}&\multicolumn{1}{l@{}}{LATIN LETTER OPEN E}\\%
\multicolumn{1}{@{}l}{U+0196}&\multicolumn{1}{l}{U+0269}&\multicolumn{1}{l}{Ɩ}&\multicolumn{1}{l}{ɩ}&\multicolumn{1}{l@{}}{LATIN LETTER IOTA}\\%
\multicolumn{1}{@{}l}{U+01B2}&\multicolumn{1}{l}{U+028B}&\multicolumn{1}{l}{Ʋ}&\multicolumn{1}{l}{ʋ}&\multicolumn{1}{l@{}}{LATIN LETTER V WITH HOOK}\\%
\multicolumn{1}{@{}l}{U+01B2 U+0308}&\multicolumn{1}{l}{U+028B U+0308}&\multicolumn{1}{l}{Ʋ̈}&\multicolumn{1}{l}{ʋ̈}&\multicolumn{1}{p{2in}@{}}{\raggedright LATIN LETTER V WITH HOOK with COMBINING DIAERESIS}\\%
\multicolumn{1}{@{}l}{N/a}&\multicolumn{1}{l}{U+02BC}&\multicolumn{1}{l}{ʼ}&\multicolumn{1}{l}{}&\multicolumn{1}{l@{}}{MODIFIER LETTER APOSTROPHE}\\%
\multicolumn{1}{@{}l}{N/a}&\multicolumn{1}{l}{U+02D7}&\multicolumn{1}{l}{˗}&\multicolumn{1}{l}{}&\multicolumn{1}{l@{}}{MODIFIER LETTER MINUS SIGN}\\%
\multicolumn{1}{@{}l}{N/a}&\multicolumn{1}{l}{U+02EE}&\multicolumn{1}{l}{ˮ}&\multicolumn{1}{l}{}&\multicolumn{1}{p{2in}@{}}{\raggedright MODIFIER LETTER DOUBLE APOSTROPHE}\\%
\multicolumn{1}{@{}l}{N/a}&\multicolumn{1}{l}{U+A78A}&\multicolumn{1}{l}{꞊}&\multicolumn{1}{l}{}&\multicolumn{1}{p{2in}@{}}{\raggedright MODIFIER LETTER SHORT EQUALS SIGN}\\\bottomrule%
\end{longtable}
}
}\indent While it is not important from a graphical-to-sound correspondence, the encoding of characters at a technical level is very important for software to be fully supportive of language based features. e.g. spell checking. Table \hyperlink{tEasternDanCharactersdiacritics}{19} includes characters which may have more than one representation. Many people assume that there is only one correct representation. According to Unicode both representations (NFC and NFD) should be treated equally as correct. However, it is up to the software vendor to implement canonical equivalence.\par{}\vspace{11pt plus 2pt minus 1pt}\XLingPaperneedspace{3\baselineskip}\protect\hypertarget{tEasternDanCharactersdiacritics}{}\XLingPaperaddtocontents{tEasternDanCharactersdiacritics}{\protect\raggedright{\singlespacing
{Table }{19.}{  List of characters whose composition includes diacritics\\}}}\vspace{0pt}{\singlespacing
\hspace*{.25in}{\setcounter{footnote}{20}\singlespacing\vspace*{-3\baselineskip}
\begin{longtable}
[l]{@{}ccccp{2.5in}@{}}\toprule\multicolumn{1}{@{}c}{\textbf{Uppercase}}&\multicolumn{1}{c}{\textbf{Lowercase}}&\multicolumn{2}{c}{\multirow{2}{*}[1.25ex]{\textbf{Glyph}}}&\multicolumn{1}{p{2.5in}@{}}{\raggedright\textbf{Approximate Unicode Name\footnote{{See footnote }\hyperlink{nUnicodeName}{20} in chapter 3.}}}\\\multicolumn{1}{@{}c}{\textbf{NFC Encoding}}&\multicolumn{1}{c}{\textbf{NFC Encoding}}&&&\multicolumn{1}{p{2.5in}@{}}{\raggedright\textbf{}}\\\midrule\endhead \multicolumn{1}{@{}l}{U+00CB}&\multicolumn{1}{l}{U+00EB}&\multicolumn{1}{l}{Ë}&\multicolumn{1}{l}{ë}&\multicolumn{1}{l@{}}{LATIN LETTER E WITH DIAERESIS}\\%
\multicolumn{1}{@{}l}{U+00D6}&\multicolumn{1}{l}{U+00F6}&\multicolumn{1}{l}{Ö}&\multicolumn{1}{l}{ö}&\multicolumn{1}{l@{}}{LATIN LETTER O WITH DIAERESIS}\\%
\multicolumn{1}{@{}l}{U+00DC}&\multicolumn{1}{l}{U+00FC}&\multicolumn{1}{l}{Ü}&\multicolumn{1}{l}{ü}&\multicolumn{1}{l@{}}{LATIN LETTER U WITH DIAERESIS}\\%
\multicolumn{1}{@{}l}{None\protect\footnote{{\leftskip0pt\parindent1em\raisebox{\baselineskip}[0pt]{\protect\hypertarget{nNoNFC}{}} There is No NFC encoding for this character. It must be encoded as NFD.}}}&\multicolumn{1}{l}{None}&\multicolumn{1}{l}{Ʋ̈}&\multicolumn{1}{l}{ʋ̈}&\multicolumn{1}{p{2in}@{}}{\raggedright LATIN LETTER V WITH HOOK with COMBINING DIAERESIS (NFD and NFC are the same as this combination can only be generated in a composed form)}\\\bottomrule%
\end{longtable}
}
}{\vspace{10pt}\XLingPaperneedspace{3\baselineskip}\noindent
\fontsize{13}{15.6}\selectfont \textit{{\noindent
\raisebox{\baselineskip}[0pt]{\pdfbookmark[3]{{3.4.2 } Functional units}{sFunctional-Units}}\raisebox{\baselineskip}[0pt]{\protect\hypertarget{sFunctional-Units}{}}{3.4.2 }Functional units}}\markboth{Functional units}{Eastern Dan writing system}\XLingPaperaddtocontents{sFunctional-Units}}\par{}
\penalty10000\vspace{10pt}\penalty10000\indent Given the data available, table \hyperlink{ntFuntionalUnitsList}{20} lists the known functional units for Eastern Dan. Functional units which refer to the segmental tier phonemes are unmarked for type. Diphthongs are labeled with a {\XLingPaperCambriaZMathFontFamily{\textup{\textmd{⟨ {\XLingPaperCharisZSILFontFamily{\textup{\textup{\textmd{D}}}}} ⟩}}}} and tonal patters are labeled with a {\XLingPaperCambriaZMathFontFamily{\textup{\textmd{⟨ {\XLingPaperCharisZSILFontFamily{\textup{\textup{\textmd{T}}}}} ⟩}}}}. Tonal units have an em dash {\XLingPaperCambriaZMathFontFamily{\textup{\textmd{⟨ {\XLingPaperCharisZSILFontFamily{\textup{\textup{\textmd{—}}}}} ⟩}}}} where the typographical baseline glyphs would occur. Where no mark would occur then null marker {\XLingPaperCambriaZMathFontFamily{\textup{\textmd{⟨ {\XLingPaperCharisZSILFontFamily{\textup{\textup{\textmd{∅}}}}} ⟩}}}} is used. The rational for showing a null marker is that readers will have to process the distinction of not just the presence of a symbol, but also the absence of a symbol. Therefore the functional unit is not just the presence of the marks indicating tone but rather the arrangement of those marks. If a reader can intuit the tonal pattern without looking at the beginning and the end of a word then that needs to be accounted for in the reading model.\par{}\indent Note that none of the consonant clusters mentioned in section \hyperlink{sConsonants}{3.2.2} are considered functional units. That is because I analyze them as \hyperlink{gtConsonantBlend}{{\textit{consonant blends}}}. All of the phonetic material is there for these consonant clusters, however the timing tier (if one assumes a metrical analysis) dictates that these phonetic attributes must fit into a single "timing slot".\par{}\vspace{11pt plus 2pt minus 1pt}\XLingPaperneedspace{3\baselineskip}\protect\hypertarget{ntFuntionalUnitsList}{}\XLingPaperaddtocontents{ntFuntionalUnitsList}{\protect\raggedright{\singlespacing
{Table }{20.}{  List of functional units\\}}}\vspace{0pt}{\singlespacing
\hspace*{.25in}{\setcounter{footnote}{22}
\XLingPaperminmaxcellincolumn{Uppercase}{\XLingPapermincola}{\textbf{Uppercase}}{\XLingPapermaxcola}{+0\tabcolsep}
\XLingPaperminmaxcellincolumn{Lowercase}{\XLingPapermincolb}{\textbf{Lowercase}}{\XLingPapermaxcolb}{+0\tabcolsep}
\XLingPaperminmaxcellincolumn{}{\XLingPapermincolc}{\textbf{}}{\XLingPapermaxcolc}{+0\tabcolsep}
\XLingPaperminmaxcellincolumn{Uppercase}{\XLingPapermincold}{\textbf{Uppercase}}{\XLingPapermaxcold}{+0\tabcolsep}
\XLingPaperminmaxcellincolumn{Lowercase}{\XLingPapermincole}{\textbf{Lowercase}}{\XLingPapermaxcole}{+0\tabcolsep}
\XLingPaperminmaxcellincolumn{}{\XLingPapermincolf}{\textbf{}}{\XLingPapermaxcolf}{+0\tabcolsep}
\XLingPaperminmaxcellincolumn{Uppercase}{\XLingPapermincolg}{\textbf{Uppercase}}{\XLingPapermaxcolg}{+0\tabcolsep}
\XLingPaperminmaxcellincolumn{Lowercase}{\XLingPapermincolh}{\textbf{Lowercase}}{\XLingPapermaxcolh}{+0\tabcolsep}
\XLingPaperminmaxcellincolumn{}{\XLingPapermincoli}{\textbf{}}{\XLingPapermaxcoli}{+0\tabcolsep}
\XLingPaperminmaxcellincolumn{A}{\XLingPapermincola}{A}{\XLingPapermaxcola}{+0\tabcolsep}
\XLingPaperminmaxcellincolumn{a}{\XLingPapermincolb}{a}{\XLingPapermaxcolb}{+0\tabcolsep}
\XLingPaperminmaxcellincolumn{}{\XLingPapermincolc}{}{\XLingPapermaxcolc}{+0\tabcolsep}
\XLingPaperminmaxcellincolumn{Ɩ}{\XLingPapermincold}{Ɩ}{\XLingPapermaxcold}{+0\tabcolsep}
\XLingPaperminmaxcellincolumn{ɩ}{\XLingPapermincole}{ɩ}{\XLingPapermaxcole}{+0\tabcolsep}
\XLingPaperminmaxcellincolumn{}{\XLingPapermincolf}{}{\XLingPapermaxcolf}{+0\tabcolsep}
\XLingPaperminmaxcellincolumn{V}{\XLingPapermincolg}{V}{\XLingPapermaxcolg}{+0\tabcolsep}
\XLingPaperminmaxcellincolumn{v}{\XLingPapermincolh}{v}{\XLingPapermaxcolh}{+0\tabcolsep}
\XLingPaperminmaxcellincolumn{}{\XLingPapermincoli}{}{\XLingPapermaxcoli}{+0\tabcolsep}
\XLingPaperminmaxcellincolumn{Aa}{\XLingPapermincola}{Aa}{\XLingPapermaxcola}{+0\tabcolsep}
\XLingPaperminmaxcellincolumn{aa}{\XLingPapermincolb}{aa}{\XLingPapermaxcolb}{+0\tabcolsep}
\XLingPaperminmaxcellincolumn{}{\XLingPapermincolc}{}{\XLingPapermaxcolc}{+0\tabcolsep}
\XLingPaperminmaxcellincolumn{Ɩɩ}{\XLingPapermincold}{Ɩɩ}{\XLingPapermaxcold}{+0\tabcolsep}
\XLingPaperminmaxcellincolumn{ɩɩ}{\XLingPapermincole}{ɩɩ}{\XLingPapermaxcole}{+0\tabcolsep}
\XLingPaperminmaxcellincolumn{}{\XLingPapermincolf}{}{\XLingPapermaxcolf}{+0\tabcolsep}
\XLingPaperminmaxcellincolumn{W}{\XLingPapermincolg}{W}{\XLingPapermaxcolg}{+0\tabcolsep}
\XLingPaperminmaxcellincolumn{w}{\XLingPapermincolh}{w}{\XLingPapermaxcolh}{+0\tabcolsep}
\XLingPaperminmaxcellincolumn{}{\XLingPapermincoli}{}{\XLingPapermaxcoli}{+0\tabcolsep}
\XLingPaperminmaxcellincolumn{An}{\XLingPapermincola}{An}{\XLingPapermaxcola}{+0\tabcolsep}
\XLingPaperminmaxcellincolumn{an}{\XLingPapermincolb}{an}{\XLingPapermaxcolb}{+0\tabcolsep}
\XLingPaperminmaxcellincolumn{}{\XLingPapermincolc}{}{\XLingPapermaxcolc}{+0\tabcolsep}
\XLingPaperminmaxcellincolumn{K}{\XLingPapermincold}{K}{\XLingPapermaxcold}{+0\tabcolsep}
\XLingPaperminmaxcellincolumn{k}{\XLingPapermincole}{k}{\XLingPapermaxcole}{+0\tabcolsep}
\XLingPaperminmaxcellincolumn{}{\XLingPapermincolf}{}{\XLingPapermaxcolf}{+0\tabcolsep}
\XLingPaperminmaxcellincolumn{Y}{\XLingPapermincolg}{Y}{\XLingPapermaxcolg}{+0\tabcolsep}
\XLingPaperminmaxcellincolumn{y}{\XLingPapermincolh}{y}{\XLingPapermaxcolh}{+0\tabcolsep}
\XLingPaperminmaxcellincolumn{}{\XLingPapermincoli}{}{\XLingPapermaxcoli}{+0\tabcolsep}
\XLingPaperminmaxcellincolumn{Aan}{\XLingPapermincola}{Aan}{\XLingPapermaxcola}{+0\tabcolsep}
\XLingPaperminmaxcellincolumn{aan}{\XLingPapermincolb}{aan}{\XLingPapermaxcolb}{+0\tabcolsep}
\XLingPaperminmaxcellincolumn{}{\XLingPapermincolc}{}{\XLingPapermaxcolc}{+0\tabcolsep}
\XLingPaperminmaxcellincolumn{Kp}{\XLingPapermincold}{Kp}{\XLingPapermaxcold}{+0\tabcolsep}
\XLingPaperminmaxcellincolumn{kp}{\XLingPapermincole}{kp}{\XLingPapermaxcole}{+0\tabcolsep}
\XLingPaperminmaxcellincolumn{}{\XLingPapermincolf}{}{\XLingPapermaxcolf}{+0\tabcolsep}
\XLingPaperminmaxcellincolumn{Z}{\XLingPapermincolg}{Z}{\XLingPapermaxcolg}{+0\tabcolsep}
\XLingPaperminmaxcellincolumn{z}{\XLingPapermincolh}{z}{\XLingPapermaxcolh}{+0\tabcolsep}
\XLingPaperminmaxcellincolumn{}{\XLingPapermincoli}{}{\XLingPapermaxcoli}{+0\tabcolsep}
\XLingPaperminmaxcellincolumn{Aɔ}{\XLingPapermincola}{Aɔ}{\XLingPapermaxcola}{+0\tabcolsep}
\XLingPaperminmaxcellincolumn{aɔ}{\XLingPapermincolb}{aɔ}{\XLingPapermaxcolb}{+0\tabcolsep}
\XLingPaperminmaxcellincolumn{}{\XLingPapermincolc}{}{\XLingPapermaxcolc}{+0\tabcolsep}
\XLingPaperminmaxcellincolumn{Kw}{\XLingPapermincold}{Kw}{\XLingPapermaxcold}{+0\tabcolsep}
\XLingPaperminmaxcellincolumn{kw}{\XLingPapermincole}{kw}{\XLingPapermaxcole}{+0\tabcolsep}
\XLingPaperminmaxcellincolumn{}{\XLingPapermincolf}{}{\XLingPapermaxcolf}{+0\tabcolsep}
\XLingPaperminmaxcellincolumn{Ʋ}{\XLingPapermincolg}{Ʋ}{\XLingPapermaxcolg}{+0\tabcolsep}
\XLingPaperminmaxcellincolumn{ʋ}{\XLingPapermincolh}{ʋ}{\XLingPapermaxcolh}{+0\tabcolsep}
\XLingPaperminmaxcellincolumn{}{\XLingPapermincoli}{}{\XLingPapermaxcoli}{+0\tabcolsep}
\XLingPaperminmaxcellincolumn{Aɔn}{\XLingPapermincola}{Aɔn}{\XLingPapermaxcola}{+0\tabcolsep}
\XLingPaperminmaxcellincolumn{aɔn}{\XLingPapermincolb}{aɔn}{\XLingPapermaxcolb}{+0\tabcolsep}
\XLingPaperminmaxcellincolumn{}{\XLingPapermincolc}{}{\XLingPapermaxcolc}{+0\tabcolsep}
\XLingPaperminmaxcellincolumn{L}{\XLingPapermincold}{L}{\XLingPapermaxcold}{+0\tabcolsep}
\XLingPaperminmaxcellincolumn{l}{\XLingPapermincole}{l}{\XLingPapermaxcole}{+0\tabcolsep}
\XLingPaperminmaxcellincolumn{}{\XLingPapermincolf}{}{\XLingPapermaxcolf}{+0\tabcolsep}
\XLingPaperminmaxcellincolumn{Ʋʋ}{\XLingPapermincolg}{Ʋʋ}{\XLingPapermaxcolg}{+0\tabcolsep}
\XLingPaperminmaxcellincolumn{ʋʋ}{\XLingPapermincolh}{ʋʋ}{\XLingPapermaxcolh}{+0\tabcolsep}
\XLingPaperminmaxcellincolumn{}{\XLingPapermincoli}{}{\XLingPapermaxcoli}{+0\tabcolsep}
\XLingPaperminmaxcellincolumn{Bh}{\XLingPapermincola}{Bh}{\XLingPapermaxcola}{+0\tabcolsep}
\XLingPaperminmaxcellincolumn{bh}{\XLingPapermincolb}{bh}{\XLingPapermaxcolb}{+0\tabcolsep}
\XLingPaperminmaxcellincolumn{}{\XLingPapermincolc}{}{\XLingPapermaxcolc}{+0\tabcolsep}
\XLingPaperminmaxcellincolumn{M}{\XLingPapermincold}{M}{\XLingPapermaxcold}{+0\tabcolsep}
\XLingPaperminmaxcellincolumn{m}{\XLingPapermincole}{m}{\XLingPapermaxcole}{+0\tabcolsep}
\XLingPaperminmaxcellincolumn{}{\XLingPapermincolf}{}{\XLingPapermaxcolf}{+0\tabcolsep}
\XLingPaperminmaxcellincolumn{Ʋ̈}{\XLingPapermincolg}{Ʋ̈}{\XLingPapermaxcolg}{+0\tabcolsep}
\XLingPaperminmaxcellincolumn{ʋ̈}{\XLingPapermincolh}{ʋ̈}{\XLingPapermaxcolh}{+0\tabcolsep}
\XLingPaperminmaxcellincolumn{}{\XLingPapermincoli}{}{\XLingPapermaxcoli}{+0\tabcolsep}
\XLingPaperminmaxcellincolumn{D}{\XLingPapermincola}{D}{\XLingPapermaxcola}{+0\tabcolsep}
\XLingPaperminmaxcellincolumn{d}{\XLingPapermincolb}{d}{\XLingPapermaxcolb}{+0\tabcolsep}
\XLingPaperminmaxcellincolumn{}{\XLingPapermincolc}{}{\XLingPapermaxcolc}{+0\tabcolsep}
\XLingPaperminmaxcellincolumn{N}{\XLingPapermincold}{N}{\XLingPapermaxcold}{+0\tabcolsep}
\XLingPaperminmaxcellincolumn{n}{\XLingPapermincole}{n}{\XLingPapermaxcole}{+0\tabcolsep}
\XLingPaperminmaxcellincolumn{}{\XLingPapermincolf}{}{\XLingPapermaxcolf}{+0\tabcolsep}
\XLingPaperminmaxcellincolumn{Ʋ̈ʋ̈}{\XLingPapermincolg}{Ʋ̈ʋ̈}{\XLingPapermaxcolg}{+0\tabcolsep}
\XLingPaperminmaxcellincolumn{ʋ̈ʋ̈}{\XLingPapermincolh}{ʋ̈ʋ̈}{\XLingPapermaxcolh}{+0\tabcolsep}
\XLingPaperminmaxcellincolumn{}{\XLingPapermincoli}{}{\XLingPapermaxcoli}{+0\tabcolsep}
\XLingPaperminmaxcellincolumn{Dh}{\XLingPapermincola}{Dh}{\XLingPapermaxcola}{+0\tabcolsep}
\XLingPaperminmaxcellincolumn{dh}{\XLingPapermincolb}{dh}{\XLingPapermaxcolb}{+0\tabcolsep}
\XLingPaperminmaxcellincolumn{}{\XLingPapermincolc}{}{\XLingPapermaxcolc}{+0\tabcolsep}
\XLingPaperminmaxcellincolumn{Ng}{\XLingPapermincold}{Ng}{\XLingPapermaxcold}{+0\tabcolsep}
\XLingPaperminmaxcellincolumn{ng}{\XLingPapermincole}{ng}{\XLingPapermaxcole}{+0\tabcolsep}
\XLingPaperminmaxcellincolumn{}{\XLingPapermincolf}{}{\XLingPapermaxcolf}{+0\tabcolsep}
\XLingPaperminmaxcellincolumn{uppercase}{\XLingPapermincolg}{}{\XLingPapermaxcolg}{+0\tabcolsep}
\XLingPaperminmaxcellincolumn{iʋ̈}{\XLingPapermincolh}{iʋ̈}{\XLingPapermaxcolh}{+0\tabcolsep}
\XLingPaperminmaxcellincolumn{D}{\XLingPapermincoli}{D}{\XLingPapermaxcoli}{+0\tabcolsep}
\XLingPaperminmaxcellincolumn{E}{\XLingPapermincola}{E}{\XLingPapermaxcola}{+0\tabcolsep}
\XLingPaperminmaxcellincolumn{e}{\XLingPapermincolb}{e}{\XLingPapermaxcolb}{+0\tabcolsep}
\XLingPaperminmaxcellincolumn{}{\XLingPapermincolc}{}{\XLingPapermaxcolc}{+0\tabcolsep}
\XLingPaperminmaxcellincolumn{O}{\XLingPapermincold}{O}{\XLingPapermaxcold}{+0\tabcolsep}
\XLingPaperminmaxcellincolumn{o}{\XLingPapermincole}{o}{\XLingPapermaxcole}{+0\tabcolsep}
\XLingPaperminmaxcellincolumn{}{\XLingPapermincolf}{}{\XLingPapermaxcolf}{+0\tabcolsep}
\XLingPaperminmaxcellincolumn{}{\XLingPapermincolg}{}{\XLingPapermaxcolg}{+0\tabcolsep}
\XLingPaperminmaxcellincolumn{iö}{\XLingPapermincolh}{iö}{\XLingPapermaxcolh}{+0\tabcolsep}
\XLingPaperminmaxcellincolumn{D}{\XLingPapermincoli}{D}{\XLingPapermaxcoli}{+0\tabcolsep}
\XLingPaperminmaxcellincolumn{Ee}{\XLingPapermincola}{Ee}{\XLingPapermaxcola}{+0\tabcolsep}
\XLingPaperminmaxcellincolumn{ee}{\XLingPapermincolb}{ee}{\XLingPapermaxcolb}{+0\tabcolsep}
\XLingPaperminmaxcellincolumn{}{\XLingPapermincolc}{}{\XLingPapermaxcolc}{+0\tabcolsep}
\XLingPaperminmaxcellincolumn{Oo}{\XLingPapermincold}{Oo}{\XLingPapermaxcold}{+0\tabcolsep}
\XLingPaperminmaxcellincolumn{oo}{\XLingPapermincole}{oo}{\XLingPapermaxcole}{+0\tabcolsep}
\XLingPaperminmaxcellincolumn{}{\XLingPapermincolf}{}{\XLingPapermaxcolf}{+0\tabcolsep}
\XLingPaperminmaxcellincolumn{}{\XLingPapermincolg}{}{\XLingPapermaxcolg}{+0\tabcolsep}
\XLingPaperminmaxcellincolumn{ië}{\XLingPapermincolh}{ië}{\XLingPapermaxcolh}{+0\tabcolsep}
\XLingPaperminmaxcellincolumn{D}{\XLingPapermincoli}{D}{\XLingPapermaxcoli}{+0\tabcolsep}
\XLingPaperminmaxcellincolumn{Ɛ}{\XLingPapermincola}{Ɛ}{\XLingPapermaxcola}{+0\tabcolsep}
\XLingPaperminmaxcellincolumn{ɛ}{\XLingPapermincolb}{ɛ}{\XLingPapermaxcolb}{+0\tabcolsep}
\XLingPaperminmaxcellincolumn{}{\XLingPapermincolc}{}{\XLingPapermaxcolc}{+0\tabcolsep}
\XLingPaperminmaxcellincolumn{Ö}{\XLingPapermincold}{Ö}{\XLingPapermaxcold}{+0\tabcolsep}
\XLingPaperminmaxcellincolumn{ö}{\XLingPapermincole}{ö}{\XLingPapermaxcole}{+0\tabcolsep}
\XLingPaperminmaxcellincolumn{}{\XLingPapermincolf}{}{\XLingPapermaxcolf}{+0\tabcolsep}
\XLingPaperminmaxcellincolumn{}{\XLingPapermincolg}{}{\XLingPapermaxcolg}{+0\tabcolsep}
\XLingPaperminmaxcellincolumn{ia}{\XLingPapermincolh}{ia}{\XLingPapermaxcolh}{+0\tabcolsep}
\XLingPaperminmaxcellincolumn{D}{\XLingPapermincoli}{D}{\XLingPapermaxcoli}{+0\tabcolsep}
\XLingPaperminmaxcellincolumn{Ɛɛ}{\XLingPapermincola}{Ɛɛ}{\XLingPapermaxcola}{+0\tabcolsep}
\XLingPaperminmaxcellincolumn{ɛɛ}{\XLingPapermincolb}{ɛɛ}{\XLingPapermaxcolb}{+0\tabcolsep}
\XLingPaperminmaxcellincolumn{}{\XLingPapermincolc}{}{\XLingPapermaxcolc}{+0\tabcolsep}
\XLingPaperminmaxcellincolumn{Öö}{\XLingPapermincold}{Öö}{\XLingPapermaxcold}{+0\tabcolsep}
\XLingPaperminmaxcellincolumn{öö}{\XLingPapermincole}{öö}{\XLingPapermaxcole}{+0\tabcolsep}
\XLingPaperminmaxcellincolumn{}{\XLingPapermincolf}{}{\XLingPapermaxcolf}{+0\tabcolsep}
\XLingPaperminmaxcellincolumn{}{\XLingPapermincolg}{}{\XLingPapermaxcolg}{+0\tabcolsep}
\XLingPaperminmaxcellincolumn{ian}{\XLingPapermincolh}{ian}{\XLingPapermaxcolh}{+0\tabcolsep}
\XLingPaperminmaxcellincolumn{D}{\XLingPapermincoli}{D}{\XLingPapermaxcoli}{+0\tabcolsep}
\XLingPaperminmaxcellincolumn{Ɛa}{\XLingPapermincola}{Ɛa}{\XLingPapermaxcola}{+0\tabcolsep}
\XLingPaperminmaxcellincolumn{ɛa}{\XLingPapermincolb}{ɛa}{\XLingPapermaxcolb}{+0\tabcolsep}
\XLingPaperminmaxcellincolumn{}{\XLingPapermincolc}{}{\XLingPapermaxcolc}{+0\tabcolsep}
\XLingPaperminmaxcellincolumn{Ɔ}{\XLingPapermincold}{Ɔ}{\XLingPapermaxcold}{+0\tabcolsep}
\XLingPaperminmaxcellincolumn{ɔ}{\XLingPapermincole}{ɔ}{\XLingPapermaxcole}{+0\tabcolsep}
\XLingPaperminmaxcellincolumn{}{\XLingPapermincolf}{}{\XLingPapermaxcolf}{+0\tabcolsep}
\XLingPaperminmaxcellincolumn{}{\XLingPapermincolg}{}{\XLingPapermaxcolg}{+0\tabcolsep}
\XLingPaperminmaxcellincolumn{ɩa}{\XLingPapermincolh}{ɩa}{\XLingPapermaxcolh}{+0\tabcolsep}
\XLingPaperminmaxcellincolumn{D}{\XLingPapermincoli}{D}{\XLingPapermaxcoli}{+0\tabcolsep}
\XLingPaperminmaxcellincolumn{Ɛan}{\XLingPapermincola}{Ɛan}{\XLingPapermaxcola}{+0\tabcolsep}
\XLingPaperminmaxcellincolumn{ɛan}{\XLingPapermincolb}{ɛan}{\XLingPapermaxcolb}{+0\tabcolsep}
\XLingPaperminmaxcellincolumn{}{\XLingPapermincolc}{}{\XLingPapermaxcolc}{+0\tabcolsep}
\XLingPaperminmaxcellincolumn{Ɔɔ}{\XLingPapermincold}{Ɔɔ}{\XLingPapermaxcold}{+0\tabcolsep}
\XLingPaperminmaxcellincolumn{ɔɔ}{\XLingPapermincole}{ɔɔ}{\XLingPapermaxcole}{+0\tabcolsep}
\XLingPaperminmaxcellincolumn{}{\XLingPapermincolf}{}{\XLingPapermaxcolf}{+0\tabcolsep}
\XLingPaperminmaxcellincolumn{}{\XLingPapermincolg}{}{\XLingPapermaxcolg}{+0\tabcolsep}
\XLingPaperminmaxcellincolumn{uë}{\XLingPapermincolh}{uë}{\XLingPapermaxcolh}{+0\tabcolsep}
\XLingPaperminmaxcellincolumn{D}{\XLingPapermincoli}{D}{\XLingPapermaxcoli}{+0\tabcolsep}
\XLingPaperminmaxcellincolumn{Ɛn}{\XLingPapermincola}{Ɛn}{\XLingPapermaxcola}{+0\tabcolsep}
\XLingPaperminmaxcellincolumn{ɛn}{\XLingPapermincolb}{ɛn}{\XLingPapermaxcolb}{+0\tabcolsep}
\XLingPaperminmaxcellincolumn{}{\XLingPapermincolc}{}{\XLingPapermaxcolc}{+0\tabcolsep}
\XLingPaperminmaxcellincolumn{Ɔn}{\XLingPapermincold}{Ɔn}{\XLingPapermaxcold}{+0\tabcolsep}
\XLingPaperminmaxcellincolumn{ɔn}{\XLingPapermincole}{ɔn}{\XLingPapermaxcole}{+0\tabcolsep}
\XLingPaperminmaxcellincolumn{}{\XLingPapermincolf}{}{\XLingPapermaxcolf}{+0\tabcolsep}
\XLingPaperminmaxcellincolumn{}{\XLingPapermincolg}{}{\XLingPapermaxcolg}{+0\tabcolsep}
\XLingPaperminmaxcellincolumn{ʋë}{\XLingPapermincolh}{ʋë}{\XLingPapermaxcolh}{+0\tabcolsep}
\XLingPaperminmaxcellincolumn{D}{\XLingPapermincoli}{D}{\XLingPapermaxcoli}{+0\tabcolsep}
\XLingPaperminmaxcellincolumn{Ɛɛn}{\XLingPapermincola}{Ɛɛn}{\XLingPapermaxcola}{+0\tabcolsep}
\XLingPaperminmaxcellincolumn{ɛɛn}{\XLingPapermincolb}{ɛɛn}{\XLingPapermaxcolb}{+0\tabcolsep}
\XLingPaperminmaxcellincolumn{}{\XLingPapermincolc}{}{\XLingPapermaxcolc}{+0\tabcolsep}
\XLingPaperminmaxcellincolumn{Ɔɔn}{\XLingPapermincold}{Ɔɔn}{\XLingPapermaxcold}{+0\tabcolsep}
\XLingPaperminmaxcellincolumn{ɔɔn}{\XLingPapermincole}{ɔɔn}{\XLingPapermaxcole}{+0\tabcolsep}
\XLingPaperminmaxcellincolumn{}{\XLingPapermincolf}{}{\XLingPapermaxcolf}{+0\tabcolsep}
\XLingPaperminmaxcellincolumn{}{\XLingPapermincolg}{}{\XLingPapermaxcolg}{+0\tabcolsep}
\XLingPaperminmaxcellincolumn{ʋ̈ü}{\XLingPapermincolh}{ʋ̈ü}{\XLingPapermaxcolh}{+0\tabcolsep}
\XLingPaperminmaxcellincolumn{D}{\XLingPapermincoli}{D}{\XLingPapermaxcoli}{+0\tabcolsep}
\XLingPaperminmaxcellincolumn{Ë}{\XLingPapermincola}{Ë}{\XLingPapermaxcola}{+0\tabcolsep}
\XLingPaperminmaxcellincolumn{ë}{\XLingPapermincolb}{ë}{\XLingPapermaxcolb}{+0\tabcolsep}
\XLingPaperminmaxcellincolumn{}{\XLingPapermincolc}{}{\XLingPapermaxcolc}{+0\tabcolsep}
\XLingPaperminmaxcellincolumn{P}{\XLingPapermincold}{P}{\XLingPapermaxcold}{+0\tabcolsep}
\XLingPaperminmaxcellincolumn{ɔn}{\XLingPapermincole}{ɔn}{\XLingPapermaxcole}{+0\tabcolsep}
\XLingPaperminmaxcellincolumn{}{\XLingPapermincolf}{}{\XLingPapermaxcolf}{+0\tabcolsep}
\XLingPaperminmaxcellincolumn{}{\XLingPapermincolg}{}{\XLingPapermaxcolg}{+0\tabcolsep}
\XLingPaperminmaxcellincolumn{ˮ}{\XLingPapermincolh}{ˮ — ∅}{\XLingPapermaxcolh}{+0\tabcolsep}
\XLingPaperminmaxcellincolumn{T}{\XLingPapermincoli}{T}{\XLingPapermaxcoli}{+0\tabcolsep}
\XLingPaperminmaxcellincolumn{Ëë}{\XLingPapermincola}{Ëë}{\XLingPapermaxcola}{+0\tabcolsep}
\XLingPaperminmaxcellincolumn{ëë}{\XLingPapermincolb}{ëë}{\XLingPapermaxcolb}{+0\tabcolsep}
\XLingPaperminmaxcellincolumn{}{\XLingPapermincolc}{}{\XLingPapermaxcolc}{+0\tabcolsep}
\XLingPaperminmaxcellincolumn{R}{\XLingPapermincold}{R}{\XLingPapermaxcold}{+0\tabcolsep}
\XLingPaperminmaxcellincolumn{ɔɔn}{\XLingPapermincole}{ɔɔn}{\XLingPapermaxcole}{+0\tabcolsep}
\XLingPaperminmaxcellincolumn{}{\XLingPapermincolf}{}{\XLingPapermaxcolf}{+0\tabcolsep}
\XLingPaperminmaxcellincolumn{}{\XLingPapermincolg}{}{\XLingPapermaxcolg}{+0\tabcolsep}
\XLingPaperminmaxcellincolumn{ʼ}{\XLingPapermincolh}{ʼ — ∅}{\XLingPapermaxcolh}{+0\tabcolsep}
\XLingPaperminmaxcellincolumn{T}{\XLingPapermincoli}{T}{\XLingPapermaxcoli}{+0\tabcolsep}
\XLingPaperminmaxcellincolumn{Ën}{\XLingPapermincola}{Ën}{\XLingPapermaxcola}{+0\tabcolsep}
\XLingPaperminmaxcellincolumn{ën}{\XLingPapermincolb}{ën}{\XLingPapermaxcolb}{+0\tabcolsep}
\XLingPaperminmaxcellincolumn{}{\XLingPapermincolc}{}{\XLingPapermaxcolc}{+0\tabcolsep}
\XLingPaperminmaxcellincolumn{S}{\XLingPapermincold}{S}{\XLingPapermaxcold}{+0\tabcolsep}
\XLingPaperminmaxcellincolumn{s}{\XLingPapermincole}{s}{\XLingPapermaxcole}{+0\tabcolsep}
\XLingPaperminmaxcellincolumn{}{\XLingPapermincolf}{}{\XLingPapermaxcolf}{+0\tabcolsep}
\XLingPaperminmaxcellincolumn{}{\XLingPapermincolg}{}{\XLingPapermaxcolg}{+0\tabcolsep}
\XLingPaperminmaxcellincolumn{∅}{\XLingPapermincolh}{∅ — ∅}{\XLingPapermaxcolh}{+0\tabcolsep}
\XLingPaperminmaxcellincolumn{T}{\XLingPapermincoli}{T}{\XLingPapermaxcoli}{+0\tabcolsep}
\XLingPaperminmaxcellincolumn{Ëën}{\XLingPapermincola}{Ëën}{\XLingPapermaxcola}{+0\tabcolsep}
\XLingPaperminmaxcellincolumn{ëën}{\XLingPapermincolb}{ëën}{\XLingPapermaxcolb}{+0\tabcolsep}
\XLingPaperminmaxcellincolumn{}{\XLingPapermincolc}{}{\XLingPapermaxcolc}{+0\tabcolsep}
\XLingPaperminmaxcellincolumn{T}{\XLingPapermincold}{T}{\XLingPapermaxcold}{+0\tabcolsep}
\XLingPaperminmaxcellincolumn{t}{\XLingPapermincole}{t}{\XLingPapermaxcole}{+0\tabcolsep}
\XLingPaperminmaxcellincolumn{}{\XLingPapermincolf}{}{\XLingPapermaxcolf}{+0\tabcolsep}
\XLingPaperminmaxcellincolumn{}{\XLingPapermincolg}{}{\XLingPapermaxcolg}{+0\tabcolsep}
\XLingPaperminmaxcellincolumn{꞊}{\XLingPapermincolh}{꞊ — ∅}{\XLingPapermaxcolh}{+0\tabcolsep}
\XLingPaperminmaxcellincolumn{T}{\XLingPapermincoli}{T}{\XLingPapermaxcoli}{+0\tabcolsep}
\XLingPaperminmaxcellincolumn{F}{\XLingPapermincola}{F}{\XLingPapermaxcola}{+0\tabcolsep}
\XLingPaperminmaxcellincolumn{f}{\XLingPapermincolb}{f}{\XLingPapermaxcolb}{+0\tabcolsep}
\XLingPaperminmaxcellincolumn{}{\XLingPapermincolc}{}{\XLingPapermaxcolc}{+0\tabcolsep}
\XLingPaperminmaxcellincolumn{U}{\XLingPapermincold}{U}{\XLingPapermaxcold}{+0\tabcolsep}
\XLingPaperminmaxcellincolumn{u}{\XLingPapermincole}{u}{\XLingPapermaxcole}{+0\tabcolsep}
\XLingPaperminmaxcellincolumn{}{\XLingPapermincolf}{}{\XLingPapermaxcolf}{+0\tabcolsep}
\XLingPaperminmaxcellincolumn{}{\XLingPapermincolg}{}{\XLingPapermaxcolg}{+0\tabcolsep}
\XLingPaperminmaxcellincolumn{˗}{\XLingPapermincolh}{˗ — ∅}{\XLingPapermaxcolh}{+0\tabcolsep}
\XLingPaperminmaxcellincolumn{T}{\XLingPapermincoli}{T}{\XLingPapermaxcoli}{+0\tabcolsep}
\XLingPaperminmaxcellincolumn{G}{\XLingPapermincola}{G}{\XLingPapermaxcola}{+0\tabcolsep}
\XLingPaperminmaxcellincolumn{g}{\XLingPapermincolb}{g}{\XLingPapermaxcolb}{+0\tabcolsep}
\XLingPaperminmaxcellincolumn{}{\XLingPapermincolc}{}{\XLingPapermaxcolc}{+0\tabcolsep}
\XLingPaperminmaxcellincolumn{Uu}{\XLingPapermincold}{Uu}{\XLingPapermaxcold}{+0\tabcolsep}
\XLingPaperminmaxcellincolumn{uu}{\XLingPapermincole}{uu}{\XLingPapermaxcole}{+0\tabcolsep}
\XLingPaperminmaxcellincolumn{}{\XLingPapermincolf}{}{\XLingPapermaxcolf}{+0\tabcolsep}
\XLingPaperminmaxcellincolumn{}{\XLingPapermincolg}{}{\XLingPapermaxcolg}{+0\tabcolsep}
\XLingPaperminmaxcellincolumn{ˮ}{\XLingPapermincolh}{ˮ — ˗}{\XLingPapermaxcolh}{+0\tabcolsep}
\XLingPaperminmaxcellincolumn{T}{\XLingPapermincoli}{T}{\XLingPapermaxcoli}{+0\tabcolsep}
\XLingPaperminmaxcellincolumn{Gb}{\XLingPapermincola}{Gb}{\XLingPapermaxcola}{+0\tabcolsep}
\XLingPaperminmaxcellincolumn{gb}{\XLingPapermincolb}{gb}{\XLingPapermaxcolb}{+0\tabcolsep}
\XLingPaperminmaxcellincolumn{}{\XLingPapermincolc}{}{\XLingPapermaxcolc}{+0\tabcolsep}
\XLingPaperminmaxcellincolumn{Un}{\XLingPapermincold}{Un}{\XLingPapermaxcold}{+0\tabcolsep}
\XLingPaperminmaxcellincolumn{un}{\XLingPapermincole}{un}{\XLingPapermaxcole}{+0\tabcolsep}
\XLingPaperminmaxcellincolumn{}{\XLingPapermincolf}{}{\XLingPapermaxcolf}{+0\tabcolsep}
\XLingPaperminmaxcellincolumn{}{\XLingPapermincolg}{}{\XLingPapermaxcolg}{+0\tabcolsep}
\XLingPaperminmaxcellincolumn{ʼ}{\XLingPapermincolh}{ʼ — ˗}{\XLingPapermaxcolh}{+0\tabcolsep}
\XLingPaperminmaxcellincolumn{T}{\XLingPapermincoli}{T}{\XLingPapermaxcoli}{+0\tabcolsep}
\XLingPaperminmaxcellincolumn{Gw}{\XLingPapermincola}{Gw}{\XLingPapermaxcola}{+0\tabcolsep}
\XLingPaperminmaxcellincolumn{gw}{\XLingPapermincolb}{gw}{\XLingPapermaxcolb}{+0\tabcolsep}
\XLingPaperminmaxcellincolumn{}{\XLingPapermincolc}{}{\XLingPapermaxcolc}{+0\tabcolsep}
\XLingPaperminmaxcellincolumn{Uun}{\XLingPapermincold}{Uun}{\XLingPapermaxcold}{+0\tabcolsep}
\XLingPaperminmaxcellincolumn{uun}{\XLingPapermincole}{uun}{\XLingPapermaxcole}{+0\tabcolsep}
\XLingPaperminmaxcellincolumn{}{\XLingPapermincolf}{}{\XLingPapermaxcolf}{+0\tabcolsep}
\XLingPaperminmaxcellincolumn{}{\XLingPapermincolg}{}{\XLingPapermaxcolg}{+0\tabcolsep}
\XLingPaperminmaxcellincolumn{∅}{\XLingPapermincolh}{∅ — ˗}{\XLingPapermaxcolh}{+0\tabcolsep}
\XLingPaperminmaxcellincolumn{T}{\XLingPapermincoli}{T}{\XLingPapermaxcoli}{+0\tabcolsep}
\XLingPaperminmaxcellincolumn{I}{\XLingPapermincola}{I}{\XLingPapermaxcola}{+0\tabcolsep}
\XLingPaperminmaxcellincolumn{i}{\XLingPapermincolb}{i}{\XLingPapermaxcolb}{+0\tabcolsep}
\XLingPaperminmaxcellincolumn{}{\XLingPapermincolc}{}{\XLingPapermaxcolc}{+0\tabcolsep}
\XLingPaperminmaxcellincolumn{Ü}{\XLingPapermincold}{Ü}{\XLingPapermaxcold}{+0\tabcolsep}
\XLingPaperminmaxcellincolumn{ü}{\XLingPapermincole}{ü}{\XLingPapermaxcole}{+0\tabcolsep}
\XLingPaperminmaxcellincolumn{}{\XLingPapermincolf}{}{\XLingPapermaxcolf}{+0\tabcolsep}
\XLingPaperminmaxcellincolumn{}{\XLingPapermincolg}{}{\XLingPapermaxcolg}{+0\tabcolsep}
\XLingPaperminmaxcellincolumn{꞊}{\XLingPapermincolh}{꞊ — ˗}{\XLingPapermaxcolh}{+0\tabcolsep}
\XLingPaperminmaxcellincolumn{T}{\XLingPapermincoli}{T}{\XLingPapermaxcoli}{+0\tabcolsep}
\XLingPaperminmaxcellincolumn{In}{\XLingPapermincola}{In}{\XLingPapermaxcola}{+0\tabcolsep}
\XLingPaperminmaxcellincolumn{in}{\XLingPapermincolb}{in}{\XLingPapermaxcolb}{+0\tabcolsep}
\XLingPaperminmaxcellincolumn{}{\XLingPapermincolc}{}{\XLingPapermaxcolc}{+0\tabcolsep}
\XLingPaperminmaxcellincolumn{Üü}{\XLingPapermincold}{Üü}{\XLingPapermaxcold}{+0\tabcolsep}
\XLingPaperminmaxcellincolumn{üü}{\XLingPapermincole}{üü}{\XLingPapermaxcole}{+0\tabcolsep}
\XLingPaperminmaxcellincolumn{}{\XLingPapermincolf}{}{\XLingPapermaxcolf}{+0\tabcolsep}
\XLingPaperminmaxcellincolumn{}{\XLingPapermincolg}{}{\XLingPapermaxcolg}{+0\tabcolsep}
\XLingPaperminmaxcellincolumn{∅}{\XLingPapermincolh}{∅ — ʼ}{\XLingPapermaxcolh}{+0\tabcolsep}
\XLingPaperminmaxcellincolumn{T}{\XLingPapermincoli}{T}{\XLingPapermaxcoli}{+0\tabcolsep}
\XLingPaperminmaxcellincolumn{Ii}{\XLingPapermincola}{Ii}{\XLingPapermaxcola}{+0\tabcolsep}
\XLingPaperminmaxcellincolumn{ii}{\XLingPapermincolb}{ii}{\XLingPapermaxcolb}{+0\tabcolsep}
\XLingPaperminmaxcellincolumn{}{\XLingPapermincolc}{}{\XLingPapermaxcolc}{+0\tabcolsep}
\XLingPaperminmaxcellincolumn{Ün}{\XLingPapermincold}{Ün}{\XLingPapermaxcold}{+0\tabcolsep}
\XLingPaperminmaxcellincolumn{ün}{\XLingPapermincole}{ün}{\XLingPapermaxcole}{+0\tabcolsep}
\XLingPaperminmaxcellincolumn{}{\XLingPapermincolf}{}{\XLingPapermaxcolf}{+0\tabcolsep}
\XLingPaperminmaxcellincolumn{}{\XLingPapermincolg}{}{\XLingPapermaxcolg}{+0\tabcolsep}
\XLingPaperminmaxcellincolumn{∅}{\XLingPapermincolh}{∅ — ˮ}{\XLingPapermaxcolh}{+0\tabcolsep}
\XLingPaperminmaxcellincolumn{T}{\XLingPapermincoli}{T}{\XLingPapermaxcoli}{+0\tabcolsep}
\XLingPaperminmaxcellincolumn{Iin}{\XLingPapermincola}{Iin}{\XLingPapermaxcola}{+0\tabcolsep}
\XLingPaperminmaxcellincolumn{iin}{\XLingPapermincolb}{iin}{\XLingPapermaxcolb}{+0\tabcolsep}
\XLingPaperminmaxcellincolumn{}{\XLingPapermincolc}{}{\XLingPapermaxcolc}{+0\tabcolsep}
\XLingPaperminmaxcellincolumn{Üün}{\XLingPapermincold}{Üün}{\XLingPapermaxcold}{+0\tabcolsep}
\XLingPaperminmaxcellincolumn{üün}{\XLingPapermincole}{üün}{\XLingPapermaxcole}{+0\tabcolsep}
\XLingPaperminmaxcellincolumn{}{\XLingPapermincolf}{}{\XLingPapermaxcolf}{+0\tabcolsep}
\XLingPaperminmaxcellincolumn{}{\XLingPapermincolg}{}{\XLingPapermaxcolg}{+0\tabcolsep}
\XLingPaperminmaxcellincolumn{}{\XLingPapermincolh}{}{\XLingPapermaxcolh}{+0\tabcolsep}
\XLingPaperminmaxcellincolumn{}{\XLingPapermincoli}{}{\XLingPapermaxcoli}{+0\tabcolsep}
\setlength{\XLingPaperavailabletablewidth}{433.62pt}
\setlength{\XLingPapertableminwidth}{\XLingPapermincola+\XLingPapermincolb+\XLingPapermincolc+\XLingPapermincold+\XLingPapermincole+\XLingPapermincolf+\XLingPapermincolg+\XLingPapermincolh+\XLingPapermincoli}
\setlength{\XLingPapertablemaxwidth}{\XLingPapermaxcola+\XLingPapermaxcolb+\XLingPapermaxcolc+\XLingPapermaxcold+\XLingPapermaxcole+\XLingPapermaxcolf+\XLingPapermaxcolg+\XLingPapermaxcolh+\XLingPapermaxcoli}
\XLingPapercalculatetablewidthratio{}
\XLingPapersetcolumnwidth{\XLingPapercolawidth}{\XLingPapermincola}{\XLingPapermaxcola}{-0\tabcolsep}
\XLingPapersetcolumnwidth{\XLingPapercolbwidth}{\XLingPapermincolb}{\XLingPapermaxcolb}{-2\tabcolsep}
\XLingPapersetcolumnwidth{\XLingPapercolcwidth}{\XLingPapermincolc}{\XLingPapermaxcolc}{-2\tabcolsep}
\XLingPapersetcolumnwidth{\XLingPapercoldwidth}{\XLingPapermincold}{\XLingPapermaxcold}{-2\tabcolsep}
\XLingPapersetcolumnwidth{\XLingPapercolewidth}{\XLingPapermincole}{\XLingPapermaxcole}{-2\tabcolsep}
\XLingPapersetcolumnwidth{\XLingPapercolfwidth}{\XLingPapermincolf}{\XLingPapermaxcolf}{-2\tabcolsep}
\XLingPapersetcolumnwidth{\XLingPapercolgwidth}{\XLingPapermincolg}{\XLingPapermaxcolg}{-2\tabcolsep}
\XLingPapersetcolumnwidth{\XLingPapercolhwidth}{\XLingPapermincolh}{\XLingPapermaxcolh}{-2\tabcolsep}
\XLingPapersetcolumnwidth{\XLingPapercoliwidth}{\XLingPapermincoli}{\XLingPapermaxcoli}{-2\tabcolsep}\setcounter{footnote}{22}\singlespacing\vspace*{-3\baselineskip}
\begin{longtable}
[l]{@{}p{\XLingPapercolawidth}p{\XLingPapercolbwidth}p{\XLingPapercolcwidth}p{\XLingPapercoldwidth}p{\XLingPapercolewidth}p{\XLingPapercolfwidth}p{\XLingPapercolgwidth}p{\XLingPapercolhwidth}p{\XLingPapercoliwidth}@{}}\toprule\multicolumn{1}{@{}p{\XLingPapercolawidth}}{\textbf{Uppercase}}&\multicolumn{1}{p{\XLingPapercolbwidth}}{\textbf{Lowercase}}&\multicolumn{1}{p{\XLingPapercolcwidth}}{\textbf{}}&\multicolumn{1}{p{\XLingPapercoldwidth}}{\textbf{Uppercase}}&\multicolumn{1}{p{\XLingPapercolewidth}}{\textbf{Lowercase}}&\multicolumn{1}{p{\XLingPapercolfwidth}}{\textbf{}}&\multicolumn{1}{p{\XLingPapercolgwidth}}{\textbf{Uppercase}}&\multicolumn{1}{p{\XLingPapercolhwidth}}{\textbf{Lowercase}}&\multicolumn{1}{p{\XLingPapercoliwidth}@{}}{\textbf{}}\\%
\midrule\endhead \multicolumn{1}{@{}p{\XLingPapercolawidth}}{A}&\multicolumn{1}{p{\XLingPapercolbwidth}}{a}&\multicolumn{1}{p{\XLingPapercolcwidth}}{}&\multicolumn{1}{p{\XLingPapercoldwidth}}{Ɩ}&\multicolumn{1}{p{\XLingPapercolewidth}}{ɩ}&\multicolumn{1}{p{\XLingPapercolfwidth}}{}&\multicolumn{1}{p{\XLingPapercolgwidth}}{V}&\multicolumn{1}{p{\XLingPapercolhwidth}}{v}&\multicolumn{1}{p{\XLingPapercoliwidth}@{}}{}\\%
\multicolumn{1}{@{}p{\XLingPapercolawidth}}{Aa}&\multicolumn{1}{p{\XLingPapercolbwidth}}{aa}&\multicolumn{1}{p{\XLingPapercolcwidth}}{}&\multicolumn{1}{p{\XLingPapercoldwidth}}{Ɩɩ}&\multicolumn{1}{p{\XLingPapercolewidth}}{ɩɩ}&\multicolumn{1}{p{\XLingPapercolfwidth}}{}&\multicolumn{1}{p{\XLingPapercolgwidth}}{W}&\multicolumn{1}{p{\XLingPapercolhwidth}}{w}&\multicolumn{1}{p{\XLingPapercoliwidth}@{}}{}\\%
\multicolumn{1}{@{}p{\XLingPapercolawidth}}{An}&\multicolumn{1}{p{\XLingPapercolbwidth}}{an}&\multicolumn{1}{p{\XLingPapercolcwidth}}{}&\multicolumn{1}{p{\XLingPapercoldwidth}}{K}&\multicolumn{1}{p{\XLingPapercolewidth}}{k}&\multicolumn{1}{p{\XLingPapercolfwidth}}{}&\multicolumn{1}{p{\XLingPapercolgwidth}}{Y}&\multicolumn{1}{p{\XLingPapercolhwidth}}{y}&\multicolumn{1}{p{\XLingPapercoliwidth}@{}}{}\\%
\multicolumn{1}{@{}p{\XLingPapercolawidth}}{Aan}&\multicolumn{1}{p{\XLingPapercolbwidth}}{aan}&\multicolumn{1}{p{\XLingPapercolcwidth}}{}&\multicolumn{1}{p{\XLingPapercoldwidth}}{Kp}&\multicolumn{1}{p{\XLingPapercolewidth}}{kp}&\multicolumn{1}{p{\XLingPapercolfwidth}}{}&\multicolumn{1}{p{\XLingPapercolgwidth}}{Z}&\multicolumn{1}{p{\XLingPapercolhwidth}}{z}&\multicolumn{1}{p{\XLingPapercoliwidth}@{}}{}\\%
\multicolumn{1}{@{}p{\XLingPapercolawidth}}{Aɔ}&\multicolumn{1}{p{\XLingPapercolbwidth}}{aɔ}&\multicolumn{1}{p{\XLingPapercolcwidth}}{}&\multicolumn{1}{p{\XLingPapercoldwidth}}{Kw}&\multicolumn{1}{p{\XLingPapercolewidth}}{kw}&\multicolumn{1}{p{\XLingPapercolfwidth}}{}&\multicolumn{1}{p{\XLingPapercolgwidth}}{Ʋ}&\multicolumn{1}{p{\XLingPapercolhwidth}}{ʋ}&\multicolumn{1}{p{\XLingPapercoliwidth}@{}}{}\\%
\multicolumn{1}{@{}p{\XLingPapercolawidth}}{Aɔn}&\multicolumn{1}{p{\XLingPapercolbwidth}}{aɔn}&\multicolumn{1}{p{\XLingPapercolcwidth}}{}&\multicolumn{1}{p{\XLingPapercoldwidth}}{L}&\multicolumn{1}{p{\XLingPapercolewidth}}{l}&\multicolumn{1}{p{\XLingPapercolfwidth}}{}&\multicolumn{1}{p{\XLingPapercolgwidth}}{Ʋʋ}&\multicolumn{1}{p{\XLingPapercolhwidth}}{ʋʋ}&\multicolumn{1}{p{\XLingPapercoliwidth}@{}}{}\\%
\multicolumn{1}{@{}p{\XLingPapercolawidth}}{Bh}&\multicolumn{1}{p{\XLingPapercolbwidth}}{bh}&\multicolumn{1}{p{\XLingPapercolcwidth}}{}&\multicolumn{1}{p{\XLingPapercoldwidth}}{M}&\multicolumn{1}{p{\XLingPapercolewidth}}{m}&\multicolumn{1}{p{\XLingPapercolfwidth}}{}&\multicolumn{1}{p{\XLingPapercolgwidth}}{Ʋ̈}&\multicolumn{1}{p{\XLingPapercolhwidth}}{ʋ̈}&\multicolumn{1}{p{\XLingPapercoliwidth}@{}}{}\\%
\multicolumn{1}{@{}p{\XLingPapercolawidth}}{D}&\multicolumn{1}{p{\XLingPapercolbwidth}}{d}&\multicolumn{1}{p{\XLingPapercolcwidth}}{}&\multicolumn{1}{p{\XLingPapercoldwidth}}{N}&\multicolumn{1}{p{\XLingPapercolewidth}}{n}&\multicolumn{1}{p{\XLingPapercolfwidth}}{}&\multicolumn{1}{p{\XLingPapercolgwidth}}{Ʋ̈ʋ̈}&\multicolumn{1}{p{\XLingPapercolhwidth}}{ʋ̈ʋ̈}&\multicolumn{1}{p{\XLingPapercoliwidth}@{}}{}\\%
\multicolumn{1}{@{}p{\XLingPapercolawidth}}{Dh}&\multicolumn{1}{p{\XLingPapercolbwidth}}{dh}&\multicolumn{1}{p{\XLingPapercolcwidth}}{}&\multicolumn{1}{p{\XLingPapercoldwidth}}{Ng}&\multicolumn{1}{p{\XLingPapercolewidth}}{ng}&\multicolumn{1}{p{\XLingPapercolfwidth}}{}&\multicolumn{1}{p{\XLingPapercolgwidth}}{\protect\footnote{{\leftskip0pt\parindent1em\raisebox{\baselineskip}[0pt]{\protect\hypertarget{nNoCapitalDipthong}{}} No uppercase version of the diphthong is presented because it is unlikely that they need to be capitalized due to phonotactics. However, if they do occur in words which are written in uppercase, then all the glyphs are already available in the writing system.}}}&\multicolumn{1}{p{\XLingPapercolhwidth}}{iʋ̈}&\multicolumn{1}{p{\XLingPapercoliwidth}@{}}{D}\\%
\multicolumn{1}{@{}p{\XLingPapercolawidth}}{E}&\multicolumn{1}{p{\XLingPapercolbwidth}}{e}&\multicolumn{1}{p{\XLingPapercolcwidth}}{}&\multicolumn{1}{p{\XLingPapercoldwidth}}{O}&\multicolumn{1}{p{\XLingPapercolewidth}}{o}&\multicolumn{1}{p{\XLingPapercolfwidth}}{}&\multicolumn{1}{p{\XLingPapercolgwidth}}{}&\multicolumn{1}{p{\XLingPapercolhwidth}}{iö}&\multicolumn{1}{p{\XLingPapercoliwidth}@{}}{D}\\%
\multicolumn{1}{@{}p{\XLingPapercolawidth}}{Ee}&\multicolumn{1}{p{\XLingPapercolbwidth}}{ee}&\multicolumn{1}{p{\XLingPapercolcwidth}}{}&\multicolumn{1}{p{\XLingPapercoldwidth}}{Oo}&\multicolumn{1}{p{\XLingPapercolewidth}}{oo}&\multicolumn{1}{p{\XLingPapercolfwidth}}{}&\multicolumn{1}{p{\XLingPapercolgwidth}}{}&\multicolumn{1}{p{\XLingPapercolhwidth}}{ië}&\multicolumn{1}{p{\XLingPapercoliwidth}@{}}{D}\\%
\multicolumn{1}{@{}p{\XLingPapercolawidth}}{Ɛ}&\multicolumn{1}{p{\XLingPapercolbwidth}}{ɛ}&\multicolumn{1}{p{\XLingPapercolcwidth}}{}&\multicolumn{1}{p{\XLingPapercoldwidth}}{Ö}&\multicolumn{1}{p{\XLingPapercolewidth}}{ö}&\multicolumn{1}{p{\XLingPapercolfwidth}}{}&\multicolumn{1}{p{\XLingPapercolgwidth}}{}&\multicolumn{1}{p{\XLingPapercolhwidth}}{ia}&\multicolumn{1}{p{\XLingPapercoliwidth}@{}}{D}\\%
\multicolumn{1}{@{}p{\XLingPapercolawidth}}{Ɛɛ}&\multicolumn{1}{p{\XLingPapercolbwidth}}{ɛɛ}&\multicolumn{1}{p{\XLingPapercolcwidth}}{}&\multicolumn{1}{p{\XLingPapercoldwidth}}{Öö}&\multicolumn{1}{p{\XLingPapercolewidth}}{öö}&\multicolumn{1}{p{\XLingPapercolfwidth}}{}&\multicolumn{1}{p{\XLingPapercolgwidth}}{}&\multicolumn{1}{p{\XLingPapercolhwidth}}{ian}&\multicolumn{1}{p{\XLingPapercoliwidth}@{}}{D}\\%
\multicolumn{1}{@{}p{\XLingPapercolawidth}}{Ɛa}&\multicolumn{1}{p{\XLingPapercolbwidth}}{ɛa}&\multicolumn{1}{p{\XLingPapercolcwidth}}{}&\multicolumn{1}{p{\XLingPapercoldwidth}}{Ɔ}&\multicolumn{1}{p{\XLingPapercolewidth}}{ɔ}&\multicolumn{1}{p{\XLingPapercolfwidth}}{}&\multicolumn{1}{p{\XLingPapercolgwidth}}{}&\multicolumn{1}{p{\XLingPapercolhwidth}}{ɩa}&\multicolumn{1}{p{\XLingPapercoliwidth}@{}}{D}\\%
\multicolumn{1}{@{}p{\XLingPapercolawidth}}{Ɛan}&\multicolumn{1}{p{\XLingPapercolbwidth}}{ɛan}&\multicolumn{1}{p{\XLingPapercolcwidth}}{}&\multicolumn{1}{p{\XLingPapercoldwidth}}{Ɔɔ}&\multicolumn{1}{p{\XLingPapercolewidth}}{ɔɔ}&\multicolumn{1}{p{\XLingPapercolfwidth}}{}&\multicolumn{1}{p{\XLingPapercolgwidth}}{}&\multicolumn{1}{p{\XLingPapercolhwidth}}{uë}&\multicolumn{1}{p{\XLingPapercoliwidth}@{}}{D}\\%
\multicolumn{1}{@{}p{\XLingPapercolawidth}}{Ɛn}&\multicolumn{1}{p{\XLingPapercolbwidth}}{ɛn}&\multicolumn{1}{p{\XLingPapercolcwidth}}{}&\multicolumn{1}{p{\XLingPapercoldwidth}}{Ɔn}&\multicolumn{1}{p{\XLingPapercolewidth}}{ɔn}&\multicolumn{1}{p{\XLingPapercolfwidth}}{}&\multicolumn{1}{p{\XLingPapercolgwidth}}{}&\multicolumn{1}{p{\XLingPapercolhwidth}}{ʋë}&\multicolumn{1}{p{\XLingPapercoliwidth}@{}}{D}\\%
\multicolumn{1}{@{}p{\XLingPapercolawidth}}{Ɛɛn}&\multicolumn{1}{p{\XLingPapercolbwidth}}{ɛɛn}&\multicolumn{1}{p{\XLingPapercolcwidth}}{}&\multicolumn{1}{p{\XLingPapercoldwidth}}{Ɔɔn}&\multicolumn{1}{p{\XLingPapercolewidth}}{ɔɔn}&\multicolumn{1}{p{\XLingPapercolfwidth}}{}&\multicolumn{1}{p{\XLingPapercolgwidth}}{}&\multicolumn{1}{p{\XLingPapercolhwidth}}{ʋ̈ü}&\multicolumn{1}{p{\XLingPapercoliwidth}@{}}{D}\\%
\multicolumn{1}{@{}p{\XLingPapercolawidth}}{Ë}&\multicolumn{1}{p{\XLingPapercolbwidth}}{ë}&\multicolumn{1}{p{\XLingPapercolcwidth}}{}&\multicolumn{1}{p{\XLingPapercoldwidth}}{P}&\multicolumn{1}{p{\XLingPapercolewidth}}{ɔn}&\multicolumn{1}{p{\XLingPapercolfwidth}}{}&\multicolumn{1}{p{\XLingPapercolgwidth}}{}&\multicolumn{1}{p{\XLingPapercolhwidth}}{ˮ — ∅}&\multicolumn{1}{p{\XLingPapercoliwidth}@{}}{T}\\%
\multicolumn{1}{@{}p{\XLingPapercolawidth}}{Ëë}&\multicolumn{1}{p{\XLingPapercolbwidth}}{ëë}&\multicolumn{1}{p{\XLingPapercolcwidth}}{}&\multicolumn{1}{p{\XLingPapercoldwidth}}{R}&\multicolumn{1}{p{\XLingPapercolewidth}}{ɔɔn}&\multicolumn{1}{p{\XLingPapercolfwidth}}{}&\multicolumn{1}{p{\XLingPapercolgwidth}}{}&\multicolumn{1}{p{\XLingPapercolhwidth}}{ʼ — ∅}&\multicolumn{1}{p{\XLingPapercoliwidth}@{}}{T}\\%
\multicolumn{1}{@{}p{\XLingPapercolawidth}}{Ën}&\multicolumn{1}{p{\XLingPapercolbwidth}}{ën}&\multicolumn{1}{p{\XLingPapercolcwidth}}{}&\multicolumn{1}{p{\XLingPapercoldwidth}}{S}&\multicolumn{1}{p{\XLingPapercolewidth}}{s}&\multicolumn{1}{p{\XLingPapercolfwidth}}{}&\multicolumn{1}{p{\XLingPapercolgwidth}}{}&\multicolumn{1}{p{\XLingPapercolhwidth}}{∅ — ∅}&\multicolumn{1}{p{\XLingPapercoliwidth}@{}}{T}\\%
\multicolumn{1}{@{}p{\XLingPapercolawidth}}{Ëën}&\multicolumn{1}{p{\XLingPapercolbwidth}}{ëën}&\multicolumn{1}{p{\XLingPapercolcwidth}}{}&\multicolumn{1}{p{\XLingPapercoldwidth}}{T}&\multicolumn{1}{p{\XLingPapercolewidth}}{t}&\multicolumn{1}{p{\XLingPapercolfwidth}}{}&\multicolumn{1}{p{\XLingPapercolgwidth}}{}&\multicolumn{1}{p{\XLingPapercolhwidth}}{꞊ — ∅}&\multicolumn{1}{p{\XLingPapercoliwidth}@{}}{T}\\%
\multicolumn{1}{@{}p{\XLingPapercolawidth}}{F}&\multicolumn{1}{p{\XLingPapercolbwidth}}{f}&\multicolumn{1}{p{\XLingPapercolcwidth}}{}&\multicolumn{1}{p{\XLingPapercoldwidth}}{U}&\multicolumn{1}{p{\XLingPapercolewidth}}{u}&\multicolumn{1}{p{\XLingPapercolfwidth}}{}&\multicolumn{1}{p{\XLingPapercolgwidth}}{}&\multicolumn{1}{p{\XLingPapercolhwidth}}{˗ — ∅}&\multicolumn{1}{p{\XLingPapercoliwidth}@{}}{T}\\%
\multicolumn{1}{@{}p{\XLingPapercolawidth}}{G}&\multicolumn{1}{p{\XLingPapercolbwidth}}{g}&\multicolumn{1}{p{\XLingPapercolcwidth}}{}&\multicolumn{1}{p{\XLingPapercoldwidth}}{Uu}&\multicolumn{1}{p{\XLingPapercolewidth}}{uu}&\multicolumn{1}{p{\XLingPapercolfwidth}}{}&\multicolumn{1}{p{\XLingPapercolgwidth}}{}&\multicolumn{1}{p{\XLingPapercolhwidth}}{ˮ — ˗}&\multicolumn{1}{p{\XLingPapercoliwidth}@{}}{T}\\%
\multicolumn{1}{@{}p{\XLingPapercolawidth}}{Gb}&\multicolumn{1}{p{\XLingPapercolbwidth}}{gb}&\multicolumn{1}{p{\XLingPapercolcwidth}}{}&\multicolumn{1}{p{\XLingPapercoldwidth}}{Un}&\multicolumn{1}{p{\XLingPapercolewidth}}{un}&\multicolumn{1}{p{\XLingPapercolfwidth}}{}&\multicolumn{1}{p{\XLingPapercolgwidth}}{}&\multicolumn{1}{p{\XLingPapercolhwidth}}{ʼ — ˗}&\multicolumn{1}{p{\XLingPapercoliwidth}@{}}{T}\\%
\multicolumn{1}{@{}p{\XLingPapercolawidth}}{Gw}&\multicolumn{1}{p{\XLingPapercolbwidth}}{gw}&\multicolumn{1}{p{\XLingPapercolcwidth}}{}&\multicolumn{1}{p{\XLingPapercoldwidth}}{Uun}&\multicolumn{1}{p{\XLingPapercolewidth}}{uun}&\multicolumn{1}{p{\XLingPapercolfwidth}}{}&\multicolumn{1}{p{\XLingPapercolgwidth}}{}&\multicolumn{1}{p{\XLingPapercolhwidth}}{∅ — ˗}&\multicolumn{1}{p{\XLingPapercoliwidth}@{}}{T}\\%
\multicolumn{1}{@{}p{\XLingPapercolawidth}}{I}&\multicolumn{1}{p{\XLingPapercolbwidth}}{i}&\multicolumn{1}{p{\XLingPapercolcwidth}}{}&\multicolumn{1}{p{\XLingPapercoldwidth}}{Ü}&\multicolumn{1}{p{\XLingPapercolewidth}}{ü}&\multicolumn{1}{p{\XLingPapercolfwidth}}{}&\multicolumn{1}{p{\XLingPapercolgwidth}}{}&\multicolumn{1}{p{\XLingPapercolhwidth}}{꞊ — ˗}&\multicolumn{1}{p{\XLingPapercoliwidth}@{}}{T}\\%
\multicolumn{1}{@{}p{\XLingPapercolawidth}}{In}&\multicolumn{1}{p{\XLingPapercolbwidth}}{in}&\multicolumn{1}{p{\XLingPapercolcwidth}}{}&\multicolumn{1}{p{\XLingPapercoldwidth}}{Üü}&\multicolumn{1}{p{\XLingPapercolewidth}}{üü}&\multicolumn{1}{p{\XLingPapercolfwidth}}{}&\multicolumn{1}{p{\XLingPapercolgwidth}}{}&\multicolumn{1}{p{\XLingPapercolhwidth}}{∅ — ʼ}&\multicolumn{1}{p{\XLingPapercoliwidth}@{}}{T}\\%
\multicolumn{1}{@{}p{\XLingPapercolawidth}}{Ii}&\multicolumn{1}{p{\XLingPapercolbwidth}}{ii}&\multicolumn{1}{p{\XLingPapercolcwidth}}{}&\multicolumn{1}{p{\XLingPapercoldwidth}}{Ün}&\multicolumn{1}{p{\XLingPapercolewidth}}{ün}&\multicolumn{1}{p{\XLingPapercolfwidth}}{}&\multicolumn{1}{p{\XLingPapercolgwidth}}{}&\multicolumn{1}{p{\XLingPapercolhwidth}}{∅ — ˮ}&\multicolumn{1}{p{\XLingPapercoliwidth}@{}}{T}\\%
\multicolumn{1}{@{}p{\XLingPapercolawidth}}{Iin}&\multicolumn{1}{p{\XLingPapercolbwidth}}{iin}&\multicolumn{1}{p{\XLingPapercolcwidth}}{}&\multicolumn{1}{p{\XLingPapercoldwidth}}{Üün}&\multicolumn{1}{p{\XLingPapercolewidth}}{üün}&\multicolumn{1}{p{\XLingPapercolfwidth}}{}&\multicolumn{1}{p{\XLingPapercolgwidth}}{}&\multicolumn{1}{p{\XLingPapercolhwidth}}{}&\multicolumn{1}{p{\XLingPapercoliwidth}@{}}{}\\\bottomrule%
\end{longtable}
}
}{\vspace{15pt}\XLingPaperneedspace{3\baselineskip}\noindent
\fontsize{13}{15.6}\selectfont \textbf{{\noindent
\raisebox{\baselineskip}[0pt]{\pdfbookmark[2]{{3.5 } Numbers}{Numbers}}\raisebox{\baselineskip}[0pt]{\protect\hypertarget{Numbers}{}}{3.5 }Numbers}}\markboth{Numbers}{Eastern Dan writing system}\XLingPaperaddtocontents{Numbers}}\par{}
\penalty10000\vspace{10pt}\penalty10000\indent Numbers are an important part of any writing system and even more so with the ubiquitous nature of mobile phone technology. \hyperlink{rBurmeisterJonathanL.1987Numbe}{Burmeister (1987)} even before the age of personal digital communication indicates that there is a strong interest in numeracy (within Côte d'Ivoire) prior to literacy. Scripts generally each have their own graphical expressions for numbers. However, just as the terminology for numbers is easily borrowed between languages, so also are the glyphs which represent those numbers. Hindu–Arabic numerals are especially invasive, and now they have the power of the various Internet protocols behind them. Many orthography statements fail to include numbers, which are an incredibly important part of culture and communication. The Eastern Dan 1994 Primer includes Hindu–Arabic numerals and these are seen in the corpus as presented in table \hyperlink{NumberCharacters}{21}.\par{}\vspace{11pt plus 2pt minus 1pt}\XLingPaperneedspace{3\baselineskip}\protect\hypertarget{NumberCharacters}{}\XLingPaperaddtocontents{NumberCharacters}{\protect\raggedright{\singlespacing
{Table }{21.}{  List of number characters used in Eastern Dan\\}}}\vspace{0pt}{\singlespacing
\hspace*{.25in}{
\XLingPaperminmaxcellincolumn{Codepoint}{\XLingPapermincola}{\textbf{Codepoint}}{\XLingPapermaxcola}{+0\tabcolsep}
\XLingPaperminmaxcellincolumn{Glyph}{\XLingPapermincolb}{\textbf{Glyph}}{\XLingPapermaxcolb}{+0\tabcolsep}
\XLingPaperminmaxcellincolumn{U+0030}{\XLingPapermincola}{U+0030}{\XLingPapermaxcola}{+0\tabcolsep}
\XLingPaperminmaxcellincolumn{0}{\XLingPapermincolb}{0}{\XLingPapermaxcolb}{+0\tabcolsep}
\XLingPaperminmaxcellincolumn{U+0031}{\XLingPapermincola}{U+0031}{\XLingPapermaxcola}{+0\tabcolsep}
\XLingPaperminmaxcellincolumn{1}{\XLingPapermincolb}{1}{\XLingPapermaxcolb}{+0\tabcolsep}
\XLingPaperminmaxcellincolumn{U+0032}{\XLingPapermincola}{U+0032}{\XLingPapermaxcola}{+0\tabcolsep}
\XLingPaperminmaxcellincolumn{2}{\XLingPapermincolb}{2}{\XLingPapermaxcolb}{+0\tabcolsep}
\XLingPaperminmaxcellincolumn{U+0033}{\XLingPapermincola}{U+0033}{\XLingPapermaxcola}{+0\tabcolsep}
\XLingPaperminmaxcellincolumn{3}{\XLingPapermincolb}{3}{\XLingPapermaxcolb}{+0\tabcolsep}
\XLingPaperminmaxcellincolumn{U+0034}{\XLingPapermincola}{U+0034}{\XLingPapermaxcola}{+0\tabcolsep}
\XLingPaperminmaxcellincolumn{4}{\XLingPapermincolb}{4}{\XLingPapermaxcolb}{+0\tabcolsep}
\XLingPaperminmaxcellincolumn{U+0035}{\XLingPapermincola}{U+0035}{\XLingPapermaxcola}{+0\tabcolsep}
\XLingPaperminmaxcellincolumn{5}{\XLingPapermincolb}{5}{\XLingPapermaxcolb}{+0\tabcolsep}
\XLingPaperminmaxcellincolumn{U+0036}{\XLingPapermincola}{U+0036}{\XLingPapermaxcola}{+0\tabcolsep}
\XLingPaperminmaxcellincolumn{6}{\XLingPapermincolb}{6}{\XLingPapermaxcolb}{+0\tabcolsep}
\XLingPaperminmaxcellincolumn{U+0037}{\XLingPapermincola}{U+0037}{\XLingPapermaxcola}{+0\tabcolsep}
\XLingPaperminmaxcellincolumn{7}{\XLingPapermincolb}{7}{\XLingPapermaxcolb}{+0\tabcolsep}
\XLingPaperminmaxcellincolumn{U+0038}{\XLingPapermincola}{U+0038}{\XLingPapermaxcola}{+0\tabcolsep}
\XLingPaperminmaxcellincolumn{8}{\XLingPapermincolb}{8}{\XLingPapermaxcolb}{+0\tabcolsep}
\XLingPaperminmaxcellincolumn{U+0039}{\XLingPapermincola}{U+0039}{\XLingPapermaxcola}{+0\tabcolsep}
\XLingPaperminmaxcellincolumn{9}{\XLingPapermincolb}{9}{\XLingPapermaxcolb}{+0\tabcolsep}
\setlength{\XLingPaperavailabletablewidth}{433.62pt}
\setlength{\XLingPapertableminwidth}{\XLingPapermincola+\XLingPapermincolb}
\setlength{\XLingPapertablemaxwidth}{\XLingPapermaxcola+\XLingPapermaxcolb}
\XLingPapercalculatetablewidthratio{}
\XLingPapersetcolumnwidth{\XLingPapercolawidth}{\XLingPapermincola}{\XLingPapermaxcola}{-0\tabcolsep}
\XLingPapersetcolumnwidth{\XLingPapercolbwidth}{\XLingPapermincolb}{\XLingPapermaxcolb}{-2\tabcolsep}\singlespacing\vspace*{-3\baselineskip}
\begin{longtable}
[l]{@{}p{\XLingPapercolawidth}p{\XLingPapercolbwidth}@{}}\toprule\multicolumn{1}{@{}p{\XLingPapercolawidth}}{\textbf{Codepoint}}&\multicolumn{1}{p{\XLingPapercolbwidth}@{}}{\textbf{Glyph}}\\%
\midrule\endhead \multicolumn{1}{@{}p{\XLingPapercolawidth}}{U+0030}&\multicolumn{1}{p{\XLingPapercolbwidth}@{}}{0}\\%
\multicolumn{1}{@{}p{\XLingPapercolawidth}}{U+0031}&\multicolumn{1}{p{\XLingPapercolbwidth}@{}}{1}\\%
\multicolumn{1}{@{}p{\XLingPapercolawidth}}{U+0032}&\multicolumn{1}{p{\XLingPapercolbwidth}@{}}{2}\\%
\multicolumn{1}{@{}p{\XLingPapercolawidth}}{U+0033}&\multicolumn{1}{p{\XLingPapercolbwidth}@{}}{3}\\%
\multicolumn{1}{@{}p{\XLingPapercolawidth}}{U+0034}&\multicolumn{1}{p{\XLingPapercolbwidth}@{}}{4}\\%
\multicolumn{1}{@{}p{\XLingPapercolawidth}}{U+0035}&\multicolumn{1}{p{\XLingPapercolbwidth}@{}}{5}\\%
\multicolumn{1}{@{}p{\XLingPapercolawidth}}{U+0036}&\multicolumn{1}{p{\XLingPapercolbwidth}@{}}{6}\\%
\multicolumn{1}{@{}p{\XLingPapercolawidth}}{U+0037}&\multicolumn{1}{p{\XLingPapercolbwidth}@{}}{7}\\%
\multicolumn{1}{@{}p{\XLingPapercolawidth}}{U+0038}&\multicolumn{1}{p{\XLingPapercolbwidth}@{}}{8}\\%
\multicolumn{1}{@{}p{\XLingPapercolawidth}}{U+0039}&\multicolumn{1}{p{\XLingPapercolbwidth}@{}}{9}\\\bottomrule%
\end{longtable}
}
}\indent \hyperlink{Hosken}{Hosken (2003)} points out that some languages have multiple set of glyphs for representing numbers\protect\footnote[24]{{\leftskip0pt\parindent1em\raisebox{\baselineskip}[0pt]{\protect\hypertarget{splitNumberSets}{}} A common split is where one set is the Hindu–Arabic numeral set and another set is part of the more local script.}}. This is not the case in Dan as far as I can tell. \hyperlink{rSternstein2008}{Sternstein (2008)} indicates that Dan has a base ten system with a with sub-base five system — meaning the Hindu–Arabic numeral system fits nicely.\par{}\indent It is useful to discuss some of the context around numeral use and where further research would be helpful. \hyperlink{rSternstein2008}{Sternstein (2008)}, who experienced Dan as it is used in Liberia, discusses units of measurement such as length, area, volume, and time as part of mathematics and makes a modest effort to describe some of the units he encountered. These units (and perhaps glyphs to represent them) are not included here because it is not clear to me if they are actually presented with enough detail to be integrated into some larger work, like a \hyperlink{gtLocale}{{\textit{Locale}}}. For a Locale description the Dan orthographical representation both numerically and as letters (word form) need to be presented. Concepts around Zero and null sets need to be expounded upon. \hyperlink{rSternstein2008}{Sternstein (2008)} mentioned three variations on null or zero but it is not entirely clear in which contexts they would be used. Also, a Locale needs cardinal, ordinal, and inflected variants of numbers in both textual and numerical expressions i.e. {\XLingPaperCambriaZMathFontFamily{\textup{\textmd{⟨ {\XLingPaperCharisZSILFontFamily{\textup{\textup{\textmd{1st, first, 1., (1)}}}}} ⟩}}}}.\par{}\indent The following observations were made based on numbers and their environments in the corpus I used.\par{}{\parskip .5pt plus 1pt minus 1pt
                    
\vspace{\baselineskip}

{\setlength{\XLingPapertempdim}{\XLingPapersingledigitlistitemwidth+\parindent{}}\leftskip\XLingPapertempdim\relax
\interlinepenalty10000
\XLingPaperlistitem{\parindent{}}{\XLingPapersingledigitlistitemwidth}{1.}{Thousands separator is {\XLingPaperCambriaZMathFontFamily{⟨ {\XLingPaperCharisZSILFontFamily{\textup{\textup{\textmd{.}}}}} ⟩}} U+002E 'FULL STOP'.}}
{\setlength{\XLingPapertempdim}{\XLingPapersingledigitlistitemwidth+\parindent{}}\leftskip\XLingPapertempdim\relax
\interlinepenalty10000
\XLingPaperlistitem{\parindent{}}{\XLingPapersingledigitlistitemwidth}{2.}{Telephone numbers are written in series of two digits. These digits can be separated with {\XLingPaperCambriaZMathFontFamily{⟨ {\XLingPaperCharisZSILFontFamily{\textup{\textup{\textmd{.}}}}} ⟩}} U+002E or spaces.}}
{\setlength{\XLingPapertempdim}{\XLingPapersingledigitlistitemwidth+\parindent{}}\leftskip\XLingPapertempdim\relax
\interlinepenalty10000
\XLingPaperlistitem{\parindent{}}{\XLingPapersingledigitlistitemwidth}{3.}{A list of numbers is separated by a comma and a space. e.g. {\XLingPaperCambriaZMathFontFamily{\textup{\textmd{⟨ {\XLingPaperCharisZSILFontFamily{\textup{\textup{\textmd{1, 2, 3}}}}} ⟩}}}}.}}
{\setlength{\XLingPapertempdim}{\XLingPapersingledigitlistitemwidth+\parindent{}}\leftskip\XLingPapertempdim\relax
\interlinepenalty10000
\XLingPaperlistitem{\parindent{}}{\XLingPapersingledigitlistitemwidth}{4.}{There is a shortened form of the word "number" in many transcription traditions. Unicode has a special character for this {\XLingPaperCambriaZMathFontFamily{⟨ {\XLingPaperCharisZSILFontFamily{\textup{\textup{\textmd{№}}}}} ⟩}} U+2116 'NUMERO SIGN'. Typographical practice in Dan appear to follows French social practice, rather than best practice for encoding. This was evidenced only one time in the corpus and is the source of {\XLingPaperCambriaZMathFontFamily{⟨ {\XLingPaperCharisZSILFontFamily{\textup{\textup{\textmd{°}}}}} ⟩}} U+00B0 'DEGREE SIGN', and likely deserves further investigation before strong claims are made about what method should be used in Eastern Dan writing.\protect\footnote[25]{{\leftskip0pt\parindent1em\raisebox{\baselineskip}[0pt]{\protect\hypertarget{nNumeroSign}{}} \hyperlink{rWikiNumeroSign}{Wikipedia (2018)} suggests that: "the numero symbol is not in common use in France and does not appear on a standard AZERTY keyboard. Instead, the French Imprimerie nationale recommends the use of the form {\XLingPaperCambriaZMathFontFamily{\textup{\textmd{⟨ {\XLingPaperCharisZSILFontFamily{\textup{\textup{\textmd{no}}}}} ⟩}}}} (an {\XLingPaperCambriaZMathFontFamily{\textup{\textmd{⟨ {\XLingPaperCharisZSILFontFamily{\textup{\textup{\textmd{n}}}}} ⟩}}}} followed by a superscript lowercase {\XLingPaperCambriaZMathFontFamily{\textup{\textmd{⟨ {\XLingPaperCharisZSILFontFamily{\textup{\textup{\textmd{o}}}}} ⟩}}}}). The plural form {\XLingPaperCambriaZMathFontFamily{\textup{\textmd{⟨ {\XLingPaperCharisZSILFontFamily{\textup{\textup{\textmd{nos}}}}} ⟩}}}} can also be used. In practice, the {\XLingPaperCambriaZMathFontFamily{\textup{\textmd{⟨ {\XLingPaperCharisZSILFontFamily{\textup{\textup{\textmd{o}}}}} ⟩}}}} is often replaced by the degree symbol {\XLingPaperCambriaZMathFontFamily{\textup{\textmd{⟨ {\XLingPaperCharisZSILFontFamily{\textup{\textup{\textmd{°}}}}} ⟩}}}}, which is visually similar to the superscript {\XLingPaperCambriaZMathFontFamily{\textup{\textmd{⟨ {\XLingPaperCharisZSILFontFamily{\textup{\textup{\textmd{o}}}}} ⟩}}}} and is easily accessible on an AZERTY keyboard." If the Numero sign is a glyph or function which is frequently occurring in Côte d'Ivorian literacy culture, then the proper typographical access should be considered when designing text input solutions.}}}}
\vspace{\baselineskip}
}\indent Left undressed are issues like the representation of money units, fractions, and decimal units. Presumably Dan would follow Côte d'Ivorian norms (if these differ from those of French expression in France), but this is not guaranteed as Dan is also used in Liberia, and would equally need access to units of measure, money, and time as is generally expressed in the national context of Liberia.\par{}{\vspace{15pt}\XLingPaperneedspace{3\baselineskip}\noindent
\fontsize{13}{15.6}\selectfont \textbf{{\noindent
\raisebox{\baselineskip}[0pt]{\pdfbookmark[2]{{3.6 } Punctuation}{Punctuation}}\raisebox{\baselineskip}[0pt]{\protect\hypertarget{Punctuation}{}}{3.6 }Punctuation}}\markboth{Punctuation}{Eastern Dan writing system}\XLingPaperaddtocontents{Punctuation}}\par{}
\penalty10000\vspace{10pt}\penalty10000\indent \hyperlink{rBolli1994Cours}{Bolli \& Flik (1994)} presents the punctuation and usage explanations in table \hyperlink{PunctuationCharacters}{22} for Eastern Dan. It is a great application of basic French punctuation to an indigenous language.\par{}\vspace{11pt plus 2pt minus 1pt}\XLingPaperneedspace{3\baselineskip}\protect\hypertarget{PunctuationCharacters}{}\XLingPaperaddtocontents{PunctuationCharacters}{\protect\raggedright{\singlespacing
{Table }{22.}{  List of punctuation characters used in Eastern Dan\\}}}\vspace{0pt}{\singlespacing
\hspace*{.25in}{\setcounter{footnote}{25}
\XLingPaperminmaxcellincolumn{Codepoint}{\XLingPapermincola}{\textbf{Codepoint}}{\XLingPapermaxcola}{+0\tabcolsep}
\XLingPaperminmaxcellincolumn{Glyph}{\XLingPapermincolb}{\textbf{Glyph}}{\XLingPapermaxcolb}{+0\tabcolsep}
\XLingPaperminmaxcellincolumn{Usage}{\XLingPapermincolc}{\textbf{Usage}}{\XLingPapermaxcolc}{+0\tabcolsep}
\XLingPaperminmaxcellincolumn{U+00AB}{\XLingPapermincola}{U+00AB}{\XLingPapermaxcola}{+0\tabcolsep}
\XLingPaperminmaxcellincolumn{«}{\XLingPapermincolb}{«}{\XLingPapermaxcolb}{+0\tabcolsep}
\XLingPaperminmaxcellincolumn{indicator}{\XLingPapermincolc}{\vbox{\hbox{\strut{}{les guillemets ouvrant et} }\hbox{\strut{}(tr. \textsquarebracketleft{}eng\textsquarebracketright{}: opening indicator for marking a quote)}}}{\XLingPapermaxcolc}{+0\tabcolsep}
\XLingPaperminmaxcellincolumn{U+00BB}{\XLingPapermincola}{U+00BB}{\XLingPapermaxcola}{+0\tabcolsep}
\XLingPaperminmaxcellincolumn{»}{\XLingPapermincolb}{»}{\XLingPapermaxcolb}{+0\tabcolsep}
\XLingPaperminmaxcellincolumn{discourse}{\XLingPapermincolc}{\vbox{\hbox{\strut{}{fermant un discourse direct} }\hbox{\strut{}(tr. \textsquarebracketleft{}eng\textsquarebracketright{}: closing indicator for marking a quote)}}}{\XLingPapermaxcolc}{+0\tabcolsep}
\XLingPaperminmaxcellincolumn{U+0021}{\XLingPapermincola}{U+0021}{\XLingPapermaxcola}{+0\tabcolsep}
\XLingPaperminmaxcellincolumn{!}{\XLingPapermincolb}{!}{\XLingPapermaxcolb}{+0\tabcolsep}
\XLingPaperminmaxcellincolumn{following}{\XLingPapermincolc}{\vbox{\hbox{\strut{}{le point d'interrogation marque la présence d'une exclamation} }\hbox{\strut{}(tr. \textsquarebracketleft{}eng\textsquarebracketright{}: following an exclamation)}}}{\XLingPapermaxcolc}{+0\tabcolsep}
\XLingPaperminmaxcellincolumn{U+003B}{\XLingPapermincola}{U+003B}{\XLingPapermaxcola}{+0\tabcolsep}
\XLingPaperminmaxcellincolumn{;}{\XLingPapermincolb}{;}{\XLingPapermaxcolb}{+0\tabcolsep}
\XLingPaperminmaxcellincolumn{phrases)}{\XLingPapermincolc}{\vbox{\hbox{\strut{}{\vbox{\hbox{\strut{}le point-virgule entrecoupe deux parties d'une longue phrasele}\hbox{\strut{}point-virgule entrecoupe deux parties d'une longue phrase}}} }\hbox{\strut{}(tr. \textsquarebracketleft{}eng\textsquarebracketright{}: joins two long phrases)}}}{\XLingPapermaxcolc}{+0\tabcolsep}
\XLingPaperminmaxcellincolumn{U+003C}{\XLingPapermincola}{U+003C}{\XLingPapermaxcola}{+0\tabcolsep}
\XLingPaperminmaxcellincolumn{described}{\XLingPapermincolb}{\textless{}}{\XLingPapermaxcolb}{+0\tabcolsep}
\XLingPaperminmaxcellincolumn{indicator}{\XLingPapermincolc}{\vbox{\hbox{\strut{}{les guillemets simples ouvrant et} }\hbox{\strut{}(tr. \textsquarebracketleft{}eng\textsquarebracketright{}: opening indicator for marking a quote inside a quote)}}}{\XLingPapermaxcolc}{+0\tabcolsep}
\XLingPaperminmaxcellincolumn{U+003E}{\XLingPapermincola}{U+003E}{\XLingPapermaxcola}{+0\tabcolsep}
\XLingPaperminmaxcellincolumn{\textgreater{}}{\XLingPapermincolb}{\textgreater{}}{\XLingPapermaxcolb}{+0\tabcolsep}
\XLingPaperminmaxcellincolumn{discourse}{\XLingPapermincolc}{\vbox{\hbox{\strut{}{fermant un discourse direct placé dans un autre discourse direct} }\hbox{\strut{}(tr. \textsquarebracketleft{}eng\textsquarebracketright{}: closing indicator for marking a quote inside a quote)}}}{\XLingPapermaxcolc}{+0\tabcolsep}
\XLingPaperminmaxcellincolumn{U+003F}{\XLingPapermincola}{U+003F}{\XLingPapermaxcola}{+0\tabcolsep}
\XLingPaperminmaxcellincolumn{?}{\XLingPapermincolb}{?}{\XLingPapermaxcolb}{+0\tabcolsep}
\XLingPaperminmaxcellincolumn{following}{\XLingPapermincolc}{\vbox{\hbox{\strut{}{le point d'interrogation marque la présence d'une question}}\hbox{\strut{}(tr. \textsquarebracketleft{}eng\textsquarebracketright{}: following a question)}}}{\XLingPapermaxcolc}{+0\tabcolsep}
\XLingPaperminmaxcellincolumn{U+002E}{\XLingPapermincola}{U+002E}{\XLingPapermaxcola}{+0\tabcolsep}
\XLingPaperminmaxcellincolumn{.}{\XLingPapermincolb}{.}{\XLingPapermaxcolb}{+0\tabcolsep}
\XLingPaperminmaxcellincolumn{finishing}{\XLingPapermincolc}{\vbox{\hbox{\strut{}{le point marquant la fin d'une pensée}}\hbox{\strut{}(tr. \textsquarebracketleft{}eng\textsquarebracketright{}: finishing a thought)}}}{\XLingPapermaxcolc}{+0\tabcolsep}
\XLingPaperminmaxcellincolumn{U+002C}{\XLingPapermincola}{U+002C}{\XLingPapermaxcola}{+0\tabcolsep}
\XLingPaperminmaxcellincolumn{,}{\XLingPapermincolb}{,}{\XLingPapermaxcolb}{+0\tabcolsep}
\XLingPaperminmaxcellincolumn{virgule}{\XLingPapermincolc}{\vbox{\hbox{\strut{}{la virgule donne l'occasion de prendre haleine}}\hbox{\strut{}(tr. \textsquarebracketleft{}eng\textsquarebracketright{}: taking a breath)}}}{\XLingPapermaxcolc}{+0\tabcolsep}
\XLingPaperminmaxcellincolumn{U+003A}{\XLingPapermincola}{U+003A}{\XLingPapermaxcola}{+0\tabcolsep}
\XLingPaperminmaxcellincolumn{:}{\XLingPapermincolb}{:}{\XLingPapermaxcolb}{+0\tabcolsep}
\XLingPaperminmaxcellincolumn{discourse}{\XLingPapermincolc}{\vbox{\hbox{\strut{}{le double point marque le début d'un discourse direct}}\hbox{\strut{}(tr. \textsquarebracketleft{}eng\textsquarebracketright{}: marking the start of a quote)}}}{\XLingPapermaxcolc}{+0\tabcolsep}
\setlength{\XLingPaperavailabletablewidth}{433.62pt}
\setlength{\XLingPapertableminwidth}{\XLingPapermincola+\XLingPapermincolb+\XLingPapermincolc}
\setlength{\XLingPapertablemaxwidth}{\XLingPapermaxcola+\XLingPapermaxcolb+\XLingPapermaxcolc}
\XLingPapercalculatetablewidthratio{}
\XLingPapersetcolumnwidth{\XLingPapercolawidth}{\XLingPapermincola}{\XLingPapermaxcola}{-0\tabcolsep}
\XLingPapersetcolumnwidth{\XLingPapercolbwidth}{\XLingPapermincolb}{\XLingPapermaxcolb}{-2\tabcolsep}
\XLingPapersetcolumnwidth{\XLingPapercolcwidth}{\XLingPapermincolc}{\XLingPapermaxcolc}{-2\tabcolsep}\setcounter{footnote}{25}\singlespacing\vspace*{-3\baselineskip}
\begin{longtable}
[l]{@{}p{\XLingPapercolawidth}p{\XLingPapercolbwidth}p{\XLingPapercolcwidth}@{}}\toprule\multicolumn{1}{@{}p{\XLingPapercolawidth}}{\textbf{Codepoint}}&\multicolumn{1}{p{\XLingPapercolbwidth}}{\textbf{Glyph}}&\multicolumn{1}{p{\XLingPapercolcwidth}@{}}{\textbf{Usage}}\\%
\midrule\endhead \multicolumn{1}{@{}p{\XLingPapercolawidth}}{U+00AB}&\multicolumn{1}{p{\XLingPapercolbwidth}}{«}&\multicolumn{1}{p{\XLingPapercolcwidth}@{}}{\vbox{\hbox{\strut{}{les guillemets ouvrant et} }\hbox{\strut{}(tr. \textsquarebracketleft{}eng\textsquarebracketright{}: opening indicator for marking a quote)}}}\\%
\multicolumn{1}{@{}p{\XLingPapercolawidth}}{U+00BB}&\multicolumn{1}{p{\XLingPapercolbwidth}}{»}&\multicolumn{1}{p{\XLingPapercolcwidth}@{}}{\vbox{\hbox{\strut{}{fermant un discourse direct} }\hbox{\strut{}(tr. \textsquarebracketleft{}eng\textsquarebracketright{}: closing indicator for marking a quote)}}}\\%
\multicolumn{1}{@{}p{\XLingPapercolawidth}}{U+0021}&\multicolumn{1}{p{\XLingPapercolbwidth}}{!}&\multicolumn{1}{p{\XLingPapercolcwidth}@{}}{\vbox{\hbox{\strut{}{le point d'interrogation marque la présence d'une exclamation} }\hbox{\strut{}(tr. \textsquarebracketleft{}eng\textsquarebracketright{}: following an exclamation)}}}\\%
\multicolumn{1}{@{}p{\XLingPapercolawidth}}{U+003B}&\multicolumn{1}{p{\XLingPapercolbwidth}}{;}&\multicolumn{1}{p{\XLingPapercolcwidth}@{}}{\vbox{\hbox{\strut{}{\vbox{\hbox{\strut{}le point-virgule entrecoupe deux parties d'une longue phrasele}\hbox{\strut{}point-virgule entrecoupe deux parties d'une longue phrase}}} }\hbox{\strut{}(tr. \textsquarebracketleft{}eng\textsquarebracketright{}: joins two long phrases)}}}\\%
\multicolumn{1}{@{}p{\XLingPapercolawidth}}{U+003C}&\multicolumn{1}{p{\XLingPapercolbwidth}}{\textless{}\protect\footnote{{\leftskip0pt\parindent1em\raisebox{\baselineskip}[0pt]{\protect\hypertarget{nIncorectAngleBrackets}{}} My impression is that {\XLingPaperCambriaZMathFontFamily{\textup{\textmd{⟨ {\XLingPaperCharisZSILFontFamily{\textup{\textup{\textmd{\textless{}}}}}} ⟩}}}} U+003C and {\XLingPaperCambriaZMathFontFamily{\textup{\textmd{⟨ {\XLingPaperCharisZSILFontFamily{\textup{\textup{\textmd{\textgreater{}}}}}} ⟩}}}} U+003E should be {\XLingPaperCambriaZMathFontFamily{\textup{\textmd{⟨ {\XLingPaperCharisZSILFontFamily{\textup{\textup{\textmd{‹}}}}} ⟩}}}} U+2039 and {\XLingPaperCambriaZMathFontFamily{\textup{\textmd{⟨ {\XLingPaperCharisZSILFontFamily{\textup{\textup{\textmd{›}}}}} ⟩}}}} U+203A. U+2039 and U+203A are the proper characters in Unicode for the function described by Bolli and Flik. The misused glyphs in typesetting and typing is understandable given the text input challenge.}}}&\multicolumn{1}{p{\XLingPapercolcwidth}@{}}{\vbox{\hbox{\strut{}{les guillemets simples ouvrant et} }\hbox{\strut{}(tr. \textsquarebracketleft{}eng\textsquarebracketright{}: opening indicator for marking a quote inside a quote)}}}\\%
\multicolumn{1}{@{}p{\XLingPapercolawidth}}{U+003E}&\multicolumn{1}{p{\XLingPapercolbwidth}}{\textgreater{}}&\multicolumn{1}{p{\XLingPapercolcwidth}@{}}{\vbox{\hbox{\strut{}{fermant un discourse direct placé dans un autre discourse direct} }\hbox{\strut{}(tr. \textsquarebracketleft{}eng\textsquarebracketright{}: closing indicator for marking a quote inside a quote)}}}\\%
\multicolumn{1}{@{}p{\XLingPapercolawidth}}{U+003F}&\multicolumn{1}{p{\XLingPapercolbwidth}}{?}&\multicolumn{1}{p{\XLingPapercolcwidth}@{}}{\vbox{\hbox{\strut{}{le point d'interrogation marque la présence d'une question}}\hbox{\strut{}(tr. \textsquarebracketleft{}eng\textsquarebracketright{}: following a question)}}}\\%
\multicolumn{1}{@{}p{\XLingPapercolawidth}}{U+002E}&\multicolumn{1}{p{\XLingPapercolbwidth}}{.}&\multicolumn{1}{p{\XLingPapercolcwidth}@{}}{\vbox{\hbox{\strut{}{le point marquant la fin d'une pensée}}\hbox{\strut{}(tr. \textsquarebracketleft{}eng\textsquarebracketright{}: finishing a thought)}}}\\%
\multicolumn{1}{@{}p{\XLingPapercolawidth}}{U+002C}&\multicolumn{1}{p{\XLingPapercolbwidth}}{,}&\multicolumn{1}{p{\XLingPapercolcwidth}@{}}{\vbox{\hbox{\strut{}{la virgule donne l'occasion de prendre haleine}}\hbox{\strut{}(tr. \textsquarebracketleft{}eng\textsquarebracketright{}: taking a breath)}}}\\%
\multicolumn{1}{@{}p{\XLingPapercolawidth}}{U+003A}&\multicolumn{1}{p{\XLingPapercolbwidth}}{:}&\multicolumn{1}{p{\XLingPapercolcwidth}@{}}{\vbox{\hbox{\strut{}{le double point marque le début d'un discourse direct}}\hbox{\strut{}(tr. \textsquarebracketleft{}eng\textsquarebracketright{}: marking the start of a quote)}}}\\\bottomrule%
\end{longtable}
}
}\indent There are no clear guides or rules for what punctuation a language needs. The literature specifically dedicated to the process of selecting punctuation in newly created orthographies for previously unwritten languages is sparse. Looking at language documentation literature, some authors \hyperlink{rGrenobleLenoreALindsayJWhaley2006Ortho}{(Grenoble \& Whaley  2006:142, 157}, \hyperlink{rSeifartFrank2006Ortho}{Seifart  2006:277)}, \hyperlink{rCahillKaren2008}{Cahill \& Karan  2008}, \hyperlink{Jany}{Jany  2010:251}, and \hyperlink{rLFCpke}{Lüpke  2011:322, 334)} acknowledge that issues like word formation and punctuation need to be addressed, but they reserve discussing the points with any detail; choosing rather to treat the issue as an advanced topic. Other authors \hyperlink{GuE9rin}{(Guérin  2008}, \hyperlink{rChelliahShobhanaLakshmiWillemJosephdeReuse2011Handb}{Chelliah \& de Reuse  2011}, \hyperlink{rRastogiKavita20151231RajiO}{Rastogi  2015)} fail to mention punctuation usage at all. Punctuation as a topic sees a little more discussion in literacy and pedagogical venues. \hyperlink{rLarson1976Punct}{Larson (1976)} argues from experience with Aguaruna {\textsquarebracketleft{}agr\textsquarebracketright{}}, that a language might not need more than breath or pause marks; mapping the role of punctuation to the phonological phrase groups of a language. \hyperlink{rParks1993}{Parkes (1993)} comes to a similar position in his treatment of the historical usage and development of punctuation in Graeco-Roman, and later Latin scripts of western Europe. Parkes makes an interesting observation though, in that the need for punctuation rises as the users of a language increase who can not natively detect intended phrase groupings. While the story of Latin punctuation developed over centuries, orthographies which are designed from the start to be pan-lectical may be targeting a community of users which already represents a significant diversity  in their phonological phrase groupings.\par{}\indent In contrast to Larson, \hyperlink{rKoffi1995Indig}{Koffi (1995)} agues for a strong approach to marking grammar and suggests that languages (speaking from an African perspective) only need the following six punctuation symbols to accomplish this: {\XLingPaperCambriaZMathFontFamily{\textup{\textmd{⟨ {\XLingPaperCharisZSILFontFamily{\textup{\textup{\textmd{,}}}}} ⟩}}}}, {\XLingPaperCambriaZMathFontFamily{\textup{\textmd{⟨ {\XLingPaperCharisZSILFontFamily{\textup{\textup{\textmd{.}}}}} ⟩}}}}, {\XLingPaperCambriaZMathFontFamily{\textup{\textmd{⟨ {\XLingPaperCharisZSILFontFamily{\textup{\textup{\textmd{:}}}}} ⟩}}}}, {\XLingPaperCambriaZMathFontFamily{\textup{\textmd{⟨ {\XLingPaperCharisZSILFontFamily{\textup{\textup{\textmd{!}}}}} ⟩}}}}, {\XLingPaperCambriaZMathFontFamily{\textup{\textmd{⟨ {\XLingPaperCharisZSILFontFamily{\textup{\textup{\textmd{?}}}}} ⟩}}}}, {\XLingPaperCambriaZMathFontFamily{\textup{\textmd{⟨ {\XLingPaperCharisZSILFontFamily{\textup{\textup{\textmd{" "}}}}} ⟩}}}}\protect\footnote[27]{{\leftskip0pt\parindent1em\raisebox{\baselineskip}[0pt]{\protect\hypertarget{nQuoteMarks}{}} Koffi, does not specify what kind of quote marks. In Dan, since tone is marked with apostrophe like characters, guillemets {\XLingPaperCambriaZMathFontFamily{\textup{\textmd{⟨ {\XLingPaperCharisZSILFontFamily{\textup{\textup{\textmd{« , »}}}}} ⟩}}}} or French quotes which lower the quote marks to the mid-line of the character stream is important visual distinction decreasing ambiguity between tone marks and quote marks.}}. Koffi's suggestion seems to be the prevailing thought as demonstrated in the various language specific orthography guides which SIL produces in Cameroon (e.g. \hyperlink{rDavison2009Wehor}{Davison  2009}, \hyperlink{rAnderson2013IsuOr}{Anderson  2013)}.\par{}\indent To determine if a language needs more or less punctuation one needs to look back to the purpose of punctuation, which as it has been discussed to this point has been to create clear oral readings. However, not only are orthographies being designed to be pan-lectial, their use is now in context where oral readings are not be assumed! The ubiquitous nature of mobile communication has changed the nature of the kinds of written content,and the way it is read.  \par{}\indent A writing system needs glyphs. An orthography statement needs correct examples of usage with interlinear grammatical notation. Pedagogical instruments need clear examples and analogies.\par{}\indent Punctuation is both a class of glyphs and at the same time several systems of combining those glyphs with other sets of glyphs in the writing system. This definition veers from long standing definitions of punctuation as held in the language development industry. For instance, \hyperlink{rLarson1976Punct}{Larson (1976)} argues that punctuation should be free from the dictates of the prescriptive rules of national languages, and should be used to help the readers find the breath groups necessary to produce clear readings. The assumption is that these breath groups will, when properly expressed, provide an unambiguous grammatical reading. This is a reasonable assumption if the social context of reading from paper objects. It begins to become more challenging when writing – especially in multi-lectical contexts where prosody is a dialectical feature. It completely breaks down when we move the writing context to mobile digital communications where the text is "read" but not uttered.\par{}\indent The function of punctuation is to assist readers to experience the rhythm of a text; rhythm of the dynamic communicative stream, rhythm between conceptual groups (clause internal, sentence internal, and narrative internal), rhythm between visual groups, and rhythm between aural groups. One important area where written text fails to display the dynamic of face-to-face orality is in the expression of emotion. This "reading" of emotion, in face-to-face communication is important to the successful communicative event.\par{}\indent Therefore, punctuation is one sub-system of a writing system which needs to be defined in three different ways: (1) the components of the system, (2) the unique function and peculiarities of the intended meaning of the components, and (3) the one or more ways in which the interarrangment of punctuation works as a system to express meaning. Unlike the representation of regular sounds in a language (though punctuation may also prompt or impact the use of vocalizations), punctuation is an open set of symbols. These symbols have a systematic usage. To illustrate the third part of how a punctuation system needs to be described, I'll illustrate with {\XLingPaperCambriaZMathFontFamily{⟨ {\XLingPaperDejaVuZSerifFontFamily{‽}} ⟩}} U+203D 'INTERROBANG'. Interrobang is an infrequently used punctuation mark in English. Though when shown to any English writer, they would intuitively know that it belongs at the end of a sentence; and any Spanish writer would know to look for an additional but inverted version to place at the beginning of the sentence. In this way the "system" of punctuation has a normalcy about it which is independent of the characters which might be used.\par{}\indent Globally and historically writing systems have had different kinds and different quantities of punctuation marks. This is to say that, not all languages in their written form use commas (or any particular mark).For instance, many languages don't use question marks because they have grammatical particle which indicates a question\protect\footnote[28]{{\leftskip0pt\parindent1em\raisebox{\baselineskip}[0pt]{\protect\hypertarget{nUtmainTone}{}} ut-Ma'in {\textsquarebracketleft{}gel\textsquarebracketright{}}, a Kanji language of Nigeria is a tonal language and also has a particle at the end of the language. In some cases the tone at the beginning of the sentence changes because of the question. So, some sort of punctuation at the beginning of the sentence is needed in addition to the sentence final grammatical particle.}}. The application of punctuation marks, even within a single language, is one area of textual representation which is highly subject to variation. Within print media many publishing houses determine their own methods for applying punctuation including matching the punctuation to grammatical features and which graphical representations of punctuation are permissible. Within social media and SMS, punctuation can take on a very different set of connotations. For instance, Norwegian receivers of text messages perceive messages with periods {\XLingPaperCambriaZMathFontFamily{\textup{\textmd{⟨ . ⟩}}}} at the end of them as being more rude, this behavior is assumed because the act of sending the SMS is indicative that the sentence is finished\hyperlink{r}{ ()}. A second method in which punctuation is communicative, and which is more challenging to describe is the use of typographical emoticons (not to be confused with emojis {\XLingPaperCambriaZMathFontFamily{\textup{\textmd{⟨}}}} \vspace*{1pt}\leavevmode{}{\XeTeXpicfile "../Resources/smile.png" scaled 75}{\XLingPaperCambriaZMathFontFamily{\textup{\textmd{⟩}}}} )\hyperlink{r}{ ()}. The description of typographical emoticons such as {\XLingPaperCambriaZMathFontFamily{\textup{\textmd{⟨ :-) ⟩}}}}, {\XLingPaperCambriaZMathFontFamily{\textup{\textmd{⟨ \textgreater{}\textless{}\textgreater{} ⟩}}}}, or {\XLingPaperCambriaZMathFontFamily{\textup{\textmd{⟨ :-P ⟩}}}} are perhaps safe to understand FRENCH \hyperlink{r}{ ()}.\par{}\indent of In this sense, punctuation may sometimes best be viewed as belonging to the domain of the writing system and the application of punctuation to the sytlesheet – a component of writing below the orthography. Never-the-less, punctuation is needed and used in both pedagogy and digital support tools. \hyperlink{rBolli1994Cours}{Bolli \& Flik (1994)} present the content of table \hyperlink{PunctuationCharacters}{22} as Eastern Dan punctuation.\par{}\indent From a writing systems perspective, all that is needed are the glyphs, however, from the perspective of developing resources for under-resourced languages a full description of how glyphs are used is needed. This should include references to their intended usage and the intended meaning, and the grammatical or phrasal constructs which invoke their use.\par{}\indent \hyperlink{rPavalanathan}{Pavalanathan \& Eisenstein (2016)} {\XLingPaperCambriaZMathFontFamily{\textup{\textmd{⟨ {\XLingPaperCharisZSILFontFamily{\textup{\textup{\textmd{¯\textbackslash{}\_({\XLingPaperSimSunFontFamily{\fontsize{12}{14.399999999999999}\selectfont ツ}})\_/¯}}}}} ⟩}}}}{\XLingPaperCambriaZMathFontFamily{\textup{\textmd{⟨ {\XLingPaperCharisZSILFontFamily{\textup{\textup{\textmd{}}}}} ⟩}}}}\par{}{\vspace{15pt}\XLingPaperneedspace{3\baselineskip}\noindent
\fontsize{13}{15.6}\selectfont \textbf{{\noindent
\raisebox{\baselineskip}[0pt]{\pdfbookmark[2]{{3.7 } Internet characters}{sInternetUnicodeCharacters}}\raisebox{\baselineskip}[0pt]{\protect\hypertarget{sInternetUnicodeCharacters}{}}{3.7 }Internet characters}}\markboth{Internet characters}{Eastern Dan writing system}\XLingPaperaddtocontents{sInternetUnicodeCharacters}}\par{}
\penalty10000\vspace{10pt}\penalty10000\indent According to RFC 3986 \hyperlink{rURISyntax2005}{(Berners-Lee et al.  2005)} the following characters are needed for reasonable Internet use in the URL and URI syntax. In the Internet domain these characters can sometimes have a reserved meaning. Therefore they should be given appropriate consideration in all orthographies. So while their typographical function may or may not be present in the everyday writing of Eastern Dan, as Eastern Dan speakers become more digitally active with their language, these characters will increase in their usage by Eastern Dan language users. This does not preclude any language based denotation that the orthography may make on these characters. For instance there is a long typographical history in Eastern Dan of using {\XLingPaperCambriaZMathFontFamily{\textup{\textmd{⟨ {\XLingPaperCharisZSILFontFamily{\textup{\textup{\textmd{=}}}}} ⟩}}}} U+003D 'EQUALS SIGN' as a tone marking character. It is even the case that the original text of this corpus was encoded with this character, no doubt for practical reasons of keyboard accessibility. However the more appropriate character is {\XLingPaperCambriaZMathFontFamily{\textup{\textmd{⟨ {\XLingPaperCharisZSILFontFamily{\textup{\textup{\textmd{꞊}}}}} ⟩}}}} U+A78A 'MODIFIER LETTER SHORT EQUALS SIGN'. Typographically across fonts, it is common that these characters appear the same, however their Unicode properties are different. U+A78A can not be substituted for Internet use and U+003D will not properly join with other text to form words in text processing software. By way of analogy, just because the Internet does not use the same quote marks that French and Eastern Dan do, does not mean that these languages need to change, only that accessing these characters and their social contribution is a needed consideration in orthography statements and written language development. Unmentioned in RFC 3986 is the use of {\XLingPaperCambriaZMathFontFamily{\textup{\textmd{⟨ {\XLingPaperCharisZSILFontFamily{\textup{\textup{\textmd{"}}}}} ⟩}}}} U+0022 'QUOTATION MARK', {\XLingPaperCambriaZMathFontFamily{\textup{\textmd{⟨ {\XLingPaperCharisZSILFontFamily{\textup{\textup{\textmd{\textgreater{}}}}}} ⟩}}}} U+003E 'GREATER-THAN SIGN', and {\XLingPaperCambriaZMathFontFamily{\textup{\textmd{⟨ {\XLingPaperCharisZSILFontFamily{\textup{\textup{\textmd{\textless{}}}}}} ⟩}}}} U+003C 'GREATER-THAN SIGN' which are all highly important in various mark-ups like HTML5 \hyperlink{rHTML5}{(W3C  2017)}. Markdown \hyperlink{rGFM2017}{(GitHub Engineering  2017)}, a common text markup language, requires {\XLingPaperCambriaZMathFontFamily{\textup{\textmd{⟨ {\XLingPaperCharisZSILFontFamily{\textup{\textup{\textmd{`}}}}} ⟩}}}} U+0060 'GRAVE ACCENT', {\XLingPaperCambriaZMathFontFamily{\textup{\textmd{⟨ {\XLingPaperCharisZSILFontFamily{\textup{\textup{\textmd{\textbar{}}}}}} ⟩}}}} U+007C 'VERTICAL LINE', and {\XLingPaperCambriaZMathFontFamily{\textup{\textmd{⟨ {\XLingPaperCharisZSILFontFamily{\textup{\textup{\textmd{\textbackslash{}}}}}} ⟩}}}} U+005C 'REVERSE SOLIDUS'. The following table represents RFC 3986 plus {\XLingPaperCambriaZMathFontFamily{\textup{\textmd{⟨ {\XLingPaperCharisZSILFontFamily{\textup{\textup{\textmd{", \textless{}, \textgreater{}, \textbar{}, `, \textbackslash{} }}}}} ⟩}}}}. Many of these characters are evidenced in the corpus. However some are not evidenced. \par{}\vspace{11pt plus 2pt minus 1pt}\XLingPaperneedspace{3\baselineskip}\protect\hypertarget{InternetUnicodeCharacters}{}\XLingPaperaddtocontents{InternetUnicodeCharacters}{\protect\raggedright{\singlespacing
{Table }{23.}{  List of Internet characters which need to be accessible in Eastern Dan\\}}}\vspace{0pt}{\singlespacing
\hspace*{.25in}{
\XLingPaperminmaxcellincolumn{Codepoint}{\XLingPapermincola}{\textbf{Codepoint}}{\XLingPapermaxcola}{+0\tabcolsep}
\XLingPaperminmaxcellincolumn{Glyph}{\XLingPapermincolb}{\textbf{Glyph}}{\XLingPapermaxcolb}{+0\tabcolsep}
\XLingPaperminmaxcellincolumn{Corpus}{\XLingPapermincolc}{\textbf{Corpus count}}{\XLingPapermaxcolc}{+0\tabcolsep}
\XLingPaperminmaxcellincolumn{U+0021}{\XLingPapermincola}{U+0021}{\XLingPapermaxcola}{+0\tabcolsep}
\XLingPaperminmaxcellincolumn{!}{\XLingPapermincolb}{!}{\XLingPapermaxcolb}{+0\tabcolsep}
\XLingPaperminmaxcellincolumn{}{\XLingPapermincolc}{}{\XLingPapermaxcolc}{+0\tabcolsep}
\XLingPaperminmaxcellincolumn{U+0022}{\XLingPapermincola}{U+0022}{\XLingPapermaxcola}{+0\tabcolsep}
\XLingPaperminmaxcellincolumn{"}{\XLingPapermincolb}{"}{\XLingPapermaxcolb}{+0\tabcolsep}
\XLingPaperminmaxcellincolumn{}{\XLingPapermincolc}{}{\XLingPapermaxcolc}{+0\tabcolsep}
\XLingPaperminmaxcellincolumn{U+0023}{\XLingPapermincola}{U+0023}{\XLingPapermaxcola}{+0\tabcolsep}
\XLingPaperminmaxcellincolumn{\#}{\XLingPapermincolb}{\#}{\XLingPapermaxcolb}{+0\tabcolsep}
\XLingPaperminmaxcellincolumn{}{\XLingPapermincolc}{}{\XLingPapermaxcolc}{+0\tabcolsep}
\XLingPaperminmaxcellincolumn{U+0024}{\XLingPapermincola}{U+0024}{\XLingPapermaxcola}{+0\tabcolsep}
\XLingPaperminmaxcellincolumn{\textdollar{}}{\XLingPapermincolb}{\textdollar{}}{\XLingPapermaxcolb}{+0\tabcolsep}
\XLingPaperminmaxcellincolumn{}{\XLingPapermincolc}{}{\XLingPapermaxcolc}{+0\tabcolsep}
\XLingPaperminmaxcellincolumn{U+0025}{\XLingPapermincola}{U+0025}{\XLingPapermaxcola}{+0\tabcolsep}
\XLingPaperminmaxcellincolumn{\%}{\XLingPapermincolb}{\%}{\XLingPapermaxcolb}{+0\tabcolsep}
\XLingPaperminmaxcellincolumn{}{\XLingPapermincolc}{}{\XLingPapermaxcolc}{+0\tabcolsep}
\XLingPaperminmaxcellincolumn{U+0026}{\XLingPapermincola}{U+0026}{\XLingPapermaxcola}{+0\tabcolsep}
\XLingPaperminmaxcellincolumn{\&}{\XLingPapermincolb}{\&}{\XLingPapermaxcolb}{+0\tabcolsep}
\XLingPaperminmaxcellincolumn{}{\XLingPapermincolc}{}{\XLingPapermaxcolc}{+0\tabcolsep}
\XLingPaperminmaxcellincolumn{U+0027}{\XLingPapermincola}{U+0027}{\XLingPapermaxcola}{+0\tabcolsep}
\XLingPaperminmaxcellincolumn{'}{\XLingPapermincolb}{'}{\XLingPapermaxcolb}{+0\tabcolsep}
\XLingPaperminmaxcellincolumn{}{\XLingPapermincolc}{}{\XLingPapermaxcolc}{+0\tabcolsep}
\XLingPaperminmaxcellincolumn{U+0028}{\XLingPapermincola}{U+0028}{\XLingPapermaxcola}{+0\tabcolsep}
\XLingPaperminmaxcellincolumn{(}{\XLingPapermincolb}{(}{\XLingPapermaxcolb}{+0\tabcolsep}
\XLingPaperminmaxcellincolumn{}{\XLingPapermincolc}{}{\XLingPapermaxcolc}{+0\tabcolsep}
\XLingPaperminmaxcellincolumn{U+0029}{\XLingPapermincola}{U+0029}{\XLingPapermaxcola}{+0\tabcolsep}
\XLingPaperminmaxcellincolumn{)}{\XLingPapermincolb}{)}{\XLingPapermaxcolb}{+0\tabcolsep}
\XLingPaperminmaxcellincolumn{}{\XLingPapermincolc}{}{\XLingPapermaxcolc}{+0\tabcolsep}
\XLingPaperminmaxcellincolumn{U+002A}{\XLingPapermincola}{U+002A}{\XLingPapermaxcola}{+0\tabcolsep}
\XLingPaperminmaxcellincolumn{*}{\XLingPapermincolb}{*}{\XLingPapermaxcolb}{+0\tabcolsep}
\XLingPaperminmaxcellincolumn{}{\XLingPapermincolc}{}{\XLingPapermaxcolc}{+0\tabcolsep}
\XLingPaperminmaxcellincolumn{U+002B}{\XLingPapermincola}{U+002B}{\XLingPapermaxcola}{+0\tabcolsep}
\XLingPaperminmaxcellincolumn{+}{\XLingPapermincolb}{+}{\XLingPapermaxcolb}{+0\tabcolsep}
\XLingPaperminmaxcellincolumn{}{\XLingPapermincolc}{}{\XLingPapermaxcolc}{+0\tabcolsep}
\XLingPaperminmaxcellincolumn{U+002C}{\XLingPapermincola}{U+002C}{\XLingPapermaxcola}{+0\tabcolsep}
\XLingPaperminmaxcellincolumn{,}{\XLingPapermincolb}{,}{\XLingPapermaxcolb}{+0\tabcolsep}
\XLingPaperminmaxcellincolumn{}{\XLingPapermincolc}{}{\XLingPapermaxcolc}{+0\tabcolsep}
\XLingPaperminmaxcellincolumn{U+002D}{\XLingPapermincola}{U+002D}{\XLingPapermaxcola}{+0\tabcolsep}
\XLingPaperminmaxcellincolumn{-}{\XLingPapermincolb}{-}{\XLingPapermaxcolb}{+0\tabcolsep}
\XLingPaperminmaxcellincolumn{}{\XLingPapermincolc}{}{\XLingPapermaxcolc}{+0\tabcolsep}
\XLingPaperminmaxcellincolumn{U+002E}{\XLingPapermincola}{U+002E}{\XLingPapermaxcola}{+0\tabcolsep}
\XLingPaperminmaxcellincolumn{.}{\XLingPapermincolb}{.}{\XLingPapermaxcolb}{+0\tabcolsep}
\XLingPaperminmaxcellincolumn{}{\XLingPapermincolc}{}{\XLingPapermaxcolc}{+0\tabcolsep}
\XLingPaperminmaxcellincolumn{U+002F}{\XLingPapermincola}{U+002F}{\XLingPapermaxcola}{+0\tabcolsep}
\XLingPaperminmaxcellincolumn{/}{\XLingPapermincolb}{/}{\XLingPapermaxcolb}{+0\tabcolsep}
\XLingPaperminmaxcellincolumn{}{\XLingPapermincolc}{}{\XLingPapermaxcolc}{+0\tabcolsep}
\XLingPaperminmaxcellincolumn{U+003A}{\XLingPapermincola}{U+003A}{\XLingPapermaxcola}{+0\tabcolsep}
\XLingPaperminmaxcellincolumn{:}{\XLingPapermincolb}{:}{\XLingPapermaxcolb}{+0\tabcolsep}
\XLingPaperminmaxcellincolumn{}{\XLingPapermincolc}{}{\XLingPapermaxcolc}{+0\tabcolsep}
\XLingPaperminmaxcellincolumn{U+003B}{\XLingPapermincola}{U+003B}{\XLingPapermaxcola}{+0\tabcolsep}
\XLingPaperminmaxcellincolumn{;}{\XLingPapermincolb}{;}{\XLingPapermaxcolb}{+0\tabcolsep}
\XLingPaperminmaxcellincolumn{}{\XLingPapermincolc}{}{\XLingPapermaxcolc}{+0\tabcolsep}
\XLingPaperminmaxcellincolumn{U+003C}{\XLingPapermincola}{U+003C}{\XLingPapermaxcola}{+0\tabcolsep}
\XLingPaperminmaxcellincolumn{\textless{}}{\XLingPapermincolb}{\textless{}}{\XLingPapermaxcolb}{+0\tabcolsep}
\XLingPaperminmaxcellincolumn{}{\XLingPapermincolc}{}{\XLingPapermaxcolc}{+0\tabcolsep}
\XLingPaperminmaxcellincolumn{U+003D}{\XLingPapermincola}{U+003D}{\XLingPapermaxcola}{+0\tabcolsep}
\XLingPaperminmaxcellincolumn{=}{\XLingPapermincolb}{=}{\XLingPapermaxcolb}{+0\tabcolsep}
\XLingPaperminmaxcellincolumn{}{\XLingPapermincolc}{}{\XLingPapermaxcolc}{+0\tabcolsep}
\XLingPaperminmaxcellincolumn{U+003E}{\XLingPapermincola}{U+003E}{\XLingPapermaxcola}{+0\tabcolsep}
\XLingPaperminmaxcellincolumn{\textgreater{}}{\XLingPapermincolb}{\textgreater{}}{\XLingPapermaxcolb}{+0\tabcolsep}
\XLingPaperminmaxcellincolumn{}{\XLingPapermincolc}{}{\XLingPapermaxcolc}{+0\tabcolsep}
\XLingPaperminmaxcellincolumn{U+003F}{\XLingPapermincola}{U+003F}{\XLingPapermaxcola}{+0\tabcolsep}
\XLingPaperminmaxcellincolumn{?}{\XLingPapermincolb}{?}{\XLingPapermaxcolb}{+0\tabcolsep}
\XLingPaperminmaxcellincolumn{}{\XLingPapermincolc}{}{\XLingPapermaxcolc}{+0\tabcolsep}
\XLingPaperminmaxcellincolumn{U+0040}{\XLingPapermincola}{U+0040}{\XLingPapermaxcola}{+0\tabcolsep}
\XLingPaperminmaxcellincolumn{@}{\XLingPapermincolb}{@}{\XLingPapermaxcolb}{+0\tabcolsep}
\XLingPaperminmaxcellincolumn{}{\XLingPapermincolc}{}{\XLingPapermaxcolc}{+0\tabcolsep}
\XLingPaperminmaxcellincolumn{U+005C}{\XLingPapermincola}{U+005C}{\XLingPapermaxcola}{+0\tabcolsep}
\XLingPaperminmaxcellincolumn{\textbackslash{}}{\XLingPapermincolb}{\textbackslash{}}{\XLingPapermaxcolb}{+0\tabcolsep}
\XLingPaperminmaxcellincolumn{}{\XLingPapermincolc}{}{\XLingPapermaxcolc}{+0\tabcolsep}
\XLingPaperminmaxcellincolumn{U+005B}{\XLingPapermincola}{U+005B}{\XLingPapermaxcola}{+0\tabcolsep}
\XLingPaperminmaxcellincolumn{\textsquarebracketleft{}}{\XLingPapermincolb}{\textsquarebracketleft{}}{\XLingPapermaxcolb}{+0\tabcolsep}
\XLingPaperminmaxcellincolumn{}{\XLingPapermincolc}{}{\XLingPapermaxcolc}{+0\tabcolsep}
\XLingPaperminmaxcellincolumn{U+005D}{\XLingPapermincola}{U+005D}{\XLingPapermaxcola}{+0\tabcolsep}
\XLingPaperminmaxcellincolumn{\textsquarebracketright{}}{\XLingPapermincolb}{\textsquarebracketright{}}{\XLingPapermaxcolb}{+0\tabcolsep}
\XLingPaperminmaxcellincolumn{}{\XLingPapermincolc}{}{\XLingPapermaxcolc}{+0\tabcolsep}
\XLingPaperminmaxcellincolumn{U+005F}{\XLingPapermincola}{U+005F}{\XLingPapermaxcola}{+0\tabcolsep}
\XLingPaperminmaxcellincolumn{\_}{\XLingPapermincolb}{\_}{\XLingPapermaxcolb}{+0\tabcolsep}
\XLingPaperminmaxcellincolumn{}{\XLingPapermincolc}{}{\XLingPapermaxcolc}{+0\tabcolsep}
\XLingPaperminmaxcellincolumn{U+0060}{\XLingPapermincola}{U+0060}{\XLingPapermaxcola}{+0\tabcolsep}
\XLingPaperminmaxcellincolumn{`}{\XLingPapermincolb}{`}{\XLingPapermaxcolb}{+0\tabcolsep}
\XLingPaperminmaxcellincolumn{}{\XLingPapermincolc}{}{\XLingPapermaxcolc}{+0\tabcolsep}
\XLingPaperminmaxcellincolumn{U+007C}{\XLingPapermincola}{U+007C}{\XLingPapermaxcola}{+0\tabcolsep}
\XLingPaperminmaxcellincolumn{\textbar{}}{\XLingPapermincolb}{\textbar{}}{\XLingPapermaxcolb}{+0\tabcolsep}
\XLingPaperminmaxcellincolumn{}{\XLingPapermincolc}{}{\XLingPapermaxcolc}{+0\tabcolsep}
\XLingPaperminmaxcellincolumn{U+007E}{\XLingPapermincola}{U+007E}{\XLingPapermaxcola}{+0\tabcolsep}
\XLingPaperminmaxcellincolumn{\textasciitilde{}}{\XLingPapermincolb}{\textasciitilde{}}{\XLingPapermaxcolb}{+0\tabcolsep}
\XLingPaperminmaxcellincolumn{}{\XLingPapermincolc}{}{\XLingPapermaxcolc}{+0\tabcolsep}
\setlength{\XLingPaperavailabletablewidth}{433.62pt}
\setlength{\XLingPapertableminwidth}{\XLingPapermincola+\XLingPapermincolb+\XLingPapermincolc}
\setlength{\XLingPapertablemaxwidth}{\XLingPapermaxcola+\XLingPapermaxcolb+\XLingPapermaxcolc}
\XLingPapercalculatetablewidthratio{}
\XLingPapersetcolumnwidth{\XLingPapercolawidth}{\XLingPapermincola}{\XLingPapermaxcola}{-0\tabcolsep}
\XLingPapersetcolumnwidth{\XLingPapercolbwidth}{\XLingPapermincolb}{\XLingPapermaxcolb}{-2\tabcolsep}
\XLingPapersetcolumnwidth{\XLingPapercolcwidth}{\XLingPapermincolc}{\XLingPapermaxcolc}{-2\tabcolsep}\singlespacing\vspace*{-3\baselineskip}
\begin{longtable}
[l]{@{}p{\XLingPapercolawidth}p{\XLingPapercolbwidth}p{\XLingPapercolcwidth}@{}}\toprule\multicolumn{1}{@{}p{\XLingPapercolawidth}}{\textbf{Codepoint}}&\multicolumn{1}{p{\XLingPapercolbwidth}}{\textbf{Glyph}}&\multicolumn{1}{p{\XLingPapercolcwidth}@{}}{\textbf{Corpus count}}\\%
\midrule\endhead \multicolumn{1}{@{}p{\XLingPapercolawidth}}{U+0021}&\multicolumn{1}{p{\XLingPapercolbwidth}}{!}&\multicolumn{1}{p{\XLingPapercolcwidth}@{}}{}\\%
\multicolumn{1}{@{}p{\XLingPapercolawidth}}{U+0022}&\multicolumn{1}{p{\XLingPapercolbwidth}}{"}&\multicolumn{1}{p{\XLingPapercolcwidth}@{}}{}\\%
\multicolumn{1}{@{}p{\XLingPapercolawidth}}{U+0023}&\multicolumn{1}{p{\XLingPapercolbwidth}}{\#}&\multicolumn{1}{p{\XLingPapercolcwidth}@{}}{}\\%
\multicolumn{1}{@{}p{\XLingPapercolawidth}}{U+0024}&\multicolumn{1}{p{\XLingPapercolbwidth}}{\textdollar{}}&\multicolumn{1}{p{\XLingPapercolcwidth}@{}}{}\\%
\multicolumn{1}{@{}p{\XLingPapercolawidth}}{U+0025}&\multicolumn{1}{p{\XLingPapercolbwidth}}{\%}&\multicolumn{1}{p{\XLingPapercolcwidth}@{}}{}\\%
\multicolumn{1}{@{}p{\XLingPapercolawidth}}{U+0026}&\multicolumn{1}{p{\XLingPapercolbwidth}}{\&}&\multicolumn{1}{p{\XLingPapercolcwidth}@{}}{}\\%
\multicolumn{1}{@{}p{\XLingPapercolawidth}}{U+0027}&\multicolumn{1}{p{\XLingPapercolbwidth}}{'}&\multicolumn{1}{p{\XLingPapercolcwidth}@{}}{}\\%
\multicolumn{1}{@{}p{\XLingPapercolawidth}}{U+0028}&\multicolumn{1}{p{\XLingPapercolbwidth}}{(}&\multicolumn{1}{p{\XLingPapercolcwidth}@{}}{}\\%
\multicolumn{1}{@{}p{\XLingPapercolawidth}}{U+0029}&\multicolumn{1}{p{\XLingPapercolbwidth}}{)}&\multicolumn{1}{p{\XLingPapercolcwidth}@{}}{}\\%
\multicolumn{1}{@{}p{\XLingPapercolawidth}}{U+002A}&\multicolumn{1}{p{\XLingPapercolbwidth}}{*}&\multicolumn{1}{p{\XLingPapercolcwidth}@{}}{}\\%
\multicolumn{1}{@{}p{\XLingPapercolawidth}}{U+002B}&\multicolumn{1}{p{\XLingPapercolbwidth}}{+}&\multicolumn{1}{p{\XLingPapercolcwidth}@{}}{}\\%
\multicolumn{1}{@{}p{\XLingPapercolawidth}}{U+002C}&\multicolumn{1}{p{\XLingPapercolbwidth}}{,}&\multicolumn{1}{p{\XLingPapercolcwidth}@{}}{}\\%
\multicolumn{1}{@{}p{\XLingPapercolawidth}}{U+002D}&\multicolumn{1}{p{\XLingPapercolbwidth}}{-}&\multicolumn{1}{p{\XLingPapercolcwidth}@{}}{}\\%
\multicolumn{1}{@{}p{\XLingPapercolawidth}}{U+002E}&\multicolumn{1}{p{\XLingPapercolbwidth}}{.}&\multicolumn{1}{p{\XLingPapercolcwidth}@{}}{}\\%
\multicolumn{1}{@{}p{\XLingPapercolawidth}}{U+002F}&\multicolumn{1}{p{\XLingPapercolbwidth}}{/}&\multicolumn{1}{p{\XLingPapercolcwidth}@{}}{}\\%
\multicolumn{1}{@{}p{\XLingPapercolawidth}}{U+003A}&\multicolumn{1}{p{\XLingPapercolbwidth}}{:}&\multicolumn{1}{p{\XLingPapercolcwidth}@{}}{}\\%
\multicolumn{1}{@{}p{\XLingPapercolawidth}}{U+003B}&\multicolumn{1}{p{\XLingPapercolbwidth}}{;}&\multicolumn{1}{p{\XLingPapercolcwidth}@{}}{}\\%
\multicolumn{1}{@{}p{\XLingPapercolawidth}}{U+003C}&\multicolumn{1}{p{\XLingPapercolbwidth}}{\textless{}}&\multicolumn{1}{p{\XLingPapercolcwidth}@{}}{}\\%
\multicolumn{1}{@{}p{\XLingPapercolawidth}}{U+003D}&\multicolumn{1}{p{\XLingPapercolbwidth}}{=}&\multicolumn{1}{p{\XLingPapercolcwidth}@{}}{}\\%
\multicolumn{1}{@{}p{\XLingPapercolawidth}}{U+003E}&\multicolumn{1}{p{\XLingPapercolbwidth}}{\textgreater{}}&\multicolumn{1}{p{\XLingPapercolcwidth}@{}}{}\\%
\multicolumn{1}{@{}p{\XLingPapercolawidth}}{U+003F}&\multicolumn{1}{p{\XLingPapercolbwidth}}{?}&\multicolumn{1}{p{\XLingPapercolcwidth}@{}}{}\\%
\multicolumn{1}{@{}p{\XLingPapercolawidth}}{U+0040}&\multicolumn{1}{p{\XLingPapercolbwidth}}{@}&\multicolumn{1}{p{\XLingPapercolcwidth}@{}}{}\\%
\multicolumn{1}{@{}p{\XLingPapercolawidth}}{U+005C}&\multicolumn{1}{p{\XLingPapercolbwidth}}{\textbackslash{}}&\multicolumn{1}{p{\XLingPapercolcwidth}@{}}{}\\%
\multicolumn{1}{@{}p{\XLingPapercolawidth}}{U+005B}&\multicolumn{1}{p{\XLingPapercolbwidth}}{\textsquarebracketleft{}}&\multicolumn{1}{p{\XLingPapercolcwidth}@{}}{}\\%
\multicolumn{1}{@{}p{\XLingPapercolawidth}}{U+005D}&\multicolumn{1}{p{\XLingPapercolbwidth}}{\textsquarebracketright{}}&\multicolumn{1}{p{\XLingPapercolcwidth}@{}}{}\\%
\multicolumn{1}{@{}p{\XLingPapercolawidth}}{U+005F}&\multicolumn{1}{p{\XLingPapercolbwidth}}{\_}&\multicolumn{1}{p{\XLingPapercolcwidth}@{}}{}\\%
\multicolumn{1}{@{}p{\XLingPapercolawidth}}{U+0060}&\multicolumn{1}{p{\XLingPapercolbwidth}}{`}&\multicolumn{1}{p{\XLingPapercolcwidth}@{}}{}\\%
\multicolumn{1}{@{}p{\XLingPapercolawidth}}{U+007C}&\multicolumn{1}{p{\XLingPapercolbwidth}}{\textbar{}}&\multicolumn{1}{p{\XLingPapercolcwidth}@{}}{}\\%
\multicolumn{1}{@{}p{\XLingPapercolawidth}}{U+007E}&\multicolumn{1}{p{\XLingPapercolbwidth}}{\textasciitilde{}}&\multicolumn{1}{p{\XLingPapercolcwidth}@{}}{}\\\bottomrule%
\end{longtable}
}
}{\vspace{15pt}\XLingPaperneedspace{3\baselineskip}\noindent
\fontsize{13}{15.6}\selectfont \textbf{{\noindent
\raisebox{\baselineskip}[0pt]{\pdfbookmark[2]{{3.8 } Casing rules}{sCasing}}\raisebox{\baselineskip}[0pt]{\protect\hypertarget{sCasing}{}}{3.8 }Casing rules}}\markboth{Casing rules}{Eastern Dan writing system}\XLingPaperaddtocontents{sCasing}}\par{}
\penalty10000\vspace{10pt}\penalty10000\indent When a new writing system is established it is important to establish if letters will have a case correspondence between uppercase and lowercase. For Eastern Dan this is shown in the 1994 reader. However what is not discussed is any other occasions that uppercase (capitalization) is supposed to occur in the language. Based on data within the corpus as originally delivered, casing rules appear to follow general French casing norms. Explicit observations are listed below. Numbers 1-5 are congruent with French casing norms, whereas numbers 6-7 contrast with French casing norms\protect\footnote[29]{{\leftskip0pt\parindent1em\raisebox{\baselineskip}[0pt]{\protect\hypertarget{nFrenchNorms}{}} I am not an expert in the norms or regional variations of French typographic tradition. Eastern Dan is spoken in the Ivory Coast, which has French as its official language, and may have its own regional norms for French usage.}}.\par{}{\parskip .5pt plus 1pt minus 1pt
                    
\vspace{\baselineskip}

{\setlength{\XLingPapertempdim}{\XLingPapersingledigitlistitemwidth+\parindent{}}\leftskip\XLingPapertempdim\relax
\interlinepenalty10000
\XLingPaperlistitem{\parindent{}}{\XLingPapersingledigitlistitemwidth}{1.}{Tone marks preceding the non-tone mark portion of the word do not get capitalized, but the characters following the tone marks {\XLingPaperCambriaZMathFontFamily{\textup{\textmd{⟨ {\XLingPaperCharisZSILFontFamily{\textup{\textup{\textmd{a-zA-Z}}}}} ⟩}}}}\protect\footnote[30]{{\leftskip0pt\parindent1em\raisebox{\baselineskip}[0pt]{\protect\hypertarget{nCharacterInclusive}{}} Inclusive of characters borrowed from the IPA. Using Unicode attributes Lm characters are not case sensitive. All }} do get capitalized. Yet tone marks are considered part of the word and should not have word breaks between them and the words they belong with.}}
{\setlength{\XLingPapertempdim}{\XLingPapersingledigitlistitemwidth+\parindent{}}\leftskip\XLingPapertempdim\relax
\interlinepenalty10000
\XLingPaperlistitem{\parindent{}}{\XLingPapersingledigitlistitemwidth}{2.}{The first word of a sentence is capitalized.}}
{\setlength{\XLingPapertempdim}{\XLingPapersingledigitlistitemwidth+\parindent{}}\leftskip\XLingPapertempdim\relax
\interlinepenalty10000
\XLingPaperlistitem{\parindent{}}{\XLingPapersingledigitlistitemwidth}{3.}{Proper nouns are capitalized.}}
{\setlength{\XLingPapertempdim}{\XLingPapersingledigitlistitemwidth+\parindent{}}\leftskip\XLingPapertempdim\relax
\interlinepenalty10000
\XLingPaperlistitem{\parindent{}}{\XLingPapersingledigitlistitemwidth}{4.}{Uppercase can be used as a style choice in titles of creative works, much as is the case in many languages, which use a Latin script.}}
{\setlength{\XLingPapertempdim}{\XLingPapersingledigitlistitemwidth+\parindent{}}\leftskip\XLingPapertempdim\relax
\interlinepenalty10000
\XLingPaperlistitem{\parindent{}}{\XLingPapersingledigitlistitemwidth}{5.}{Only the first letter of a digraph is capitalized. i.e. {\XLingPaperCambriaZMathFontFamily{\textup{\textmd{⟨ {\XLingPaperCharisZSILFontFamily{\textup{\textup{\textmd{"Ɛa-}}}}} ⟩}}}} is correct whereas {\XLingPaperCambriaZMathFontFamily{\textup{\textmd{⟨ {\XLingPaperCharisZSILFontFamily{\textup{\textup{\textmd{"ƐA-}}}}} ⟩}}}} is not.}}
{\setlength{\XLingPapertempdim}{\XLingPapersingledigitlistitemwidth+\parindent{}}\leftskip\XLingPapertempdim\relax
\interlinepenalty10000
\XLingPaperlistitem{\parindent{}}{\XLingPapersingledigitlistitemwidth}{6.}{Unlike French where, when an article is the first word of a sentence both the first word and the second word are capitalized, in Eastern Dan only the first word is capitalized.}}
{\setlength{\XLingPapertempdim}{\XLingPapersingledigitlistitemwidth+\parindent{}}\leftskip\XLingPapertempdim\relax
\interlinepenalty10000
\XLingPaperlistitem{\parindent{}}{\XLingPapersingledigitlistitemwidth}{7.}{Surnames are not capitalized as is the custom in French literature.}}
\vspace{\baselineskip}
}{\vspace{15pt}\XLingPaperneedspace{3\baselineskip}\noindent
\fontsize{13}{15.6}\selectfont \textbf{{\noindent
\raisebox{\baselineskip}[0pt]{\pdfbookmark[2]{{3.9 } Word breaking behavior}{sWordBreaks}}\raisebox{\baselineskip}[0pt]{\protect\hypertarget{sWordBreaks}{}}{3.9 }Word breaking behavior}}\markboth{Word breaking behavior}{Eastern Dan writing system}\XLingPaperaddtocontents{sWordBreaks}}\par{}
\penalty10000\vspace{10pt}\penalty10000\indent Word break rules are not discussed in the 1994 primer, but reading is taught with single words bounded by spaces. This seems to indicate that., as is normal for Latin script\protect\footnote[31]{{\leftskip0pt\parindent1em\raisebox{\baselineskip}[0pt]{\protect\hypertarget{nLatinScriptWordBreaks}{}} Historically, in Greek and Latin scripts word spacing was on practiced. Even today, some east Asian scripts do not use inter-word spacing.}}, that words are bounded or indicated via space. Sentences are also padded with spaces (no double spaces between sentences were detected in the corpus). Hyphenation, the practice of splitting a single typographical word across two lines is not addressed by Bolli and Flik. However, one may assume that tone marks should never separate from the rest of their word. It would just be weird to insert a hyphen into a word that uses a hyphen looking glyph as a letter. So presumably hyphenation is not allowed in this writing system either; the same might be said for en dash or em dash too. That is due to issues related to visual glyph character disambiguation, typographical practices using horizontal bar like characters should be avoided.\par{}\indent Generally in Latin scripts U+0020 ‘SPACE’ is used to separate words. This character has a  Because Eastern Dan uses ‘MODIFIER LETTER’ characters which look like punctuation but are really letters, typists often use actual punctuation characters. This creates a problem for writers to the point that it is has been common practice to overcompensate to keep characters representing tone attached to the rest of the string that represents the word. Various kinds of special characters are used in the corpus as it was originally delivered, to prevent word breaks in undesired places. Sometimes {\XLingPaperCambriaZMathFontFamily{⟨ {\XLingPaperCharisZSILFontFamily{\textup{\textup{\textmd{ }}}}} ⟩}} U+00A0 'NO-BREAK SPACE' and sometimes {\XLingPaperCambriaZMathFontFamily{⟨ {\XLingPaperCharisZSILFontFamily{\textup{\textup{\textmd{‑}}}}} ⟩}} U+2011 'NON-BREAKING HYPHEN' was used to control line and word breaking behavior. These special typesetting characters could be avoided if 'MODIFIER LETTER' equivalent characters for tone marks were used instead of characters with punctuation attributes. It should be one of the goals of a well implemented keyboard layout to make the correct Unicode characters available to typists.\par{}{\vspace{15pt}\XLingPaperneedspace{3\baselineskip}\noindent
\fontsize{13}{15.6}\selectfont \textbf{{\noindent
\raisebox{\baselineskip}[0pt]{\pdfbookmark[2]{{3.10 } Loan Words and borrowed letters}{sLoanWords}}\raisebox{\baselineskip}[0pt]{\protect\hypertarget{sLoanWords}{}}{3.10 }Loan Words and borrowed letters}}\markboth{Loan Words and borrowed letters}{Eastern Dan writing system}\XLingPaperaddtocontents{sLoanWords}}\par{}
\penalty10000\vspace{10pt}\penalty10000\indent \hyperlink{Hosken}{Hosken (2003:11)} makes the point that “one of the hardest things for an orthography to deal with is how to spell words which are imported from another language”. There are two categories of loan words as they relate to print media. Loan words which are loan words at the oral level (borrowings in speech) and loan words which are introduced directly at the written level (borrowings in print). Loan words at the oral level may or may not be borrowed without phonological changes from the donor language to the recipient language. Loan words in print therefore may be loan words which are borrowed orally and and changed at the oral level, and then put into the Eastern Dan orthography or they may be direct print borrowings. The difference is that if they are direct print borrowings they may contain characters from the donor language's orthography, loan words which are borrowed orally are candidates for modification and reanalysis per the rules of Eastern Dan. However, not all oral borrowings are always reanalysed according to the recipient's phonology. Eastern Dan communities are multilingual and as such may be inserting words from other languages in a form of mix-speech or code switching. In print media, the names of places, people, brands, and organizations tend to be highly stable. One such case is the city name {\XLingPaperCambriaZMathFontFamily{\textup{\textmd{⟨ {\XLingPaperCharisZSILFontFamily{\textup{\textup{\textmd{Abidjan}}}}} ⟩}}}} – which is spelled in French with a {\XLingPaperCambriaZMathFontFamily{\textup{\textmd{⟨ {\XLingPaperCharisZSILFontFamily{\textup{\textup{\textmd{j}}}}} ⟩}}}} but there is no {\XLingPaperCambriaZMathFontFamily{\textup{\textmd{⟨ {\XLingPaperCharisZSILFontFamily{\textup{\textup{\textmd{j}}}}} ⟩}}}} in the Eastern Dan orthography. Alternatively oral routes of borrowing, words may be borrowed directly from print sources. In the corpus on hand, the donor language is not always clear, it could be English, French, or Russian, or a combination of these. The reason for this is that these loan words are often seen in the print culture of with letters which are not part of the language's writing system. The sounds which are represented by those foreign letters may or may not represent sounds which are phonology This is also true for Dan. French is the national language, there are other Mande languages in the region and there are obviously English sub-strata in some terms. My French language skills put me at a disadvantage for detecting borrowings which appear in the text. However, the examples in table \hyperlink{ntLoanLetters}{24} demonstrate the kinds of examples found in the text.\par{}\vspace{11pt plus 2pt minus 1pt}\XLingPaperneedspace{3\baselineskip}\protect\hypertarget{ntLoanLetters}{}\XLingPaperaddtocontents{ntLoanLetters}{\protect\raggedright{\singlespacing
{Table }{24.}{  Loan word and print borrwing examples and examples with borrowed letters\\}}}\vspace{0pt}{\singlespacing
\hspace*{.25in}{\setcounter{footnote}{31}
\XLingPaperminmaxcellincolumn{Catagory}{\XLingPapermincola}{\textbf{Catagory}}{\XLingPapermaxcola}{+0\tabcolsep}
\XLingPaperminmaxcellincolumn{Loaned}{\XLingPapermincolb}{\textbf{Loaned Word}}{\XLingPapermaxcolb}{+0\tabcolsep}
\XLingPaperminmaxcellincolumn{Dan}{\XLingPapermincolc}{\textbf{Dan Transcription}}{\XLingPapermaxcolc}{+0\tabcolsep}
\XLingPaperminmaxcellincolumn{Proper}{\XLingPapermincola}{Proper Name}{\XLingPapermaxcola}{+0\tabcolsep}
\XLingPaperminmaxcellincolumn{Graham}{\XLingPapermincolb}{Billy Graham}{\XLingPapermaxcolb}{+0\tabcolsep}
\XLingPaperminmaxcellincolumn{Grahamö)}{\XLingPapermincolc}{Bili ‘Gramö (Billy Grahamö)}{\XLingPapermaxcolc}{+0\tabcolsep}
\XLingPaperminmaxcellincolumn{}{\XLingPapermincola}{}{\XLingPapermaxcola}{+0\tabcolsep}
\XLingPaperminmaxcellincolumn{Elena}{\XLingPapermincolb}{Elena Perekhvalskaya}{\XLingPapermaxcolb}{+0\tabcolsep}
\XLingPaperminmaxcellincolumn{Edhɛna}{\XLingPapermincolc}{Edhɛna Pɛdhɛkwalösökaya}{\XLingPapermaxcolc}{+0\tabcolsep}
\XLingPaperminmaxcellincolumn{}{\XLingPapermincola}{}{\XLingPapermaxcola}{+0\tabcolsep}
\XLingPaperminmaxcellincolumn{Valentin}{\XLingPapermincolb}{Valentin Vydrin}{\XLingPapermaxcolb}{+0\tabcolsep}
\XLingPaperminmaxcellincolumn{Valangtɛn}{\XLingPapermincolc}{Valangtɛn Vidrinö}{\XLingPapermaxcolc}{+0\tabcolsep}
\XLingPaperminmaxcellincolumn{}{\XLingPapermincola}{}{\XLingPapermaxcola}{+0\tabcolsep}
\XLingPaperminmaxcellincolumn{}{\XLingPapermincolb}{}{\XLingPapermaxcolb}{+0\tabcolsep}
\XLingPaperminmaxcellincolumn{}{\XLingPapermincolc}{}{\XLingPapermaxcolc}{+0\tabcolsep}
\XLingPaperminmaxcellincolumn{Places}{\XLingPapermincola}{Places}{\XLingPapermaxcola}{+0\tabcolsep}
\XLingPaperminmaxcellincolumn{Abidjan}{\XLingPapermincolb}{Abidjan}{\XLingPapermaxcolb}{+0\tabcolsep}
\XLingPaperminmaxcellincolumn{Abidjan}{\XLingPapermincolc}{Abidjan}{\XLingPapermaxcolc}{+0\tabcolsep}
\XLingPaperminmaxcellincolumn{}{\XLingPapermincola}{}{\XLingPapermaxcola}{+0\tabcolsep}
\XLingPaperminmaxcellincolumn{Sénégal}{\XLingPapermincolb}{Sénégal (fra)}{\XLingPapermaxcolb}{+0\tabcolsep}
\XLingPaperminmaxcellincolumn{Senegalë}{\XLingPapermincolc}{Senegalë}{\XLingPapermaxcolc}{+0\tabcolsep}
\XLingPaperminmaxcellincolumn{}{\XLingPapermincola}{}{\XLingPapermaxcola}{+0\tabcolsep}
\XLingPaperminmaxcellincolumn{(fra)}{\XLingPapermincolb}{Mali (fra)}{\XLingPapermaxcolb}{+0\tabcolsep}
\XLingPaperminmaxcellincolumn{Mali}{\XLingPapermincolc}{Mali}{\XLingPapermaxcolc}{+0\tabcolsep}
\XLingPaperminmaxcellincolumn{}{\XLingPapermincola}{}{\XLingPapermaxcola}{+0\tabcolsep}
\XLingPaperminmaxcellincolumn{(fra)}{\XLingPapermincolb}{Bangladesh (fra)}{\XLingPapermaxcolb}{+0\tabcolsep}
\XLingPaperminmaxcellincolumn{Bɛnggladɛsü}{\XLingPapermincolc}{Bɛnggladɛsü}{\XLingPapermaxcolc}{+0\tabcolsep}
\XLingPaperminmaxcellincolumn{}{\XLingPapermincola}{}{\XLingPapermaxcola}{+0\tabcolsep}
\XLingPaperminmaxcellincolumn{Soudan}{\XLingPapermincolb}{Soudan (fra)}{\XLingPapermaxcolb}{+0\tabcolsep}
\XLingPaperminmaxcellincolumn{Sudan}{\XLingPapermincolc}{Sudan}{\XLingPapermaxcolc}{+0\tabcolsep}
\XLingPaperminmaxcellincolumn{}{\XLingPapermincola}{}{\XLingPapermaxcola}{+0\tabcolsep}
\XLingPaperminmaxcellincolumn{d'Ivoire}{\XLingPapermincolb}{Côte d'Ivoire (fra)}{\XLingPapermaxcolb}{+0\tabcolsep}
\XLingPaperminmaxcellincolumn{Kɔdivuaa}{\XLingPapermincolc}{'Kɔtö Divuaa \& Kɔdivuaa}{\XLingPapermaxcolc}{+0\tabcolsep}
\XLingPaperminmaxcellincolumn{}{\XLingPapermincola}{}{\XLingPapermaxcola}{+0\tabcolsep}
\XLingPaperminmaxcellincolumn{}{\XLingPapermincolb}{}{\XLingPapermaxcolb}{+0\tabcolsep}
\XLingPaperminmaxcellincolumn{“Yʋngbhɔklɔɔ}{\XLingPapermincolc}{“Yʋngbhɔklɔɔ (Dimbokro)}{\XLingPapermaxcolc}{+0\tabcolsep}
\XLingPaperminmaxcellincolumn{}{\XLingPapermincola}{}{\XLingPapermaxcola}{+0\tabcolsep}
\XLingPaperminmaxcellincolumn{}{\XLingPapermincolb}{}{\XLingPapermaxcolb}{+0\tabcolsep}
\XLingPaperminmaxcellincolumn{-Zuan}{\XLingPapermincolc}{-Zuan yuö ( Zouan-Hien )}{\XLingPapermaxcolc}{+0\tabcolsep}
\XLingPaperminmaxcellincolumn{}{\XLingPapermincola}{}{\XLingPapermaxcola}{+0\tabcolsep}
\XLingPaperminmaxcellincolumn{}{\XLingPapermincolb}{}{\XLingPapermaxcolb}{+0\tabcolsep}
\XLingPaperminmaxcellincolumn{‘Utö}{\XLingPapermincolc}{‘Utö ( Août )}{\XLingPapermaxcolc}{+0\tabcolsep}
\XLingPaperminmaxcellincolumn{Africa}{\XLingPapermincola}{Abbreviations}{\XLingPapermaxcola}{+0\tabcolsep}
\XLingPaperminmaxcellincolumn{}{\XLingPapermincolb}{}{\XLingPapermaxcolb}{+0\tabcolsep}
\XLingPaperminmaxcellincolumn{UNICEF}{\XLingPapermincolc}{UNICEF}{\XLingPapermaxcolc}{+0\tabcolsep}
\XLingPaperminmaxcellincolumn{}{\XLingPapermincola}{}{\XLingPapermaxcola}{+0\tabcolsep}
\XLingPaperminmaxcellincolumn{}{\XLingPapermincolb}{}{\XLingPapermaxcolb}{+0\tabcolsep}
\XLingPaperminmaxcellincolumn{UEESO}{\XLingPapermincolc}{UEESO}{\XLingPapermaxcolc}{+0\tabcolsep}
\XLingPaperminmaxcellincolumn{}{\XLingPapermincola}{}{\XLingPapermaxcola}{+0\tabcolsep}
\XLingPaperminmaxcellincolumn{}{\XLingPapermincolb}{}{\XLingPapermaxcolb}{+0\tabcolsep}
\XLingPaperminmaxcellincolumn{BCEAO}{\XLingPapermincolc}{BCEAO}{\XLingPapermaxcolc}{+0\tabcolsep}
\XLingPaperminmaxcellincolumn{}{\XLingPapermincola}{}{\XLingPapermaxcola}{+0\tabcolsep}
\XLingPaperminmaxcellincolumn{}{\XLingPapermincolb}{}{\XLingPapermaxcolb}{+0\tabcolsep}
\XLingPaperminmaxcellincolumn{}{\XLingPapermincolc}{}{\XLingPapermaxcolc}{+0\tabcolsep}
\XLingPaperminmaxcellincolumn{Concepts}{\XLingPapermincola}{Concepts}{\XLingPapermaxcola}{+0\tabcolsep}
\XLingPaperminmaxcellincolumn{(eng-fra)}{\XLingPapermincolb}{4 x 4 (eng-fra) - type of vehicle}{\XLingPapermaxcolb}{+0\tabcolsep}
\XLingPaperminmaxcellincolumn{4}{\XLingPapermincolc}{4 x 4}{\XLingPapermaxcolc}{+0\tabcolsep}
\XLingPaperminmaxcellincolumn{}{\XLingPapermincola}{}{\XLingPapermaxcola}{+0\tabcolsep}
\XLingPaperminmaxcellincolumn{(eng-fra)}{\XLingPapermincolb}{Million (eng-fra)}{\XLingPapermaxcolb}{+0\tabcolsep}
\XLingPaperminmaxcellincolumn{miliɔn}{\XLingPapermincolc}{miliɔn}{\XLingPapermaxcolc}{+0\tabcolsep}
\setlength{\XLingPaperavailabletablewidth}{433.62pt}
\setlength{\XLingPapertableminwidth}{\XLingPapermincola+\XLingPapermincolb+\XLingPapermincolc}
\setlength{\XLingPapertablemaxwidth}{\XLingPapermaxcola+\XLingPapermaxcolb+\XLingPapermaxcolc}
\XLingPapercalculatetablewidthratio{}
\XLingPapersetcolumnwidth{\XLingPapercolawidth}{\XLingPapermincola}{\XLingPapermaxcola}{-0\tabcolsep}
\XLingPapersetcolumnwidth{\XLingPapercolbwidth}{\XLingPapermincolb}{\XLingPapermaxcolb}{-2\tabcolsep}
\XLingPapersetcolumnwidth{\XLingPapercolcwidth}{\XLingPapermincolc}{\XLingPapermaxcolc}{-2\tabcolsep}\setcounter{footnote}{31}\singlespacing\vspace*{-3\baselineskip}
\begin{longtable}
[l]{@{}p{\XLingPapercolawidth}p{\XLingPapercolbwidth}p{\XLingPapercolcwidth}@{}}\toprule\multicolumn{1}{@{}p{\XLingPapercolawidth}}{\textbf{Catagory}}&\multicolumn{1}{p{\XLingPapercolbwidth}}{\textbf{Loaned Word}}&\multicolumn{1}{p{\XLingPapercolcwidth}@{}}{\textbf{Dan Transcription}}\\%
\midrule\endhead \multicolumn{1}{@{}p{\XLingPapercolawidth}}{Proper Name}&\multicolumn{1}{p{\XLingPapercolbwidth}}{Billy Graham}&\multicolumn{1}{p{\XLingPapercolcwidth}@{}}{Bili ‘Gramö (Billy Grahamö)}\\%
\multicolumn{1}{@{}p{\XLingPapercolawidth}}{}&\multicolumn{1}{p{\XLingPapercolbwidth}}{Elena Perekhvalskaya}&\multicolumn{1}{p{\XLingPapercolcwidth}@{}}{Edhɛna Pɛdhɛkwalösökaya}\\%
\multicolumn{1}{@{}p{\XLingPapercolawidth}}{}&\multicolumn{1}{p{\XLingPapercolbwidth}}{Valentin Vydrin}&\multicolumn{1}{p{\XLingPapercolcwidth}@{}}{Valangtɛn Vidrinö}\\%
\multicolumn{1}{@{}p{\XLingPapercolawidth}}{}&\multicolumn{1}{p{\XLingPapercolbwidth}}{}&\multicolumn{1}{p{\XLingPapercolcwidth}@{}}{}\\%
\multicolumn{1}{@{}p{\XLingPapercolawidth}}{Places}&\multicolumn{1}{p{\XLingPapercolbwidth}}{Abidjan}&\multicolumn{1}{p{\XLingPapercolcwidth}@{}}{Abidjan}\\%
\multicolumn{1}{@{}p{\XLingPapercolawidth}}{}&\multicolumn{1}{p{\XLingPapercolbwidth}}{Sénégal (fra)}&\multicolumn{1}{p{\XLingPapercolcwidth}@{}}{Senegalë}\\%
\multicolumn{1}{@{}p{\XLingPapercolawidth}}{}&\multicolumn{1}{p{\XLingPapercolbwidth}}{Mali (fra)}&\multicolumn{1}{p{\XLingPapercolcwidth}@{}}{Mali}\\%
\multicolumn{1}{@{}p{\XLingPapercolawidth}}{}&\multicolumn{1}{p{\XLingPapercolbwidth}}{Bangladesh (fra)}&\multicolumn{1}{p{\XLingPapercolcwidth}@{}}{Bɛnggladɛsü}\\%
\multicolumn{1}{@{}p{\XLingPapercolawidth}}{}&\multicolumn{1}{p{\XLingPapercolbwidth}}{Soudan (fra)}&\multicolumn{1}{p{\XLingPapercolcwidth}@{}}{Sudan}\\%
\multicolumn{1}{@{}p{\XLingPapercolawidth}}{}&\multicolumn{1}{p{\XLingPapercolbwidth}}{Côte d'Ivoire (fra)}&\multicolumn{1}{p{\XLingPapercolcwidth}@{}}{'Kɔtö Divuaa \& Kɔdivuaa}\\%
\multicolumn{1}{@{}p{\XLingPapercolawidth}}{}&\multicolumn{1}{p{\XLingPapercolbwidth}}{}&\multicolumn{1}{p{\XLingPapercolcwidth}@{}}{“Yʋngbhɔklɔɔ (Dimbokro)}\\%
\multicolumn{1}{@{}p{\XLingPapercolawidth}}{}&\multicolumn{1}{p{\XLingPapercolbwidth}}{}&\multicolumn{1}{p{\XLingPapercolcwidth}@{}}{-Zuan yuö ( Zouan-Hien )}\\%
\multicolumn{1}{@{}p{\XLingPapercolawidth}}{}&\multicolumn{1}{p{\XLingPapercolbwidth}}{}&\multicolumn{1}{p{\XLingPapercolcwidth}@{}}{‘Utö ( Août )}\\%
\multicolumn{1}{@{}p{\XLingPapercolawidth}}{Abbreviations\protect\footnote{{\leftskip0pt\parindent1em\raisebox{\baselineskip}[0pt]{\protect\hypertarget{nAbbreviations}{}} In West Africa it is common for the abbreviation of the name of an organization to come to be used as the name.}}}&\multicolumn{1}{p{\XLingPapercolbwidth}}{}&\multicolumn{1}{p{\XLingPapercolcwidth}@{}}{UNICEF}\\%
\multicolumn{1}{@{}p{\XLingPapercolawidth}}{}&\multicolumn{1}{p{\XLingPapercolbwidth}}{}&\multicolumn{1}{p{\XLingPapercolcwidth}@{}}{UEESO}\\%
\multicolumn{1}{@{}p{\XLingPapercolawidth}}{}&\multicolumn{1}{p{\XLingPapercolbwidth}}{}&\multicolumn{1}{p{\XLingPapercolcwidth}@{}}{BCEAO}\\%
\multicolumn{1}{@{}p{\XLingPapercolawidth}}{}&\multicolumn{1}{p{\XLingPapercolbwidth}}{}&\multicolumn{1}{p{\XLingPapercolcwidth}@{}}{}\\%
\multicolumn{1}{@{}p{\XLingPapercolawidth}}{Concepts}&\multicolumn{1}{p{\XLingPapercolbwidth}}{4 x 4 (eng-fra) - type of vehicle}&\multicolumn{1}{p{\XLingPapercolcwidth}@{}}{4 x 4}\\%
\multicolumn{1}{@{}p{\XLingPapercolawidth}}{}&\multicolumn{1}{p{\XLingPapercolbwidth}}{Million (eng-fra)}&\multicolumn{1}{p{\XLingPapercolcwidth}@{}}{miliɔn}\\\bottomrule%
\end{longtable}
}
}{\vspace{10pt}\XLingPaperneedspace{3\baselineskip}\noindent
\fontsize{13}{15.6}\selectfont \textit{{\noindent
\raisebox{\baselineskip}[0pt]{\pdfbookmark[3]{{3.10.1 } Auxiliary characters}{sAuxiliaryCharacters}}\raisebox{\baselineskip}[0pt]{\protect\hypertarget{sAuxiliaryCharacters}{}}{3.10.1 }Auxiliary characters}}\markboth{Auxiliary characters}{Eastern Dan writing system}\XLingPaperaddtocontents{sAuxiliaryCharacters}}\par{}
\penalty10000\vspace{10pt}\penalty10000\indent So if we were to include characters which are not frequently used in Dan, but are in some way needed in the writing system we might come close to some sort of statement like that of auxiliary characters. Auxiliary characters are characters which are not in an alphabet, might not be in a sort order but are needed in a writing system. Unicode informally (not via the properties of the {\hyperlink{vUCD}{{UCD}}}) defines five categories of characters in TR35 \hyperlink{rUmaoka}{(Umaoka et al.  2018)}. These are shown in table \hyperlink{ntAuxCharacters}{26}.\par{}\vspace{11pt plus 2pt minus 1pt}\XLingPaperneedspace{3\baselineskip}\protect\hypertarget{ntUnicodeCatagories}{}\XLingPaperaddtocontents{ntUnicodeCatagories}{\protect\raggedright{\singlespacing
{Table }{25.}{  Unicode Categories\\}}}\vspace{0pt}{\singlespacing
\hspace*{.25in}{
\XLingPaperminmaxcellincolumn{Type}{\XLingPapermincola}{\textbf{Type}}{\XLingPapermaxcola}{+0\tabcolsep}
\XLingPaperminmaxcellincolumn{Description}{\XLingPapermincolb}{\textbf{Description}}{\XLingPapermaxcolb}{+0\tabcolsep}
\XLingPaperminmaxcellincolumn{standard}{\XLingPapermincola}{\XLingPaperCharisZSILFontFamily{\fontsize{8}{9.6}\selectfont \textup{\textup{Main / standard}}}}{\XLingPapermaxcola}{+0\tabcolsep}
\XLingPaperminmaxcellincolumn{language}{\XLingPapermincolb}{\XLingPaperCharisZSILFontFamily{\fontsize{8}{9.6}\selectfont \textup{\textup{Main letters used in the language}}}}{\XLingPapermaxcolb}{+0\tabcolsep}
\XLingPaperminmaxcellincolumn{Auxiliary}{\XLingPapermincola}{\XLingPaperCharisZSILFontFamily{\fontsize{8}{9.6}\selectfont \textup{\textup{Auxiliary}}}}{\XLingPapermaxcola}{+0\tabcolsep}
\XLingPaperminmaxcellincolumn{technical}{\XLingPapermincolb}{\XLingPaperCharisZSILFontFamily{\fontsize{8}{9.6}\selectfont \textup{\textup{Additional characters for common foreign words, technical usage}}}}{\XLingPapermaxcolb}{+0\tabcolsep}
\XLingPaperminmaxcellincolumn{Index}{\XLingPapermincola}{\XLingPaperCharisZSILFontFamily{\fontsize{8}{9.6}\selectfont \textup{\textup{Index}}}}{\XLingPapermaxcola}{+0\tabcolsep}
\XLingPaperminmaxcellincolumn{header}{\XLingPapermincolb}{\XLingPaperCharisZSILFontFamily{\fontsize{8}{9.6}\selectfont \textup{\textup{Characters for the header of an index}}}}{\XLingPapermaxcolb}{+0\tabcolsep}
\XLingPaperminmaxcellincolumn{Punctuation}{\XLingPapermincola}{\XLingPaperCharisZSILFontFamily{\fontsize{8}{9.6}\selectfont \textup{\textup{Punctuation}}}}{\XLingPapermaxcola}{+0\tabcolsep}
\XLingPaperminmaxcellincolumn{Common}{\XLingPapermincolb}{\XLingPaperCharisZSILFontFamily{\fontsize{8}{9.6}\selectfont \textup{\textup{Common punctuation}}}}{\XLingPapermaxcolb}{+0\tabcolsep}
\XLingPaperminmaxcellincolumn{Numbers}{\XLingPapermincola}{\XLingPaperCharisZSILFontFamily{\fontsize{8}{9.6}\selectfont \textup{\textup{Numbers}}}}{\XLingPapermaxcola}{+0\tabcolsep}
\XLingPaperminmaxcellincolumn{currency.}{\XLingPapermincolb}{\XLingPaperCharisZSILFontFamily{\fontsize{8}{9.6}\selectfont \textup{\textup{The characters needed to display the common number formats: decimal, percent, and currency.}}}}{\XLingPapermaxcolb}{+0\tabcolsep}
\setlength{\XLingPaperavailabletablewidth}{433.62pt}
\setlength{\XLingPapertableminwidth}{\XLingPapermincola+\XLingPapermincolb}
\setlength{\XLingPapertablemaxwidth}{\XLingPapermaxcola+\XLingPapermaxcolb}
\XLingPapercalculatetablewidthratio{}
\XLingPapersetcolumnwidth{\XLingPapercolawidth}{\XLingPapermincola}{\XLingPapermaxcola}{-0\tabcolsep}
\XLingPapersetcolumnwidth{\XLingPapercolbwidth}{\XLingPapermincolb}{\XLingPapermaxcolb}{-2\tabcolsep}\singlespacing\vspace*{-3\baselineskip}
\begin{longtable}
[l]{@{}p{\XLingPapercolawidth}p{\XLingPapercolbwidth}@{}}\toprule\multicolumn{1}{@{}p{\XLingPapercolawidth}}{\textbf{Type}}&\multicolumn{1}{p{\XLingPapercolbwidth}@{}}{\textbf{Description}}\\%
\midrule\endhead \multicolumn{1}{@{}p{\XLingPapercolawidth}}{\XLingPaperCharisZSILFontFamily{\fontsize{8}{9.6}\selectfont \textup{\textup{Main / standard}}}}&\multicolumn{1}{p{\XLingPapercolbwidth}@{}}{\XLingPaperCharisZSILFontFamily{\fontsize{8}{9.6}\selectfont \textup{\textup{Main letters used in the language}}}}\\[1.5pt]%
\multicolumn{1}{@{}p{\XLingPapercolawidth}}{\XLingPaperCharisZSILFontFamily{\fontsize{8}{9.6}\selectfont \textup{\textup{Auxiliary}}}}&\multicolumn{1}{p{\XLingPapercolbwidth}@{}}{\XLingPaperCharisZSILFontFamily{\fontsize{8}{9.6}\selectfont \textup{\textup{Additional characters for common foreign words, technical usage}}}}\\[1.5pt]%
\multicolumn{1}{@{}p{\XLingPapercolawidth}}{\XLingPaperCharisZSILFontFamily{\fontsize{8}{9.6}\selectfont \textup{\textup{Index}}}}&\multicolumn{1}{p{\XLingPapercolbwidth}@{}}{\XLingPaperCharisZSILFontFamily{\fontsize{8}{9.6}\selectfont \textup{\textup{Characters for the header of an index}}}}\\[1.5pt]%
\multicolumn{1}{@{}p{\XLingPapercolawidth}}{\XLingPaperCharisZSILFontFamily{\fontsize{8}{9.6}\selectfont \textup{\textup{Punctuation}}}}&\multicolumn{1}{p{\XLingPapercolbwidth}@{}}{\XLingPaperCharisZSILFontFamily{\fontsize{8}{9.6}\selectfont \textup{\textup{Common punctuation}}}}\\[1.5pt]%
\multicolumn{1}{@{}p{\XLingPapercolawidth}}{\XLingPaperCharisZSILFontFamily{\fontsize{8}{9.6}\selectfont \textup{\textup{Numbers}}}}&\multicolumn{1}{p{\XLingPapercolbwidth}@{}}{\XLingPaperCharisZSILFontFamily{\fontsize{8}{9.6}\selectfont \textup{\textup{The characters needed to display the common number formats: decimal, percent, and currency.}}}}\\[1.5pt]\bottomrule%
\end{longtable}
}
}\indent Based on the full Dan corpus, it would appear that the following characters should be added to the writing system as auxiliary characters. This does not mean that they are part of the orthography, only that technical devices should include them when working with Eastern Dan data.\par{}\vspace{11pt plus 2pt minus 1pt}\XLingPaperneedspace{3\baselineskip}\protect\hypertarget{ntAuxCharacters}{}\XLingPaperaddtocontents{ntAuxCharacters}{\protect\raggedright{\singlespacing
{Table }{26.}{  List of auxiliary characters used in Eastern Dan\\}}}\vspace{0pt}{\singlespacing
\hspace*{.25in}{
\XLingPaperminmaxcellincolumn{Codepoint}{\XLingPapermincola}{\textbf{Codepoint}}{\XLingPapermaxcola}{+0\tabcolsep}
\XLingPaperminmaxcellincolumn{Glyph}{\XLingPapermincolb}{\textbf{Glyph}}{\XLingPapermaxcolb}{+0\tabcolsep}
\XLingPaperminmaxcellincolumn{U+}{\XLingPapermincola}{U+}{\XLingPapermaxcola}{+0\tabcolsep}
\XLingPaperminmaxcellincolumn{}{\XLingPapermincolb}{}{\XLingPapermaxcolb}{+0\tabcolsep}
\XLingPaperminmaxcellincolumn{U+}{\XLingPapermincola}{U+}{\XLingPapermaxcola}{+0\tabcolsep}
\XLingPaperminmaxcellincolumn{}{\XLingPapermincolb}{}{\XLingPapermaxcolb}{+0\tabcolsep}
\XLingPaperminmaxcellincolumn{U+}{\XLingPapermincola}{U+}{\XLingPapermaxcola}{+0\tabcolsep}
\XLingPaperminmaxcellincolumn{}{\XLingPapermincolb}{}{\XLingPapermaxcolb}{+0\tabcolsep}
\XLingPaperminmaxcellincolumn{U+}{\XLingPapermincola}{U+}{\XLingPapermaxcola}{+0\tabcolsep}
\XLingPaperminmaxcellincolumn{}{\XLingPapermincolb}{}{\XLingPapermaxcolb}{+0\tabcolsep}
\XLingPaperminmaxcellincolumn{U+}{\XLingPapermincola}{U+}{\XLingPapermaxcola}{+0\tabcolsep}
\XLingPaperminmaxcellincolumn{}{\XLingPapermincolb}{}{\XLingPapermaxcolb}{+0\tabcolsep}
\XLingPaperminmaxcellincolumn{U+}{\XLingPapermincola}{U+}{\XLingPapermaxcola}{+0\tabcolsep}
\XLingPaperminmaxcellincolumn{}{\XLingPapermincolb}{}{\XLingPapermaxcolb}{+0\tabcolsep}
\XLingPaperminmaxcellincolumn{U+}{\XLingPapermincola}{U+}{\XLingPapermaxcola}{+0\tabcolsep}
\XLingPaperminmaxcellincolumn{}{\XLingPapermincolb}{}{\XLingPapermaxcolb}{+0\tabcolsep}
\XLingPaperminmaxcellincolumn{U+}{\XLingPapermincola}{U+}{\XLingPapermaxcola}{+0\tabcolsep}
\XLingPaperminmaxcellincolumn{}{\XLingPapermincolb}{}{\XLingPapermaxcolb}{+0\tabcolsep}
\XLingPaperminmaxcellincolumn{U+}{\XLingPapermincola}{U+}{\XLingPapermaxcola}{+0\tabcolsep}
\XLingPaperminmaxcellincolumn{}{\XLingPapermincolb}{}{\XLingPapermaxcolb}{+0\tabcolsep}
\XLingPaperminmaxcellincolumn{U+}{\XLingPapermincola}{U+}{\XLingPapermaxcola}{+0\tabcolsep}
\XLingPaperminmaxcellincolumn{}{\XLingPapermincolb}{}{\XLingPapermaxcolb}{+0\tabcolsep}
\setlength{\XLingPaperavailabletablewidth}{433.62pt}
\setlength{\XLingPapertableminwidth}{\XLingPapermincola+\XLingPapermincolb}
\setlength{\XLingPapertablemaxwidth}{\XLingPapermaxcola+\XLingPapermaxcolb}
\XLingPapercalculatetablewidthratio{}
\XLingPapersetcolumnwidth{\XLingPapercolawidth}{\XLingPapermincola}{\XLingPapermaxcola}{-0\tabcolsep}
\XLingPapersetcolumnwidth{\XLingPapercolbwidth}{\XLingPapermincolb}{\XLingPapermaxcolb}{-2\tabcolsep}\singlespacing\vspace*{-3\baselineskip}
\begin{longtable}
[l]{@{}p{\XLingPapercolawidth}p{\XLingPapercolbwidth}@{}}\toprule\multicolumn{1}{@{}p{\XLingPapercolawidth}}{\textbf{Codepoint}}&\multicolumn{1}{p{\XLingPapercolbwidth}@{}}{\textbf{Glyph}}\\%
\midrule\endhead \multicolumn{1}{@{}p{\XLingPapercolawidth}}{U+}&\multicolumn{1}{p{\XLingPapercolbwidth}@{}}{}\\%
\multicolumn{1}{@{}p{\XLingPapercolawidth}}{U+}&\multicolumn{1}{p{\XLingPapercolbwidth}@{}}{}\\%
\multicolumn{1}{@{}p{\XLingPapercolawidth}}{U+}&\multicolumn{1}{p{\XLingPapercolbwidth}@{}}{}\\%
\multicolumn{1}{@{}p{\XLingPapercolawidth}}{U+}&\multicolumn{1}{p{\XLingPapercolbwidth}@{}}{}\\%
\multicolumn{1}{@{}p{\XLingPapercolawidth}}{U+}&\multicolumn{1}{p{\XLingPapercolbwidth}@{}}{}\\%
\multicolumn{1}{@{}p{\XLingPapercolawidth}}{U+}&\multicolumn{1}{p{\XLingPapercolbwidth}@{}}{}\\%
\multicolumn{1}{@{}p{\XLingPapercolawidth}}{U+}&\multicolumn{1}{p{\XLingPapercolbwidth}@{}}{}\\%
\multicolumn{1}{@{}p{\XLingPapercolawidth}}{U+}&\multicolumn{1}{p{\XLingPapercolbwidth}@{}}{}\\%
\multicolumn{1}{@{}p{\XLingPapercolawidth}}{U+}&\multicolumn{1}{p{\XLingPapercolbwidth}@{}}{}\\%
\multicolumn{1}{@{}p{\XLingPapercolawidth}}{U+}&\multicolumn{1}{p{\XLingPapercolbwidth}@{}}{}\\\bottomrule%
\end{longtable}
}
}\vspace{11pt plus 2pt minus 1pt}\vspace{11pt plus 2pt minus 1pt}\XLingPaperneedspace{3\baselineskip}\protect\hypertarget{ntFrenchCharacters}{}\XLingPaperaddtocontents{ntFrenchCharacters}{\protect\raggedright{\singlespacing
{Table }{27.}{  French Characters\\}}}\vspace{0pt}{\singlespacing
\hspace*{.25in}{
\XLingPaperminmaxcellincolumn{Type}{\XLingPapermincola}{\textbf{Type}}{\XLingPapermaxcola}{+0\tabcolsep}
\XLingPaperminmaxcellincolumn{Description}{\XLingPapermincolb}{\textbf{Description}}{\XLingPapermaxcolb}{+0\tabcolsep}
\XLingPaperminmaxcellincolumn{Glyphs}{\XLingPapermincolc}{\textbf{Glyphs}}{\XLingPapermaxcolc}{+0\tabcolsep}
\XLingPaperminmaxcellincolumn{standard}{\XLingPapermincola}{\XLingPaperCharisZSILFontFamily{\fontsize{8}{9.6}\selectfont \textup{\textup{Main / standard}}}}{\XLingPapermaxcola}{+0\tabcolsep}
\XLingPaperminmaxcellincolumn{language}{\XLingPapermincolb}{\XLingPaperCharisZSILFontFamily{\fontsize{8}{9.6}\selectfont \textup{\textup{Main letters used in the language}}}}{\XLingPapermaxcolb}{+0\tabcolsep}
\XLingPaperminmaxcellincolumn{}{\XLingPapermincolc}{\XLingPaperCharisZSILFontFamily{\fontsize{8}{9.6}\selectfont \textup{\textup{}}}}{\XLingPapermaxcolc}{+0\tabcolsep}
\XLingPaperminmaxcellincolumn{Auxiliary}{\XLingPapermincola}{\XLingPaperCharisZSILFontFamily{\fontsize{8}{9.6}\selectfont \textup{\textup{Auxiliary}}}}{\XLingPapermaxcola}{+0\tabcolsep}
\XLingPaperminmaxcellincolumn{technical}{\XLingPapermincolb}{\XLingPaperCharisZSILFontFamily{\fontsize{8}{9.6}\selectfont \textup{\textup{Additional characters for common foreign words, technical usage}}}}{\XLingPapermaxcolb}{+0\tabcolsep}
\XLingPaperminmaxcellincolumn{}{\XLingPapermincolc}{\XLingPaperCharisZSILFontFamily{\fontsize{8}{9.6}\selectfont \textup{\textup{}}}}{\XLingPapermaxcolc}{+0\tabcolsep}
\XLingPaperminmaxcellincolumn{Index}{\XLingPapermincola}{\XLingPaperCharisZSILFontFamily{\fontsize{8}{9.6}\selectfont \textup{\textup{Index}}}}{\XLingPapermaxcola}{+0\tabcolsep}
\XLingPaperminmaxcellincolumn{header}{\XLingPapermincolb}{\XLingPaperCharisZSILFontFamily{\fontsize{8}{9.6}\selectfont \textup{\textup{Characters for the header of an index}}}}{\XLingPapermaxcolb}{+0\tabcolsep}
\XLingPaperminmaxcellincolumn{}{\XLingPapermincolc}{\XLingPaperCharisZSILFontFamily{\fontsize{8}{9.6}\selectfont \textup{\textup{}}}}{\XLingPapermaxcolc}{+0\tabcolsep}
\XLingPaperminmaxcellincolumn{Punctuation}{\XLingPapermincola}{\XLingPaperCharisZSILFontFamily{\fontsize{8}{9.6}\selectfont \textup{\textup{Punctuation}}}}{\XLingPapermaxcola}{+0\tabcolsep}
\XLingPaperminmaxcellincolumn{Common}{\XLingPapermincolb}{\XLingPaperCharisZSILFontFamily{\fontsize{8}{9.6}\selectfont \textup{\textup{Common punctuation}}}}{\XLingPapermaxcolb}{+0\tabcolsep}
\XLingPaperminmaxcellincolumn{}{\XLingPapermincolc}{\XLingPaperCharisZSILFontFamily{\fontsize{8}{9.6}\selectfont \textup{\textup{}}}}{\XLingPapermaxcolc}{+0\tabcolsep}
\XLingPaperminmaxcellincolumn{Numbers}{\XLingPapermincola}{\XLingPaperCharisZSILFontFamily{\fontsize{8}{9.6}\selectfont \textup{\textup{Numbers}}}}{\XLingPapermaxcola}{+0\tabcolsep}
\XLingPaperminmaxcellincolumn{currency.}{\XLingPapermincolb}{\XLingPaperCharisZSILFontFamily{\fontsize{8}{9.6}\selectfont \textup{\textup{The characters needed to display the common number formats: decimal, percent, and currency.}}}}{\XLingPapermaxcolb}{+0\tabcolsep}
\XLingPaperminmaxcellincolumn{}{\XLingPapermincolc}{\XLingPaperCharisZSILFontFamily{\fontsize{8}{9.6}\selectfont \textup{\textup{}}}}{\XLingPapermaxcolc}{+0\tabcolsep}
\setlength{\XLingPaperavailabletablewidth}{433.62pt}
\setlength{\XLingPapertableminwidth}{\XLingPapermincola+\XLingPapermincolb+\XLingPapermincolc}
\setlength{\XLingPapertablemaxwidth}{\XLingPapermaxcola+\XLingPapermaxcolb+\XLingPapermaxcolc}
\XLingPapercalculatetablewidthratio{}
\XLingPapersetcolumnwidth{\XLingPapercolawidth}{\XLingPapermincola}{\XLingPapermaxcola}{-0\tabcolsep}
\XLingPapersetcolumnwidth{\XLingPapercolbwidth}{\XLingPapermincolb}{\XLingPapermaxcolb}{-2\tabcolsep}
\XLingPapersetcolumnwidth{\XLingPapercolcwidth}{\XLingPapermincolc}{\XLingPapermaxcolc}{-2\tabcolsep}\singlespacing\vspace*{-3\baselineskip}
\begin{longtable}
[l]{@{}p{\XLingPapercolawidth}p{\XLingPapercolbwidth}p{\XLingPapercolcwidth}@{}}\toprule\multicolumn{1}{@{}p{\XLingPapercolawidth}}{\textbf{Type}}&\multicolumn{1}{p{\XLingPapercolbwidth}}{\textbf{Description}}&\multicolumn{1}{p{\XLingPapercolcwidth}@{}}{\textbf{Glyphs}}\\%
\midrule\endhead \multicolumn{1}{@{}p{\XLingPapercolawidth}}{\XLingPaperCharisZSILFontFamily{\fontsize{8}{9.6}\selectfont \textup{\textup{Main / standard}}}}&\multicolumn{1}{p{\XLingPapercolbwidth}}{\XLingPaperCharisZSILFontFamily{\fontsize{8}{9.6}\selectfont \textup{\textup{Main letters used in the language}}}}&\multicolumn{1}{p{\XLingPapercolcwidth}@{}}{\XLingPaperCharisZSILFontFamily{\fontsize{8}{9.6}\selectfont \textup{\textup{}}}}\\[1.5pt]%
\multicolumn{1}{@{}p{\XLingPapercolawidth}}{\XLingPaperCharisZSILFontFamily{\fontsize{8}{9.6}\selectfont \textup{\textup{Auxiliary}}}}&\multicolumn{1}{p{\XLingPapercolbwidth}}{\XLingPaperCharisZSILFontFamily{\fontsize{8}{9.6}\selectfont \textup{\textup{Additional characters for common foreign words, technical usage}}}}&\multicolumn{1}{p{\XLingPapercolcwidth}@{}}{\XLingPaperCharisZSILFontFamily{\fontsize{8}{9.6}\selectfont \textup{\textup{}}}}\\[1.5pt]%
\multicolumn{1}{@{}p{\XLingPapercolawidth}}{\XLingPaperCharisZSILFontFamily{\fontsize{8}{9.6}\selectfont \textup{\textup{Index}}}}&\multicolumn{1}{p{\XLingPapercolbwidth}}{\XLingPaperCharisZSILFontFamily{\fontsize{8}{9.6}\selectfont \textup{\textup{Characters for the header of an index}}}}&\multicolumn{1}{p{\XLingPapercolcwidth}@{}}{\XLingPaperCharisZSILFontFamily{\fontsize{8}{9.6}\selectfont \textup{\textup{}}}}\\[1.5pt]%
\multicolumn{1}{@{}p{\XLingPapercolawidth}}{\XLingPaperCharisZSILFontFamily{\fontsize{8}{9.6}\selectfont \textup{\textup{Punctuation}}}}&\multicolumn{1}{p{\XLingPapercolbwidth}}{\XLingPaperCharisZSILFontFamily{\fontsize{8}{9.6}\selectfont \textup{\textup{Common punctuation}}}}&\multicolumn{1}{p{\XLingPapercolcwidth}@{}}{\XLingPaperCharisZSILFontFamily{\fontsize{8}{9.6}\selectfont \textup{\textup{}}}}\\[1.5pt]%
\multicolumn{1}{@{}p{\XLingPapercolawidth}}{\XLingPaperCharisZSILFontFamily{\fontsize{8}{9.6}\selectfont \textup{\textup{Numbers}}}}&\multicolumn{1}{p{\XLingPapercolbwidth}}{\XLingPaperCharisZSILFontFamily{\fontsize{8}{9.6}\selectfont \textup{\textup{The characters needed to display the common number formats: decimal, percent, and currency.}}}}&\multicolumn{1}{p{\XLingPapercolcwidth}@{}}{\XLingPaperCharisZSILFontFamily{\fontsize{8}{9.6}\selectfont \textup{\textup{}}}}\\[1.5pt]\bottomrule%
\end{longtable}
}
}\vspace{11pt plus 2pt minus 1pt}\vspace{11pt plus 2pt minus 1pt}\XLingPaperneedspace{3\baselineskip}\protect\hypertarget{ntJulaCharacters}{}\XLingPaperaddtocontents{ntJulaCharacters}{\protect\raggedright{\singlespacing
{Table }{28.}{  Jula Characters\\}}}\vspace{0pt}{\singlespacing
\hspace*{.25in}{
\XLingPaperminmaxcellincolumn{Type}{\XLingPapermincola}{\textbf{Type}}{\XLingPapermaxcola}{+0\tabcolsep}
\XLingPaperminmaxcellincolumn{Description}{\XLingPapermincolb}{\textbf{Description}}{\XLingPapermaxcolb}{+0\tabcolsep}
\XLingPaperminmaxcellincolumn{Glyphs}{\XLingPapermincolc}{\textbf{Glyphs}}{\XLingPapermaxcolc}{+0\tabcolsep}
\XLingPaperminmaxcellincolumn{standard}{\XLingPapermincola}{Main / standard}{\XLingPapermaxcola}{+0\tabcolsep}
\XLingPaperminmaxcellincolumn{language}{\XLingPapermincolb}{Main letters used in the language}{\XLingPapermaxcolb}{+0\tabcolsep}
\XLingPaperminmaxcellincolumn{}{\XLingPapermincolc}{}{\XLingPapermaxcolc}{+0\tabcolsep}
\XLingPaperminmaxcellincolumn{}{\XLingPapermincola}{}{\XLingPapermaxcola}{+0\tabcolsep}
\XLingPaperminmaxcellincolumn{}{\XLingPapermincolb}{}{\XLingPapermaxcolb}{+0\tabcolsep}
\XLingPaperminmaxcellincolumn{}{\XLingPapermincolc}{}{\XLingPapermaxcolc}{+0\tabcolsep}
\XLingPaperminmaxcellincolumn{Auxiliary}{\XLingPapermincola}{Auxiliary}{\XLingPapermaxcola}{+0\tabcolsep}
\XLingPaperminmaxcellincolumn{technical}{\XLingPapermincolb}{Additional characters for common foreign words, technical usage}{\XLingPapermaxcolb}{+0\tabcolsep}
\XLingPaperminmaxcellincolumn{ }{\XLingPapermincolc}{ }{\XLingPapermaxcolc}{+0\tabcolsep}
\XLingPaperminmaxcellincolumn{}{\XLingPapermincola}{}{\XLingPapermaxcola}{+0\tabcolsep}
\XLingPaperminmaxcellincolumn{}{\XLingPapermincolb}{}{\XLingPapermaxcolb}{+0\tabcolsep}
\XLingPaperminmaxcellincolumn{}{\XLingPapermincolc}{}{\XLingPapermaxcolc}{+0\tabcolsep}
\XLingPaperminmaxcellincolumn{Index}{\XLingPapermincola}{Index}{\XLingPapermaxcola}{+0\tabcolsep}
\XLingPaperminmaxcellincolumn{header}{\XLingPapermincolb}{Characters for the header of an index}{\XLingPapermaxcolb}{+0\tabcolsep}
\XLingPaperminmaxcellincolumn{}{\XLingPapermincolc}{}{\XLingPapermaxcolc}{+0\tabcolsep}
\XLingPaperminmaxcellincolumn{}{\XLingPapermincola}{}{\XLingPapermaxcola}{+0\tabcolsep}
\XLingPaperminmaxcellincolumn{}{\XLingPapermincolb}{}{\XLingPapermaxcolb}{+0\tabcolsep}
\XLingPaperminmaxcellincolumn{}{\XLingPapermincolc}{}{\XLingPapermaxcolc}{+0\tabcolsep}
\XLingPaperminmaxcellincolumn{Punctuation}{\XLingPapermincola}{Punctuation}{\XLingPapermaxcola}{+0\tabcolsep}
\XLingPaperminmaxcellincolumn{Common}{\XLingPapermincolb}{Common punctuation}{\XLingPapermaxcolb}{+0\tabcolsep}
\XLingPaperminmaxcellincolumn{}{\XLingPapermincolc}{}{\XLingPapermaxcolc}{+0\tabcolsep}
\XLingPaperminmaxcellincolumn{}{\XLingPapermincola}{}{\XLingPapermaxcola}{+0\tabcolsep}
\XLingPaperminmaxcellincolumn{}{\XLingPapermincolb}{}{\XLingPapermaxcolb}{+0\tabcolsep}
\XLingPaperminmaxcellincolumn{}{\XLingPapermincolc}{}{\XLingPapermaxcolc}{+0\tabcolsep}
\XLingPaperminmaxcellincolumn{Numbers}{\XLingPapermincola}{Numbers}{\XLingPapermaxcola}{+0\tabcolsep}
\XLingPaperminmaxcellincolumn{currency.}{\XLingPapermincolb}{The characters needed to display the common number formats: decimal, percent, and currency.}{\XLingPapermaxcolb}{+0\tabcolsep}
\XLingPaperminmaxcellincolumn{}{\XLingPapermincolc}{}{\XLingPapermaxcolc}{+0\tabcolsep}
\setlength{\XLingPaperavailabletablewidth}{433.62pt}
\setlength{\XLingPapertableminwidth}{\XLingPapermincola+\XLingPapermincolb+\XLingPapermincolc}
\setlength{\XLingPapertablemaxwidth}{\XLingPapermaxcola+\XLingPapermaxcolb+\XLingPapermaxcolc}
\XLingPapercalculatetablewidthratio{}
\XLingPapersetcolumnwidth{\XLingPapercolawidth}{\XLingPapermincola}{\XLingPapermaxcola}{-0\tabcolsep}
\XLingPapersetcolumnwidth{\XLingPapercolbwidth}{\XLingPapermincolb}{\XLingPapermaxcolb}{-2\tabcolsep}
\XLingPapersetcolumnwidth{\XLingPapercolcwidth}{\XLingPapermincolc}{\XLingPapermaxcolc}{-2\tabcolsep}\singlespacing\vspace*{-3\baselineskip}
\begin{longtable}
[l]{@{}p{\XLingPapercolawidth}p{\XLingPapercolbwidth}p{\XLingPapercolcwidth}@{}}\toprule\multicolumn{1}{@{}p{\XLingPapercolawidth}}{\textbf{Type}}&\multicolumn{1}{p{\XLingPapercolbwidth}}{\textbf{Description}}&\multicolumn{1}{p{\XLingPapercolcwidth}@{}}{\textbf{Glyphs}}\\%
\midrule\endhead \multicolumn{1}{@{}p{\XLingPapercolawidth}}{Main / standard}&\multicolumn{1}{p{\XLingPapercolbwidth}}{Main letters used in the language}&\multicolumn{1}{p{\XLingPapercolcwidth}@{}}{}\\%
\multicolumn{1}{@{}p{\XLingPapercolawidth}}{}&\multicolumn{1}{p{\XLingPapercolbwidth}}{}&\multicolumn{1}{p{\XLingPapercolcwidth}@{}}{}\\%
\multicolumn{1}{@{}p{\XLingPapercolawidth}}{Auxiliary}&\multicolumn{1}{p{\XLingPapercolbwidth}}{Additional characters for common foreign words, technical usage}&\multicolumn{1}{p{\XLingPapercolcwidth}@{}}{ }\\%
\multicolumn{1}{@{}p{\XLingPapercolawidth}}{}&\multicolumn{1}{p{\XLingPapercolbwidth}}{}&\multicolumn{1}{p{\XLingPapercolcwidth}@{}}{}\\%
\multicolumn{1}{@{}p{\XLingPapercolawidth}}{Index}&\multicolumn{1}{p{\XLingPapercolbwidth}}{Characters for the header of an index}&\multicolumn{1}{p{\XLingPapercolcwidth}@{}}{}\\%
\multicolumn{1}{@{}p{\XLingPapercolawidth}}{}&\multicolumn{1}{p{\XLingPapercolbwidth}}{}&\multicolumn{1}{p{\XLingPapercolcwidth}@{}}{}\\%
\multicolumn{1}{@{}p{\XLingPapercolawidth}}{Punctuation}&\multicolumn{1}{p{\XLingPapercolbwidth}}{Common punctuation}&\multicolumn{1}{p{\XLingPapercolcwidth}@{}}{}\\%
\multicolumn{1}{@{}p{\XLingPapercolawidth}}{}&\multicolumn{1}{p{\XLingPapercolbwidth}}{}&\multicolumn{1}{p{\XLingPapercolcwidth}@{}}{}\\%
\multicolumn{1}{@{}p{\XLingPapercolawidth}}{Numbers}&\multicolumn{1}{p{\XLingPapercolbwidth}}{The characters needed to display the common number formats: decimal, percent, and currency.}&\multicolumn{1}{p{\XLingPapercolcwidth}@{}}{}\\\bottomrule%
\end{longtable}
}
}\vspace{11pt plus 2pt minus 1pt}\vspace{11pt plus 2pt minus 1pt}\XLingPaperneedspace{3\baselineskip}\protect\hypertarget{ntEnglishCharacters}{}\XLingPaperaddtocontents{ntEnglishCharacters}{\protect\raggedright{\singlespacing
{Table }{29.}{  English Characters\\}}}\vspace{0pt}{\singlespacing
\hspace*{.25in}{
\XLingPaperminmaxcellincolumn{Type}{\XLingPapermincola}{\textbf{Type}}{\XLingPapermaxcola}{+0\tabcolsep}
\XLingPaperminmaxcellincolumn{Description}{\XLingPapermincolb}{\textbf{Description}}{\XLingPapermaxcolb}{+0\tabcolsep}
\XLingPaperminmaxcellincolumn{Glyphs}{\XLingPapermincolc}{\textbf{Glyphs}}{\XLingPapermaxcolc}{+0\tabcolsep}
\XLingPaperminmaxcellincolumn{standard}{\XLingPapermincola}{Main / standard}{\XLingPapermaxcola}{+0\tabcolsep}
\XLingPaperminmaxcellincolumn{language}{\XLingPapermincolb}{Main letters used in the language}{\XLingPapermaxcolb}{+0\tabcolsep}
\XLingPaperminmaxcellincolumn{}{\XLingPapermincolc}{}{\XLingPapermaxcolc}{+0\tabcolsep}
\XLingPaperminmaxcellincolumn{}{\XLingPapermincola}{}{\XLingPapermaxcola}{+0\tabcolsep}
\XLingPaperminmaxcellincolumn{}{\XLingPapermincolb}{}{\XLingPapermaxcolb}{+0\tabcolsep}
\XLingPaperminmaxcellincolumn{}{\XLingPapermincolc}{}{\XLingPapermaxcolc}{+0\tabcolsep}
\XLingPaperminmaxcellincolumn{Auxiliary}{\XLingPapermincola}{Auxiliary}{\XLingPapermaxcola}{+0\tabcolsep}
\XLingPaperminmaxcellincolumn{technical}{\XLingPapermincolb}{Additional characters for common foreign words, technical usage}{\XLingPapermaxcolb}{+0\tabcolsep}
\XLingPaperminmaxcellincolumn{}{\XLingPapermincolc}{}{\XLingPapermaxcolc}{+0\tabcolsep}
\XLingPaperminmaxcellincolumn{}{\XLingPapermincola}{}{\XLingPapermaxcola}{+0\tabcolsep}
\XLingPaperminmaxcellincolumn{}{\XLingPapermincolb}{}{\XLingPapermaxcolb}{+0\tabcolsep}
\XLingPaperminmaxcellincolumn{}{\XLingPapermincolc}{}{\XLingPapermaxcolc}{+0\tabcolsep}
\XLingPaperminmaxcellincolumn{Index}{\XLingPapermincola}{Index}{\XLingPapermaxcola}{+0\tabcolsep}
\XLingPaperminmaxcellincolumn{header}{\XLingPapermincolb}{Characters for the header of an index}{\XLingPapermaxcolb}{+0\tabcolsep}
\XLingPaperminmaxcellincolumn{}{\XLingPapermincolc}{}{\XLingPapermaxcolc}{+0\tabcolsep}
\XLingPaperminmaxcellincolumn{}{\XLingPapermincola}{}{\XLingPapermaxcola}{+0\tabcolsep}
\XLingPaperminmaxcellincolumn{}{\XLingPapermincolb}{}{\XLingPapermaxcolb}{+0\tabcolsep}
\XLingPaperminmaxcellincolumn{}{\XLingPapermincolc}{}{\XLingPapermaxcolc}{+0\tabcolsep}
\XLingPaperminmaxcellincolumn{Punctuation}{\XLingPapermincola}{Punctuation}{\XLingPapermaxcola}{+0\tabcolsep}
\XLingPaperminmaxcellincolumn{Common}{\XLingPapermincolb}{Common punctuation}{\XLingPapermaxcolb}{+0\tabcolsep}
\XLingPaperminmaxcellincolumn{}{\XLingPapermincolc}{}{\XLingPapermaxcolc}{+0\tabcolsep}
\XLingPaperminmaxcellincolumn{}{\XLingPapermincola}{}{\XLingPapermaxcola}{+0\tabcolsep}
\XLingPaperminmaxcellincolumn{}{\XLingPapermincolb}{}{\XLingPapermaxcolb}{+0\tabcolsep}
\XLingPaperminmaxcellincolumn{}{\XLingPapermincolc}{}{\XLingPapermaxcolc}{+0\tabcolsep}
\XLingPaperminmaxcellincolumn{Numbers}{\XLingPapermincola}{Numbers}{\XLingPapermaxcola}{+0\tabcolsep}
\XLingPaperminmaxcellincolumn{currency.}{\XLingPapermincolb}{The characters needed to display the common number formats: decimal, percent, and currency.}{\XLingPapermaxcolb}{+0\tabcolsep}
\XLingPaperminmaxcellincolumn{}{\XLingPapermincolc}{}{\XLingPapermaxcolc}{+0\tabcolsep}
\setlength{\XLingPaperavailabletablewidth}{433.62pt}
\setlength{\XLingPapertableminwidth}{\XLingPapermincola+\XLingPapermincolb+\XLingPapermincolc}
\setlength{\XLingPapertablemaxwidth}{\XLingPapermaxcola+\XLingPapermaxcolb+\XLingPapermaxcolc}
\XLingPapercalculatetablewidthratio{}
\XLingPapersetcolumnwidth{\XLingPapercolawidth}{\XLingPapermincola}{\XLingPapermaxcola}{-0\tabcolsep}
\XLingPapersetcolumnwidth{\XLingPapercolbwidth}{\XLingPapermincolb}{\XLingPapermaxcolb}{-2\tabcolsep}
\XLingPapersetcolumnwidth{\XLingPapercolcwidth}{\XLingPapermincolc}{\XLingPapermaxcolc}{-2\tabcolsep}\singlespacing\vspace*{-3\baselineskip}
\begin{longtable}
[l]{@{}p{\XLingPapercolawidth}p{\XLingPapercolbwidth}p{\XLingPapercolcwidth}@{}}\toprule\multicolumn{1}{@{}p{\XLingPapercolawidth}}{\textbf{Type}}&\multicolumn{1}{p{\XLingPapercolbwidth}}{\textbf{Description}}&\multicolumn{1}{p{\XLingPapercolcwidth}@{}}{\textbf{Glyphs}}\\%
\midrule\endhead \multicolumn{1}{@{}p{\XLingPapercolawidth}}{Main / standard}&\multicolumn{1}{p{\XLingPapercolbwidth}}{Main letters used in the language}&\multicolumn{1}{p{\XLingPapercolcwidth}@{}}{}\\%
\multicolumn{1}{@{}p{\XLingPapercolawidth}}{}&\multicolumn{1}{p{\XLingPapercolbwidth}}{}&\multicolumn{1}{p{\XLingPapercolcwidth}@{}}{}\\%
\multicolumn{1}{@{}p{\XLingPapercolawidth}}{Auxiliary}&\multicolumn{1}{p{\XLingPapercolbwidth}}{Additional characters for common foreign words, technical usage}&\multicolumn{1}{p{\XLingPapercolcwidth}@{}}{}\\%
\multicolumn{1}{@{}p{\XLingPapercolawidth}}{}&\multicolumn{1}{p{\XLingPapercolbwidth}}{}&\multicolumn{1}{p{\XLingPapercolcwidth}@{}}{}\\%
\multicolumn{1}{@{}p{\XLingPapercolawidth}}{Index}&\multicolumn{1}{p{\XLingPapercolbwidth}}{Characters for the header of an index}&\multicolumn{1}{p{\XLingPapercolcwidth}@{}}{}\\%
\multicolumn{1}{@{}p{\XLingPapercolawidth}}{}&\multicolumn{1}{p{\XLingPapercolbwidth}}{}&\multicolumn{1}{p{\XLingPapercolcwidth}@{}}{}\\%
\multicolumn{1}{@{}p{\XLingPapercolawidth}}{Punctuation}&\multicolumn{1}{p{\XLingPapercolbwidth}}{Common punctuation}&\multicolumn{1}{p{\XLingPapercolcwidth}@{}}{}\\%
\multicolumn{1}{@{}p{\XLingPapercolawidth}}{}&\multicolumn{1}{p{\XLingPapercolbwidth}}{}&\multicolumn{1}{p{\XLingPapercolcwidth}@{}}{}\\%
\multicolumn{1}{@{}p{\XLingPapercolawidth}}{Numbers}&\multicolumn{1}{p{\XLingPapercolbwidth}}{The characters needed to display the common number formats: decimal, percent, and currency.}&\multicolumn{1}{p{\XLingPapercolcwidth}@{}}{}\\\bottomrule%
\end{longtable}
}
}\vspace{11pt plus 2pt minus 1pt}\vspace{11pt plus 2pt minus 1pt}\XLingPaperneedspace{3\baselineskip}\protect\hypertarget{ntHausaCharacters}{}\XLingPaperaddtocontents{ntHausaCharacters}{\protect\raggedright{\singlespacing
{Table }{30.}{  Hausa Characters\\}}}\vspace{0pt}{\singlespacing
\hspace*{.25in}{
\XLingPaperminmaxcellincolumn{Type}{\XLingPapermincola}{\textbf{Type}}{\XLingPapermaxcola}{+0\tabcolsep}
\XLingPaperminmaxcellincolumn{Description}{\XLingPapermincolb}{\textbf{Description}}{\XLingPapermaxcolb}{+0\tabcolsep}
\XLingPaperminmaxcellincolumn{Glyphs}{\XLingPapermincolc}{\textbf{Glyphs}}{\XLingPapermaxcolc}{+0\tabcolsep}
\XLingPaperminmaxcellincolumn{standard}{\XLingPapermincola}{\XLingPaperCharisZSILFontFamily{\fontsize{8}{9.6}\selectfont \textup{\textup{Main / standard}}}}{\XLingPapermaxcola}{+0\tabcolsep}
\XLingPaperminmaxcellincolumn{language}{\XLingPapermincolb}{\XLingPaperCharisZSILFontFamily{\fontsize{8}{9.6}\selectfont \textup{\textup{Main letters used in the language}}}}{\XLingPapermaxcolb}{+0\tabcolsep}
\XLingPaperminmaxcellincolumn{a b ɓ c d ɗ e f g h i j k ƙ l m n o r s \{sh\} t \{ts\} u w y \{ʼy\} z ʼ}{\XLingPapermincolc}{\XLingPaperCharisZSILFontFamily{\fontsize{8}{9.6}\selectfont \textup{\textup{a b ɓ c d ɗ e f g h i j k ƙ l m n o r s \{sh\} t \{ts\} u w y \{ʼy\} z ʼ}}}}{\XLingPapermaxcolc}{+0\tabcolsep}
\XLingPaperminmaxcellincolumn{Auxiliary}{\XLingPapermincola}{\XLingPaperCharisZSILFontFamily{\fontsize{8}{9.6}\selectfont \textup{\textup{Auxiliary}}}}{\XLingPapermaxcola}{+0\tabcolsep}
\XLingPaperminmaxcellincolumn{technical}{\XLingPapermincolb}{\XLingPaperCharisZSILFontFamily{\fontsize{8}{9.6}\selectfont \textup{\textup{Additional characters for common foreign words, technical usage}}}}{\XLingPapermaxcolb}{+0\tabcolsep}
\XLingPaperminmaxcellincolumn{á à â é è ê í ì î ó ò ô p q \{r\textbackslash{}u0303\} ú ù û v x ƴ}{\XLingPapermincolc}{\XLingPaperCharisZSILFontFamily{\fontsize{8}{9.6}\selectfont \textup{\textup{á à â é è ê í ì î ó ò ô p q \{r\textbackslash{}u0303\} ú ù û v x ƴ}}}}{\XLingPapermaxcolc}{+0\tabcolsep}
\XLingPaperminmaxcellincolumn{Index}{\XLingPapermincola}{\XLingPaperCharisZSILFontFamily{\fontsize{8}{9.6}\selectfont \textup{\textup{Index}}}}{\XLingPapermaxcola}{+0\tabcolsep}
\XLingPaperminmaxcellincolumn{header}{\XLingPapermincolb}{\XLingPaperCharisZSILFontFamily{\fontsize{8}{9.6}\selectfont \textup{\textup{Characters for the header of an index}}}}{\XLingPapermaxcolb}{+0\tabcolsep}
\XLingPaperminmaxcellincolumn{}{\XLingPapermincolc}{\XLingPaperCharisZSILFontFamily{\fontsize{8}{9.6}\selectfont \textup{\textup{}}}}{\XLingPapermaxcolc}{+0\tabcolsep}
\XLingPaperminmaxcellincolumn{Punctuation}{\XLingPapermincola}{\XLingPaperCharisZSILFontFamily{\fontsize{8}{9.6}\selectfont \textup{\textup{Punctuation}}}}{\XLingPapermaxcola}{+0\tabcolsep}
\XLingPaperminmaxcellincolumn{Common}{\XLingPapermincolb}{\XLingPaperCharisZSILFontFamily{\fontsize{8}{9.6}\selectfont \textup{\textup{Common punctuation}}}}{\XLingPapermaxcolb}{+0\tabcolsep}
\XLingPaperminmaxcellincolumn{}{\XLingPapermincolc}{\XLingPaperCharisZSILFontFamily{\fontsize{8}{9.6}\selectfont \textup{\textup{}}}}{\XLingPapermaxcolc}{+0\tabcolsep}
\XLingPaperminmaxcellincolumn{Numbers}{\XLingPapermincola}{\XLingPaperCharisZSILFontFamily{\fontsize{8}{9.6}\selectfont \textup{\textup{Numbers}}}}{\XLingPapermaxcola}{+0\tabcolsep}
\XLingPaperminmaxcellincolumn{currency.}{\XLingPapermincolb}{\XLingPaperCharisZSILFontFamily{\fontsize{8}{9.6}\selectfont \textup{\textup{The characters needed to display the common number formats: decimal, percent, and currency.}}}}{\XLingPapermaxcolb}{+0\tabcolsep}
\XLingPaperminmaxcellincolumn{}{\XLingPapermincolc}{\XLingPaperCharisZSILFontFamily{\fontsize{8}{9.6}\selectfont \textup{\textup{}}}}{\XLingPapermaxcolc}{+0\tabcolsep}
\setlength{\XLingPaperavailabletablewidth}{433.62pt}
\setlength{\XLingPapertableminwidth}{\XLingPapermincola+\XLingPapermincolb+\XLingPapermincolc}
\setlength{\XLingPapertablemaxwidth}{\XLingPapermaxcola+\XLingPapermaxcolb+\XLingPapermaxcolc}
\XLingPapercalculatetablewidthratio{}
\XLingPapersetcolumnwidth{\XLingPapercolawidth}{\XLingPapermincola}{\XLingPapermaxcola}{-0\tabcolsep}
\XLingPapersetcolumnwidth{\XLingPapercolbwidth}{\XLingPapermincolb}{\XLingPapermaxcolb}{-2\tabcolsep}
\XLingPapersetcolumnwidth{\XLingPapercolcwidth}{\XLingPapermincolc}{\XLingPapermaxcolc}{-2\tabcolsep}\singlespacing\vspace*{-3\baselineskip}
\begin{longtable}
[l]{@{}p{\XLingPapercolawidth}p{\XLingPapercolbwidth}p{\XLingPapercolcwidth}@{}}\toprule\multicolumn{1}{@{}p{\XLingPapercolawidth}}{\textbf{Type}}&\multicolumn{1}{p{\XLingPapercolbwidth}}{\textbf{Description}}&\multicolumn{1}{p{\XLingPapercolcwidth}@{}}{\textbf{Glyphs}}\\%
\midrule\endhead \multicolumn{1}{@{}p{\XLingPapercolawidth}}{\XLingPaperCharisZSILFontFamily{\fontsize{8}{9.6}\selectfont \textup{\textup{Main / standard}}}}&\multicolumn{1}{p{\XLingPapercolbwidth}}{\XLingPaperCharisZSILFontFamily{\fontsize{8}{9.6}\selectfont \textup{\textup{Main letters used in the language}}}}&\multicolumn{1}{p{\XLingPapercolcwidth}@{}}{\XLingPaperCharisZSILFontFamily{\fontsize{8}{9.6}\selectfont \textup{\textup{a b ɓ c d ɗ e f g h i j k ƙ l m n o r s \{sh\} t \{ts\} u w y \{ʼy\} z ʼ}}}}\\[1.5pt]%
\multicolumn{1}{@{}p{\XLingPapercolawidth}}{\XLingPaperCharisZSILFontFamily{\fontsize{8}{9.6}\selectfont \textup{\textup{Auxiliary}}}}&\multicolumn{1}{p{\XLingPapercolbwidth}}{\XLingPaperCharisZSILFontFamily{\fontsize{8}{9.6}\selectfont \textup{\textup{Additional characters for common foreign words, technical usage}}}}&\multicolumn{1}{p{\XLingPapercolcwidth}@{}}{\XLingPaperCharisZSILFontFamily{\fontsize{8}{9.6}\selectfont \textup{\textup{á à â é è ê í ì î ó ò ô p q \{r\textbackslash{}u0303\} ú ù û v x ƴ}}}}\\[1.5pt]%
\multicolumn{1}{@{}p{\XLingPapercolawidth}}{\XLingPaperCharisZSILFontFamily{\fontsize{8}{9.6}\selectfont \textup{\textup{Index}}}}&\multicolumn{1}{p{\XLingPapercolbwidth}}{\XLingPaperCharisZSILFontFamily{\fontsize{8}{9.6}\selectfont \textup{\textup{Characters for the header of an index}}}}&\multicolumn{1}{p{\XLingPapercolcwidth}@{}}{\XLingPaperCharisZSILFontFamily{\fontsize{8}{9.6}\selectfont \textup{\textup{}}}}\\[1.5pt]%
\multicolumn{1}{@{}p{\XLingPapercolawidth}}{\XLingPaperCharisZSILFontFamily{\fontsize{8}{9.6}\selectfont \textup{\textup{Punctuation}}}}&\multicolumn{1}{p{\XLingPapercolbwidth}}{\XLingPaperCharisZSILFontFamily{\fontsize{8}{9.6}\selectfont \textup{\textup{Common punctuation}}}}&\multicolumn{1}{p{\XLingPapercolcwidth}@{}}{\XLingPaperCharisZSILFontFamily{\fontsize{8}{9.6}\selectfont \textup{\textup{}}}}\\[1.5pt]%
\multicolumn{1}{@{}p{\XLingPapercolawidth}}{\XLingPaperCharisZSILFontFamily{\fontsize{8}{9.6}\selectfont \textup{\textup{Numbers}}}}&\multicolumn{1}{p{\XLingPapercolbwidth}}{\XLingPaperCharisZSILFontFamily{\fontsize{8}{9.6}\selectfont \textup{\textup{The characters needed to display the common number formats: decimal, percent, and currency.}}}}&\multicolumn{1}{p{\XLingPapercolcwidth}@{}}{\XLingPaperCharisZSILFontFamily{\fontsize{8}{9.6}\selectfont \textup{\textup{}}}}\\[1.5pt]\bottomrule%
\end{longtable}
}
}\clearpage
\thispagestyle{bodyfirstpage}\markboth{Methodology}{Methodology}
\XLingPaperaddtocontents{cMethods}{\vspace*{.65in}\XLingPaperneedspace{3\baselineskip}\noindent
\fontsize{14}{16.8}\selectfont \textbf{{\centering
CHAPTER \raisebox{\baselineskip}[0pt]{\protect\hypertarget{cMethods}{}}\raisebox{\baselineskip}[0pt]{\pdfbookmark[1]{4 Methodology}{cMethods}}4\protect\\}}}\par{}
{\XLingPaperneedspace{3\baselineskip}\noindent
\fontsize{14}{16.8}\selectfont \textbf{{\centering
Methodology\protect\\}}}\par{}
\vspace{16pt}\indent For the experiments presented in chapter \hyperlink{cExperiments}{5} three elements were needed: corpora, keyboard layouts, and software implementing an optimization algorithm. These are the methods used to create the corpora and the tools used. To describe the higher level view of the process figure \hyperlink{fMethodProcess1}{20} illustrates the first part of the process. A suitable corpus needed to be found. Text input methods needed to be procured. Software to do the computation needed to be found or created. The corpora needed to be massaged to be compatible with the software. And the writing systems and orthographies of the corpora needed to be understood. The specifics of each of these steps are discussed with greater detail in the following sections.\par{}\vspace{11pt plus 2pt minus 1pt}\setbox0=\vbox{\protect\centering \leavevmode
\vspace*{0pt}{\XeTeXpdffile "../Resources/UpDatedProcess1-2.pdf" scaled 490}\\[0pt]\protect\hypertarget{fMethodProcess1}{}\XLingPaperaddtocontents{fMethodProcess1}{\singlespacing
{Figure }{20.}{ Research Process\\}}}\box0\par{}\vspace{11pt plus 2pt minus 1pt}{\vspace{15pt}\XLingPaperneedspace{3\baselineskip}\noindent
\fontsize{13}{15.6}\selectfont \textbf{{\noindent
\raisebox{\baselineskip}[0pt]{\pdfbookmark[2]{{4.1 } Corpus creation and description}{sCorpus}}\raisebox{\baselineskip}[0pt]{\protect\hypertarget{sCorpus}{}}{4.1 }Corpus creation and description}}\markboth{Corpus creation and description}{Methodology}\XLingPaperaddtocontents{sCorpus}}\par{}
\penalty10000\vspace{10pt}\penalty10000\indent Two classes of corpora were compiled. A larger corpus exclusively for Eastern Dan, and a smaller but parallel corpus\protect\footnote[1]{{\leftskip0pt\parindent1em\raisebox{\baselineskip}[0pt]{\protect\hypertarget{nLagnagues.}{}} The languages include: Eastern Dan \textsquarebracketleft{}dnj\textsquarebracketright{}, French \textsquarebracketleft{}fra\textsquarebracketright{}, English \textsquarebracketleft{}eng\textsquarebracketright{}, Spanish \textsquarebracketleft{}spa\textsquarebracketright{}, and Malinaltepec Meꞌphaa \textsquarebracketleft{}tcf\textsquarebracketright{}.}} for which covered several languages including Eastern Dan. As is the case with many minority languages finding a parallel corpus is a challenge\protect\footnote[2]{{\leftskip0pt\parindent1em\raisebox{\baselineskip}[0pt]{\protect\hypertarget{nOpenParalleCorpus}{}} Even “Corpus... The open parallel copus” is dominated by the presence of European languages. See \href{http://opus.nlpl.eu}{\textcolor[rgb]{0,0,0}{http://opus.nlpl.eu}}.}}. Others have approach the challenge in one of four ways: (1) use the translation of Christian scriptures \hyperlink{rResnikPhilipMariBromanOlsenMonaDiab1999TheBi}{(Resnik, Olsen \& Diab  1999}; \hyperlink{rD6stlingRobert2014Bayes}{Östling  2014}, \hyperlink{rD6stlingRobert2015Bayes}{2015)}, (2) use translated government documents \hyperlink{rSteinbergeretal}{(Steinberger et al.  2014)}, (3) use translated movie scripts \hyperlink{rLevshinaNatalia2016Verbs}{(Levshina  2016}; \hyperlink{rZhang2014}{Zhang et al.  2014)} or (4) use academic paper publishing metadata \hyperlink{rYepes2013}{(Yepes et al.  2013)}. For the languages explored, the only available parallel text was from the Christian New Testament; the epistle of James was chosen as that was the only portion which was commonly available across all the languages compared.\par{}\indent The lager Eastern Dan corpus consists of several texts from the Eastern Dan newspaper {\textit{\textbf{\textit{˗Pamɛbhamɛ}}}}, along with medical consuls which had been translated into Eastern Dan by \hyperlink{rKessE9gbeu2007}{Kességbeu (2007)}. The Eastern Dan texts were regularized with scripts to clean up various typographical inconsistencies, some of which are discussed in chapter \hyperlink{cDiscussion}{7}.\par{}\indent The purpose of the corpora is to create digram counts upon which finger movements can be modeled. Since at least \hyperlink{r}{ ()} it has been established that larger texts are better for the exact frequency count of writing units which are created less frequently\protect\footnote[3]{{\leftskip0pt\parindent1em\raisebox{\baselineskip}[0pt]{\protect\hypertarget{nEnglish-Counts}{}} The primere studies in English for letter based digram counting are \hyperlink{rMayzner1965}{Mayzner \& Tresselt (1965)} and \hyperlink{rNorvig2012}{Norvig (2012)}. The problem with these kinds of studies which look at digram letter counting is that for full optimization of keyboards one needs all the typographical units, not just letters. Puntuation characters also need to be represented in the frequencies.}}. Within the keyboard optimization literature, it also typical to use very large corpora so that the ratios between all possible digram frequencies have an opportunity to present themselves with a proper weighting. A sufficiently large corpus which is (1) representative of natural typing, and (2) is a parallel corpora across several languages is not to be found.\par{}\indent For under-resourced and resource scarce languages it is nearly impossible to find corpora large enough to work with, of the size which is needed to find the frequency of all typographical writing units relative to each other. To work around this limitation for the Eastern Dan keyboard, the solution was to run the experiment on two small but separate corpora. The smaller corpora (dnj-james) was used to provide equivalence with other languages, while the larger corpus (dnj-full) was used to test for keyboard layouts.\par{}\indent There are linguistic and orthographic differences between the languages in the parallel corpus. As far as I can tell there is no recommended way to statistically describe the parts of a parallel corpus. My judgement was to assume that if a writer of a minority language would have the option to type the same content in two languages that the content would need to be semantically similar. Translated works are an obvious choice. Corpora for establishing the frequency of typing digrams should contain all the possible digrams in the writing system with their relative frequencies. The smaller parallel corpus consisting of translations of James do not present the needed characters for any of the languages.\par{}\indent A statistical description of the corpora provides one dimension of their size, shown in table \hyperlink{ntCorpusStats}{31}. The word count was conducted via the GNU {\XLingPaperCharisZSILFontFamily{\textit{wc}}} utility. Counting orthographic characters required the creation of a new utility which implemented Unicode’s TR29 grapheme cluster algorithm.\par{}\vspace{11pt plus 2pt minus 1pt}\XLingPaperneedspace{3\baselineskip}\protect\hypertarget{ntCorpusStats}{}\XLingPaperaddtocontents{ntCorpusStats}{\protect\raggedright{\singlespacing
{Table }{31.}{  Corpora Statistics\\}}}\vspace{0pt}{\singlespacing
\hspace*{.25in}{
\XLingPaperminmaxcellincolumn{Corpus}{\XLingPapermincola}{\textbf{Corpus}}{\XLingPapermaxcola}{+0\tabcolsep}
\XLingPaperminmaxcellincolumn{Words}{\XLingPapermincolb}{\textbf{Words}}{\XLingPapermaxcolb}{+0\tabcolsep}
\XLingPaperminmaxcellincolumn{Grapheme}{\XLingPapermincolc}{\textbf{Unicode Grapheme Clusters}}{\XLingPapermaxcolc}{+0\tabcolsep}
\XLingPaperminmaxcellincolumn{Unique}{\XLingPapermincold}{\textbf{Unique Characters}}{\XLingPapermaxcold}{+0\tabcolsep}
\XLingPaperminmaxcellincolumn{eng-James}{\XLingPapermincola}{eng-James}{\XLingPapermaxcola}{+0\tabcolsep}
\XLingPaperminmaxcellincolumn{2531}{\XLingPapermincolb}{2531}{\XLingPapermaxcolb}{+0\tabcolsep}
\XLingPaperminmaxcellincolumn{13,371}{\XLingPapermincolc}{13,371}{\XLingPapermaxcolc}{+0\tabcolsep}
\XLingPaperminmaxcellincolumn{61}{\XLingPapermincold}{61}{\XLingPapermaxcold}{+0\tabcolsep}
\XLingPaperminmaxcellincolumn{fra-James}{\XLingPapermincola}{fra-James}{\XLingPapermaxcola}{+0\tabcolsep}
\XLingPaperminmaxcellincolumn{2378}{\XLingPapermincolb}{2378}{\XLingPapermaxcolb}{+0\tabcolsep}
\XLingPaperminmaxcellincolumn{13,372}{\XLingPapermincolc}{13,372}{\XLingPapermaxcolc}{+0\tabcolsep}
\XLingPaperminmaxcellincolumn{68}{\XLingPapermincold}{68}{\XLingPapermaxcold}{+0\tabcolsep}
\XLingPaperminmaxcellincolumn{spaj-James}{\XLingPapermincola}{spaj-James}{\XLingPapermaxcola}{+0\tabcolsep}
\XLingPaperminmaxcellincolumn{2165}{\XLingPapermincolb}{2165}{\XLingPapermaxcolb}{+0\tabcolsep}
\XLingPaperminmaxcellincolumn{12,174}{\XLingPapermincolc}{12,174}{\XLingPapermaxcolc}{+0\tabcolsep}
\XLingPaperminmaxcellincolumn{}{\XLingPapermincold}{}{\XLingPapermaxcold}{+0\tabcolsep}
\XLingPaperminmaxcellincolumn{tcf-James}{\XLingPapermincola}{tcf-James}{\XLingPapermaxcola}{+0\tabcolsep}
\XLingPaperminmaxcellincolumn{2856}{\XLingPapermincolb}{2856}{\XLingPapermaxcolb}{+0\tabcolsep}
\XLingPaperminmaxcellincolumn{18,128}{\XLingPapermincolc}{18,128}{\XLingPapermaxcolc}{+0\tabcolsep}
\XLingPaperminmaxcellincolumn{}{\XLingPapermincold}{}{\XLingPapermaxcold}{+0\tabcolsep}
\XLingPaperminmaxcellincolumn{dnj-James}{\XLingPapermincola}{dnj-James}{\XLingPapermaxcola}{+0\tabcolsep}
\XLingPaperminmaxcellincolumn{5197}{\XLingPapermincolb}{5197}{\XLingPapermaxcolb}{+0\tabcolsep}
\XLingPaperminmaxcellincolumn{23,958}{\XLingPapermincolc}{23,958}{\XLingPapermaxcolc}{+0\tabcolsep}
\XLingPaperminmaxcellincolumn{61}{\XLingPapermincold}{61}{\XLingPapermaxcold}{+0\tabcolsep}
\XLingPaperminmaxcellincolumn{dnj-full}{\XLingPapermincola}{dnj-full}{\XLingPapermaxcola}{+0\tabcolsep}
\XLingPaperminmaxcellincolumn{84268}{\XLingPapermincolb}{84268}{\XLingPapermaxcolb}{+0\tabcolsep}
\XLingPaperminmaxcellincolumn{(401,875)}{\XLingPapermincolc}{399,971 (401,875)}{\XLingPapermaxcolc}{+0\tabcolsep}
\XLingPaperminmaxcellincolumn{99}{\XLingPapermincold}{99}{\XLingPapermaxcold}{+0\tabcolsep}
\setlength{\XLingPaperavailabletablewidth}{433.62pt}
\setlength{\XLingPapertableminwidth}{\XLingPapermincola+\XLingPapermincolb+\XLingPapermincolc+\XLingPapermincold}
\setlength{\XLingPapertablemaxwidth}{\XLingPapermaxcola+\XLingPapermaxcolb+\XLingPapermaxcolc+\XLingPapermaxcold}
\XLingPapercalculatetablewidthratio{}
\XLingPapersetcolumnwidth{\XLingPapercolawidth}{\XLingPapermincola}{\XLingPapermaxcola}{-0\tabcolsep}
\XLingPapersetcolumnwidth{\XLingPapercolbwidth}{\XLingPapermincolb}{\XLingPapermaxcolb}{-2\tabcolsep}
\XLingPapersetcolumnwidth{\XLingPapercolcwidth}{\XLingPapermincolc}{\XLingPapermaxcolc}{-2\tabcolsep}
\XLingPapersetcolumnwidth{\XLingPapercoldwidth}{\XLingPapermincold}{\XLingPapermaxcold}{-2\tabcolsep}\singlespacing\vspace*{-3\baselineskip}
\begin{longtable}
[l]{@{}p{\XLingPapercolawidth}p{\XLingPapercolbwidth}p{\XLingPapercolcwidth}p{\XLingPapercoldwidth}@{}}\toprule\multicolumn{1}{@{}p{\XLingPapercolawidth}}{\textbf{Corpus}}&\multicolumn{1}{p{\XLingPapercolbwidth}}{\textbf{Words}}&\multicolumn{1}{p{\XLingPapercolcwidth}}{\textbf{Unicode Grapheme Clusters}}&\multicolumn{1}{p{\XLingPapercoldwidth}@{}}{\textbf{Unique Characters}}\\%
\midrule\endhead \multicolumn{1}{@{}p{\XLingPapercolawidth}}{eng-James}&\multicolumn{1}{p{\XLingPapercolbwidth}}{2531}&\multicolumn{1}{p{\XLingPapercolcwidth}}{13,371}&\multicolumn{1}{p{\XLingPapercoldwidth}@{}}{61}\\%
\multicolumn{1}{@{}p{\XLingPapercolawidth}}{fra-James}&\multicolumn{1}{p{\XLingPapercolbwidth}}{2378}&\multicolumn{1}{p{\XLingPapercolcwidth}}{13,372}&\multicolumn{1}{p{\XLingPapercoldwidth}@{}}{68}\\%
\multicolumn{1}{@{}p{\XLingPapercolawidth}}{spaj-James}&\multicolumn{1}{p{\XLingPapercolbwidth}}{2165}&\multicolumn{1}{p{\XLingPapercolcwidth}}{12,174}&\multicolumn{1}{p{\XLingPapercoldwidth}@{}}{}\\%
\multicolumn{1}{@{}p{\XLingPapercolawidth}}{tcf-James}&\multicolumn{1}{p{\XLingPapercolbwidth}}{2856}&\multicolumn{1}{p{\XLingPapercolcwidth}}{18,128}&\multicolumn{1}{p{\XLingPapercoldwidth}@{}}{}\\%
\multicolumn{1}{@{}p{\XLingPapercolawidth}}{dnj-James}&\multicolumn{1}{p{\XLingPapercolbwidth}}{5197}&\multicolumn{1}{p{\XLingPapercolcwidth}}{23,958}&\multicolumn{1}{p{\XLingPapercoldwidth}@{}}{61}\\%
\multicolumn{1}{@{}p{\XLingPapercolawidth}}{dnj-full}&\multicolumn{1}{p{\XLingPapercolbwidth}}{84268}&\multicolumn{1}{p{\XLingPapercolcwidth}}{399,971 (401,875)}&\multicolumn{1}{p{\XLingPapercoldwidth}@{}}{99}\\\bottomrule%
\end{longtable}
}
}\indent The following sub-points are gleaned and clarified as work was done on the small Eastern Dan corpus, to bring it from “folk writing” to “conforming to Unicode best practices”, so that the corpus could inform an analysis on how to best arrange a keyboard layout for Eastern Dan (which is presented in chapter \hyperlink{cResults}{6}).\par{}\vspace{11pt plus 2pt minus 1pt}\setbox0=\vbox{\protect\centering \leavevmode
\vspace*{0pt}{\XeTeXpicfile "../Resources/rose-diagram.png" scaled 400}\\[0pt]\protect\hypertarget{fEasternDanCharacterFrequency}{}\XLingPaperaddtocontents{fEasternDanCharacterFrequency}{\singlespacing
{Figure }{21.}{ Eastern Dan Character Frequency\\}}}\box0\par{}\vspace{11pt plus 2pt minus 1pt}\indent Plot with chart like this: https://stackoverflow.com/questions/25898523/how-to-plot-points-on-a-clock\par{}{\vspace{15pt}\XLingPaperneedspace{3\baselineskip}\noindent
\fontsize{13}{15.6}\selectfont \textbf{{\noindent
\raisebox{\baselineskip}[0pt]{\pdfbookmark[2]{{4.2 } Keyboards}{sKeyboards}}\raisebox{\baselineskip}[0pt]{\protect\hypertarget{sKeyboards}{}}{4.2 }Keyboards}}\markboth{Keyboards}{Methodology}\XLingPaperaddtocontents{sKeyboards}}\par{}
\penalty10000\vspace{10pt}\penalty10000\indent Two reference keyboards for typing English are presented along with two layouts for typing French and two layouts for typing Eastern Dan, as shown in table \hyperlink{ntKeyboardsLanguage}{32}.\par{}\vspace{11pt plus 2pt minus 1pt}\XLingPaperneedspace{3\baselineskip}\protect\hypertarget{ntKeyboardsLanguage}{}\XLingPaperaddtocontents{ntKeyboardsLanguage}{\protect\raggedright{\singlespacing
{Table }{32.}{  Keyboard layouts by language\\}}}\vspace{0pt}{\singlespacing
\hspace*{.25in}{
\XLingPaperminmaxcellincolumn{Keyboard}{\XLingPapermincola}{\textbf{Keyboard Layout}}{\XLingPapermaxcola}{+0\tabcolsep}
\XLingPaperminmaxcellincolumn{Language}{\XLingPapermincolb}{\textbf{Language}}{\XLingPapermaxcolb}{+0\tabcolsep}
\XLingPaperminmaxcellincolumn{QWERTY}{\XLingPapermincola}{QWERTY}{\XLingPapermaxcola}{+0\tabcolsep}
\XLingPaperminmaxcellincolumn{eng}{\XLingPapermincolb}{eng}{\XLingPapermaxcolb}{+0\tabcolsep}
\XLingPaperminmaxcellincolumn{Dvorak}{\XLingPapermincola}{Dvorak}{\XLingPapermaxcola}{+0\tabcolsep}
\XLingPaperminmaxcellincolumn{eng}{\XLingPapermincolb}{eng}{\XLingPapermaxcolb}{+0\tabcolsep}
\XLingPaperminmaxcellincolumn{AZERTY}{\XLingPapermincola}{AZERTY}{\XLingPapermaxcola}{+0\tabcolsep}
\XLingPaperminmaxcellincolumn{fra}{\XLingPapermincolb}{fra}{\XLingPapermaxcolb}{+0\tabcolsep}
\XLingPaperminmaxcellincolumn{Bépo}{\XLingPapermincola}{Bépo}{\XLingPapermaxcola}{+0\tabcolsep}
\XLingPaperminmaxcellincolumn{fra}{\XLingPapermincolb}{fra}{\XLingPapermaxcolb}{+0\tabcolsep}
\XLingPaperminmaxcellincolumn{AFU}{\XLingPapermincola}{AFU}{\XLingPapermaxcola}{+0\tabcolsep}
\XLingPaperminmaxcellincolumn{dnj}{\XLingPapermincolb}{dnj}{\XLingPapermaxcolb}{+0\tabcolsep}
\XLingPaperminmaxcellincolumn{Trans-Mande}{\XLingPapermincola}{Trans-Mande}{\XLingPapermaxcola}{+0\tabcolsep}
\XLingPaperminmaxcellincolumn{dnj}{\XLingPapermincolb}{dnj}{\XLingPapermaxcolb}{+0\tabcolsep}
\XLingPaperminmaxcellincolumn{}{\XLingPapermincola}{}{\XLingPapermaxcola}{+0\tabcolsep}
\XLingPaperminmaxcellincolumn{}{\XLingPapermincolb}{}{\XLingPapermaxcolb}{+0\tabcolsep}
\XLingPaperminmaxcellincolumn{}{\XLingPapermincola}{}{\XLingPapermaxcola}{+0\tabcolsep}
\XLingPaperminmaxcellincolumn{}{\XLingPapermincolb}{}{\XLingPapermaxcolb}{+0\tabcolsep}
\XLingPaperminmaxcellincolumn{}{\XLingPapermincola}{}{\XLingPapermaxcola}{+0\tabcolsep}
\XLingPaperminmaxcellincolumn{}{\XLingPapermincolb}{}{\XLingPapermaxcolb}{+0\tabcolsep}
\setlength{\XLingPaperavailabletablewidth}{433.62pt}
\setlength{\XLingPapertableminwidth}{\XLingPapermincola+\XLingPapermincolb}
\setlength{\XLingPapertablemaxwidth}{\XLingPapermaxcola+\XLingPapermaxcolb}
\XLingPapercalculatetablewidthratio{}
\XLingPapersetcolumnwidth{\XLingPapercolawidth}{\XLingPapermincola}{\XLingPapermaxcola}{-0\tabcolsep}
\XLingPapersetcolumnwidth{\XLingPapercolbwidth}{\XLingPapermincolb}{\XLingPapermaxcolb}{-2\tabcolsep}\singlespacing\vspace*{-3\baselineskip}
\begin{longtable}
[l]{@{}p{\XLingPapercolawidth}p{\XLingPapercolbwidth}@{}}\toprule\multicolumn{1}{@{}p{\XLingPapercolawidth}}{\textbf{Keyboard Layout}}&\multicolumn{1}{p{\XLingPapercolbwidth}@{}}{\textbf{Language}}\\%
\midrule\endhead \multicolumn{1}{@{}p{\XLingPapercolawidth}}{QWERTY}&\multicolumn{1}{p{\XLingPapercolbwidth}@{}}{eng}\\%
\multicolumn{1}{@{}p{\XLingPapercolawidth}}{Dvorak}&\multicolumn{1}{p{\XLingPapercolbwidth}@{}}{eng}\\%
\multicolumn{1}{@{}p{\XLingPapercolawidth}}{AZERTY}&\multicolumn{1}{p{\XLingPapercolbwidth}@{}}{fra}\\%
\multicolumn{1}{@{}p{\XLingPapercolawidth}}{Bépo}&\multicolumn{1}{p{\XLingPapercolbwidth}@{}}{fra}\\%
\multicolumn{1}{@{}p{\XLingPapercolawidth}}{AFU}&\multicolumn{1}{p{\XLingPapercolbwidth}@{}}{dnj}\\%
\multicolumn{1}{@{}p{\XLingPapercolawidth}}{Trans-Mande}&\multicolumn{1}{p{\XLingPapercolbwidth}@{}}{dnj}\\%
\multicolumn{1}{@{}p{\XLingPapercolawidth}}{}&\multicolumn{1}{p{\XLingPapercolbwidth}@{}}{}\\%
\multicolumn{1}{@{}p{\XLingPapercolawidth}}{}&\multicolumn{1}{p{\XLingPapercolbwidth}@{}}{}\\%
\multicolumn{1}{@{}p{\XLingPapercolawidth}}{}&\multicolumn{1}{p{\XLingPapercolbwidth}@{}}{}\\\bottomrule%
\end{longtable}
}
}\noindent Most of the English typing world is familiar with and uses the QWERTY keyboard layout. Many are aware of the existence of the Dvorak keyboard layout (1935) and the ongoing “efficient keyboard layout” discussion. Fitness scores for these keyboards are provided (cf. section 3) so that the scope of the typing task in minority languages can be compared within this more popular debate.\par{}\noindent AZERTY is the default keyboard which has shipped with computers sold in French language markets. As a result it is somewhat of a “standard” keyboard for the French language. Many French language typists disdain the peculiarities and deficiencies of the AZERTY layout. One optimization effort for the French keyboard has produced the Bépo layout\protect\footnote[4]{{\leftskip0pt\parindent1em\raisebox{\baselineskip}[0pt]{\protect\hypertarget{nBE9po}{}} Bépo: \href{http://bepo.fr}{\textcolor[rgb]{0,0,0}{http://bepo.fr}}}} as an alternative to AZERTY. In contrast to the QWERTY-Dvorak debate there are some characters needed for typing French which are not available at all on AZERTY, Bépo addresses this issue in addition to providing an efficient arrangement of keys.\par{}\noindent Both keyboards in use for Eastern Dan are adaptations of existing keyboards. AFU, is a keyboard with a variety of characters pertinent to linguists and is distributed by CNRS-LLACAN. AFU is based on the AZERTY layout. Trans-Mande is also a keyboard designed for use by linguists, and was designed for researchers working in more than one Mande language. Trans-Mande is based on QWERTY. I received my copy from {Valentin Vydrin}.\par{}{\vspace{15pt}\XLingPaperneedspace{3\baselineskip}\noindent
\fontsize{13}{15.6}\selectfont \textbf{{\noindent
\raisebox{\baselineskip}[0pt]{\pdfbookmark[2]{{4.3 } Software}{sSoftware}}\raisebox{\baselineskip}[0pt]{\protect\hypertarget{sSoftware}{}}{4.3 }Software}}\markboth{Software}{Methodology}\XLingPaperaddtocontents{sSoftware}}\par{}
\penalty10000\vspace{10pt}\penalty10000\indent A variety of open source software exists for keyboard optimization. {\XLingPaperCharisZSILFontFamily{\textit{Typing}}} by Michael Dickens \hyperlink{rDickens2016}{Dickens (2016)} was chosen because it was written in C and compiled cleanly on MacOS 10.13.5 and Linux (Ubuntu 16.04). {\XLingPaperCharisZSILFontFamily{\textit{Typing}}} has a ranking system with allows it to produce a fitness score which is tied to a keyboard layout and a given corpus. {\XLingPaperCharisZSILFontFamily{\textit{Typing}}} also has two desirable features: (1) it can solve for an optimized keyboard layout using and evolutionary algorithm; (2) it can score a given keyboard layout per a specified corpus using the same fitness model that it uses to evaluate potential optimized options.\par{}\indent {\XLingPaperCharisZSILFontFamily{\textit{Typing}}} has some limitations. Most restrictive is its lack of ability to process characters which do not occur on the default and shift state of the QWERTY keyboard (non-ASCII). A second limitation is that it can only optimize two states: default and shift. This means that free modeling of dead keys is not possible. This is a problem that \hyperlink{rEggers2003Optim}{Eggers et al. (2003b)} solve by re-encoding the corpus so that characters which would have been created by dead keys are actually encoded as what they would be if the same keys were struck but the dead key function was not active. I follow the same methodology in these experiments. One last restriction that {\XLingPaperCharisZSILFontFamily{\textit{Typing}}} has is that it is currently only capable of evaluating ANSI keyboard layouts. Many keyboard layouts designed for European markets are ISO layouts. These experiments use AFU as it appears on MacOS. The difference between the AFU documentation\protect\footnote[5]{{\leftskip0pt\parindent1em\raisebox{\baselineskip}[0pt]{\protect\hypertarget{nAFUdocument}{}} \href{http://llacan.vjf.cnrs.fr/res\_manuels\_en.php}{\textcolor[rgb]{0,0,0}{http://llacan.vjf.cnrs.fr/res\_manuels\_en.php}}}} and the MacOS version is that the ISO 9995 key B00 is moved to the E00 position.\par{}\indent For these experiments the {\XLingPaperCharisZSILFontFamily{\textit{Typing}}} setting {\XLingPaperMonospacedFontFamily{\textup{\textmd{setksize}}}} was set to {\XLingPaperMonospacedFontFamily{\textup{\textmd{standard}}}} so that the full keyboard would be accessible to the optimizer.{\XLingPaperLateefFontFamily{\textit{\hyperlink{r62A62763164A62E62F63164A62764162An.d.64A64368664A646}{پذيرش, تاريخ دريافت \& تاريخ  (1394 (2015))}}}}\par{}\clearpage
\thispagestyle{bodyfirstpage}\markboth{Experiments}{Experiments}
\XLingPaperaddtocontents{cExperiments}{\vspace*{.65in}\XLingPaperneedspace{3\baselineskip}\noindent
\fontsize{14}{16.8}\selectfont \textbf{{\centering
CHAPTER \raisebox{\baselineskip}[0pt]{\protect\hypertarget{cExperiments}{}}\raisebox{\baselineskip}[0pt]{\pdfbookmark[1]{5 Experiments}{cExperiments}}5\protect\\}}}\par{}
{\XLingPaperneedspace{3\baselineskip}\noindent
\fontsize{14}{16.8}\selectfont \textbf{{\centering
Experiments\protect\\}}}\par{}
\vspace{16pt}\indent For this thesis a total of 51 separate experiments were run. A summary is provided in figure \hyperlink{TotalResults}{25} in appendix \hyperlink{ChartResults}{A}. There are essentially two types of experiments: (1) some experiments take a given text and then generate a keyboard optimized on the digrams generated from that text, (2) other experiments take a given keyboard and a text and generate a fitness score. Experiments of type two can further be broken down by their purpose in the overall research question: (1) experiments were needed to establish some sort of baseline, (2) they were testing the output of generated experiments.\par{}\indent The keyboards used to generate baselines performance scores are listed in table \hyperlink{ntBaseline}{33}:\par{}\vspace{11pt plus 2pt minus 1pt}\XLingPaperneedspace{3\baselineskip}\protect\hypertarget{ntBaseline}{}\XLingPaperaddtocontents{ntBaseline}{\protect\raggedright{\singlespacing
{Table }{33.}{  Baseline Scores\\}}}\vspace{0pt}{\singlespacing
\hspace*{.25in}{
\XLingPaperminmaxcellincolumn{Keyboard}{\XLingPapermincola}{\textbf{Keyboard}}{\XLingPapermaxcola}{+0\tabcolsep}
\XLingPaperminmaxcellincolumn{Corpus}{\XLingPapermincolb}{\textbf{Corpus}}{\XLingPapermaxcolb}{+0\tabcolsep}
\XLingPaperminmaxcellincolumn{Language}{\XLingPapermincolc}{\textbf{Language}}{\XLingPapermaxcolc}{+0\tabcolsep}
\XLingPaperminmaxcellincolumn{Fitness}{\XLingPapermincold}{\textbf{Fitness Score}}{\XLingPapermaxcold}{+0\tabcolsep}
\XLingPaperminmaxcellincolumn{ID}{\XLingPapermincole}{\textbf{Experiment ID}}{\XLingPapermaxcole}{+0\tabcolsep}
\XLingPaperminmaxcellincolumn{QWERTY}{\XLingPapermincola}{QWERTY}{\XLingPapermaxcola}{+0\tabcolsep}
\XLingPaperminmaxcellincolumn{eng-James}{\XLingPapermincolb}{eng-James}{\XLingPapermaxcolb}{+0\tabcolsep}
\XLingPaperminmaxcellincolumn{eng}{\XLingPapermincolc}{eng}{\XLingPapermaxcolc}{+0\tabcolsep}
\XLingPaperminmaxcellincolumn{3362}{\XLingPapermincold}{3362}{\XLingPapermaxcold}{+0\tabcolsep}
\XLingPaperminmaxcellincolumn{28}{\XLingPapermincole}{28}{\XLingPapermaxcole}{+0\tabcolsep}
\XLingPaperminmaxcellincolumn{Dvorak}{\XLingPapermincola}{Dvorak}{\XLingPapermaxcola}{+0\tabcolsep}
\XLingPaperminmaxcellincolumn{eng-James}{\XLingPapermincolb}{eng-James}{\XLingPapermaxcolb}{+0\tabcolsep}
\XLingPaperminmaxcellincolumn{eng}{\XLingPapermincolc}{eng}{\XLingPapermaxcolc}{+0\tabcolsep}
\XLingPaperminmaxcellincolumn{1642}{\XLingPapermincold}{1642}{\XLingPapermaxcold}{+0\tabcolsep}
\XLingPaperminmaxcellincolumn{31}{\XLingPapermincole}{31}{\XLingPapermaxcole}{+0\tabcolsep}
\XLingPaperminmaxcellincolumn{Capewell}{\XLingPapermincola}{Capewell}{\XLingPapermaxcola}{+0\tabcolsep}
\XLingPaperminmaxcellincolumn{eng-James}{\XLingPapermincolb}{eng-James}{\XLingPapermaxcolb}{+0\tabcolsep}
\XLingPaperminmaxcellincolumn{eng}{\XLingPapermincolc}{eng}{\XLingPapermaxcolc}{+0\tabcolsep}
\XLingPaperminmaxcellincolumn{1747}{\XLingPapermincold}{1747}{\XLingPapermaxcold}{+0\tabcolsep}
\XLingPaperminmaxcellincolumn{30}{\XLingPapermincole}{30}{\XLingPapermaxcole}{+0\tabcolsep}
\XLingPaperminmaxcellincolumn{Carpalx}{\XLingPapermincola}{Carpalx}{\XLingPapermaxcola}{+0\tabcolsep}
\XLingPaperminmaxcellincolumn{eng-James}{\XLingPapermincolb}{eng-James}{\XLingPapermaxcolb}{+0\tabcolsep}
\XLingPaperminmaxcellincolumn{eng}{\XLingPapermincolc}{eng}{\XLingPapermaxcolc}{+0\tabcolsep}
\XLingPaperminmaxcellincolumn{1820}{\XLingPapermincold}{1820}{\XLingPapermaxcold}{+0\tabcolsep}
\XLingPaperminmaxcellincolumn{29}{\XLingPapermincole}{29}{\XLingPapermaxcole}{+0\tabcolsep}
\XLingPaperminmaxcellincolumn{Colemak}{\XLingPapermincola}{Colemak}{\XLingPapermaxcola}{+0\tabcolsep}
\XLingPaperminmaxcellincolumn{eng-James}{\XLingPapermincolb}{eng-James}{\XLingPapermaxcolb}{+0\tabcolsep}
\XLingPaperminmaxcellincolumn{eng}{\XLingPapermincolc}{eng}{\XLingPapermaxcolc}{+0\tabcolsep}
\XLingPaperminmaxcellincolumn{1568}{\XLingPapermincold}{1568}{\XLingPapermaxcold}{+0\tabcolsep}
\XLingPaperminmaxcellincolumn{33}{\XLingPapermincole}{33}{\XLingPapermaxcole}{+0\tabcolsep}
\XLingPaperminmaxcellincolumn{Workman}{\XLingPapermincola}{Workman}{\XLingPapermaxcola}{+0\tabcolsep}
\XLingPaperminmaxcellincolumn{eng-James}{\XLingPapermincolb}{eng-James}{\XLingPapermaxcolb}{+0\tabcolsep}
\XLingPaperminmaxcellincolumn{eng}{\XLingPapermincolc}{eng}{\XLingPapermaxcolc}{+0\tabcolsep}
\XLingPaperminmaxcellincolumn{1519}{\XLingPapermincold}{1519}{\XLingPapermaxcold}{+0\tabcolsep}
\XLingPaperminmaxcellincolumn{34}{\XLingPapermincole}{34}{\XLingPapermaxcole}{+0\tabcolsep}
\XLingPaperminmaxcellincolumn{Norman}{\XLingPapermincola}{Norman}{\XLingPapermaxcola}{+0\tabcolsep}
\XLingPaperminmaxcellincolumn{eng-James}{\XLingPapermincolb}{eng-James}{\XLingPapermaxcolb}{+0\tabcolsep}
\XLingPaperminmaxcellincolumn{eng}{\XLingPapermincolc}{eng}{\XLingPapermaxcolc}{+0\tabcolsep}
\XLingPaperminmaxcellincolumn{1517}{\XLingPapermincold}{1517}{\XLingPapermaxcold}{+0\tabcolsep}
\XLingPaperminmaxcellincolumn{35}{\XLingPapermincole}{35}{\XLingPapermaxcole}{+0\tabcolsep}
\XLingPaperminmaxcellincolumn{AZERTY}{\XLingPapermincola}{AZERTY}{\XLingPapermaxcola}{+0\tabcolsep}
\XLingPaperminmaxcellincolumn{fra-James}{\XLingPapermincolb}{fra-James}{\XLingPapermaxcolb}{+0\tabcolsep}
\XLingPaperminmaxcellincolumn{fra}{\XLingPapermincolc}{fra}{\XLingPapermaxcolc}{+0\tabcolsep}
\XLingPaperminmaxcellincolumn{3358}{\XLingPapermincold}{3358}{\XLingPapermaxcold}{+0\tabcolsep}
\XLingPaperminmaxcellincolumn{8}{\XLingPapermincole}{8}{\XLingPapermaxcole}{+0\tabcolsep}
\XLingPaperminmaxcellincolumn{Bépo}{\XLingPapermincola}{Bépo}{\XLingPapermaxcola}{+0\tabcolsep}
\XLingPaperminmaxcellincolumn{fra-James}{\XLingPapermincolb}{fra-James}{\XLingPapermaxcolb}{+0\tabcolsep}
\XLingPaperminmaxcellincolumn{fra}{\XLingPapermincolc}{fra}{\XLingPapermaxcolc}{+0\tabcolsep}
\XLingPaperminmaxcellincolumn{1472}{\XLingPapermincold}{1472}{\XLingPapermaxcold}{+0\tabcolsep}
\XLingPaperminmaxcellincolumn{12}{\XLingPapermincole}{12}{\XLingPapermaxcole}{+0\tabcolsep}
\XLingPaperminmaxcellincolumn{AFU}{\XLingPapermincola}{AFU}{\XLingPapermaxcola}{+0\tabcolsep}
\XLingPaperminmaxcellincolumn{fra-James}{\XLingPapermincolb}{fra-James}{\XLingPapermaxcolb}{+0\tabcolsep}
\XLingPaperminmaxcellincolumn{fra}{\XLingPapermincolc}{fra}{\XLingPapermaxcolc}{+0\tabcolsep}
\XLingPaperminmaxcellincolumn{3345}{\XLingPapermincold}{3345}{\XLingPapermaxcold}{+0\tabcolsep}
\XLingPaperminmaxcellincolumn{2}{\XLingPapermincole}{2}{\XLingPapermaxcole}{+0\tabcolsep}
\XLingPaperminmaxcellincolumn{Trans-Mande}{\XLingPapermincola}{Trans-Mande}{\XLingPapermaxcola}{+0\tabcolsep}
\XLingPaperminmaxcellincolumn{fra-James}{\XLingPapermincolb}{fra-James}{\XLingPapermaxcolb}{+0\tabcolsep}
\XLingPaperminmaxcellincolumn{fra}{\XLingPapermincolc}{fra}{\XLingPapermaxcolc}{+0\tabcolsep}
\XLingPaperminmaxcellincolumn{3588}{\XLingPapermincold}{3588}{\XLingPapermaxcold}{+0\tabcolsep}
\XLingPaperminmaxcellincolumn{22}{\XLingPapermincole}{22}{\XLingPapermaxcole}{+0\tabcolsep}
\XLingPaperminmaxcellincolumn{Spanish}{\XLingPapermincola}{Spanish}{\XLingPapermaxcola}{+0\tabcolsep}
\XLingPaperminmaxcellincolumn{spa-James}{\XLingPapermincolb}{spa-James}{\XLingPapermaxcolb}{+0\tabcolsep}
\XLingPaperminmaxcellincolumn{spa}{\XLingPapermincolc}{spa}{\XLingPapermaxcolc}{+0\tabcolsep}
\XLingPaperminmaxcellincolumn{3089}{\XLingPapermincold}{3089}{\XLingPapermaxcold}{+0\tabcolsep}
\XLingPaperminmaxcellincolumn{17}{\XLingPapermincole}{17}{\XLingPapermaxcole}{+0\tabcolsep}
\XLingPaperminmaxcellincolumn{Me'phaa}{\XLingPapermincola}{SIL {Me'phaa}}{\XLingPapermaxcola}{+0\tabcolsep}
\XLingPaperminmaxcellincolumn{tcf-James}{\XLingPapermincolb}{tcf-James}{\XLingPapermaxcolb}{+0\tabcolsep}
\XLingPaperminmaxcellincolumn{tcf}{\XLingPapermincolc}{tcf}{\XLingPapermaxcolc}{+0\tabcolsep}
\XLingPaperminmaxcellincolumn{10463}{\XLingPapermincold}{10463}{\XLingPapermaxcold}{+0\tabcolsep}
\XLingPaperminmaxcellincolumn{14}{\XLingPapermincole}{14}{\XLingPapermaxcole}{+0\tabcolsep}
\XLingPaperminmaxcellincolumn{Trans-Mande}{\XLingPapermincola}{Trans-Mande}{\XLingPapermaxcola}{+0\tabcolsep}
\XLingPaperminmaxcellincolumn{dnj-James}{\XLingPapermincolb}{dnj-James}{\XLingPapermaxcolb}{+0\tabcolsep}
\XLingPaperminmaxcellincolumn{dnj}{\XLingPapermincolc}{dnj}{\XLingPapermaxcolc}{+0\tabcolsep}
\XLingPaperminmaxcellincolumn{17922}{\XLingPapermincold}{17922}{\XLingPapermaxcold}{+0\tabcolsep}
\XLingPaperminmaxcellincolumn{21}{\XLingPapermincole}{21}{\XLingPapermaxcole}{+0\tabcolsep}
\XLingPaperminmaxcellincolumn{AFU}{\XLingPapermincola}{AFU}{\XLingPapermaxcola}{+0\tabcolsep}
\XLingPaperminmaxcellincolumn{dnj-James}{\XLingPapermincolb}{dnj-James}{\XLingPapermaxcolb}{+0\tabcolsep}
\XLingPaperminmaxcellincolumn{dnj}{\XLingPapermincolc}{dnj}{\XLingPapermaxcolc}{+0\tabcolsep}
\XLingPaperminmaxcellincolumn{22723}{\XLingPapermincold}{22723}{\XLingPapermaxcold}{+0\tabcolsep}
\XLingPaperminmaxcellincolumn{1}{\XLingPapermincole}{1}{\XLingPapermaxcole}{+0\tabcolsep}
\XLingPaperminmaxcellincolumn{Trans-Mande}{\XLingPapermincola}{Trans-Mande}{\XLingPapermaxcola}{+0\tabcolsep}
\XLingPaperminmaxcellincolumn{dnj-full}{\XLingPapermincolb}{dnj-full}{\XLingPapermaxcolb}{+0\tabcolsep}
\XLingPaperminmaxcellincolumn{dnj}{\XLingPapermincolc}{dnj}{\XLingPapermaxcolc}{+0\tabcolsep}
\XLingPaperminmaxcellincolumn{376242}{\XLingPapermincold}{376242}{\XLingPapermaxcold}{+0\tabcolsep}
\XLingPaperminmaxcellincolumn{20}{\XLingPapermincole}{20}{\XLingPapermaxcole}{+0\tabcolsep}
\XLingPaperminmaxcellincolumn{AFU}{\XLingPapermincola}{AFU}{\XLingPapermaxcola}{+0\tabcolsep}
\XLingPaperminmaxcellincolumn{dnj-full}{\XLingPapermincolb}{dnj-full}{\XLingPapermaxcolb}{+0\tabcolsep}
\XLingPaperminmaxcellincolumn{dnj}{\XLingPapermincolc}{dnj}{\XLingPapermaxcolc}{+0\tabcolsep}
\XLingPaperminmaxcellincolumn{470136}{\XLingPapermincold}{470136}{\XLingPapermaxcold}{+0\tabcolsep}
\XLingPaperminmaxcellincolumn{44}{\XLingPapermincole}{44}{\XLingPapermaxcole}{+0\tabcolsep}
\setlength{\XLingPaperavailabletablewidth}{433.62pt}
\setlength{\XLingPapertableminwidth}{\XLingPapermincola+\XLingPapermincolb+\XLingPapermincolc+\XLingPapermincold+\XLingPapermincole}
\setlength{\XLingPapertablemaxwidth}{\XLingPapermaxcola+\XLingPapermaxcolb+\XLingPapermaxcolc+\XLingPapermaxcold+\XLingPapermaxcole}
\XLingPapercalculatetablewidthratio{}
\XLingPapersetcolumnwidth{\XLingPapercolawidth}{\XLingPapermincola}{\XLingPapermaxcola}{-0\tabcolsep}
\XLingPapersetcolumnwidth{\XLingPapercolbwidth}{\XLingPapermincolb}{\XLingPapermaxcolb}{-2\tabcolsep}
\XLingPapersetcolumnwidth{\XLingPapercolcwidth}{\XLingPapermincolc}{\XLingPapermaxcolc}{-2\tabcolsep}
\XLingPapersetcolumnwidth{\XLingPapercoldwidth}{\XLingPapermincold}{\XLingPapermaxcold}{-2\tabcolsep}
\XLingPapersetcolumnwidth{\XLingPapercolewidth}{\XLingPapermincole}{\XLingPapermaxcole}{-2\tabcolsep}\singlespacing\vspace*{-3\baselineskip}
\begin{longtable}
[l]{@{}p{\XLingPapercolawidth}p{\XLingPapercolbwidth}>{\centering}p{\XLingPapercolcwidth}p{\XLingPapercoldwidth}p{\XLingPapercolewidth}@{}}\toprule\multicolumn{1}{@{}p{\XLingPapercolawidth}}{\textbf{Keyboard}}&\multicolumn{1}{p{\XLingPapercolbwidth}}{\textbf{Corpus}}&\multicolumn{1}{>{\centering}p{\XLingPapercolcwidth}}{\textbf{Language}}&\multicolumn{1}{p{\XLingPapercoldwidth}}{\textbf{Fitness Score}}&\multicolumn{1}{p{\XLingPapercolewidth}@{}}{\textbf{Experiment ID}}\\%
\midrule\endhead \multicolumn{1}{@{}p{\XLingPapercolawidth}}{QWERTY}&\multicolumn{1}{p{\XLingPapercolbwidth}}{eng-James}&\multicolumn{1}{>{\centering}p{\XLingPapercolcwidth}}{eng}&\multicolumn{1}{>{\raggedleft}p{\XLingPapercoldwidth}}{3362}&\multicolumn{1}{>{\centering}p{\XLingPapercolewidth}@{}}{28}\\%
\multicolumn{1}{@{}p{\XLingPapercolawidth}}{Dvorak}&\multicolumn{1}{p{\XLingPapercolbwidth}}{eng-James}&\multicolumn{1}{>{\centering}p{\XLingPapercolcwidth}}{eng}&\multicolumn{1}{>{\raggedleft}p{\XLingPapercoldwidth}}{1642}&\multicolumn{1}{>{\centering}p{\XLingPapercolewidth}@{}}{31}\\%
\multicolumn{1}{@{}p{\XLingPapercolawidth}}{Capewell}&\multicolumn{1}{p{\XLingPapercolbwidth}}{eng-James}&\multicolumn{1}{>{\centering}p{\XLingPapercolcwidth}}{eng}&\multicolumn{1}{>{\raggedleft}p{\XLingPapercoldwidth}}{1747}&\multicolumn{1}{>{\centering}p{\XLingPapercolewidth}@{}}{30}\\%
\multicolumn{1}{@{}p{\XLingPapercolawidth}}{Carpalx}&\multicolumn{1}{p{\XLingPapercolbwidth}}{eng-James}&\multicolumn{1}{>{\centering}p{\XLingPapercolcwidth}}{eng}&\multicolumn{1}{>{\raggedleft}p{\XLingPapercoldwidth}}{1820}&\multicolumn{1}{>{\centering}p{\XLingPapercolewidth}@{}}{29}\\%
\multicolumn{1}{@{}p{\XLingPapercolawidth}}{Colemak}&\multicolumn{1}{p{\XLingPapercolbwidth}}{eng-James}&\multicolumn{1}{>{\centering}p{\XLingPapercolcwidth}}{eng}&\multicolumn{1}{>{\raggedleft}p{\XLingPapercoldwidth}}{1568}&\multicolumn{1}{>{\centering}p{\XLingPapercolewidth}@{}}{33}\\%
\multicolumn{1}{@{}p{\XLingPapercolawidth}}{Workman}&\multicolumn{1}{p{\XLingPapercolbwidth}}{eng-James}&\multicolumn{1}{>{\centering}p{\XLingPapercolcwidth}}{eng}&\multicolumn{1}{>{\raggedleft}p{\XLingPapercoldwidth}}{1519}&\multicolumn{1}{>{\centering}p{\XLingPapercolewidth}@{}}{34}\\%
\multicolumn{1}{@{}p{\XLingPapercolawidth}}{Norman}&\multicolumn{1}{p{\XLingPapercolbwidth}}{eng-James}&\multicolumn{1}{>{\centering}p{\XLingPapercolcwidth}}{eng}&\multicolumn{1}{>{\raggedleft}p{\XLingPapercoldwidth}}{1517}&\multicolumn{1}{>{\centering}p{\XLingPapercolewidth}@{}}{35}\\%
\multicolumn{1}{@{}p{\XLingPapercolawidth}}{AZERTY}&\multicolumn{1}{p{\XLingPapercolbwidth}}{fra-James}&\multicolumn{1}{>{\centering}p{\XLingPapercolcwidth}}{fra}&\multicolumn{1}{>{\raggedleft}p{\XLingPapercoldwidth}}{3358}&\multicolumn{1}{>{\centering}p{\XLingPapercolewidth}@{}}{8}\\%
\multicolumn{1}{@{}p{\XLingPapercolawidth}}{Bépo}&\multicolumn{1}{p{\XLingPapercolbwidth}}{fra-James}&\multicolumn{1}{>{\centering}p{\XLingPapercolcwidth}}{fra}&\multicolumn{1}{>{\raggedleft}p{\XLingPapercoldwidth}}{1472}&\multicolumn{1}{>{\centering}p{\XLingPapercolewidth}@{}}{12}\\%
\multicolumn{1}{@{}p{\XLingPapercolawidth}}{AFU}&\multicolumn{1}{p{\XLingPapercolbwidth}}{fra-James}&\multicolumn{1}{>{\centering}p{\XLingPapercolcwidth}}{fra}&\multicolumn{1}{>{\raggedleft}p{\XLingPapercoldwidth}}{3345}&\multicolumn{1}{>{\centering}p{\XLingPapercolewidth}@{}}{2}\\%
\multicolumn{1}{@{}p{\XLingPapercolawidth}}{Trans-Mande}&\multicolumn{1}{p{\XLingPapercolbwidth}}{fra-James}&\multicolumn{1}{>{\centering}p{\XLingPapercolcwidth}}{fra}&\multicolumn{1}{>{\raggedleft}p{\XLingPapercoldwidth}}{3588}&\multicolumn{1}{>{\centering}p{\XLingPapercolewidth}@{}}{22}\\%
\multicolumn{1}{@{}p{\XLingPapercolawidth}}{Spanish}&\multicolumn{1}{p{\XLingPapercolbwidth}}{spa-James}&\multicolumn{1}{>{\centering}p{\XLingPapercolcwidth}}{spa}&\multicolumn{1}{>{\raggedleft}p{\XLingPapercoldwidth}}{3089}&\multicolumn{1}{>{\centering}p{\XLingPapercolewidth}@{}}{17}\\%
\multicolumn{1}{@{}p{\XLingPapercolawidth}}{SIL {Me'phaa}}&\multicolumn{1}{p{\XLingPapercolbwidth}}{tcf-James}&\multicolumn{1}{>{\centering}p{\XLingPapercolcwidth}}{tcf}&\multicolumn{1}{>{\raggedleft}p{\XLingPapercoldwidth}}{10463}&\multicolumn{1}{>{\centering}p{\XLingPapercolewidth}@{}}{14}\\%
\multicolumn{1}{@{}p{\XLingPapercolawidth}}{Trans-Mande}&\multicolumn{1}{p{\XLingPapercolbwidth}}{dnj-James}&\multicolumn{1}{>{\centering}p{\XLingPapercolcwidth}}{dnj}&\multicolumn{1}{>{\raggedleft}p{\XLingPapercoldwidth}}{17922}&\multicolumn{1}{>{\centering}p{\XLingPapercolewidth}@{}}{21}\\%
\multicolumn{1}{@{}p{\XLingPapercolawidth}}{AFU}&\multicolumn{1}{p{\XLingPapercolbwidth}}{dnj-James}&\multicolumn{1}{>{\centering}p{\XLingPapercolcwidth}}{dnj}&\multicolumn{1}{>{\raggedleft}p{\XLingPapercoldwidth}}{22723}&\multicolumn{1}{>{\centering}p{\XLingPapercolewidth}@{}}{1}\\%
\multicolumn{1}{@{}p{\XLingPapercolawidth}}{Trans-Mande}&\multicolumn{1}{p{\XLingPapercolbwidth}}{dnj-full}&\multicolumn{1}{>{\centering}p{\XLingPapercolcwidth}}{dnj}&\multicolumn{1}{>{\raggedleft}p{\XLingPapercoldwidth}}{376242}&\multicolumn{1}{>{\centering}p{\XLingPapercolewidth}@{}}{20}\\%
\multicolumn{1}{@{}p{\XLingPapercolawidth}}{AFU}&\multicolumn{1}{p{\XLingPapercolbwidth}}{dnj-full}&\multicolumn{1}{>{\centering}p{\XLingPapercolcwidth}}{dnj}&\multicolumn{1}{>{\raggedleft}p{\XLingPapercoldwidth}}{470136}&\multicolumn{1}{>{\centering}p{\XLingPapercolewidth}@{}}{44}\\\bottomrule%
\end{longtable}
}
}\noindent Some keyboards are listed multiple times. The keyboard layout is the same, but the corpus and possibly language changes generating a different fitness score. To make the tracking and discussion of keyboards easier through the rest of the work a Keyboard ID will be added. The Keyboard ID will be consistent to the particular keyboard layout arrangement regardless of the corpus or language used in the experiment.\par{}\indent The following charts illustrate the interrelatedness of various experiments. Figure \hyperlink{fAFU}{22} illustrates the 13 experiments with the AFU keyboard which all had the Dan Language.\par{}\vspace{11pt plus 2pt minus 1pt}\setbox0=\vbox{\protect\raggedright\leavevmode
\vspace*{0pt}{\XeTeXpdffile "../Resources/AFUTree.pdf" scaled 260}\\[0pt]\protect\hypertarget{fAFU}{}\XLingPaperaddtocontents{fAFU}{\singlespacing
{Figure }{22.}{ AFU Experiments\\}}}\box0\par{}\vspace{11pt plus 2pt minus 1pt}\vspace{11pt plus 2pt minus 1pt}\setbox0=\vbox{\protect\raggedright\leavevmode
\vspace*{0pt}{\XeTeXpdffile "../Resources/Trans-MandeTree.pdf" scaled 260}\\[0pt]\protect\hypertarget{fTrans-Mande}{}\XLingPaperaddtocontents{fTrans-Mande}{\singlespacing
{Figure }{23.}{ Trans-Mande Experiments\\}}}\box0\par{}\vspace{11pt plus 2pt minus 1pt}\clearpage
\thispagestyle{bodyfirstpage}\markboth{Results}{Results}
\XLingPaperaddtocontents{cResults}{\vspace*{.65in}\XLingPaperneedspace{3\baselineskip}\noindent
\fontsize{14}{16.8}\selectfont \textbf{{\centering
CHAPTER \raisebox{\baselineskip}[0pt]{\protect\hypertarget{cResults}{}}\raisebox{\baselineskip}[0pt]{\pdfbookmark[1]{6 Results}{cResults}}6\protect\\}}}\par{}
{\XLingPaperneedspace{3\baselineskip}\noindent
\fontsize{14}{16.8}\selectfont \textbf{{\centering
Results\protect\\}}}\par{}
\vspace{16pt}\indent The experimental result are presented in tables by language. The tables indicate the keyboard, the fitness score, a Keyboard ID, and an experiment ID. The Keyboard ID and the Experiment ID are keyed to a larger data set of experiments (and languages) on keyboards which is yet to be published. Keyboard IDs are consistent across tables for keyboard layouts which are the same. Lower fitness scores represent better performance scores.\par{}\vspace{11pt plus 2pt minus 1pt}\XLingPaperneedspace{3\baselineskip}\protect\hypertarget{ntResults}{}\XLingPaperaddtocontents{ntResults}{\protect\raggedright{\singlespacing
{Table }{34.}{  Comparison of English Fitness Scores\\}}}\vspace{0pt}{\singlespacing
\hspace*{.25in}{
\XLingPaperminmaxcellincolumn{Keyboard}{\XLingPapermincola}{\textbf{Keyboard Name}}{\XLingPapermaxcola}{+0\tabcolsep}
\XLingPaperminmaxcellincolumn{Fitness}{\XLingPapermincolb}{\textbf{Fitness Score}}{\XLingPapermaxcolb}{+0\tabcolsep}
\XLingPaperminmaxcellincolumn{Keyboard}{\XLingPapermincolc}{\textbf{Keyboard ID}}{\XLingPapermaxcolc}{+0\tabcolsep}
\XLingPaperminmaxcellincolumn{ID}{\XLingPapermincold}{\textbf{Experiment ID}}{\XLingPapermaxcold}{+0\tabcolsep}
\XLingPaperminmaxcellincolumn{Dvorak}{\XLingPapermincola}{Dvorak}{\XLingPapermaxcola}{+0\tabcolsep}
\XLingPaperminmaxcellincolumn{1642}{\XLingPapermincolb}{1642}{\XLingPapermaxcolb}{+0\tabcolsep}
\XLingPaperminmaxcellincolumn{27}{\XLingPapermincolc}{27}{\XLingPapermaxcolc}{+0\tabcolsep}
\XLingPaperminmaxcellincolumn{31}{\XLingPapermincold}{31}{\XLingPapermaxcold}{+0\tabcolsep}
\XLingPaperminmaxcellincolumn{QWERTY}{\XLingPapermincola}{QWERTY}{\XLingPapermaxcola}{+0\tabcolsep}
\XLingPaperminmaxcellincolumn{3362}{\XLingPapermincolb}{3362}{\XLingPapermaxcolb}{+0\tabcolsep}
\XLingPaperminmaxcellincolumn{24}{\XLingPapermincolc}{24}{\XLingPapermaxcolc}{+0\tabcolsep}
\XLingPaperminmaxcellincolumn{28}{\XLingPapermincold}{28}{\XLingPapermaxcold}{+0\tabcolsep}
\setlength{\XLingPaperavailabletablewidth}{433.62pt}
\setlength{\XLingPapertableminwidth}{\XLingPapermincola+\XLingPapermincolb+\XLingPapermincolc+\XLingPapermincold}
\setlength{\XLingPapertablemaxwidth}{\XLingPapermaxcola+\XLingPapermaxcolb+\XLingPapermaxcolc+\XLingPapermaxcold}
\XLingPapercalculatetablewidthratio{}
\XLingPapersetcolumnwidth{\XLingPapercolawidth}{\XLingPapermincola}{\XLingPapermaxcola}{-0\tabcolsep}
\XLingPapersetcolumnwidth{\XLingPapercolbwidth}{\XLingPapermincolb}{\XLingPapermaxcolb}{-2\tabcolsep}
\XLingPapersetcolumnwidth{\XLingPapercolcwidth}{\XLingPapermincolc}{\XLingPapermaxcolc}{-2\tabcolsep}
\XLingPapersetcolumnwidth{\XLingPapercoldwidth}{\XLingPapermincold}{\XLingPapermaxcold}{-2\tabcolsep}\singlespacing\vspace*{-3\baselineskip}
\begin{longtable}
[l]{@{}p{\XLingPapercolawidth}p{\XLingPapercolbwidth}p{\XLingPapercolcwidth}p{\XLingPapercoldwidth}@{}}\toprule\multicolumn{1}{@{}p{\XLingPapercolawidth}}{\textbf{Keyboard Name}}&\multicolumn{1}{p{\XLingPapercolbwidth}}{\textbf{Fitness Score}}&\multicolumn{1}{p{\XLingPapercolcwidth}}{\textbf{Keyboard ID}}&\multicolumn{1}{p{\XLingPapercoldwidth}@{}}{\textbf{Experiment ID}}\\%
\midrule\endhead \multicolumn{1}{@{}p{\XLingPapercolawidth}}{Dvorak}&\multicolumn{1}{p{\XLingPapercolbwidth}}{1642}&\multicolumn{1}{p{\XLingPapercolcwidth}}{27}&\multicolumn{1}{p{\XLingPapercoldwidth}@{}}{31}\\%
\multicolumn{1}{@{}p{\XLingPapercolawidth}}{QWERTY}&\multicolumn{1}{p{\XLingPapercolbwidth}}{3362}&\multicolumn{1}{p{\XLingPapercolcwidth}}{24}&\multicolumn{1}{p{\XLingPapercoldwidth}@{}}{28}\\\bottomrule%
\end{longtable}
}
}\vspace{11pt plus 2pt minus 1pt}\setbox0=\vbox{\protect\centering \leavevmode
\vspace*{0pt}{\XeTeXpicfile "../Resources/James-Scale.png" scaled 400}\\[0pt]\protect\hypertarget{fResultsComparison}{}\XLingPaperaddtocontents{fResultsComparison}{\singlespacing
{Figure }{24.}{ Results of the two evaluations\\}}}\box0\par{}\vspace{11pt plus 2pt minus 1pt}\clearpage
\thispagestyle{bodyfirstpage}\markboth{Discussion}{Discussion}
\XLingPaperaddtocontents{cDiscussion}{\vspace*{.65in}\XLingPaperneedspace{3\baselineskip}\noindent
\fontsize{14}{16.8}\selectfont \textbf{{\centering
CHAPTER \raisebox{\baselineskip}[0pt]{\protect\hypertarget{cDiscussion}{}}\raisebox{\baselineskip}[0pt]{\pdfbookmark[1]{7 Discussion}{cDiscussion}}7\protect\\}}}\par{}
{\XLingPaperneedspace{3\baselineskip}\noindent
\fontsize{14}{16.8}\selectfont \textbf{{\centering
Discussion\protect\\}}}\par{}
\vspace{16pt}\indent Based on control corpus, i.e., the parallel corpus, the QWERTY-Dvorak difference is cast within a range of 1,720 fitness points. Prior research situates complex issues such as fewer typing errors, less typing fatigue, greater typing speed, and carpal tunnel due to repetitive motion injury within this range. The distinction between the French keyboards, AZERTY and Bépo, is greater at 1,886 fitness points. However, when we look at the challenges faced by Eastern Dan typists, the difference between typing with the Trans-Mande keyboard in Dan versus AZERTY in French is an improvement of 14,564 fitness points. In this case switching languages from Dan to French is an improvement in the time cost and effort cost of communication.\par{}\vspace{11pt plus 2pt minus 1pt}\XLingPaperneedspace{3\baselineskip}\protect\hypertarget{ntBaselineAndOptimize}{}\XLingPaperaddtocontents{ntBaselineAndOptimize}{\protect\raggedright{\singlespacing
{Table }{35.}{  Baseline versus Optimized Keyboard Scores\\}}}\vspace{0pt}{\singlespacing
\hspace*{.25in}{
\XLingPaperminmaxcellincolumn{}{\XLingPapermincola}{\textbf{}}{\XLingPapermaxcola}{+0\tabcolsep}
\XLingPaperminmaxcellincolumn{}{\XLingPapermincolb}{\textbf{}}{\XLingPapermaxcolb}{+0\tabcolsep}
\XLingPaperminmaxcellincolumn{}{\XLingPapermincolc}{\textbf{}}{\XLingPapermaxcolc}{+0\tabcolsep}
\XLingPaperminmaxcellincolumn{}{\XLingPapermincola}{\textbf{}}{\XLingPapermaxcola}{+0\tabcolsep}
\XLingPaperminmaxcellincolumn{}{\XLingPapermincolb}{}{\XLingPapermaxcolb}{+0\tabcolsep}
\XLingPaperminmaxcellincolumn{}{\XLingPapermincolc}{}{\XLingPapermaxcolc}{+0\tabcolsep}
\XLingPaperminmaxcellincolumn{}{\XLingPapermincola}{\textbf{}}{\XLingPapermaxcola}{+0\tabcolsep}
\XLingPaperminmaxcellincolumn{}{\XLingPapermincolb}{}{\XLingPapermaxcolb}{+0\tabcolsep}
\XLingPaperminmaxcellincolumn{}{\XLingPapermincolc}{}{\XLingPapermaxcolc}{+0\tabcolsep}
\setlength{\XLingPaperavailabletablewidth}{433.62pt}
\setlength{\XLingPapertableminwidth}{\XLingPapermincola+\XLingPapermincolb+\XLingPapermincolc}
\setlength{\XLingPapertablemaxwidth}{\XLingPapermaxcola+\XLingPapermaxcolb+\XLingPapermaxcolc}
\XLingPapercalculatetablewidthratio{}
\XLingPapersetcolumnwidth{\XLingPapercolawidth}{\XLingPapermincola}{\XLingPapermaxcola}{-0\tabcolsep}
\XLingPapersetcolumnwidth{\XLingPapercolbwidth}{\XLingPapermincolb}{\XLingPapermaxcolb}{-2\tabcolsep}
\XLingPapersetcolumnwidth{\XLingPapercolcwidth}{\XLingPapermincolc}{\XLingPapermaxcolc}{-2\tabcolsep}\singlespacing\vspace*{-3\baselineskip}
\begin{longtable}
[l]{@{}p{\XLingPapercolawidth}p{\XLingPapercolbwidth}p{\XLingPapercolcwidth}@{}}\toprule\multicolumn{1}{@{}p{\XLingPapercolawidth}}{\textbf{}}&\multicolumn{1}{p{\XLingPapercolbwidth}}{\textbf{}}&\multicolumn{1}{p{\XLingPapercolcwidth}@{}}{\textbf{}}\\%
\midrule\endhead \multicolumn{1}{@{}p{\XLingPapercolawidth}}{\textbf{}}&\multicolumn{1}{p{\XLingPapercolbwidth}}{}&\multicolumn{1}{p{\XLingPapercolcwidth}@{}}{}\\%
\multicolumn{1}{@{}p{\XLingPapercolawidth}}{\textbf{}}&\multicolumn{1}{p{\XLingPapercolbwidth}}{}&\multicolumn{1}{p{\XLingPapercolcwidth}@{}}{}\\\bottomrule%
\end{longtable}
}
}\noindent {\XLingPaperCharisZSILFontFamily{\textit{Typing}}} discovered an optimized Trans-Mande keyboard layout for the Eastern Dan language which gives us a keyboard layout that is an improvement of 15,201 fitness points over the existing Trans-Mande keyboard layout. This result is a bit more optimal than AZERTY for French. With a few more tweaks such as optimizing the relationships of characters available under dead keys even greater optimization can be achieved.\par{}{\vspace{15pt}\XLingPaperneedspace{3\baselineskip}\noindent
\fontsize{13}{15.6}\selectfont \textbf{{\noindent
\raisebox{\baselineskip}[0pt]{\pdfbookmark[2]{{7.1 } Conclusion}{sConclusion}}\raisebox{\baselineskip}[0pt]{\protect\hypertarget{sConclusion}{}}{7.1 }Conclusion}}\markboth{Conclusion}{Discussion}\XLingPaperaddtocontents{sConclusion}}\par{}
\penalty10000\vspace{10pt}\penalty10000\indent The Typing program in conjunction with a parallel corpus offers a means of cross-language comparison. As shown in section 4, the typing-difficulty expressed as a fitness score for any given keyboard layout can be assessed based on a corpus of any particular language. Further, the typing-difficulty can be actively mitigated by optimizing keyboard layouts for particular languages.\par{}\indent I propose based on the presented models that the intergenerational transmission of writing skills within the Eastern Dan community was not broken solely because of civil wars, but because of the socio-technical change around the activity of writing. Design of efficient typing experiences could revolutionize writing for a new generation of Victor Hugos.\par{}{\vspace{15pt}\XLingPaperneedspace{3\baselineskip}\noindent
\fontsize{13}{15.6}\selectfont \textbf{{\noindent
\raisebox{\baselineskip}[0pt]{\pdfbookmark[2]{{7.2 } Implications for future work}{sApplications}}\raisebox{\baselineskip}[0pt]{\protect\hypertarget{sApplications}{}}{7.2 }Implications for future work}}\markboth{Implications for future work}{Discussion}\XLingPaperaddtocontents{sApplications}}\par{}
\penalty10000\vspace{10pt}\penalty10000{\vspace{10pt}\XLingPaperneedspace{3\baselineskip}\noindent
\fontsize{13}{15.6}\selectfont \textit{{\noindent
\raisebox{\baselineskip}[0pt]{\pdfbookmark[3]{{7.2.1 } Implications for live typing experiments}{sLiveType}}\raisebox{\baselineskip}[0pt]{\protect\hypertarget{sLiveType}{}}{7.2.1 }Implications for live typing experiments}}\markboth{Implications for live typing experiments}{Discussion}\XLingPaperaddtocontents{sLiveType}}\par{}
\penalty10000\vspace{10pt}\penalty10000\indent Anna about morphology While typing\par{}\indent Things to test, Speed of typing in both languages\par{}\indent Hand shapes actually used.\par{}\indent Modleing learning times https://users.aalto.fi/\textasciitilde{}jokinej10/visual-search/\par{}{\vspace{10pt}\XLingPaperneedspace{3\baselineskip}\noindent
\fontsize{13}{15.6}\selectfont \textit{{\noindent
\raisebox{\baselineskip}[0pt]{\pdfbookmark[3]{{7.2.2 } Some thoughts on Alphabets}{sAlphabets}}\raisebox{\baselineskip}[0pt]{\protect\hypertarget{sAlphabets}{}}{7.2.2 }Some thoughts on Alphabets}}\markboth{Some thoughts on Alphabets}{Discussion}\XLingPaperaddtocontents{sAlphabets}}\par{}
\penalty10000\vspace{10pt}\penalty10000\indent Bird suggests that there might be something like a diacritic density. However, Does eastern dan have Diacritics? And if it doesn't then what is the best way to classify the tone marks. They are not solely alphabetical... does dan have an alphabet? does {\textit{Me'phaa}} have an alphabet? If I were to ask either speaker to tell me their alphabet would they go through all the vowels with a single pitch height, or would they add each melody to their alphabet? or would they add the diacritic marks sans any thought of the alphabet? The name "tone marks" does not seem reasonable, because Spanish and French use these marks without tone, and other languages around Vietnam use these marks for things other than tone. If we had one diacritic per melody as they do in Mexico would that help? I think not, not in the sense of typing easier because it is the same number of letters. But with technology can we fast forward and type with two instead of one characters per keystroke? does it make sense to have a high/low key divide? Not really... Reference discussion herein the terms section and \hyperlink{sEDWritingSystem}{3}\par{}{\vspace{10pt}\XLingPaperneedspace{3\baselineskip}\noindent
\fontsize{13}{15.6}\selectfont \textit{{\noindent
\raisebox{\baselineskip}[0pt]{\pdfbookmark[3]{{7.2.3 } Some thoughts on}{sSome}}\raisebox{\baselineskip}[0pt]{\protect\hypertarget{sSome}{}}{7.2.3 }Some thoughts on}}\markboth{Some thoughts on}{Discussion}\XLingPaperaddtocontents{sSome}}\par{}
\penalty10000\vspace{10pt}\penalty10000{\vspace{10pt}\XLingPaperneedspace{3\baselineskip}\noindent
{\noindent
\raisebox{\baselineskip}[0pt]{\pdfbookmark[4]{{7.2.3.1 } Text input and security}{s}}\raisebox{\baselineskip}[0pt]{\protect\hypertarget{s}{}}{7.2.3.1 }Text input and security}\markboth{Text input and security}{Discussion}\XLingPaperaddtocontents{s}}\par{}
\penalty10000\vspace{10pt}\penalty10000\indent Keyman manipulates input as you type So it is more than a keyboard layout.\par{}\indent Gives you more control than OS keyboards. But this is the point it is a pre-processing text input engine. Kyeboards on OSes are only text input engines.\par{}\indent Comprehensive development tools... Where the OSes fall flat on their face...\par{}\indent But for platforms like android and iOS, we don't know what else is in the app that we download. We end up trusting a third party with our text input decisions. We are already trusting the OS developer that they are not looping that data to somewhere else. viruses are open source too.\par{}\indent when OSes do not implement language aware features, it creates opportunity for a third party to find a solution. These solutions. Deserve money and require trust.\par{}\pagestyle{body}\clearpage
\thispagestyle{empty}{\clearpage
\vspace*{4.075in}\XLingPaperneedspace{3\baselineskip}\noindent
{\centering
\raisebox{\baselineskip}[0pt]{\protect\hypertarget{rXLingPapAppendiciesPage}{}}APPENDICES\protect\\}\markboth{APPENDICES}{APPENDICES}
\XLingPaperaddtocontents{rXLingPapAppendiciesPage}}\penalty10000\par{}
\vfil
\raisebox{\baselineskip}[0pt]{\pdfbookmark[1]{APPENDICES}{rXLingPapAppendiciesPage}}\clearpage
\thispagestyle{bodyfirstpage}\markboth{Chart of all keyboard experiment results}{Chart of all keyboard experiment results}
\XLingPaperaddtocontents{ChartResults}{\vspace*{.65in}\XLingPaperneedspace{3\baselineskip}\noindent
{\centering
APPENDIX \raisebox{\baselineskip}[0pt]{\protect\hypertarget{ChartResults}{}}\raisebox{\baselineskip}[0pt]{\pdfbookmark[1]{A  Chart of all keyboard experiment results}{ChartResults}}A\protect\\}}\par{}
{\XLingPaperneedspace{3\baselineskip}\noindent
{\centering
Chart of all keyboard experiment results\protect\\}}\par{}
\vspace{16pt}\landscape
\vspace{11pt plus 2pt minus 1pt}\setbox0=\vbox{\protect\centering \vspace*{-4.75ex}\leavevmode{}{\XeTeXpdffile "../Resources/AppendixA.pdf" scaled 750}\\[0pt]\protect\hypertarget{TotalResults}{}\XLingPaperaddtocontents{TotalResults}{\singlespacing
{Figure }{25.}{ Results of all experiments by ordered by Experiment ID\\}}}\box0\par{}\vspace{11pt plus 2pt minus 1pt}\endlandscape
\clearpage
\thispagestyle{bodyfirstpage}\markboth{Glossary of technical terms}{Glossary of technical terms}
\XLingPaperaddtocontents{DefinedTerms}{\vspace*{.65in}\XLingPaperneedspace{3\baselineskip}\noindent
{\centering
APPENDIX \raisebox{\baselineskip}[0pt]{\protect\hypertarget{DefinedTerms}{}}\raisebox{\baselineskip}[0pt]{\pdfbookmark[1]{B  Glossary of technical terms}{DefinedTerms}}B\protect\\}}\par{}
{\XLingPaperneedspace{3\baselineskip}\noindent
{\centering
Glossary of technical terms\protect\\}}\par{}
\vspace{16pt}{\singlespacing
{{\hangafter1\relax
\hangindent.3in\relax
\noindent{}\raisebox{\baselineskip}[0pt]{\protect\hypertarget{gtAbstractUnicodeCharacter}{}}{\textbf{Abstract Character}}: a unit of information used for the organization, control or representation of textual data. Abstract characters may be non-graphic characters used in textual information systems to control the organization of textual data (e.g. U+FFF9 INTERLINEAR ANNOTATION ANCHOR), or to control the presentation of textual data (e.g. U+200D ZERO WIDTH JOINER).\par{}
\vspace{4pt}}{\hangafter1\relax
\hangindent.3in\relax
\noindent{}\raisebox{\baselineskip}[0pt]{\protect\hypertarget{gtAlphabet}{}}{\textbf{Alphabet}}: is a list of \hyperlink{gtLetter}{{\textit{letters}}} used in the transcription of a language. Alphabets usually have an order for pedagogical and dictionary sorting purposes. At a technical level, \hyperlink{NRSIGlossary}{Lyons (2001)} provides this definition: {\XLingPaperCharisZSILFontFamily{\textit{a segmental writing system having symbols for individual sounds, rather than for syllables or morphemes. In a true alphabet, consonants and vowels are written as independent letters, in contrast to an abugida or an abjad. In a perfectly phonemic alphabet, phonemes and letters would be predictable in both directions; that is, the sound of a word could be predicted from its spelling and vice-versa. A phonetic alphabet is also predictable in this way, however it uses separate letters for separate allophones, whereas a phonemic alphabet may describe allophones of the same phoneme using a single letter.}}} Many times in newly written languages their "alphabet" is based on a list of phonemes, in part that is what is required to write the language. But to the extent that two typographical characters are used together as a multi-graphs, an alphabet might have fewer {\textbf{letters}}/components than a list of phonemes in the same language.\par{}
\vspace{4pt}}{\hangafter1\relax
\hangindent.3in\relax
\noindent{}\raisebox{\baselineskip}[0pt]{\protect\hypertarget{gtAntColonyOptimization}{}}{\textbf{Ant Colony Optimization}}: according to \hyperlink{rSarbapriyaRay2016}{Sarbapriya (2016)} the ant colony optimization algorithm is a probabilistic technique for solving computational problems which can be reduced to finding good paths through graphs. In the case of keyboards a good path would be a strong (high frequency of use) edge between nodes (keys). So, if the key sequence of {\XLingPaperKeyboardZKeysExZExpandedFontFamily{\textup{\textmd{T}}}}{\XLingPaperKeyboardZKeysExZExpandedFontFamily{\textup{\textmd{H}}}} is stronger (more frequent) than {\XLingPaperKeyboardZKeysExZExpandedFontFamily{\textup{\textmd{S}}}}{\XLingPaperKeyboardZKeysExZExpandedFontFamily{\textup{\textmd{H}}}} then it might be defined as a better path, and something special might be done because of this.\par{}
\vspace{4pt}}{\hangafter1\relax
\hangindent.3in\relax
\noindent{}\raisebox{\baselineskip}[0pt]{\protect\hypertarget{gtBaseCharacter}{}}{\textbf{Base Character}}: the main part of an orthographic character; the {\XLingPaperCambriaZMathFontFamily{⟨ {\XLingPaperCharisZSILFontFamily{\textup{\textup{\textmd{a}}}}} ⟩}} in {\XLingPaperCambriaZMathFontFamily{⟨ {\XLingPaperCharisZSILFontFamily{\textup{\textup{\textmd{á}}}}} ⟩}} would be the base character. According to \hyperlink{rUnicodeGlossary}{Unicode (2017)}, a base character is: Any graphic character except for those with the General Category of Combining Mark (Mn). (See \href{http://www.unicode.org/versions/Unicode11.0.0/ch03.pdf\#G30602}{\textcolor[rgb]{0,0,0}{definition D51 in Section 3.6, Combination}} in \hyperlink{rUnicode11Standard}{Unicode (2018)}). In a combining character sequence, the base character is the initial character, to which the combining marks are applied.\par{}
\vspace{4pt}}{\hangafter1\relax
\hangindent.3in\relax
\noindent{}\raisebox{\baselineskip}[0pt]{\protect\hypertarget{gtBaseline}{}}{\textbf{Baseline}}: is a typography term for the invisible line which anchors the drawing of glyphs. Figures \hyperlink{fDiacritics}{1} and \hyperlink{fNTR-CSIL}{27} have illustrated baselines.\par{}
\vspace{4pt}}{\hangafter1\relax
\hangindent.3in\relax
\noindent{}\raisebox{\baselineskip}[0pt]{\protect\hypertarget{gtCharacterProperties}{}}{\textbf{Character Properties}}: in Unicode are properties assigned to each code point. These properties control text processing behaviors. Character properties are discussed in \hyperlink{rWhistler}{Whistler \& Freytag (2015)}. Examples are provided in Unicode \href{https://www.unicode.org/reports/tr23/\#CodePointProperties}{\textcolor[rgb]{0,0,0}{TR \#23 § 2.4}}. Character Properties are propagated in the {\hyperlink{vUCD}{{UCD}}} which is discussed in the Unicode Standard \href{http://www.unicode.org/versions/Unicode11.0.0/ch04.pdf\#G39}{\textcolor[rgb]{0,0,0}{Chapter 4}} and in \href{http://unicode.org/reports/tr44/}{\textcolor[rgb]{0,0,0}{TR \#44}}.\par{}
\vspace{4pt}}{\hangafter1\relax
\hangindent.3in\relax
\noindent{}\raisebox{\baselineskip}[0pt]{\protect\hypertarget{gtCodePoint}{}}{\textbf{Code point}}: in this thesis, because the discussion of characters is limited to Unicode, is a code point is the assigned location of reference for a Unicode character. It takes the form of U+xxxx(xx). More broadly a \hyperlink{gtCodePoint}{{\textit{code point}}} is a numeric value used as an encoded representation of some \hyperlink{gtAbstractUnicodeCharacter}{{\textit{abstract character}}} within a computer or information system. Code points are integer values used to represent particular characters within a particular encoding.\par{}
\vspace{4pt}}{\hangafter1\relax
\hangindent.3in\relax
\noindent{}\raisebox{\baselineskip}[0pt]{\protect\hypertarget{gtCognitiveFriction}{}}{\textbf{Cognitive Friction}}: is a term coined by Alan \hyperlink{rCooper1998}{Cooper (1998)} to describe the angst felt by a user of a digital device when the context shifts the behavior of the device in a way in which the user did not anticipate.\par{}
\vspace{4pt}}{\hangafter1\relax
\hangindent.3in\relax
\noindent{}\raisebox{\baselineskip}[0pt]{\protect\hypertarget{gtCognitiveLoad}{}}{\textbf{Cognitive load}}: is the stress placed on the brain's working memory. There are debatable ways to measure cognitive load. \hyperlink{r}{ ()} suggests that there is a significant increase in the usage of working memory after one must choose between seven options. This is used by {\hyperlink{vUX}{{UX}}} designers to suggest limits in the decision making options presented to users. Working memory is also an important function in recall, spelling, writing, and typing. The management of working memory deemed to be important to the learning process, in part because excessive cognitive load is distracting and prevents working memory from accessing other parts of storage in the brain.\par{}
\vspace{4pt}}{\hangafter1\relax
\hangindent.3in\relax
\noindent{}\raisebox{\baselineskip}[0pt]{\protect\hypertarget{gtComplexCharacter}{}}{\textbf{Complex Character}}: is also referred to as a \hyperlink{gtComposedCharacter}{{\textit{composed character}}}; is a orthographical character which in order to be created visually must use more than one Unicode code point. Sochiapam Chinantec’s {\textsquarebracketleft{}cso\textsquarebracketright{}} stressed-barred-i {\XLingPaperCambriaZMathFontFamily{⟨ {\XLingPaperCharisZSILFontFamily{\textup{\textup{\textmd{í̵}}}}} ⟩}} is an example of this as is Eastern Dan's {\XLingPaperCambriaZMathFontFamily{⟨ {\XLingPaperCharisZSILFontFamily{\textup{\textup{\textmd{ʋ̈}}}}} ⟩}} LATIN LETTER V WITH HOOK with COMBINING DIAERESIS.\par{}
\vspace{4pt}}{\hangafter1\relax
\hangindent.3in\relax
\noindent{}\raisebox{\baselineskip}[0pt]{\protect\hypertarget{gtComplexScript}{}}{\textbf{Complex Scripts}}: are scripts in which glyps change shape based on their word position (such as Greek and Arabic), or do not maintain the same direction of writing (such as Khmer).\par{}
\vspace{4pt}}{\hangafter1\relax
\hangindent.3in\relax
\noindent{}\raisebox{\baselineskip}[0pt]{\protect\hypertarget{gtComposedCharacter}{}}{\textbf{Composed Character}}: see \hyperlink{gtComplexCharacter}{{\textit{Complex Characte}}}r\par{}
\vspace{4pt}}{\hangafter1\relax
\hangindent.3in\relax
\noindent{}\raisebox{\baselineskip}[0pt]{\protect\hypertarget{gtCompositeCharacter}{}}{\textbf{Composite Character}}: also known as a pre-composed character. It is a single Unicode point which represents a character which can be broken down into multiple other characters. For example, {\XLingPaperCambriaZMathFontFamily{⟨ {\XLingPaperCharisZSILFontFamily{\textup{\textup{\textmd{á}}}}} ⟩}} can be either a composed character consisting of both the Unicode points for {\XLingPaperCambriaZMathFontFamily{⟨ ´ ⟩}} and {\XLingPaperCambriaZMathFontFamily{⟨ {\XLingPaperCharisZSILFontFamily{\textup{\textup{\textmd{a}}}}} ⟩}} or it can be a single character (precomposed) {\XLingPaperCambriaZMathFontFamily{⟨ {\XLingPaperCharisZSILFontFamily{\textup{\textup{\textmd{á}}}}} ⟩}} and represented by a single Unicode point.\par{}
\vspace{4pt}}{\hangafter1\relax
\hangindent.3in\relax
\noindent{}\raisebox{\baselineskip}[0pt]{\protect\hypertarget{gtConsonantBlend}{}}{\textbf{Consonant Blend}}: is a term often used in the teaching of reading. There are two defining attributes of a consonant blend: (1) it consists of a series of consonants which does not represent single phoneme, and (2) all of the consonants are vocalized.\par{}
\vspace{4pt}}{\hangafter1\relax
\hangindent.3in\relax
\noindent{}\raisebox{\baselineskip}[0pt]{\protect\hypertarget{gtcyberswarm}{}}{\textbf{Cyber Swarm Optimization}}: is a specialized case of {\hyperlink{vPSO}{{PSO}}} presented by \hyperlink{rYinetAl2010}{Yin et al. (2010)} which uses Scatter Search and \hyperlink{gtPathRelinking}{{\textit{path relinking}}} to create a more effective search algorithm.\par{}
\vspace{4pt}}{\hangafter1\relax
\hangindent.3in\relax
\noindent{}\raisebox{\baselineskip}[0pt]{\protect\hypertarget{gtDeadKey}{}}{\textbf{Dead Key}}: a key or combination of keys which produces no character by itself, but which in combination or sequence with another key produces a modified character, dead keys modify the out put of the character following them, rather than acting on the characters typed prior to when they are struck. The user perceives these keys as press-first-key-and-see-no-tangible-result, but press second key and get an altered result compared to what the user would have generated by just pressing the second key. This is the same pattern users who use the shift key for upper case letters experience. Shift by itself does not produce a character but impacts the next key struck. Unlike the chorded nature of using shift, dead keys are struck and released sequentially. Dead keys are commonly used to access characters with ornamentation (e.g., {\XLingPaperCambriaZMathFontFamily{\textup{\textmd{⟨ {\XLingPaperCharisZSILFontFamily{\textup{\textup{\textmd{ö, é, â}}}}} ⟩}}}}).\par{}
\vspace{4pt}}{\hangafter1\relax
\hangindent.3in\relax
\noindent{}\raisebox{\baselineskip}[0pt]{\protect\hypertarget{gtDiGraph}{}}{\textbf{Di-graph (digraph)}}: \par{}
\vspace{4pt}}{\hangafter1\relax
\hangindent.3in\relax
\noindent{}\raisebox{\baselineskip}[0pt]{\protect\hypertarget{gtDiacritic}{}}{\textbf{Diacritic Mark}}: \par{}
\vspace{4pt}}{\hangafter1\relax
\hangindent.3in\relax
\noindent{}\raisebox{\baselineskip}[0pt]{\protect\hypertarget{gtdiacritism}{}}{\textbf{D̤͓̄̈iac̤̰r͚̮̤͓it̤̱̆̈ĭ̤̺̈c̃̈̃̈ism}}: is the over indulgent use of diacritics.\par{}
\vspace{4pt}}{\hangafter1\relax
\hangindent.3in\relax
\noindent{}\raisebox{\baselineskip}[0pt]{\protect\hypertarget{gtEndangeredLanugage}{}}{\textbf{Endangered Language}}: is a language which is at risk of falling out of use.\par{}
\vspace{4pt}}{\hangafter1\relax
\hangindent.3in\relax
\noindent{}\raisebox{\baselineskip}[0pt]{\protect\hypertarget{gtevolutionaryalgorithms}{}}{\textbf{Evolutionary Algorithm}}: is a search algorithm which generates a random population of options within the search space. It then searches the population (candidates) for the best option based on a fitness (evaluation) criteria. The algorithm progresses through the search space by modulating the candidates rather than randomly seeking and evaluating new ones. As better scoring candidates are found EAs can track those changes and us that difference to focus search on sub-populations with only those traits. EAs often have a self-termination sequence after a certain number of evolutionary steps.\par{}
\vspace{4pt}}{\hangafter1\relax
\hangindent.3in\relax
\noindent{}\raisebox{\baselineskip}[0pt]{\protect\hypertarget{gtFuntionalUnit}{}}{\textbf{Functional Units}}: \par{}
\vspace{4pt}}{\hangafter1\relax
\hangindent.3in\relax
\noindent{}\raisebox{\baselineskip}[0pt]{\protect\hypertarget{gtGeneticAlgorithms}{}}{\textbf{Genetic Algorithms}}: according to \hyperlink{rSarbapriyaRay2016}{Sarbapriya (2016)} are algorithms which implement a search heuristic that mimics the process of natural selection. Genetic algorithms belong to the larger class of \hyperlink{gtevolutionaryalgorithms}{{\textit{evolutionary algorithms}}}, which generate solutions to optimization problems using techniques inspired by natural evolution, such as inheritance, mutation, selection, and crossover.\par{}
\vspace{4pt}}{\hangafter1\relax
\hangindent.3in\relax
\noindent{}\raisebox{\baselineskip}[0pt]{\protect\hypertarget{gtGlyph}{}}{\textbf{Glyph}}: recognizable abstract graphic symbol which is independent of any specific design\par{}
\vspace{4pt}}{\hangafter1\relax
\hangindent.3in\relax
\noindent{}\raisebox{\baselineskip}[0pt]{\protect\hypertarget{gtGraphemeCluster}{}}{\textbf{Grapheme Cluster}}: \par{}
\vspace{4pt}}{\hangafter1\relax
\hangindent.3in\relax
\noindent{}\raisebox{\baselineskip}[0pt]{\protect\hypertarget{gtGraphicalCharacters}{}}{\textbf{Graphical Character}}: per ISO 10646:\hyperlink{rISO10646}{2010:page 14 \#4.29} “A character, other than a control function or a format character, that has a visual representation normally handwritten, printed, or displayed”.\par{}
\vspace{4pt}}{\hangafter1\relax
\hangindent.3in\relax
\noindent{}\raisebox{\baselineskip}[0pt]{\protect\hypertarget{gtHLT}{}}{\textbf{Human Language Technologies ({\textbf{{\hyperlink{vHLT}{{HLT}}}}})}}: are technologies which allow humans to interact with machines via a language based functionality. Examples are speech commands, speech‑to‑text, typing, etc\par{}
\vspace{4pt}}{\hangafter1\relax
\hangindent.3in\relax
\noindent{}\raisebox{\baselineskip}[0pt]{\protect\hypertarget{gtISO639-3}{}}{\textbf{ISO 639-3}}: is a technical standard which provides a set of codes which refer to languages.\par{}
\vspace{4pt}}{\hangafter1\relax
\hangindent.3in\relax
\noindent{}\raisebox{\baselineskip}[0pt]{\protect\hypertarget{gtKeyCodes}{}}{\textbf{Key Codes}}: are the numeric values generated by the keyboard hardware which get sent to the operating system for processing into code points representing characters.\par{}
\vspace{4pt}}{\hangafter1\relax
\hangindent.3in\relax
\noindent{}\raisebox{\baselineskip}[0pt]{\protect\hypertarget{gtKeyboardLayout}{}}{\textbf{Keyboard Layout}}: is the abstract arrangement of characters and their relationship to the buttons of a physical or virtual keyboard.\par{}
\vspace{4pt}}{\hangafter1\relax
\hangindent.3in\relax
\noindent{}\raisebox{\baselineskip}[0pt]{\protect\hypertarget{gtLanguage}{}}{\textbf{Language}}: for the purposes of this work is any speech variety which is recognized by the ISO 639-3 standard as a distinct entity – having its own ISO 639-3 code. Languages may come and go, \hyperlink{gtEndangeredLanugage}{{\textit{Endangered languages}}} give one method of language loss, another method of language loss might be the unification of two formerly separate language into a single language at the level of the ISO 639-3 standard as is exemplified by \hyperlink{rEngland2008}{England (2008)}.\par{}
\vspace{4pt}}{\hangafter1\relax
\hangindent.3in\relax
\noindent{}\raisebox{\baselineskip}[0pt]{\protect\hypertarget{gtLWC}{}}{\textbf{Language of Wider Communication ({\textbf{{\hyperlink{vLWC}{{LWC}}}}})}}: is a language which is usually marked by use across a large geographical area. However, more importantly it is a language used by multi-linguals as they communicate with people who may not share their same ethnolinguistic identity.\par{}
\vspace{4pt}}{\hangafter1\relax
\hangindent.3in\relax
\noindent{}\raisebox{\baselineskip}[0pt]{\protect\hypertarget{gtLatinScript}{}}{\textbf{Latin Script}}: is one of the encoded scripts of ISO 15924 and Unicode. It has its history in Etruscan (Italy) alphabet, and the Phoenician phonetic system of writing based on sounds.\par{}
\vspace{4pt}}{\hangafter1\relax
\hangindent.3in\relax
\noindent{}\raisebox{\baselineskip}[0pt]{\protect\hypertarget{gtLetter}{}}{\textbf{Letter(s)}}: are typographical units for the purposes of pedagogy. These typographical units, can not be broken down into sub-components which are each or either also letters. e.g. if an alphabet wants to have a letter {\XLingPaperCambriaZMathFontFamily{\textup{\textmd{⟨ {\XLingPaperCharisZSILFontFamily{\textup{\textup{\textmd{kp}}}}} ⟩}}}} then it can not have a letter {\XLingPaperCambriaZMathFontFamily{\textup{\textmd{⟨ {\XLingPaperCharisZSILFontFamily{\textup{\textup{\textmd{k}}}}} ⟩}}}} or a letter {\XLingPaperCambriaZMathFontFamily{\textup{\textmd{⟨ {\XLingPaperCharisZSILFontFamily{\textup{\textup{\textmd{p}}}}} ⟩}}}}.\par{}
\vspace{4pt}}{\hangafter1\relax
\hangindent.3in\relax
\noindent{}\raisebox{\baselineskip}[0pt]{\protect\hypertarget{gtLocale}{}}{\textbf{Locale}}: as used in computers refers to a set of parameters used by software. The Locale settings are usually provided as part of the operating system and is used for presenting language data (texts), and cultural data (i.e. the names of the days of the week, or the currency symbol) in standard formats within the User Interface ({\hyperlink{vUI}{{UI}}}).\par{}
\vspace{4pt}}{\hangafter1\relax
\hangindent.3in\relax
\noindent{}\raisebox{\baselineskip}[0pt]{\protect\hypertarget{gtLogographicScripts}{}}{\textbf{Logographic Script}}: is a script in which each logo represents a morpheme, word, or phrase. Prime examples are Han characters used in Chinese or Japanese, and Hieroglyphics. These scripts derive a lot of meaning from the semantics attributed to the shapes of the characters, rather than the phonetics attributed to the shapes of characters.\par{}
\vspace{4pt}}{\hangafter1\relax
\hangindent.3in\relax
\noindent{}\raisebox{\baselineskip}[0pt]{\protect\hypertarget{gtLowerCase}{}}{\textbf{Lower Case}}: letters are paired with \hyperlink{gtUpperCase}{{\textit{upper case}}} letters and are use interchangeably for the purposes of spelling. Their use is generally dictated by some typographic norm established by a particular orthography or publisher style sheet. Not all scripts have upper and lower case. Not all orthographies have described norms for case usage.\par{}
\vspace{4pt}}{\hangafter1\relax
\hangindent.3in\relax
\noindent{}\raisebox{\baselineskip}[0pt]{\protect\hypertarget{gtMajorityLanguage}{}}{\textbf{Majority language}}: see \hyperlink{gtMinorityLanguage}{{\textit{minority language}}}.\par{}
\vspace{4pt}}{\hangafter1\relax
\hangindent.3in\relax
\noindent{}\raisebox{\baselineskip}[0pt]{\protect\hypertarget{gtMateriality}{}}{\textbf{Materiality}}: is a concept which relates human behavior to things. \hyperlink{rLievrouw2014}{Lievrouw (2014)} discusses relevant literature and how the concept has been defined by various authors. To summarize, materiality is the idea that the physical properties of a cultural artifact have consequences on how the object is used. This understanding of materiality has footing in modern anthropology \hyperlink{rGosdenChrisYvonneMarshall1999TheCu}{(Gosden \& Marshall  1999)}, and the study of how cultures interact with digital devices and digital content \hyperlink{rWajcmanJones2012}{(Wajcman \& Jones  2012)}.\par{}
\vspace{4pt}}{\hangafter1\relax
\hangindent.3in\relax
\noindent{}\raisebox{\baselineskip}[0pt]{\protect\hypertarget{gtMetaheuristic}{}}{\textbf{Metaheuristics}}: are processes which guide the search process. They are used when large set of possible solutions are probable and a near-accurate solution i would take less time to solve for than an absolute (accurate) solution.\par{}
\vspace{4pt}}{\hangafter1\relax
\hangindent.3in\relax
\noindent{}\raisebox{\baselineskip}[0pt]{\protect\hypertarget{gtMinorityLanguage}{}}{\textbf{Minority Languages}}: are languages in polyglossic situations where one language community will have either or both a statistical minority or a political minority status. In polyglossic social contexts the \hyperlink{gtMajorityLanguage}{{\textit{majority language}}} might also be a language of wider communication among several minority languages.\par{}
\vspace{4pt}}{\hangafter1\relax
\hangindent.3in\relax
\noindent{}\raisebox{\baselineskip}[0pt]{\protect\hypertarget{gtModifierKey}{}}{\textbf{Modifier Key}}: is a key on a keyboard which alters the default function or output of other keys on a keyboard. In this sense \hyperlink{gtDeadKey}{{\textit{dead keys}}} and \hyperlink{gtOperatorKey}{{\textit{operator keys}}} can be considered \hyperlink{gtModifierKey}{{\textit{modifier keys}}}. \hyperlink{rKacmarcik2017}{Kacmarcik \& Leithead (2017} \href{https://www.w3.org/TR/uievents-key/\#keys-modifier}{\textcolor[rgb]{0,0,0}{§ 3.2}}) lays out a nice list of modifier keys which may or may not be present on the hardware: Shift, Alt, Control, Meta, AltGraph (AltGr), CapsLock, MajLock, Hyper, Super, Fn (Function), Fn-Lock (Function-Lock), NumLock (Number Lock), ScrollLock, Symbol, SymbolLock.\par{}
\vspace{4pt}}{\hangafter1\relax
\hangindent.3in\relax
\noindent{}\raisebox{\baselineskip}[0pt]{\protect\hypertarget{gtMultiGraph}{}}{\textbf{Multi-graph (multigraph)}}: \par{}
\vspace{4pt}}{\hangafter1\relax
\hangindent.3in\relax
\noindent{}\raisebox{\baselineskip}[0pt]{\protect\hypertarget{gtNRSI}{}}{\textbf{Non-Roman Script Initiative}}: {\hyperlink{vNRSI}{{NRSI}}} functions as SIL International's foundry; providing text implementation tools, including fonts, for minority languages and scripts. In 2019 NRSI was rebranded to {\XLingPaperCharisZSILFontFamily{\textit{WSTech}}} for {\XLingPaperCharisZSILFontFamily{\textit{Writing System Technology}}}.\par{}
\vspace{4pt}}{\hangafter1\relax
\hangindent.3in\relax
\noindent{}\raisebox{\baselineskip}[0pt]{\protect\hypertarget{gtNFC}{}}{\textbf{Normalization Form C}}: Canonical Decomposition, followed by Canonical Composition. In NFC form "parts of a character" are compiled into a single Unicode code point. Note that not all complex characters have composite and decomposed forms. The transition between NFC and NFD is reversible. For the specifics on how normalization works in Unicode consult \hyperlink{rDavis2018Whistler}{Davis \& Ken Whistler (2018)}. A translation table exists for all composite characters and can be found on the Unicode website: \href{http://www.unicode.org/charts/normalization}{\textcolor[rgb]{0,0,0}{http://www.unicode.org/charts/normalization}}\par{}
\vspace{4pt}}{\hangafter1\relax
\hangindent.3in\relax
\noindent{}\raisebox{\baselineskip}[0pt]{\protect\hypertarget{gtNFD}{}}{\textbf{Normalization Form D}}: Canonical Decomposition. In NFD form "parts of a character" are de compressed into a multiple Unicode code points. Note that not all complex characters have composite and decomposed forms. The transition between NFC and NFD is reversible. For the specifics on how normalization works in Unicode consult \hyperlink{rDavis2018Whistler}{Davis \& Ken Whistler (2018)}. A translation table exists for all composite characters and can be found on the Unicode website: \href{http://www.unicode.org/charts/normalization}{\textcolor[rgb]{0,0,0}{http://www.unicode.org/charts/normalization}}\par{}
\vspace{4pt}}{\hangafter1\relax
\hangindent.3in\relax
\noindent{}\raisebox{\baselineskip}[0pt]{\protect\hypertarget{gtOperatorKey}{}}{\textbf{Operator Key}}: may be a confusing term. By popular usage, the term refers to a key which invokes a mathematical operator, such as on a calculator. However, as used in this thesis, it is a type of \hyperlink{gtModifierKey}{{\textit{modifier key}}}. \hyperlink{Paterson}{Paterson (2014)} follows \hyperlink{gtNRSI}{{\textit{NRSI}}} terminology as it is presented several times \hyperlink{rHosken2001}{Hosken (2001)} \hyperlink{rConstable2003}{Constable (2003)} and \hyperlink{rHosken2003Victor}{Hosken \& Gaultney (2003)} who contrast the interactive nature of a \hyperlink{gtDeadKey}{{\textit{dead key}}} which uses the input pattern of "diacritic"-base, with operator key which uses base-"diacritic". Therefore an operator key is one which is struck after a base character to in someway create a new character. The new character may or may not be visually similar to what would have been generated by just pressing the first key.\par{}
\vspace{4pt}}{\hangafter1\relax
\hangindent.3in\relax
\noindent{}\raisebox{\baselineskip}[0pt]{\protect\hypertarget{gtOrthographicCharacter}{}}{\textbf{Orthographic Character}}: a written symbol that is conventionally perceived as a distinct unit of writing in some orthography\par{}
\vspace{4pt}}{\hangafter1\relax
\hangindent.3in\relax
\noindent{}\raisebox{\baselineskip}[0pt]{\protect\hypertarget{gtOrthography}{}}{\textbf{Orthography}}: \par{}
\vspace{4pt}}{\hangafter1\relax
\hangindent.3in\relax
\noindent{}\raisebox{\baselineskip}[0pt]{\protect\hypertarget{gtparticleswarmoptimization}{}}{\textbf{Particle Swarm Optimization {\XLingPaperCharisZSILFontFamily{({\textbf{{\hyperlink{vPSO}{{PSO}}}}})}}}}: is an optimization algorithm which is modeled off of a flock of birds or a school of fish. It imagines each bird or fish finding the best route towards a goal. The algorithm imagines each bird or fish as a particle and then breaks down the search area assigning a single particle to each area. Each small area then in is tested for an optimal solution. The local (area) optimal solution is fed back to the larger search algorithm. It is assumed that the flock will generate an optimization and movement towards a global optimal solution.\par{}
\vspace{4pt}}{\hangafter1\relax
\hangindent.3in\relax
\noindent{}\raisebox{\baselineskip}[0pt]{\protect\hypertarget{gtPathRelinking}{}}{\textbf{Path Relinking}}: is a process of breaking down large potential solutions into their component parts, comparing these parts and keeping the best performing segments to build up an entire solution. For a full discussion and applications with search operations see \hyperlink{rGloverFredManuelLagunaRafaelMartED2000Funda}{Glover, Laguna \& Martí (2000)}.\par{}
\vspace{4pt}}{\hangafter1\relax
\hangindent.3in\relax
\noindent{}\raisebox{\baselineskip}[0pt]{\protect\hypertarget{gtPressAndHoldKey}{}}{\textbf{Press and Hold Key}}: allows a key to be pressed and held to generate a set of new options. This method gained wide spread use with its introduction via iOS and now some desktop operating systems allow for its use Images of press and hold interactions are presented in table \hyperlink{ntPressAndHold}{3}.\par{}
\vspace{4pt}}{\hangafter1\relax
\hangindent.3in\relax
\noindent{}\raisebox{\baselineskip}[0pt]{\protect\hypertarget{gtSaccade}{}}{\textbf{Saccades}}: are one of four types of eye movements. They are rapid ballistic movements between focus points. During reading, the amplitude of these movements are generally small. Focus points are generally 5-6 single width baseline characters \hyperlink{rMorrisonRaynerSacca}{(Morrison \& Rayner  1981)}.\par{}
\vspace{4pt}}{\hangafter1\relax
\hangindent.3in\relax
\noindent{}\raisebox{\baselineskip}[0pt]{\protect\hypertarget{gtScript}{}}{\textbf{Script}}: per ISO 10646:\hyperlink{rISO10646}{2010:page 16 \#4.50} and congruent with ISO 15924:\hyperlink{rISO15924}{2004} § 3.7, “a set of graphic characters used for the written form of one or more languages”. This is different from a writing system. A great example of a script-language relationship is that Russian is written with a subset of the Cyrillic Script, while Macedonian uses a different subset.\par{}
\vspace{4pt}}{\hangafter1\relax
\hangindent.3in\relax
\noindent{}\raisebox{\baselineskip}[0pt]{\protect\hypertarget{gtSMS}{}}{\textbf{Short Message Service}}: is a short text based message in telephony. In the USA these are known as \hyperlink{gtTexts}{{\textit{text messages}}} or simply \hyperlink{gtTexts}{{\textit{texts}}}. Traditionally these messages were limited to 160 \hyperlink{gtCodePoint}{{\textit{code points}}}, due to infrastructure limitations. Now it is often the case that the same social function is conducted with Internet data instead of telephone protocols. This alleviates the technological limitations of 160 characters and allows the messaging platforms to extend beyond the telephone to many kinds of devices.\par{}
\vspace{4pt}}{\hangafter1\relax
\hangindent.3in\relax
\noindent{}\raisebox{\baselineskip}[0pt]{\protect\hypertarget{gtSimulatedAnnealing}{}}{\textbf{Simulated Annealing}}: is an optimization strategy which evaluates a randomly produced solution within a search area (range of possible solutions) based on a cost framework (which needs to be defined). Then it modifies the solution slightly to check for "local" alternatives and scores those. It keeps the better scoring solution, and then repeats the process.\par{}
\vspace{4pt}}{\hangafter1\relax
\hangindent.3in\relax
\noindent{}\raisebox{\baselineskip}[0pt]{\protect\hypertarget{gtSkeuomorphism}{}}{\textbf{Skeuomorphism}}: is a design term used to describe an object (usually an imitation) and the thing that it represents (the original). Some designed objects are designed to look like another thing but take a different shape. These are Skeuomorphs. In {\hyperlink{vUI}{{UI}}} design it is possible to bring visual elements from a real world object and make the digital interaction an analogy to the real world object. This relationship between the real and the representation is Skeuomorphism. Its purpose is to immediately provide a sense of familiarity to users of new objects.\par{}
\vspace{4pt}}{\hangafter1\relax
\hangindent.3in\relax
\noindent{}\raisebox{\baselineskip}[0pt]{\protect\hypertarget{gtSpecificDesign}{}}{\textbf{Specific Design}}: when relating to a \hyperlink{gtGlyph}{{\textit{glyph}}} this term refers to graphically implemented features which might be things like stroke width, ascender and descender ratios, spacing, serif implementation, font-face (italic, oblique), font weight, etc.\par{}
\vspace{4pt}}{\hangafter1\relax
\hangindent.3in\relax
\noindent{}\raisebox{\baselineskip}[0pt]{\protect\hypertarget{gtString}{}}{\textbf{String}}: within the context of computer science is any finite sequence of characters i.e., letters, numerals, symbols and punctuation marks \hyperlink{rString}{(LINFO  2007)}.\par{}
\vspace{4pt}}{\hangafter1\relax
\hangindent.3in\relax
\noindent{}\raisebox{\baselineskip}[0pt]{\protect\hypertarget{gtTabooSearch}{}}{\textbf{Taboo (Tabu) Search}}: is a \hyperlink{gtMetaheuristic}{{\textit{metaheuristic}}} which is applied to algorithms like \hyperlink{gtSimulatedAnnealing}{{\textit{Simulated Annealing algorithms}}} which prevent the random generation of new options based on some criteria. i.e. the solution has already been checked or some sequence is known to not be permitted.\par{}
\vspace{4pt}}{\hangafter1\relax
\hangindent.3in\relax
\noindent{}\raisebox{\baselineskip}[0pt]{\protect\hypertarget{gtTexts}{}}{\textbf{Text Messages}}: see \hyperlink{gtSMS}{{\textit{Short Message Service}}}\par{}
\vspace{4pt}}{\hangafter1\relax
\hangindent.3in\relax
\noindent{}\raisebox{\baselineskip}[0pt]{\protect\hypertarget{gtTexting}{}}{\textbf{Texting}}: is the process of sending a \hyperlink{gtTexts}{{\textit{text message}}}. Usually via a reduced size keyboard on a mobile phone, or via a keypad and a text expansion tool.\par{}
\vspace{4pt}}{\hangafter1\relax
\hangindent.3in\relax
\noindent{}\raisebox{\baselineskip}[0pt]{\protect\hypertarget{gtTriGraph}{}}{\textbf{Tri-graph (trigraph)}}: \par{}
\vspace{4pt}}{\hangafter1\relax
\hangindent.3in\relax
\noindent{}\raisebox{\baselineskip}[0pt]{\protect\hypertarget{gtTypesetting}{}}{\textbf{Typesetting}}: is the process of preparing a document for publication. Usually involving page layout and information design decisions.\par{}
\vspace{4pt}}{\hangafter1\relax
\hangindent.3in\relax
\noindent{}\raisebox{\baselineskip}[0pt]{\protect\hypertarget{gtTyping}{}}{\textbf{Typing}}: is the process of creating text with a keyboard. In this work I assume a keyboard with a layout which is full size and designed for 10 digit (finger) use. Typing may be semantically differentiated from \hyperlink{gtTexting}{{\textit{texting}}} in that the assumption texting requires a mobile phone which usually do not have a full size keyboard. Both are text input processes.\par{}
\vspace{4pt}}{\hangafter1\relax
\hangindent.3in\relax
\noindent{}\raisebox{\baselineskip}[0pt]{\protect\hypertarget{gtUnderResourcedLanguage}{}}{\textbf{Under-Resourced Language}}: is defined by \hyperlink{rBesacier2014}{Besacier et al. (2014)} a language where some of (if not all) the following aspects are true: lack of a unique writing system or stable orthography, limited presence on the web, lack of linguistic expertise, lack of electronic resources for speech and language processing, such as monolingual corpora, bilingual electronic dictionaries, transcribed speech data, pronunciation dictionaries, vocabulary lists, etc. The synonyms for the same concept are: low-density languages, resource-poor languages, low-data languages, less-resourced languages. The term is credited as being introduced by \hyperlink{rKrauwerSteven2003TheBa}{Krauwer (2003)} and \hyperlink{rBermentVincent2004ME9tho}{Berment (2004)}.\par{}
\vspace{4pt}}{\hangafter1\relax
\hangindent.3in\relax
\noindent{}\raisebox{\baselineskip}[0pt]{\protect\hypertarget{gtUnicodeCharacter}{}}{\textbf{Unicode Character}}: is a character encoded in Unicode and its associated \hyperlink{gtUCD}{{\textit{properties}}}, \hyperlink{gtCodePoint}{{\textit{code point}}}, and if it has one a \hyperlink{gtGlyph}{{\textit{glyph}}}.\par{}
\vspace{4pt}}{\hangafter1\relax
\hangindent.3in\relax
\noindent{}\raisebox{\baselineskip}[0pt]{\protect\hypertarget{gtUCD}{}}{\textbf{Unicode Character Database}}: is a database containing the valid encoded characters in Unicode. What it specifically contains can be found in the \hyperlink{rUnicode11Standard}{Unicode (2018)} documentation \href{http://www.unicode.org/versions/Unicode11.0.0/ch04.pdf\#G124718}{\textcolor[rgb]{0,0,0}{v11 § 4.1}}. In summary the {\hyperlink{vUCD}{{UCD}}} contains the following properties: Name, General Category, Other important general characteristics, Display-related properties, Casing, Numeric values and types, Script and Block, Normalization properties, Age, Boundaries.\par{}
\vspace{4pt}}{\hangafter1\relax
\hangindent.3in\relax
\noindent{}\raisebox{\baselineskip}[0pt]{\protect\hypertarget{gtUTF-8}{}}{\textbf{Unicode Transformation Format – 8-bit {\XLingPaperCharisZSILFontFamily{({\textbf{{\hyperlink{vUTF-8}{{UTF-8}}}}})}}}}: is the default encoding format for the web, and is a variable width encoding format. It uses one to four 8-bit bytes to encode a character. It is capable of encoding all 1,112,064 valid \hyperlink{gtCodePoint}{{\textit{code points}}} in Unicode.\par{}
\vspace{4pt}}{\hangafter1\relax
\hangindent.3in\relax
\noindent{}\raisebox{\baselineskip}[0pt]{\protect\hypertarget{gtUpperCase}{}}{\textbf{Upper Case}}: see \hyperlink{gtLowerCase}{{\textit{lower case}}}.\par{}
\vspace{4pt}}{\hangafter1\relax
\hangindent.3in\relax
\noindent{}\raisebox{\baselineskip}[0pt]{\protect\hypertarget{gtUX}{}}{\textbf{User Experience ({\textbf{{\hyperlink{vUX}{{UX}}}}})}}: is usually a design related term which focuses on the elements needed to make some interaction appealing. It looks at the ecosystem of tasks, rewards, and interaction types. It generally, as a design philosophy term looks beyond the ability of completing a task to how the completer of the task felt while completing the task, and would they choose the same solution again if they had a choice. In some senses {\hyperlink{vUX}{{UX}}} focuses on the user's impressions and capabilities while {\hyperlink{vUI}{{UI}}} focuses on the product's presentation.\par{}
\vspace{4pt}}{\hangafter1\relax
\hangindent.3in\relax
\noindent{}\raisebox{\baselineskip}[0pt]{\protect\hypertarget{gtUI}{}}{\textbf{User Interface ({\textbf{{\hyperlink{vUI}{{UI}}}}})}}: has its grounding in graphic and page layout design. User Interface is the way that a user interacts with an item, as apposed to the experience they undergo or create with the designed item. In computer design, UI has been strongly connected with page layout and visual aesthetics. However, with {\hyperlink{vHLT}{{HLT}}} the interface may be audio/aural or motion based.\par{}
\vspace{4pt}}{\hangafter1\relax
\hangindent.3in\relax
\noindent{}\raisebox{\baselineskip}[0pt]{\protect\hypertarget{gtWritingSystem}{}}{\textbf{Writing System}}: is a superordinate category of a collection of technologies and/or metadata on how an orthography is to be implemented. Figure \hyperlink{fWritingSystemsNoCommnet}{9} visualizes the relationships between \hyperlink{gtScript}{{\textit{scripts}}}, \hyperlink{gtWritingSystem}{{\textit{writing systems}}} and \hyperlink{gtOrthography}{{\textit{orthographies}}} and is taken from \hyperlink{Constable}{Constable (2002)}, whose definition is followed throughout this thesis.\par{}
\vspace{4pt}}{\hangafter1\relax
\hangindent.3in\relax
\noindent{}\raisebox{\baselineskip}[0pt]{\protect\hypertarget{gtWSD}{}}{\textbf{Writing System Descriptions}}: are technical explanations describing sets of glyphs, and how they work together, so that they might be used by an orthography. A writing system description includes things like correspondences between upper and lower case letters, usage rules for casing, punctuation characters, usage rules for punctuation characters, letters, numbers, and characters used in Internet use, with their Unicode code points used in {\hyperlink{vUTF-8}{{UTF-8}}} or UTF-16 encoded documents. A writing system description, seeks to describe how text handling needs to occur on a computer and is more than just an orthography statement where glyphs are associated with a phonemic inventory. Technical descriptions are needed to fully support a language on digital tools like \hyperlink{gtLocale}{{\textit{Locale}}}  profiles, language specific keyboard layouts, and other natural language processing tools\protect\footnote[2]{{\leftskip0pt\parindent1em\raisebox{\baselineskip}[0pt]{\protect\hypertarget{ngWritingSystemDescriptions}{}} Three relevant resources for writing system descriptions are \hyperlink{Albright-Paper}{Albright (2000}; \hyperlink{Albright-MA}{2001)}, and \hyperlink{Hosken}{Hosken (2003)}.}}.\par{}
\vspace{4pt}}{\hangafter1\relax
\hangindent.3in\relax
\noindent{}\raisebox{\baselineskip}[0pt]{\protect\hypertarget{gtZalgo}{}}{\textbf{Zalgo}}: is like \hyperlink{gtdiacritism}{{\textit{d̤͓̄̈iac̤̰r͚̮̤͓it̤̱̆̈ĭ̤̺̈c̃̈̃̈ism}}}, it involves making the text look scary usually by adding diacritics to it. The idea is that the text is scary because the norms of visualization are not being adhered to by the diacritics and the eye is required to move up or down relative to the normal reading \hyperlink{gtSaccade}{{\textit{saccade}}} movement.\par{}
\vspace{4pt}}}}\clearpage
\thispagestyle{bodyfirstpage}\markboth{Chronological bibliography of Eastern Dan texts, descriptions, and minor mentions}{Chronological bibliography of Eastern Dan texts, descriptions, and minor mentions}
\XLingPaperaddtocontents{EasternDanBibliography}{\vspace*{.65in}\XLingPaperneedspace{3\baselineskip}\noindent
{\centering
APPENDIX \raisebox{\baselineskip}[0pt]{\protect\hypertarget{EasternDanBibliography}{}}\raisebox{\baselineskip}[0pt]{\pdfbookmark[1]{C  Chronological bibliography of Eastern Dan texts, descriptions, and minor mentions}{EasternDanBibliography}}C\protect\\}}\par{}
{\XLingPaperneedspace{3\baselineskip}\noindent
{\centering
Chronological bibliography of Eastern Dan texts, descriptions, and minor mentions\protect\\}}\par{}
\vspace{16pt}\indent This appendix has three sections (1) a short note about Eastern Dan corpora. (2) A short note about languages with similar writing systems. (3) A bibliography of works about Dan.\par{}{\vspace{15pt}\XLingPaperneedspace{3\baselineskip}\noindent
\fontsize{13}{15.6}\selectfont \textbf{{\noindent
\raisebox{\baselineskip}[0pt]{\pdfbookmark[2]{{C.1 } Eastern Dan Corpora}{sEasternDanCorpora}}\raisebox{\baselineskip}[0pt]{\protect\hypertarget{sEasternDanCorpora}{}}{C.1 }Eastern Dan Corpora}}\markboth{Eastern Dan Corpora}{Chronological bibliography of Eastern Dan texts, descriptions, and minor mentions}\XLingPaperaddtocontents{sEasternDanCorpora}}\par{}
\penalty10000\vspace{10pt}\penalty10000\indent As a clarifying note the corpora used in this experiment is different than the one presented online as \hyperlink{rVydrindnjECorpus}{Vydrin (2018)}. \hyperlink{rVydrindnjECorpus}{Vydrin (2018)} became available after work on this thesis started. It is an exciting development to see a half a million word corpus online for an under-resourced language. Working with under-resourced language data is a challenge for many reasons. One not to be ignored is the issue of intellectual property rights. Within the academy there is great deal of variation concerning the morality and applicability of copyright law. But the fact remains that most countries have signed the Bern convention and flowing agreements which legally acknowledge the rights of content creators. However, these laws also endorse the "work-for-hire" doctrine. Additionally, western ethical frameworks usually dictate that workers should be compensated, both as part of academic projects and as part of NGO projects. This creates a feeding relationship where intellectual property is rarely overtly left in the ownership of native language writers. The relationship between writing, compensation, and content ownership is frequently left undressed and not made clear to minority languages writers. This means that when someone, even a native language speaker, produces a work the default is that the work's copyright falls to the organization doing the hiring. There are several exemptions which are out of scope of the current discussion. But the impact though is that, in the overwhelming majority of cases where NGOs are involved in language development and the process of Bible translation, the generated content legally belongs to the NGO. These NGOs currently retain the copyright on large portions of the texts in minority languages around the world\protect\footnote[1]{{\leftskip0pt\parindent1em\raisebox{\baselineskip}[0pt]{\protect\hypertarget{nOrphanedWorks}{}} Whether ethical or not, having a clear copyright owner is better than having an unstated or unknown copyright holder. Working with unclaimed copyright materials, "orphaned works" has inherent risks.}}. Several notable studies using Bible translations as parallel corpora have noted the difficulty of legally using the Bible as a corpus. Obtaining permission to include Bible content is a challenge. Not only are the copyright holders slow to respond, but often the response is that they limit the portion of the text which can be included in a corpus. This is in part why my parallel corpus is so small.\par{}\indent At the time of this writing the works presented following table \hyperlink{ntVadrinCorpus}{36} are included in \hyperlink{rVydrindnjECorpus}{Vydrin (2018)}. These works are not replicated in appendix section \hyperlink{sWorksDescribing}{C.2}. Table \hyperlink{ntVadrinCorpus}{36} indicates the content composition of the corpus with a large part of the corpus coming from translated materials.\par{}\vspace{11pt plus 2pt minus 1pt}\XLingPaperneedspace{3\baselineskip}\protect\hypertarget{ntVadrinCorpus}{}\XLingPaperaddtocontents{ntVadrinCorpus}{\protect\raggedright{\singlespacing
{Table }{36.}{  The half a million word corpus.\\}}}\vspace{0pt}{\singlespacing
\hspace*{.25in}{
\XLingPaperminmaxcellincolumn{sources}{\XLingPapermincola}{\textbf{Word count of contributing sources}}{\XLingPapermaxcola}{+0\tabcolsep}
\XLingPaperminmaxcellincolumn{corpus}{\XLingPapermincolb}{\textbf{Percentage of corpus}}{\XLingPapermaxcolb}{+0\tabcolsep}
\XLingPaperminmaxcellincolumn{368093}{\XLingPapermincola}{368093}{\XLingPapermaxcola}{+0\tabcolsep}
\XLingPaperminmaxcellincolumn{82.15\%}{\XLingPapermincolb}{82.15\%}{\XLingPapermaxcolb}{+0\tabcolsep}
\XLingPaperminmaxcellincolumn{23618}{\XLingPapermincola}{23618}{\XLingPapermaxcola}{+0\tabcolsep}
\XLingPaperminmaxcellincolumn{5.27\%}{\XLingPapermincolb}{5.27\%}{\XLingPapermaxcolb}{+0\tabcolsep}
\XLingPaperminmaxcellincolumn{16441}{\XLingPapermincola}{16441}{\XLingPapermaxcola}{+0\tabcolsep}
\XLingPaperminmaxcellincolumn{3.67\%}{\XLingPapermincolb}{3.67\%}{\XLingPapermaxcolb}{+0\tabcolsep}
\XLingPaperminmaxcellincolumn{7265}{\XLingPapermincola}{7265}{\XLingPapermaxcola}{+0\tabcolsep}
\XLingPaperminmaxcellincolumn{1.62\%}{\XLingPapermincolb}{1.62\%}{\XLingPapermaxcolb}{+0\tabcolsep}
\XLingPaperminmaxcellincolumn{7520}{\XLingPapermincola}{7520}{\XLingPapermaxcola}{+0\tabcolsep}
\XLingPaperminmaxcellincolumn{1.68\%}{\XLingPapermincolb}{1.68\%}{\XLingPapermaxcolb}{+0\tabcolsep}
\XLingPaperminmaxcellincolumn{3599}{\XLingPapermincola}{3599}{\XLingPapermaxcola}{+0\tabcolsep}
\XLingPaperminmaxcellincolumn{0.80\%}{\XLingPapermincolb}{0.80\%}{\XLingPapermaxcolb}{+0\tabcolsep}
\XLingPaperminmaxcellincolumn{1698}{\XLingPapermincola}{1698}{\XLingPapermaxcola}{+0\tabcolsep}
\XLingPaperminmaxcellincolumn{0.38\%}{\XLingPapermincolb}{0.38\%}{\XLingPapermaxcolb}{+0\tabcolsep}
\XLingPaperminmaxcellincolumn{19833}{\XLingPapermincola}{19833}{\XLingPapermaxcola}{+0\tabcolsep}
\XLingPaperminmaxcellincolumn{4.43\%}{\XLingPapermincolb}{4.43\%}{\XLingPapermaxcolb}{+0\tabcolsep}
\XLingPaperminmaxcellincolumn{448067}{\XLingPapermincola}{448067}{\XLingPapermaxcola}{+0\tabcolsep}
\XLingPaperminmaxcellincolumn{100.00\%}{\XLingPapermincolb}{100.00\%}{\XLingPapermaxcolb}{+0\tabcolsep}
\setlength{\XLingPaperavailabletablewidth}{433.62pt}
\setlength{\XLingPapertableminwidth}{\XLingPapermincola+\XLingPapermincolb}
\setlength{\XLingPapertablemaxwidth}{\XLingPapermaxcola+\XLingPapermaxcolb}
\XLingPapercalculatetablewidthratio{}
\XLingPapersetcolumnwidth{\XLingPapercolawidth}{\XLingPapermincola}{\XLingPapermaxcola}{-0\tabcolsep}
\XLingPapersetcolumnwidth{\XLingPapercolbwidth}{\XLingPapermincolb}{\XLingPapermaxcolb}{-2\tabcolsep}\singlespacing\vspace*{-3\baselineskip}
\begin{longtable}
[l]{@{}p{\XLingPapercolawidth}p{\XLingPapercolbwidth}@{}}\toprule\multicolumn{1}{@{}p{\XLingPapercolawidth}}{\textbf{Word count of contributing sources}}&\multicolumn{1}{p{\XLingPapercolbwidth}@{}}{\textbf{Percentage of corpus}}\\%
\midrule\endhead \multicolumn{1}{@{}>{\raggedleft}p{\XLingPapercolawidth}}{368093}&\multicolumn{1}{p{\XLingPapercolbwidth}@{}}{82.15\%}\\%
\multicolumn{1}{@{}>{\raggedleft}p{\XLingPapercolawidth}}{23618}&\multicolumn{1}{p{\XLingPapercolbwidth}@{}}{5.27\%}\\%
\multicolumn{1}{@{}>{\raggedleft}p{\XLingPapercolawidth}}{16441}&\multicolumn{1}{p{\XLingPapercolbwidth}@{}}{3.67\%}\\%
\multicolumn{1}{@{}>{\raggedleft}p{\XLingPapercolawidth}}{7265}&\multicolumn{1}{p{\XLingPapercolbwidth}@{}}{1.62\%}\\%
\multicolumn{1}{@{}>{\raggedleft}p{\XLingPapercolawidth}}{7520}&\multicolumn{1}{p{\XLingPapercolbwidth}@{}}{1.68\%}\\%
\multicolumn{1}{@{}>{\raggedleft}p{\XLingPapercolawidth}}{3599}&\multicolumn{1}{p{\XLingPapercolbwidth}@{}}{0.80\%}\\%
\multicolumn{1}{@{}>{\raggedleft}p{\XLingPapercolawidth}}{1698}&\multicolumn{1}{p{\XLingPapercolbwidth}@{}}{0.38\%}\\%
\multicolumn{1}{@{}>{\raggedleft}p{\XLingPapercolawidth}}{19833}&\multicolumn{1}{p{\XLingPapercolbwidth}@{}}{4.43\%}\\%
\multicolumn{1}{@{}>{\raggedleft}p{\XLingPapercolawidth}}{448067}&\multicolumn{1}{p{\XLingPapercolbwidth}@{}}{100.00\%}\\\bottomrule%
\end{longtable}
}
}{\parskip .5pt plus 1pt minus 1pt

\vspace{\baselineskip}

{\setlength{\XLingPapertempdim}{\XLingPaperbulletlistitemwidth+\parindent{}}\leftskip\XLingPapertempdim\relax
\interlinepenalty10000
\XLingPaperlistitem{\parindent{}}{\XLingPaperbulletlistitemwidth}{•}{Baba, Tiémoko Sebastien \& Daniel Késsé (trans.). {\textit{Naɔ ˗së ʼö ˗gban Yesu Klisi ˗bha ʼö}} \textsquarebracketleft{}{\textit{Nouveau Testament}}\textsquarebracketright{}. Abidjan. 368093 words.}}
{\setlength{\XLingPapertempdim}{\XLingPaperbulletlistitemwidth+\parindent{}}\leftskip\XLingPapertempdim\relax
\interlinepenalty10000
\XLingPaperlistitem{\parindent{}}{\XLingPaperbulletlistitemwidth}{•}{9Baba, Tiémoko Sébastien. {\textit{˗Kwa ʼwɔn zii pö.}} \textsquarebracketleft{}Contes dan ˮgwɛɛtaawʋ.\textsquarebracketright{} vol. 2 (non-publié). 23618 words.}}
{\setlength{\XLingPapertempdim}{\XLingPaperbulletlistitemwidth+\parindent{}}\leftskip\XLingPapertempdim\relax
\interlinepenalty10000
\XLingPaperlistitem{\parindent{}}{\XLingPaperbulletlistitemwidth}{•}{Baba, Tiémoko Sebastien. {\textit{Mɛ faan dɔ -wʋ}} \textsquarebracketleft{}Livre de sensibilisation\textsquarebracketright{}. Man: La LIGUE pour la Promotion de la langue Dan. 16441 words.}}
{\setlength{\XLingPapertempdim}{\XLingPaperbulletlistitemwidth+\parindent{}}\leftskip\XLingPapertempdim\relax
\interlinepenalty10000
\XLingPaperlistitem{\parindent{}}{\XLingPaperbulletlistitemwidth}{•}{Kessé, Mongnan Alphose. 2007. {\textit{ˮSanni kö ꞊dhɔtrɔɔ ˗yö nu}} \textsquarebracketleft{}{\textit{En attendant l’arrivée du médecin}}\textsquarebracketright{}. Abidjan. 48p., 7265 words}}
{\setlength{\XLingPapertempdim}{\XLingPaperbulletlistitemwidth+\parindent{}}\leftskip\XLingPapertempdim\relax
\interlinepenalty10000
\XLingPaperlistitem{\parindent{}}{\XLingPaperbulletlistitemwidth}{•}{Kessé, Mongnan Alphonse. 2007. {\textit{ʼPë nu ˮyua ʼka}} \textsquarebracketleft{}{\textit{Ce qui apporte la maladie}}\textsquarebracketright{}. Abidjan. 7520 words.}}
{\setlength{\XLingPapertempdim}{\XLingPaperbulletlistitemwidth+\parindent{}}\leftskip\XLingPapertempdim\relax
\interlinepenalty10000
\XLingPaperlistitem{\parindent{}}{\XLingPaperbulletlistitemwidth}{•}{Kessé, Mongnan Alphonse. ˗Dhuangdhe \textsquarebracketleft{}Le miroir\textsquarebracketright{}. 3599 words.}}
{\setlength{\XLingPapertempdim}{\XLingPaperbulletlistitemwidth+\parindent{}}\leftskip\XLingPapertempdim\relax
\interlinepenalty10000
\XLingPaperlistitem{\parindent{}}{\XLingPaperbulletlistitemwidth}{•}{꞊Kesɩ ꞊Mɔyan Dhifɔnngsü (traduction). {\textit{Declaration des droits de l'homme}}). 1698 words.}}
{\setlength{\XLingPapertempdim}{\XLingPaperbulletlistitemwidth+\parindent{}}\leftskip\XLingPapertempdim\relax
\interlinepenalty10000
\XLingPaperlistitem{\parindent{}}{\XLingPaperbulletlistitemwidth}{•}{Kluubali, Misiölinö (traduction par ꞊Kesɩ ꞊Mɔyan Dhifɔnngsü). {\textit{Medɔɔ ˗bha -bin ʼgü ˗wɔn ˗nu}} \textsquarebracketleft{}{\textit{L’histoire de Medor}}\textsquarebracketright{}. Abidjan: Edilis-Éditions Livre Sud. 19833 words.}}
\vspace{\baselineskip}
}{\vspace{15pt}\XLingPaperneedspace{3\baselineskip}\noindent
\fontsize{13}{15.6}\selectfont \textbf{{\noindent
\raisebox{\baselineskip}[0pt]{\pdfbookmark[2]{{C.2 } Dan}{sWorksDescribing}}\raisebox{\baselineskip}[0pt]{\protect\hypertarget{sWorksDescribing}{}}{C.2 }Dan}}\markboth{Dan}{Chronological bibliography of Eastern Dan texts, descriptions, and minor mentions}\XLingPaperaddtocontents{sWorksDescribing}}\par{}
\penalty10000\vspace{10pt}\penalty10000\indent This is a list of known works which either describe the linguistics, the linguistic genetic positioning, writing system, writing process, or language use of Dan. References to Dan as an exemplar are also included (Dan is not required to be the primary focus of the referenced work). Works including primary language resources are also included (except those presented previously in section \hyperlink{EasternDanBibliography}{C}). The list in this is broken down into four sections: (1) a section for works explicitly or exclusively mentioning Eastern Dan as opposed to other varieties of Dan. (2) Works where the author discusses both Eastern and Western Dan or where the author does not distinguish between Eastern Dan and Western Dan. (3) A third for works explicitly mentioning Western Dan. And lastly (4) a section for works mentioning Dan as it is used in Liberia. Some works mention Dan in {\textit{Côte d'Ivoire}} as Gio, but my understanding is that Gio generally refers to the language as it is used in Liberia\protect\footnote[2]{{\leftskip0pt\parindent1em\raisebox{\baselineskip}[0pt]{\protect\hypertarget{nSlingerCorrection}{}} As a point of clarification and correction: \hyperlink{rSinglerJohnV1990}{Singler (1990)} suggests that {\XLingPaperCharisZSILFontFamily{\textit{MacKinnon, C. 1977. "The Dialect of Gio." Studia Iranica 6:221 -47.}}} is about the Liberian language Gio. However, it is really about an Iranian-Persian dialect.}}. \hyperlink{rNazam1983}{Halaoui et al. (1983:16)} present a nice list of how the language has been referred to through history by different authors up to the time of their publication. Eastern Dan was the focus of this study as such more effort was put forward to discover the pertinent literature specific to Eastern Dan. The other dialects of Dan were not in focus and so I do not assume that their literature review is exhaustive.\protect\footnote[3]{{\leftskip0pt\parindent1em\raisebox{\baselineskip}[0pt]{\protect\hypertarget{nNewspapersInDan}{}} Four topics are not included in this bibliography: (1) newspaper articles or the stories in them that were written through literacy work. (2) Preprint version of papers often significantly differ from final publications (not just in pagination but also in content). Many of the Valentin Vydrin's preprints are available online through various outlets. There is no attempt to list these, as deem them tenative and of an inferior nature to their final published versions. (3) There seems to be a large anthropology and art literature about Dan and other west Mande masks. (4) It is a challenge to find the correct attribution of Christian materials. Not only are various Chrisitan works attributed to different organizations, but also there is a myrid of inter-related media types (text, audio and video, with or without background sound tracks, all hosted at a variety of distribution outlets. The most notable of these outlets are: \href{https://joshuaproject.net/languages/dnj}{\textcolor[rgb]{0,0,0}{https://joshuaproject.net/languages/dnj}}, \href{https://www.scriptureearth.org}{\textcolor[rgb]{0,0,0}{https://www.scriptureearth.org}}, and \href{http://globalrecordings.net/en/language/dnj}{\textcolor[rgb]{0,0,0}{http://globalrecordings.net/en/language/dnj}}). Works from these categories are not indicated in this list.}} Each section is presented in chronological order as published, newest first. Reprints are also included in the timeline.\par{}{\vspace{10pt}\XLingPaperneedspace{3\baselineskip}\noindent
\fontsize{13}{15.6}\selectfont \textit{{\noindent
\raisebox{\baselineskip}[0pt]{\pdfbookmark[3]{{C.2.1 } Eastern Dan}{sEasternDan}}\raisebox{\baselineskip}[0pt]{\protect\hypertarget{sEasternDan}{}}{C.2.1 }Eastern Dan}}\markboth{Eastern Dan}{Chronological bibliography of Eastern Dan texts, descriptions, and minor mentions}\XLingPaperaddtocontents{sEasternDan}}\par{}
\penalty10000\vspace{10pt}\penalty10000\noindent{}\hangindent.25in\relax
\hangafter1\relax
{Paterson, Hugh J. III.  }{2019.  }\textit{A text input analysis of Eastern Dan.  }Grand Forks, North Dakota: University of North Dakota dissertation.  \par
\vspace{11pt plus 2pt minus 1pt}\noindent{}\hangindent.25in\relax
\hangafter1\relax
{Roberts, David,  }{Editor.  }{(submitted).  }\textit{Tone orthography and reading fluency: the voice of evidence in ten Niger-Congo languages.  }Amsterdam: John Benjamins.\par
\vspace{11pt plus 2pt minus 1pt}\noindent{}\hangindent.25in\relax
\hangafter1\relax
{Roberts, David, Ginger Boyd, Johannes Merz \& Valentin Vydrin.  }{(forthcoming).  }\textup{Quantifying written ambiguities in tone languages: a comparative study of Elip, Mbelime and Eastern Dan.  }\textit{Language Documentation \& Conservation }\textup{.  }\par
\vspace{11pt plus 2pt minus 1pt}\noindent{}\hangindent.25in\relax
\hangafter1\relax
{Roberts, David \& Valentin Vydrin.  }{(submitted).  }Chapter 10: Eastern Dan.  In Roberts, David, \textit{Tone orthography and reading fluency: the voice of evidence in ten Niger-Congo languages. }John Benjamins.\par
\vspace{11pt plus 2pt minus 1pt}\noindent{}\hangindent.25in\relax
\hangafter1\relax
{Vydrin, Valentin.  }{2018.  }Corpus dan de l'Est.    \href{http://cormand.huma-num.fr/dan/}{\textcolor[rgb]{0,0,0}{http://cormand.huma-num.fr/dan/}}  (2. December 2018.)\par
\vspace{11pt plus 2pt minus 1pt}\noindent{}\hangindent.25in\relax
\hangafter1\relax
{Roberts, David, Valentin Vydrin \& Dana Basnight-Brown.  }{2018.  }Marking tone with punctuation: an orthography experiment in Eastern Dan.  /gʁafematik/ - Graphemics in the 21st century.  Brest, France, 14 June 2018.  \href{https://www.youtube.com/watch?v=W5Pir6cPoQs}{\textcolor[rgb]{0,0,0}{https://www.youtube.com/​watch?v=W5Pir6cPoQs}}\par
\vspace{11pt plus 2pt minus 1pt}\noindent{}\hangindent.25in\relax
\hangafter1\relax
{Roberts, David.  }{2018.  }\textup{Dialogue on dialect standardization.  }\textit{Writing Systems Research }\textup{10}(1). \textup{68-71.  }  doi:\href{http://doai.io/10.1080/17586801.2018.1491440}{10.1080/17586801.2018.1491440}\par
\vspace{11pt plus 2pt minus 1pt}\noindent{}\hangindent.25in\relax
\hangafter1\relax
{Vydrin, Valentin.  }{2017.  }Quantifiers in Dan-Gweetaa (South Mande).  In Denis Paperno \& Edward L. Keenan, eds. \textit{Handbook of quantifiers in natural language, }2, 203-80. Studies in linguistics and philosophy 97.  Cham: Springer.  doi:\href{http://doai.io/10.1007/978-3-319-44330-0\_5}{10.1007/978-3-319-44330-0\_5}\par
\vspace{11pt plus 2pt minus 1pt}\noindent{}\hangindent.25in\relax
\hangafter1\relax
{Wycliffe Bible Translators, Inc.  }{2016.  }\textit{{\textit{NAƆ ‑SË 'SËËDHƐ}} {\textit{Le Nouveau Testament en Dan "Gwєєtaawʋ (Yacouba) de Côte d’Ivoire}}.  }https://www.scriptureearth.org: Wycliffe Bible Translators, Inc.\par
\vspace{11pt plus 2pt minus 1pt}\noindent{}\hangindent.25in\relax
\hangafter1\relax
{Vydrin, Valentin.  }{2016.  }Tonal inflection in Mande languages: the cases of Bamana and Dan-Gwɛɛtaa.  In Enrique L. Palancar \& Jean Léo Léonard, eds. \textit{Tone and Inflection: New facts and new perspectives, }83-105. Trends in Linguistics. Studies and Monographs \textsquarebracketleft{}TiLSM\textsquarebracketright{} 296.  Berlin, Boston: De Gruyter — Mouton.  doi:\href{http://doai.io/10.1515/9783110452754-005}{10.1515/9783110452754-005}\par
\vspace{11pt plus 2pt minus 1pt}\noindent{}\hangindent.25in\relax
\hangafter1\relax
{Osborn, Don.  }{2014-09-22.  }"Pamɛbhamɛ" article on ebola.  Beyond Niamey.    \href{http://niamey.blogspot.com/2014/09/pambham-article-on-ebola.html}{\textcolor[rgb]{0,0,0}{http://niamey.blogspot.com/2014/09/pambham-article-on-ebola.html}}  (7. November 2018.)\par
\vspace{11pt plus 2pt minus 1pt}\noindent{}\hangindent.25in\relax
\hangafter1\relax
{Anonymous.  }{2012.  }\textit{Projet Dan-est: alphabétisation, traduction, recherches. Rapport annuel d'activités du staff.  }Man, Côte d'Ivoire: Programme d'alphabétisation en langue dan.\par
\vspace{11pt plus 2pt minus 1pt}\noindent{}\hangindent.25in\relax
\hangafter1\relax
{Vydrin, Valentin \textsquarebracketleft{}Выдрин, Валентин Феодосьевич\textsquarebracketright{}.  }{2011.  }{\textit{Идиом дан-гуэта}} \textsquarebracketleft{}{\textit{Idiom Dan-Gwèètaa}}\textsquarebracketright{}.  In В. Б. Касевич \textsquarebracketleft{}Vadim Kasevich\textsquarebracketright{}, \textit{Грамматика и семантика восточного текста: Квантитативные характеристики \textsquarebracketleft{}Grammar and semantics of the oriental text: Quantitative characteristics\textsquarebracketright{}, }100-13. Санкт-Петербург, Википедия \textsquarebracketleft{}St. Petersburg, Russia\textsquarebracketright{}: Санкт-Петербургский государственный университет \textsquarebracketleft{}St. Petersburg State University\textsquarebracketright{}.\par
\vspace{11pt plus 2pt minus 1pt}\noindent{}\hangindent.25in\relax
\hangafter1\relax
{Vydrin, Valentin.  }{2011.  }{\textit{Déclinaison nominale en dan-gwèètaa (groupe mandé-sud, Côte-d’Ivoire)}}.  \textit{Faits de langues, }233-58.  Paris, France: Ophrys.\par
\vspace{11pt plus 2pt minus 1pt}\noindent{}\hangindent.25in\relax
\hangafter1\relax
{Zogbo, Lynell.  }{2011.  }\textup{NÉCROLOGIE: Kessegbeu Mongnan Alphonse.  }\textit{Le Sycomore }\textup{5}(1). \textup{50-51.  }  \href{http://www.ubs-translations.org/sycomore/autres\_numeros/syc512011/}{\textcolor[rgb]{0,0,0}{http://www.ubs-translations.org/sycomore/autres\_numeros/syc512011/}}\par
\vspace{11pt plus 2pt minus 1pt}\noindent{}\hangindent.25in\relax
\hangafter1\relax
{Vydrin, Valentin \textsquarebracketleft{}Выдрин, Валентин Феодосьевич\textsquarebracketright{}.  }{2009.  }\textup{{\textit{Морфология прилагательных в дан-гуэта (южные манде)}} \textsquarebracketleft{}Morphology of adjectives in Dan-Gweetaa (Southern Mande)\textsquarebracketright{}.  }\textit{Вестник Санкт-Петербургского университета: Востоковедение Африканистика \textsquarebracketleft{}Bulletin of St. Petersburg University: Oriental \& African Studies\textsquarebracketright{} }\textup{13}(2). \textup{120-40.  }  \href{https://cyberleninka.ru/article/n/morfologiya-prilagatelnyh-v-dan-gueta-yuzhnye-mande}{\textcolor[rgb]{0,0,0}{https://cyberleninka.ru/article/n/morfologiya-prilagatelnyh-v-dan-gueta-yuzhnye-mande}}\par
\vspace{11pt plus 2pt minus 1pt}\noindent{}\hangindent.25in\relax
\hangafter1\relax
{Vydrin, Valentin \textsquarebracketleft{}Выдрин, Валентин Феодосьевич\textsquarebracketright{}.  }{2008.  }{\textit{Стратегии релятивизации в языках манде (на примере дан-гуэта и бамана)}} \textsquarebracketleft{}Strategies of relativization in Mande languages: the cases of Dan-Gwèètaa and Bamana\textsquarebracketright{}.  In Валентин Феодосьевич Выдрин \textsquarebracketleft{}Valentin Vydrin\textsquarebracketright{}, \textit{Африканский сборник – 2007 \textsquarebracketleft{}African Collection - 2007\textsquarebracketright{}, }320-30. Санкт-Петербург, Википедия \textsquarebracketleft{}St. Petersburg, Russia\textsquarebracketright{}: Наука \textsquarebracketleft{}Science\textsquarebracketright{}.\par
\vspace{11pt plus 2pt minus 1pt}\noindent{}\hangindent.25in\relax
\hangafter1\relax
{Kességbeu, Mongnan Alphonse.  }{2008.  }{\textit{Recensement des lecteurs Dan de l’Est par village et le nombre de ceux qui savent en dan}}.  Man, Côte d'Ivoire: Unpublished, ms.\par
\vspace{11pt plus 2pt minus 1pt}\noindent{}\hangindent.25in\relax
\hangafter1\relax
{Vydrine \textsquarebracketleft{}Vydrin\textsquarebracketright{}, Valentin \& Mongnan Alphonse Kességbeu.  }{2008.  }\textit{{\textit{Dictionnaire Dan-Français (dan de l'Est) avec une esquisse de grammaire du dan de l'Est et un index français-dan}}.  }1ère edn.  St Pétersbourg, Russia: Musée d'anthropologie et d'ethnographie, Académie des sciences de la Russie — Nestor-Istoria.  \href{https://halshs.archives-ouvertes.fr/halshs-00715560}{\textcolor[rgb]{0,0,0}{https://halshs.archives-ouvertes.fr/halshs-00715560}}\par
\vspace{11pt plus 2pt minus 1pt}\noindent{}\hangindent.25in\relax
\hangafter1\relax
{Vydrine \textsquarebracketleft{}Vydrin\textsquarebracketright{}, Valentin.  }{2007.  }\textup{{\textit{Les adjectifs en dan-gwèètaa}}.  }\textit{Mandenkan }\textup{43. }\textup{77-103.  }  \href{http://llacan.vjf.cnrs.fr/PDF/Mandenkan43/vydrine-T.pdf}{\textcolor[rgb]{0,0,0}{http://llacan.vjf.cnrs.fr/PDF/Mandenkan43/vydrine-T.pdf}}\par
\vspace{11pt plus 2pt minus 1pt}\noindent{}\hangindent.25in\relax
\hangafter1\relax
{Vydrin, Valentin \textsquarebracketleft{}Выдрин, Валентин Феодосьевич\textsquarebracketright{}.  }{2005.  }Терминология родства и свойства в дан-гуэта (Кот д’Ивуар) \textsquarebracketleft{}Kinship terminology in Dan-Gweetaa (Côte d’Ivoire)\textsquarebracketright{}.  In В. Ф. Выдрин, Д. И. Раскин, В. Г. Узунова, С. Б. Чернецов \& Ю. К. Чистов \textsquarebracketleft{}Valentin Vydrin, David I. Raskin, Valentina G. Uzunova, Sevir B. Chernetsov \& Juruj K. Chistov\textsquarebracketright{}, \textit{Ad hominem. Памяти Николая Гиренко \textsquarebracketleft{}In memoria Nikolaj Girenko\textsquarebracketright{}, }41-66. Санкт-Петербург, Википедия \textsquarebracketleft{}St. Petersburg, Russia\textsquarebracketright{}: Музей антропологии и этнографии РАН (Кунсткамера) \textsquarebracketleft{}Museum of Anthropology and Ethnography of the Russian Academy of Sciences (Kunst Kamera)\textsquarebracketright{}.\par
\vspace{11pt plus 2pt minus 1pt}\noindent{}\hangindent.25in\relax
\hangafter1\relax
{Vydrine \textsquarebracketleft{}Vydrin\textsquarebracketright{}, Valentin.  }{2005.  }\textup{{\textit{Quelques recommandations méthodologiques concernant la description des langues mandé-sud}}.  }\textit{Mandenkan: Bulletin d'études linguistiques MANDE }\textup{41. }\textup{1-22.  }  \href{http://llacan.vjf.cnrs.fr/PDF/Mandenkan41/41vydr.pdf}{\textcolor[rgb]{0,0,0}{http://llacan.vjf.cnrs.fr/PDF/Mandenkan41/41vydr.pdf}}\par
\vspace{11pt plus 2pt minus 1pt}\noindent{}\hangindent.25in\relax
\hangafter1\relax
{Anonymous.  }{2002.  }\textit{{\textit{Rapport sur le cours de formation de formateurs/superviseurs pour le project dan du 27 au 30 août 2002 au college catholique “Les martyres de l’Ouganda”}}.  }Man, Côte d'Ivoire: Programme d'alphabétisation en langue dan.\par
\vspace{11pt plus 2pt minus 1pt}\noindent{}\hangindent.25in\relax
\hangafter1\relax
{Bolli, Margrit \& Eva Flik.  }{2000.  }\textit{{\textit{Cours-éclair de lecture pour les lecteurs du français apprenant à lire le dan gwɛɛtaawo}} (Quick reading course for readers of French learning to reading Eastern Dan).  }Abidjan, Ivory Coast: SIL International.\par
\vspace{11pt plus 2pt minus 1pt}\noindent{}\hangindent.25in\relax
\hangafter1\relax
{Wycliffe Bible Translators, Inc.  }{1995.  }\textit{{\textit{NAƆ ‑SË 'SËËDHƐ}} {\textit{Le Nouveau Testament en Dan "Gwєєtaawʋ (Yacouba) de Côte d’Ivoire}}.  }2nd Printing, 1st edn.  Abidjan, Côte d’Ivoire: Société Biblique Internationale en coopération avec l’Association Ivoirienne pour la Traduction de la Bible.\par
\vspace{11pt plus 2pt minus 1pt}\noindent{}\hangindent.25in\relax
\hangafter1\relax
{Bolli, Margrit \& Eva Flik.  }{1994.  }\textit{{\textit{Cours-éclair de lecture pour les lecteurs du français apprenant à lire le dan gwɛɛtaawo}} (Quick reading course for readers of French learning to reading Eastern Dan).  }Abidjan, Ivory Coast: Société Internationale de Linguistique (SIL International).  \href{https://www.sil.org/resources/archives/34670}{\textcolor[rgb]{0,0,0}{https://www.sil.org/resources/archives/34670}}\par
\vspace{11pt plus 2pt minus 1pt}\noindent{}\hangindent.25in\relax
\hangafter1\relax
{Baba, Tiémoko Sébastien, Jacob Déli Tiémoko, Margrit Bolli \& Eva Flik.  }{1994.  }\textit{=Danwʋ 'sëëdhɛ -wʋ pö -kɔ "gwɛɛtaawʋ (syllabaire dan "gwɛɛtaawʋ).  }Abidjan, Ivory Coast: SIL International.\par
\vspace{11pt plus 2pt minus 1pt}\noindent{}\hangindent.25in\relax
\hangafter1\relax
{Wycliffe Bible Translators, Inc.  }{1991.  }\textit{{\textit{NAƆ ‑SË 'SËËDHƐ}} {\textit{Le Nouveau Testament en Dan "Gwєєtaawʋ (Yacouba) de Côte d’Ivoire}}.  }1st edn.  Abidjan, Côte d’Ivoire: Société Biblique Internationale en coopération avec l’Association Ivoirienne pour la Traduction de la Bible.\par
\vspace{11pt plus 2pt minus 1pt}\noindent{}\hangindent.25in\relax
\hangafter1\relax
{Bolli, Margrit \& Eva Flik.  }{1982.  }\textit{Guide d’orthographe pour la langue dan (dialecte gwɛtaawo).  }Abidjan, Ivory Coast: Société Internationale de Linguistique (SIL International).  \href{https://www.sil.org/resources/archives/34713}{\textcolor[rgb]{0,0,0}{https://www.sil.org/resources/archives/34713}}\par
\vspace{11pt plus 2pt minus 1pt}\noindent{}\hangindent.25in\relax
\hangafter1\relax
{SIL International (Bolli, Margrit \& Eva Flik).  }{1982.  }\textit{Guide d'orthographe pour la langue dan (dialecte gwetaawo).  }Abidjan, Ivory Coast: Société Internationale de Linguistique (SIL International).  \href{https://www.sil.org/resources/archives/34713}{\textcolor[rgb]{0,0,0}{https://www.sil.org/resources/archives/34713}}\par
\vspace{11pt plus 2pt minus 1pt}\noindent{}\hangindent.25in\relax
\hangafter1\relax
{Baba, Tiémoko Sébastien.  }{1978.  }\textit{Yaobhaa -wo bhe pe -se -ya ʼgu ({\textit{Receuil de contes yacouba, ʼGwetaa -wo}}).  }Abidjan, Ivory Coast: Société Internationale de Linguistique (SIL International).  \href{https://www.sil.org/resources/archives/34532}{\textcolor[rgb]{0,0,0}{https://www.sil.org/resources/archives/34532}}\par
\vspace{11pt plus 2pt minus 1pt}\noindent{}\hangindent.25in\relax
\hangafter1\relax
{Doneux, Jean L.  }{1968.  }\textit{Esquisse grammaticale du Dan.  }(Documents linguistiques. № 15.) Dakar, Senegal: Université de Dakar.\par
\vspace{11pt plus 2pt minus 1pt}\noindent{}\hangindent.25in\relax
\hangafter1\relax
{Vendeix, M.  }{1924.  }\textup{Ethnographie du cercle de Man.  }\textit{Revue d'Ethnographie et des Traditions Populaires }\textup{5. }\textup{149-169, 287-294.  }\par
\vspace{11pt plus 2pt minus 1pt}{\vspace{10pt}\XLingPaperneedspace{3\baselineskip}\noindent
\fontsize{13}{15.6}\selectfont \textit{{\noindent
\raisebox{\baselineskip}[0pt]{\pdfbookmark[3]{{C.2.2 } Both Eastern and Western Dan}{sEandWDan}}\raisebox{\baselineskip}[0pt]{\protect\hypertarget{sEandWDan}{}}{C.2.2 }Both Eastern and Western Dan}}\markboth{Both Eastern and Western Dan}{Chronological bibliography of Eastern Dan texts, descriptions, and minor mentions}\XLingPaperaddtocontents{sEandWDan}}\par{}
\penalty10000\vspace{10pt}\penalty10000\noindent{}\hangindent.25in\relax
\hangafter1\relax
{SIL International.  }{2018.  }Dan.  In Gary F. Simons \& Charles D. Fennig, \textit{Ethnologue: Languages of the world. }21st edn.  Dallas, Texas: SIL International.  \href{https://www.ethnologue.com/language/dnj}{\textcolor[rgb]{0,0,0}{https://www.ethnologue.com/language/dnj}}\par
\vspace{11pt plus 2pt minus 1pt}\noindent{}\hangindent.25in\relax
\hangafter1\relax
{Westermann, Diedrich Hermann \& M. A. Bryan.  }{2017.  }\textit{The languages of West Africa: handbook of African languages.  }2. Linguistic surveys of Africa 14.  2.  Reprint of 1970 and 1952.  Abingdon, Oxon: Routledge.\par
\vspace{11pt plus 2pt minus 1pt}\noindent{}\hangindent.25in\relax
\hangafter1\relax
{SIL International.  }{2017.  }Dan.  In Gary F. Simons \& Charles D. Fennig, \textit{Ethnologue: Languages of the world. }20th edn.  Dallas, Texas: SIL International.  \href{https://www.ethnologue.com/20/language/dnj}{\textcolor[rgb]{0,0,0}{https://www.ethnologue.com/20/language/dnj}}\par
\vspace{11pt plus 2pt minus 1pt}\noindent{}\hangindent.25in\relax
\hangafter1\relax
{Vydrin, Valentin.  }{2016.  }\textup{Toward a Proto-Mande reconstruction and an etymological dictionary.  }\textit{Faits de langues }\textup{47}(1). \textup{109-24.  }  doi:\href{http://doai.io/10.3726/432492\_109}{10.3726/432492\_109}\par
\vspace{11pt plus 2pt minus 1pt}\noindent{}\hangindent.25in\relax
\hangafter1\relax
{Vydrin, Valentin.  }{2012.  }\textit{ISO 639-3 Change Request 2012-083.  }(Approved.) Online: ISO 639-3 Registrar (SIL International).  \href{https://iso639-3.sil.org/request/2012-083}{\textcolor[rgb]{0,0,0}{https://iso639-3.sil.org/request/2012-083}}\par
\vspace{11pt plus 2pt minus 1pt}\noindent{}\hangindent.25in\relax
\hangafter1\relax
{Vydrin, Valentin.  }{2010.  }Co-ordinative pronouns in Southern and South-Western Mande: A second compound pronouns area in Africa?  In Pozdniakov, K., Vydrin, Valentin, Zheltov, A., \textit{Personal pronouns in Niger-Congo languages: International workshop. St. Petersburg, September 13-15, 2010 – Abstracts and papers: Working materials, }164-72. St. Petersburg: St. Petersburg University Press.\par
\vspace{11pt plus 2pt minus 1pt}\noindent{}\hangindent.25in\relax
\hangafter1\relax
{SIL International.  }{2009.  }Dan.  In M. Paul Lewis., \textit{Ethnologue: Languages of the world. }16th edn.  Dallas, Texas: SIL International.  \href{https://www.ethnologue.com/16/language/daf}{\textcolor[rgb]{0,0,0}{https://www.ethnologue.com/16/language/daf}}\par
\vspace{11pt plus 2pt minus 1pt}\noindent{}\hangindent.25in\relax
\hangafter1\relax
{SIL International.  }{2005.  }Dan.  In Raymond G. Gordon Jr., \textit{Ethnologue: Languages of the world, }15th edn.  92. Dallas, Texas: SIL International.\par
\vspace{11pt plus 2pt minus 1pt}\noindent{}\hangindent.25in\relax
\hangafter1\relax
{Vydrine \textsquarebracketleft{}Vydrin\textsquarebracketright{}, Valentin.  }{2004.  }\textup{{\textit{La réduplication des adjectifs dans les langues mandé}} \textsquarebracketleft{}The reduplication of adjectives in Mande Languages\textsquarebracketright{}.  }\textit{Mandenkan }\textup{39. }\textup{61-67.  }  \href{http://llacan.vjf.cnrs.fr/PDF/Mandenkan39/39vydr.pdf}{\textcolor[rgb]{0,0,0}{http://llacan.vjf.cnrs.fr/PDF/Mandenkan39/39vydr.pdf}}\par
\vspace{11pt plus 2pt minus 1pt}\noindent{}\hangindent.25in\relax
\hangafter1\relax
{Vydrine \textsquarebracketleft{}Vydrin\textsquarebracketright{}, Valentin, Ted G. Bergman \& Matthew Benjamin.  }{2003.  }\textup{Mandé Language Family of West Africa: Location and Genetic Classification.  }\textit{SIL Electronic Survey Reports (SILER) }\textup{2000-003.  }SIL International.  \href{https://www.sil.org/resources/archives/9019}{\textcolor[rgb]{0,0,0}{https://www.sil.org/resources/archives/9019}}\par
\vspace{11pt plus 2pt minus 1pt}\noindent{}\hangindent.25in\relax
\hangafter1\relax
{Snider, Keith L.  }{1999.  }\textit{The geometry and features of tone.  }(Publications in Linguistics 133.) Dallas, Texas: Summer Institute of Linguistics.\par
\vspace{11pt plus 2pt minus 1pt}\noindent{}\hangindent.25in\relax
\hangafter1\relax
{SIL International.  }{1996.  }Dan.  In Barbara F. Grimes., \textit{Ethnologue: Languages of the world, }13th edn.  247. Dallas, Texas: Summer Institute of Linguistics, Inc.\par
\vspace{11pt plus 2pt minus 1pt}\noindent{}\hangindent.25in\relax
\hangafter1\relax
{SIL International.  }{1992.  }Dan.  In Barbara F. Grimes., \textit{Ethnologue: Languages of the world, }12th edn.  236. Dallas, Texas: Summer Institute of Linguistics, Inc.\par
\vspace{11pt plus 2pt minus 1pt}\noindent{}\hangindent.25in\relax
\hangafter1\relax
{Bolli, Margrit.  }{1991.  }\textup{Orthography difficulties to be overcome by Dan people literate in French.  }\textit{Notes on Literacy }\textup{65. }\textup{25-34.  }  \href{https://www.sil.org/resources/archives/5611}{\textcolor[rgb]{0,0,0}{https://www.sil.org/resources/archives/5611}}\par
\vspace{11pt plus 2pt minus 1pt}\noindent{}\hangindent.25in\relax
\hangafter1\relax
{Bolli, Margrit.  }{1983.  }\textup{The Victor Hugos in Dan Country: Developing a Mother-Tongue Body of Literature in a Neoliterate Society.  }\textit{Journal of Reading }\textup{27}(1). \textup{16-21.  }  \href{http://www.jstor.org/stable/40029291}{\textcolor[rgb]{0,0,0}{http://www.jstor.org/stable/40029291}}\par
\vspace{11pt plus 2pt minus 1pt}\noindent{}\hangindent.25in\relax
\hangafter1\relax
{Bolli, Margrit.  }{1983.  }\textup{The Victor Hugoes in Dan country: Developing a mother-tongue body of literature in a Neoliterate society.  }\textit{Notes on Scripture in Use }\textup{5. }\textup{3-14.  }  \href{https://www.sil.org/resources/archives/5982}{\textcolor[rgb]{0,0,0}{https://www.sil.org/resources/archives/5982}}\par
\vspace{11pt plus 2pt minus 1pt}\noindent{}\hangindent.25in\relax
\hangafter1\relax
{Alluson, M.  }{n.d.  }\textit{{\textit{Etude générale de la région de Man}}.  }4. (Etude sociologique et démographique.) Abidjan, Côte d’Ivoire: Ministère du Plan.\par
\vspace{11pt plus 2pt minus 1pt}\noindent{}\hangindent.25in\relax
\hangafter1\relax
{Halaoui, Nazam, Kalilou Tera \& Monique Trabi.  }{1983.  }\textit{{\textit{Atlas des langues Mandé-sud de Côte d'Ivoire}}.  }Abidjan: Institut de linguistique appliquée \& Agence de coopération culturelle et technique.\par
\vspace{11pt plus 2pt minus 1pt}\noindent{}\hangindent.25in\relax
\hangafter1\relax
{Lauber, Edward.  }{1983.  }\textup{The indigenisation of literacy in Dan (Yacouba).  }\textit{Notes on Literacy }\textup{37. }\textup{16-21.  }\par
\vspace{11pt plus 2pt minus 1pt}\noindent{}\hangindent.25in\relax
\hangafter1\relax
{Bolli, Margrit.  }{1982.  }The Victor Hugoes in Dan country: Developing a mother-tongue body of literature in a Neoliterate society.  International Reading Association 9th world congress on reading.  Dublin, 26-30 July 1982.\par
\vspace{11pt plus 2pt minus 1pt}\noindent{}\hangindent.25in\relax
\hangafter1\relax
{Lauber, Edward.  }{1982.  }\textup{The indigenisation of literacy in Dan (Yacouba).  }\textit{READ }\textup{17}(2). \textup{1-21.  }  \href{http://www.sil.org/resources/archives/23380}{\textcolor[rgb]{0,0,0}{http://www.sil.org/resources/archives/23380}}\par
\vspace{11pt plus 2pt minus 1pt}\noindent{}\hangindent.25in\relax
\hangafter1\relax
{Bolli, Margrit.  }{1980.  }\textup{Progress in literacy in Yacouba country.  }\textit{Notes on Literacy }\textup{31. }\textup{1-6.  }  \href{https://www.sil.org/resources/archives/5388}{\textcolor[rgb]{0,0,0}{https://www.sil.org/resources/archives/5388}}\par
\vspace{11pt plus 2pt minus 1pt}\noindent{}\hangindent.25in\relax
\hangafter1\relax
{Bolli, Margrit.  }{1980.  }\textup{Yacouba literacy report II: March 1977–February 1979.  }\textit{Notes on Literacy }\textup{31. }\textup{7-14.  }  \href{https://www.sil.org/resources/archives/5364}{\textcolor[rgb]{0,0,0}{https://www.sil.org/resources/archives/5364}}\par
\vspace{11pt plus 2pt minus 1pt}\noindent{}\hangindent.25in\relax
\hangafter1\relax
{Thomas, Paule.  }{1979.  }\textup{Alphabétisation en Yacouba.  }\textit{Cahiers ivoiriens de recherche linguistique }\textup{5. }\textup{117-35.  }Abidjan, Ivory Coast: Université Nationale de Côte d’Ivoire/Institut de Linguistique Appliquée.\par
\vspace{11pt plus 2pt minus 1pt}\noindent{}\hangindent.25in\relax
\hangafter1\relax
{Bolli, Margrit.  }{1978.  }\textup{Writing tone with punctuation marks.  }\textit{Notes on Literacy }\textup{23. }\textup{16-18.  }  \href{https://www.sil.org/resources/archives/5336}{\textcolor[rgb]{0,0,0}{https://www.sil.org/resources/archives/5336}}\par
\vspace{11pt plus 2pt minus 1pt}\noindent{}\hangindent.25in\relax
\hangafter1\relax
{Flik, Eva.  }{1977.  }\textup{Tone glides and registers in five Dan dialects.  }\textit{Linguistics }\textup{15}(201). \textup{5-60.  }  doi:\href{http://doai.io/10.1515/ling.1977.15.201.5}{10.1515/ling.1977.15.201.5}\par
\vspace{11pt plus 2pt minus 1pt}\noindent{}\hangindent.25in\relax
\hangafter1\relax
{Bolli, Margrit, Eva Flik \& John Bendor-Samuel.  }{1972.  }Testing the mutual intelligibility of dialects: Yakouba dialect survey.  Paper presented at the 10th West African Languages Congress.  Accra, 21st - 27th March 1972.  \href{https://www.sil.org/resources/archives/76376}{\textcolor[rgb]{0,0,0}{https://www.sil.org/resources/archives/76376}}\par
\vspace{11pt plus 2pt minus 1pt}\noindent{}\hangindent.25in\relax
\hangafter1\relax
{Welmers, William E.  }{1971.  }Niger-Congo, Mande.  In Jack Berry \& Thomas Albert Sebeok, eds. \textit{Linguistics in Sub-Saharan Africa. }Current Trends in Linguistics 7.  Berlin, Boston: De Gruyter.  doi:\href{http://doai.io/10.1515/9783111562520-006}{10.1515/9783111562520-006}\par
\vspace{11pt plus 2pt minus 1pt}\noindent{}\hangindent.25in\relax
\hangafter1\relax
{Welmers, William E.  }{1971.  }Appendix: Checklist of African Language and Dialect Names.  In Jack Berry \& Thomas Albert Sebeok, eds. \textit{Linguistics in Sub-Saharan Africa. }Current Trends in Linguistics 7.  Berlin, Boston: De Gruyter.  doi:\href{http://doai.io/10.1515/9783111562520-030}{10.1515/9783111562520-030}\par
\vspace{11pt plus 2pt minus 1pt}\noindent{}\hangindent.25in\relax
\hangafter1\relax
{Westermann, Diedrich Hermann \& M. A. Bryan.  }{1970.  }\textit{The languages of West Africa: handbook of African languages.  }2. Linguistic surveys of Africa 14.  2.  Reprint of 1952.  London, UK: Dawsons of Pall Mall.\par
\vspace{11pt plus 2pt minus 1pt}\noindent{}\hangindent.25in\relax
\hangafter1\relax
{Bolli, Margrit \& Eva Flik.  }{1970.  }Yakouba dialect survey.   SIL International - Ivory Coast, ms.  \href{https://www.sil.org/resources/archives/64839}{\textcolor[rgb]{0,0,0}{https://www.sil.org/resources/archives/64839}}\par
\vspace{11pt plus 2pt minus 1pt}\noindent{}\hangindent.25in\relax
\hangafter1\relax
{Bearth, Thomas \& Hugo Zemp.  }{1967.  }\textup{The phonology of Dan (Santa).  }\textit{Journal of African Languages }\textup{6}(1). \textup{9-29.  }\par
\vspace{11pt plus 2pt minus 1pt}\noindent{}\hangindent.25in\relax
\hangafter1\relax
{Koelle, Sigismund Wilhelm.  }{1963.  }\textit{Polyglotta Africana.  }Graz, Austria: Akademische Druck- u. Verlagsanstalt.Originally published in 1854.  \par
\vspace{11pt plus 2pt minus 1pt}\noindent{}\hangindent.25in\relax
\hangafter1\relax
{Murdock, George Peter.  }{1959.  }Kru and Peripheral Mande.  \textit{Africa: Its Peoples and Their Culture History, }259-64.  New York, USA: McGraw-Hill Book Company, Inc.\par
\vspace{11pt plus 2pt minus 1pt}\noindent{}\hangindent.25in\relax
\hangafter1\relax
{Welmers, William E.  }{1958.  }The Mande languages.  In William M. Austin, \textit{Report on the Ninth Annual Round Table Meeting on Linguistics and Language Studies, }9-24. Washington, D.C.: Georgetown University Press.\par
\vspace{11pt plus 2pt minus 1pt}\noindent{}\hangindent.25in\relax
\hangafter1\relax
{de Tressan, Michel de la Vergne.  }{1953.  }\textit{{\textit{Inventaire Linguistique de I'Afrique Occidentale Française et du Togo}}.  }(Mémoires de l'Institut Français d'Afrique Noire 30.) Dakar, Senegal: Institut Français/Fondamental d'Afrique Noire.\par
\vspace{11pt plus 2pt minus 1pt}\noindent{}\hangindent.25in\relax
\hangafter1\relax
{Westermann, Diedrich Hermann \& M. A. Bryan.  }{1952.  }\textit{The languages of West Africa: handbook of African languages.  }2. Linguistic surveys of Africa 14.  2.  London, UK: Oxford University Press.\par
\vspace{11pt plus 2pt minus 1pt}\noindent{}\hangindent.25in\relax
\hangafter1\relax
{Delafosse, Maurice.  }{1952.  }{\textit{Les langues du Soudan et de la Guinée}}.  In A. Meillet \& Marcel Cohen, eds. \textit{Les Langues du Monde, }735-845. Originally published in 1924.  Paris, France: Centre National de la Recherche Scientifique.\par
\vspace{11pt plus 2pt minus 1pt}\noindent{}\hangindent.25in\relax
\hangafter1\relax
{Westermann, Diedrich Hermann.  }{1927.  }\textit{{\textit{Die westlichen Sudansprachen und ihre Beziehungen zum Bantu}}.  }Berlin: Walter de Gruyter.\par
\vspace{11pt plus 2pt minus 1pt}\noindent{}\hangindent.25in\relax
\hangafter1\relax
{Delafosse, Maurice.  }{1924.  }{\textit{Les langues du Soudan et de la Guinée}}.  In A. Meillet and Marcel Cohen, \textit{Les Langues du Monde, }463-560. Paris, France: E. Champion.\par
\vspace{11pt plus 2pt minus 1pt}\noindent{}\hangindent.25in\relax
\hangafter1\relax
{Delafosse, Maurice.  }{1912.  }\textit{Haut-Sénégal-Niger.  }Paris: G.P. Maisonneuve et Larose.  \href{https://gallica.bnf.fr/ark:/12148/bpt6k103554s}{\textcolor[rgb]{0,0,0}{https://gallica.bnf.fr/ark:/12148/bpt6k103554s}}\par
\vspace{11pt plus 2pt minus 1pt}\noindent{}\hangindent.25in\relax
\hangafter1\relax
{Chevalier, A.  }{1908.  }\textup{{\textit{Les massifs montagneux du nord-ouest de la Côte d'Ivoire}}.  }\textit{Géographie }\textup{20. }\textup{207-24.  }\par
\vspace{11pt plus 2pt minus 1pt}\noindent{}\hangindent.25in\relax
\hangafter1\relax
{Delafosse, Maurice.  }{1904.  }\textit{{\textit{Vocabulaires comparatifs de plus de 60 langues ou dialectes parlés à la Côte d'Ivoire}}.  }Paris: E. Leroux.  \href{https://gallica.bnf.fr/ark:/12148/bpt6k824366/}{\textcolor[rgb]{0,0,0}{https://gallica.bnf.fr/ark:/12148/bpt6k824366/}}\par
\vspace{11pt plus 2pt minus 1pt}\noindent{}\hangindent.25in\relax
\hangafter1\relax
{Delafosse, Maurice.  }{1901.  }\textit{{\textit{Essai de manuel pratique de la langue mande ou mandingue}}.  }XIV. (Publications de l'École des Langues Orientales Vivantes III.) Paris: E. Leroux.\par
\vspace{11pt plus 2pt minus 1pt}\noindent{}\hangindent.25in\relax
\hangafter1\relax
{Koelle, Sigismund Wilhelm.  }{1854.  }\textit{Polyglotta Africana: Or a Comparative Vocabulary of Nearly Three Hundred Words and Phrases in More Than One Hundred Distinct African Languages.  }London, UK: Church Missionary House.\par
\vspace{11pt plus 2pt minus 1pt}{\vspace{10pt}\XLingPaperneedspace{3\baselineskip}\noindent
\fontsize{13}{15.6}\selectfont \textit{{\noindent
\raisebox{\baselineskip}[0pt]{\pdfbookmark[3]{{C.2.3 } Western Dan}{sWesternDan}}\raisebox{\baselineskip}[0pt]{\protect\hypertarget{sWesternDan}{}}{C.2.3 }Western Dan}}\markboth{Western Dan}{Chronological bibliography of Eastern Dan texts, descriptions, and minor mentions}\XLingPaperaddtocontents{sWesternDan}}\par{}
\penalty10000\vspace{10pt}\penalty10000\noindent{}\hangindent.25in\relax
\hangafter1\relax
{Wycliffe Bible Translators, Inc.  }{2016.  }\textit{{\textit{'WUN SË ‑NAƆ 'SËËDHƐ}}{\textit{ Le Nouveau Testament en Dan Blowo (Yacouba) de Côte d’Ivoire}}.  }https://www.scriptureearth.org: Wycliffe Bible Translators, Inc.\par
\vspace{11pt plus 2pt minus 1pt}\noindent{}\hangindent.25in\relax
\hangafter1\relax
{Wycliffe Bible Translators, Inc.  }{2013.  }\textit{'WUN SË ‑NAƆ 'SËËDHƐ {\textit{Le Nouveau Testament en Dan Blowo (Yacouba) de Côte d’Ivoire}}.  }2nd Printing, 1st edn.  Abidjan, Côte d’Ivoire: Société Biblique Internationale en coopération avec l’Association Ivoirienne pour la Traduction de la Bible.\par
\vspace{11pt plus 2pt minus 1pt}\noindent{}\hangindent.25in\relax
\hangafter1\relax
{Roberts, David.  }{2013.  }A tone orthography typology.  In Joyce, Terry, Borgwaldt, Susanne R., \textit{Typology of Writing Systems, }85-111. Amsterdam: John Benjamins.  doi:\href{http://doai.io/10.1075/bct.51.05rob}{10.1075/bct.51.05rob}\par
\vspace{11pt plus 2pt minus 1pt}\noindent{}\hangindent.25in\relax
\hangafter1\relax
{Roberts, David.  }{2011.  }\textup{A tone orthography typology.  }\textit{Written Language \& Literacy }\textup{14}(1). \textup{82-108.  }  doi:\href{http://doai.io/10.1075/wll.14.1.05rob}{10.1075/wll.14.1.05rob}\par
\vspace{11pt plus 2pt minus 1pt}\noindent{}\hangindent.25in\relax
\hangafter1\relax
{Erman, Anna.  }{2005.  }\textup{{\textit{Le grammaticalisateur -ga en dan-blo}}.  }\textit{Mandenkan: Bulletin d'études linguistiques MANDE }\textup{41. }\textup{41-61.  }  \href{http://llacan.vjf.cnrs.fr/PDF/Mandenkan41/41erma.pdf}{\textcolor[rgb]{0,0,0}{http://llacan.vjf.cnrs.fr/PDF/Mandenkan41/41erma.pdf}}\par
\vspace{11pt plus 2pt minus 1pt}\noindent{}\hangindent.25in\relax
\hangafter1\relax
{Bolli, Margrit \& Eva Flik.  }{2000.  }\textit{Zonasö (Jonah).  }Abidjan, Ivory Coast: Société Internationale de Linguistique (SIL International).SIL Language and Culture Archive ID: 40712.  \par
\vspace{11pt plus 2pt minus 1pt}\noindent{}\hangindent.25in\relax
\hangafter1\relax
{Bolli, Margrit \& Eva Flik.  }{2000.  }\textit{Rutö (Ruth).  }Abidjan, Ivory Coast: Société Internationale de Linguistique (SIL International).SIL Language and Culture Archive ID: 40701.  \par
\vspace{11pt plus 2pt minus 1pt}\noindent{}\hangindent.25in\relax
\hangafter1\relax
{Bolli, Margrit.  }{1993.  }Dan (Yacouba).  In Hartell, Rhonda L., \textit{Alphabets of Africa, }134. Dakar: UNESCO and Summer Institute of Linguistics.\par
\vspace{11pt plus 2pt minus 1pt}\noindent{}\hangindent.25in\relax
\hangafter1\relax
{Wycliffe Bible Translators, Inc.  }{1993.  }\textit{{\textit{'WUN SË ‑NAƆ 'SËËDHƐ}} {\textit{Le Nouveau Testament en Dan Blowo (Yacouba) de Côte d’Ivoire}}.  }1st edn.  Abidjan, Côte d’Ivoire: Société Biblique Internationale en coopération avec l’Association Ivoirienne pour la Traduction de la Bible.\par
\vspace{11pt plus 2pt minus 1pt}\noindent{}\hangindent.25in\relax
\hangafter1\relax
{Bolli, Margrit \& Eva Flik,  }{Editors.  }{1990.  }\textit{-Kwa =kwɛngzü pö! ʼSëëdhɛ -blɛɛzë: ʼSëëdhɛ -peedë -naa (Recueil de contes en dan blowo).  }1. 2.  Abidjan, Ivory Coast: Société Internationale de Linguistique (SIL International).  \href{https://www.sil.org/resources/archives/34560}{\textcolor[rgb]{0,0,0}{https://www.sil.org/resources/archives/34560}}\par
\vspace{11pt plus 2pt minus 1pt}\noindent{}\hangindent.25in\relax
\hangafter1\relax
{Bolli, Margrit \& Eva Flik,  }{Editors.  }{1990.  }\textit{-Kwa =kwɛngzü pö! ʼSëëdhɛ -blɛɛzë: ʼSëëdhɛ -peedë -naa (Recueil de contes en dan blowo).  }2. 2.  Abidjan, Ivory Coast: Société Internationale de Linguistique (SIL International).  \href{https://www.sil.org/resources/archives/34560}{\textcolor[rgb]{0,0,0}{https://www.sil.org/resources/archives/34560}}\par
\vspace{11pt plus 2pt minus 1pt}\noindent{}\hangindent.25in\relax
\hangafter1\relax
{Bolli, Margrit, Flik Eva, Jean D. Oumple \& Jean-Paul Zongo.  }{1982.  }\textit{{\textit{Syllabaire Dan}}.  }Pascal D. Kokora. , ed. (Collection Je lis ma langue.) Abidjan, Côte d'Ivoire : Paris, France: Les Nouvelles Éditions Africaines ; EDICEF.  \href{https://www.sil.org/resources/archives/34569}{\textcolor[rgb]{0,0,0}{https://www.sil.org/resources/archives/34569}}\par
\vspace{11pt plus 2pt minus 1pt}\noindent{}\hangindent.25in\relax
\hangafter1\relax
{SIL International (Bolli, Margrit \& Eva Flik).  }{1982.  }\textit{Guide pour les enseignants du syllabaire dan (dialecte blo -wo).  }Abidjan, Ivory Coast: Société Internationale de Linguistique (SIL International).  \href{https://www.sil.org/resources/archives/34640}{\textcolor[rgb]{0,0,0}{https://www.sil.org/resources/archives/34640}}\par
\vspace{11pt plus 2pt minus 1pt}\noindent{}\hangindent.25in\relax
\hangafter1\relax
{SIL International (Bolli, Margrit \& Eva Flik).  }{1981.  }\textit{Guide d'orthographe pour la langue dan (dialecte blo -wo).  }Abidjan, Ivory Coast: Société Internationale de Linguistique (SIL International).  \href{https://www.sil.org/resources/archives/34704}{\textcolor[rgb]{0,0,0}{https://www.sil.org/resources/archives/34704}}\par
\vspace{11pt plus 2pt minus 1pt}\noindent{}\hangindent.25in\relax
\hangafter1\relax
{Flik, Eva.  }{1978.  }Dan tense-aspect and discourse.  In Grimes, Joseph E., eds. \textit{Papers on discourse, }46-62. Summer Institute of Linguistics Publications in Linguistics and Related Fields.  51.  Dallas, Texas, USA: Summer Institute of Linguistics.\par
\vspace{11pt plus 2pt minus 1pt}\noindent{}\hangindent.25in\relax
\hangafter1\relax
{Bolli, Margrit.  }{1976.  }\textit{Etude prosodique du dan (Blossé).  }(Publications conjointes I.L.A. - S.I.L.) Abidjan: Institut de Linguistique Appliquée de l'Université d'Abidjan and Société Internationale de Linguistique (SIL International).\par
\vspace{11pt plus 2pt minus 1pt}\noindent{}\hangindent.25in\relax
\hangafter1\relax
{Bolli, Margrit \& Eva Flik.  }{1975.  }Phonological statement: Dan (Blossé).  \textit{In Two studies in Ivory Coast linguistics, }iii-51.  Language Data, African Series.  5.  Huntington Beach, California: Summer Institute of Linguistics.\par
\vspace{11pt plus 2pt minus 1pt}\noindent{}\hangindent.25in\relax
\hangafter1\relax
{Bolli, Margrit.  }{1973.  }Analyse prosodique de quelques traits phonologiques en Dan (Blossé).  Abidjan, Côte d'Ivoire: Société Internationale de Linguistique (SIL International), ms.\par
\vspace{11pt plus 2pt minus 1pt}\noindent{}\hangindent.25in\relax
\hangafter1\relax
{Flik, Eva.  }{1973.  }Dan tense-aspect and discours.  Abidjan, Côte d’Ivoire: SIL International, ms.\par
\vspace{11pt plus 2pt minus 1pt}{\vspace{10pt}\XLingPaperneedspace{3\baselineskip}\noindent
\fontsize{13}{15.6}\selectfont \textit{{\noindent
\raisebox{\baselineskip}[0pt]{\pdfbookmark[3]{{C.2.4 } Liberian Dan}{sDanLiberia}}\raisebox{\baselineskip}[0pt]{\protect\hypertarget{sDanLiberia}{}}{C.2.4 }Liberian Dan}}\markboth{Liberian Dan}{Chronological bibliography of Eastern Dan texts, descriptions, and minor mentions}\XLingPaperaddtocontents{sDanLiberia}}\par{}
\penalty10000\vspace{10pt}\penalty10000\noindent{}\hangindent.25in\relax
\hangafter1\relax
{Sternstein, Martin.  }{2008.  }\textup{Mathematics and the Dan Culture.  }\textit{The Journal of Mathematics and Culture }\textup{3}(1). \textup{1-13.  }\par
\vspace{11pt plus 2pt minus 1pt}\noindent{}\hangindent.25in\relax
\hangafter1\relax
{Sternstein, Martin.  }{1992.  }Mathematics and the Dan Culture.  23rd Annual Conference on African Linguistics.  East Lansing, Michigan: Michigan State University, 26–29 March 1992.\par
\vspace{11pt plus 2pt minus 1pt}\noindent{}\hangindent.25in\relax
\hangafter1\relax
{Singler, John V.  }{1990.  }\textup{Linguistics and Liberian Languages in the 1970s and 1980s: A Bibliography.  }\textit{Liberian Studies Journal }\textup{15}(1). \textup{108-26.  }  \href{https://scholarworks.iu.edu/journals/index.php/lsj/article/view/4124}{\textcolor[rgb]{0,0,0}{https://scholarworks.iu.edu/journals/index.php/lsj/article/view/4124}}\par
\vspace{11pt plus 2pt minus 1pt}\noindent{}\hangindent.25in\relax
\hangafter1\relax
{Holsoe, Svend E. \& Joseph J. Lauer.  }{1976.  }\textup{Who Are the Kran/Guere and the Gio/Yacouba? Ethnic Identifications along the Liberia-Ivory Coast Border.  }\textit{African Studies Review }\textup{19}(1). \textup{139-39.  }  doi:\href{http://doai.io/10.2307/523856}{10.2307/523856}\par
\vspace{11pt plus 2pt minus 1pt}\noindent{}\hangindent.25in\relax
\hangafter1\relax
{Riddell, James C., Kjell Letterstrom, Peter Dorliae \& Michael J. Hohl.  }{1971.  }\textup{Clan and Chiefdom Maps of the Ma (Mano) and Da (Gio).  }\textit{Liberian Studies Journal }\textup{4}(2). \textup{157-62.  }  \href{https://scholarworks.iu.edu/journals/index.php/lsj/article/view/4104/3731}{\textcolor[rgb]{0,0,0}{https://scholarworks.iu.edu/journals/index.php/lsj/article/view/4104/3731}}\par
\vspace{11pt plus 2pt minus 1pt}\noindent{}\hangindent.25in\relax
\hangafter1\relax
{Gay, John H., Warren L. D'Azevedo \& William E. Welmers.  }{1969.  }\textup{Language Map of Central Liberia.  }\textit{Liberian Studies Journal }\textup{2}(1). \textup{41-44.  }  \href{https://scholarworks.iu.edu/journals/index.php/lsj/article/view/4100}{\textcolor[rgb]{0,0,0}{https://scholarworks.iu.edu/journals/index.php/lsj/article/view/4100}}\par
\vspace{11pt plus 2pt minus 1pt}\noindent{}\hangindent.25in\relax
\hangafter1\relax
{Griffes, Kenneth E. \& William Everett Welmers.  }{1960.  }\textit{Gio: Structural studies and pedagogical materials and tapes.  }Hartford, Connecticut: U.S. Office of Education.  \href{https://archive.org/details/giostructuralstu00kenn}{\textcolor[rgb]{0,0,0}{https://archive.org/details/giostructuralstu00kenn}}\par
\vspace{11pt plus 2pt minus 1pt}\noindent{}\hangindent.25in\relax
\hangafter1\relax
{Griffes, Kenneth E. \& William Everett Welmers.  }{1959.  }\textit{A start in Gio.  }Cleveland: Baptist Mid-Mission Publication.\par
\vspace{11pt plus 2pt minus 1pt}\vspace{11pt plus 2pt minus 1pt}\vspace{11pt plus 2pt minus 1pt}\XLingPaperneedspace{3\baselineskip}\protect\hypertarget{EDCitations}{}\XLingPaperaddtocontents{EDCitations}\fontsize{9}{10.799999999999999}\selectfont {\protect\raggedright{\singlespacing
{Table }{37.}{  A List of works on Eastern Dan\\}}}\vspace{0pt}{\singlespacing
\hspace*{.25in}{\fontsize{9}{10.799999999999999}\selectfont \singlespacing\vspace*{-3\baselineskip}
\begin{longtable}
[l]{@{}lp{5.25in}@{}}\toprule\multicolumn{1}{@{}l}{\textbf{Dialect}}&\multicolumn{1}{p{5.25in}@{}}{\raggedright\textbf{Citation}}\\%
\midrule\endhead \multicolumn{1}{@{}l}{\raggedright }&\multicolumn{1}{p{5.25in}@{}}{\raggedright ILA (1979), Une orthographe pratique des langues ivoiriennes. Abidjan: Institut de linguistique appliquée, Université d'Abidjan.}\\%
\multicolumn{1}{@{}l}{\raggedright }&\multicolumn{1}{p{5.25in}@{}}{\raggedright Bolli, M. (1983). The Victor Hugoes in Dan country: Developing a mother-tongue body of literature in a neoliterate society. Notes on Scripture in Use, 5. 3-14. Published version of a paper presented at the International Reading Association 9th world congress on reading, Dublin, 26-30 July 1982.}\\%
\multicolumn{1}{@{}l}{\raggedright }&\multicolumn{1}{p{5.25in}@{}}{\raggedright Bolli, M. (ms 1989). The teaching of tone in the Dan language. Abidjan: SIL: Manuscript.}\\%
\multicolumn{1}{@{}l}{\raggedright }&\multicolumn{1}{p{5.25in}@{}}{\raggedright Bolli, Margrit (ms 1989), The teaching of tone in the Dan language. Abidjan: SIL.}\\%
\multicolumn{1}{@{}l}{\raggedright }&\multicolumn{1}{p{5.25in}@{}}{\raggedright Bolli, M. (1991a). Literacy and Scripture in use report: Dan project (Côte d'Ivoire).}\\%
\multicolumn{1}{@{}l}{\raggedright }&\multicolumn{1}{p{5.25in}@{}}{\raggedright Bolli, M. (1992). L'alphabétisation en yacouba (dan) de Côte d'Ivoire.}\\%
\multicolumn{1}{@{}l}{\raggedright }&\multicolumn{1}{p{5.25in}@{}}{\raggedright }\\%
\multicolumn{1}{@{}l}{\raggedright E}&\multicolumn{1}{p{5.25in}@{}}{\raggedright }\\%
\multicolumn{1}{@{}l}{\raggedright }&\multicolumn{1}{p{5.25in}@{}}{\raggedright }\\%
\multicolumn{1}{@{}l}{\raggedright }&\multicolumn{1}{p{5.25in}@{}}{\raggedright }\\%
\multicolumn{1}{@{}l}{\raggedright }&\multicolumn{1}{p{5.25in}@{}}{\raggedright }\\%
\multicolumn{1}{@{}l}{\raggedright }&\multicolumn{1}{p{5.25in}@{}}{\raggedright Vydrin, V. (2006). Emergence of morphological cases in South Mande: From the amorphous type to inflectional? In L. Kulikov, A. Malchukov, \& P. De Swart (Eds.), Case, Valency and Transitivity (pp. 49-64). Leiden and Nijmegen: John Benjamins.}\\%
\multicolumn{1}{@{}l}{\raggedright }&\multicolumn{1}{p{5.25in}@{}}{\raggedright }\\%
\multicolumn{1}{@{}l}{\raggedright }&\multicolumn{1}{p{5.25in}@{}}{\raggedright Vydrin, V. (2009a). Areal features in South Mande and Kru languages. In N. Cyffer, \& G. Ziegelmeyer (Eds.), When languages meet: Language contact and change in West Africa (pp. 91-116). Köln: Rüdiger Köppe.}\\%
\multicolumn{1}{@{}l}{\raggedright }&\multicolumn{1}{p{5.25in}@{}}{\raggedright }\\%
\multicolumn{1}{@{}l}{\raggedright }&\multicolumn{1}{p{5.25in}@{}}{\raggedright Vydrin, V. (2009b). Negation in South Mande. In N. Zyffer, E. Ebermann, \& G. Ziegelmeyer (Eds.), Negation Patterns in West African Languages and Beyond (pp. 223-260). Amsterdam: John Benjamins.}\\%
\multicolumn{1}{@{}l}{\raggedright }&\multicolumn{1}{p{5.25in}@{}}{\raggedright }\\%
\multicolumn{1}{@{}l}{\raggedright }&\multicolumn{1}{p{5.25in}@{}}{\raggedright Vydrin, V. (2009c). On the Problem of the Proto-Mande Homeland. Journal of Language Relationship (Вопросы языкового родства), 1. 107-142.}\\%
\multicolumn{1}{@{}l}{\raggedright }&\multicolumn{1}{p{5.25in}@{}}{\raggedright }\\%
\multicolumn{1}{@{}l}{\raggedright }&\multicolumn{1}{p{5.25in}@{}}{\raggedright }\\%
\multicolumn{1}{@{}l}{\raggedright }&\multicolumn{1}{p{5.25in}@{}}{\raggedright }\\%
\multicolumn{1}{@{}l}{\raggedright }&\multicolumn{1}{p{5.25in}@{}}{\raggedright }\\%
\multicolumn{1}{@{}l}{\raggedright }&\multicolumn{1}{p{5.25in}@{}}{\raggedright }\\%
\multicolumn{1}{@{}l}{\raggedright }&\multicolumn{1}{p{5.25in}@{}}{\raggedright .}\\%
\multicolumn{1}{@{}l}{\raggedright }&\multicolumn{1}{p{5.25in}@{}}{\raggedright }\\%
\multicolumn{1}{@{}l}{\raggedright E \& W}&\multicolumn{1}{p{5.25in}@{}}{\raggedright http://www.ethnologue.com/17/country/CI/languages/}\\%
\multicolumn{1}{@{}l}{\raggedright }&\multicolumn{1}{p{5.25in}@{}}{\raggedright }\\%
\multicolumn{1}{@{}l}{\raggedright }&\multicolumn{1}{p{5.25in}@{}}{\raggedright Discussion of Tone: TONATION IN THREE CHINESE WU DIALECTS}\\%
\multicolumn{1}{@{}l}{\raggedright }&\multicolumn{1}{p{5.25in}@{}}{\raggedright \vbox{\hbox{\strut{}Discussion of LIBERATO https://community.software.sil.org/t/libtralo-keyboard/1004/22}\hbox{\strut{}https://help.keyman.com/keyboard/libtralo/1.6.1/libtralo.php}\hbox{\strut{}https://keyman.com/keyboards/libtralo?embed=macos\&version=9.0.0.0}}}\\%
\multicolumn{1}{@{}l}{\raggedright }&\multicolumn{1}{p{5.25in}@{}}{\raggedright https://web.archive.org/web/20081204223715/http://www.sil.org/silesr/2000/2000-003/Dan-Tura-Mano\_map.htm}\\%
\multicolumn{1}{@{}l}{\raggedright E-W}&\multicolumn{1}{p{5.25in}@{}}{\raggedright CLDR version 22 for both}\\%
\multicolumn{1}{@{}l}{\raggedright E - W}&\multicolumn{1}{p{5.25in}@{}}{\raggedright CLDR Current for both}\\%
\multicolumn{1}{@{}l}{\raggedright }&\multicolumn{1}{p{5.25in}@{}}{\raggedright Doneux, J.-L. (1968) Notes de travail sur quelques langues de l'ouest ivoirien}\\%
\multicolumn{1}{@{}l}{\raggedright }&\multicolumn{1}{p{5.25in}@{}}{\raggedright Western Dan dictionary: http://www.africanbookscollective.com/books/dictionnaire-dan-francais-dan-de-lest-1}\\%
\multicolumn{1}{@{}l}{\raggedright }&\multicolumn{1}{p{5.25in}@{}}{\raggedright \vbox{\hbox{\strut{}https://oac.cdlib.org/findaid/ark:/13030/tf1j49n763/entire\_text/}\hbox{\strut{}William Everett Welmers Papers UCLA Library, Department of Special Collections Manuscripts Division Collection number: 1304 }\hbox{\strut{}Repository: University of California, Los Angeles. Library. Department of Special Collections. Los Angeles, California 90095-1575}\hbox{\strut{}UCLA Catalog Record ID: 4233980 }\hbox{\strut{}Box 2, Folder 7 “Tonemics, Morphotonemics, and Tonal Morphemes”: Manuscript, \textsquarebracketleft{}ca. 1959\textsquarebracketright{}.}\hbox{\strut{}Box 2, Folder 10 “Gio: Structural Studies and Pedagogical Materials”: Manuscript (in collaboration with Kenneth E. Griffes), 1960.}\hbox{\strut{}Box 7, Folder 8 Gio \textsquarebracketleft{}A Start In Gio\textsquarebracketright{}, 1959. Box 7, Folder 9 Gio \textsquarebracketleft{}A Start In Gio\textsquarebracketright{}, 1960.}\hbox{\strut{}Box 9, Folder 16 Liberian Languages; includes material on Mande languages. \textsquarebracketleft{}Liberian Languages: “T.I.L.L.”\textsquarebracketright{}, 1958, February 1, 22, April, 1974, June 21, 1976, May 4, 1977; n.d.}\hbox{\strut{}}}}\\%
\multicolumn{1}{@{}l}{\raggedright }&\multicolumn{1}{p{5.25in}@{}}{\raggedright https://hacks.mozilla.org/2017/03/internationalize-your-keyboard-controls/}\\\bottomrule%
\end{longtable}
}
}\clearpage
\thispagestyle{bodyfirstpage}\markboth{List of technical tools}{List of technical tools}
\XLingPaperaddtocontents{Technicaltools}{\vspace*{.65in}\XLingPaperneedspace{3\baselineskip}\noindent
{\centering
APPENDIX \raisebox{\baselineskip}[0pt]{\protect\hypertarget{Technicaltools}{}}\raisebox{\baselineskip}[0pt]{\pdfbookmark[1]{D  List of technical tools}{Technicaltools}}D\protect\\}}\par{}
{\XLingPaperneedspace{3\baselineskip}\noindent
{\centering
List of technical tools\protect\\}}\par{}
\vspace{16pt}\vspace{11pt plus 2pt minus 1pt}\XLingPaperneedspace{3\baselineskip}\protect\hypertarget{ntFontsUsed}{}\XLingPaperaddtocontents{ntFontsUsed}\XLingPaperCharisZSILFontFamily{\fontsize{8}{9.6}\selectfont \textup{\textup{{\protect\raggedright{\singlespacing
{Table }{38.}{  Fonts used\\}}}\vspace{0pt}{\singlespacing
\hspace*{.25in}{
\XLingPaperminmaxcellincolumn{Font}{\XLingPapermincola}{\textbf{Font}}{\XLingPapermaxcola}{+0\tabcolsep}
\XLingPaperminmaxcellincolumn{Purpose}{\XLingPapermincolb}{\textbf{Purpose}}{\XLingPapermaxcolb}{+0\tabcolsep}
\XLingPaperminmaxcellincolumn{Charis}{\XLingPapermincola}{Charis SIL}{\XLingPapermaxcola}{+0\tabcolsep}
\XLingPaperminmaxcellincolumn{document}{\XLingPapermincolb}{Main document type face}{\XLingPapermaxcolb}{+0\tabcolsep}
\XLingPaperminmaxcellincolumn{Doulos}{\XLingPapermincola}{Doulos SIL}{\XLingPapermaxcola}{+0\tabcolsep}
\XLingPaperminmaxcellincolumn{}{\XLingPapermincolb}{}{\XLingPapermaxcolb}{+0\tabcolsep}
\XLingPaperminmaxcellincolumn{GentiumAlt}{\XLingPapermincola}{GentiumAlt}{\XLingPapermaxcola}{+0\tabcolsep}
\XLingPaperminmaxcellincolumn{Examples}{\XLingPapermincolb}{Font Examples in figures}{\XLingPapermaxcolb}{+0\tabcolsep}
\XLingPaperminmaxcellincolumn{Times}{\XLingPapermincola}{Times New Roman}{\XLingPapermaxcola}{+0\tabcolsep}
\XLingPaperminmaxcellincolumn{Examples}{\XLingPapermincolb}{Font Examples in figures}{\XLingPapermaxcolb}{+0\tabcolsep}
\XLingPaperminmaxcellincolumn{Helvetica}{\XLingPapermincola}{Helvetica}{\XLingPapermaxcola}{+0\tabcolsep}
\XLingPaperminmaxcellincolumn{Examples}{\XLingPapermincolb}{Font Examples in figures}{\XLingPapermaxcolb}{+0\tabcolsep}
\XLingPaperminmaxcellincolumn{Noto}{\XLingPapermincola}{Noto Sans}{\XLingPapermaxcola}{+0\tabcolsep}
\XLingPaperminmaxcellincolumn{Examples}{\XLingPapermincolb}{Font Examples in figures}{\XLingPapermaxcolb}{+0\tabcolsep}
\XLingPaperminmaxcellincolumn{Source}{\XLingPapermincola}{Source Han Serif}{\XLingPapermaxcola}{+0\tabcolsep}
\XLingPaperminmaxcellincolumn{Japanese}{\XLingPapermincolb}{Japanese Characters}{\XLingPapermaxcolb}{+0\tabcolsep}
\XLingPaperminmaxcellincolumn{Jomolhari}{\XLingPapermincola}{\href{https://collab.its.virginia.edu/wiki/tibetan-script/Jomolhari\%20ID.html}{\textcolor[rgb]{0,0,0}{Jomolhari}}}{\XLingPapermaxcola}{+0\tabcolsep}
\XLingPaperminmaxcellincolumn{Tibetan}{\XLingPapermincolb}{Tibetan Script}{\XLingPapermaxcolb}{+0\tabcolsep}
\XLingPaperminmaxcellincolumn{}{\XLingPapermincola}{}{\XLingPapermaxcola}{+0\tabcolsep}
\XLingPaperminmaxcellincolumn{Ethiopic}{\XLingPapermincolb}{Ethiopic Script}{\XLingPapermaxcolb}{+0\tabcolsep}
\XLingPaperminmaxcellincolumn{}{\XLingPapermincola}{}{\XLingPapermaxcola}{+0\tabcolsep}
\XLingPaperminmaxcellincolumn{Google}{\XLingPapermincolb}{Methodology Chart And Google images}{\XLingPapermaxcolb}{+0\tabcolsep}
\XLingPaperminmaxcellincolumn{}{\XLingPapermincola}{}{\XLingPapermaxcola}{+0\tabcolsep}
\XLingPaperminmaxcellincolumn{Google}{\XLingPapermincolb}{Google Spread sheet}{\XLingPapermaxcolb}{+0\tabcolsep}
\XLingPaperminmaxcellincolumn{Script}{\XLingPapermincola}{xkcd Script}{\XLingPapermaxcola}{+0\tabcolsep}
\XLingPaperminmaxcellincolumn{HandWriting}{\XLingPapermincolb}{HandWriting}{\XLingPapermaxcolb}{+0\tabcolsep}
\XLingPaperminmaxcellincolumn{Droid}{\XLingPapermincola}{Droid Serif}{\XLingPapermaxcola}{+0\tabcolsep}
\XLingPaperminmaxcellincolumn{Brackets}{\XLingPapermincolb}{Unicode Characters and Angle Brackets}{\XLingPapermaxcolb}{+0\tabcolsep}
\setlength{\XLingPaperavailabletablewidth}{433.62pt}
\setlength{\XLingPapertableminwidth}{\XLingPapermincola+\XLingPapermincolb}
\setlength{\XLingPapertablemaxwidth}{\XLingPapermaxcola+\XLingPapermaxcolb}
\XLingPapercalculatetablewidthratio{}
\XLingPapersetcolumnwidth{\XLingPapercolawidth}{\XLingPapermincola}{\XLingPapermaxcola}{-0\tabcolsep}
\XLingPapersetcolumnwidth{\XLingPapercolbwidth}{\XLingPapermincolb}{\XLingPapermaxcolb}{-2\tabcolsep}\singlespacing\vspace*{-3\baselineskip}
\begin{longtable}
[l]{@{}p{\XLingPapercolawidth}p{\XLingPapercolbwidth}@{}}\multicolumn{1}{@{}p{\XLingPapercolawidth}}{\textbf{Font}}&\multicolumn{1}{p{\XLingPapercolbwidth}@{}}{\textbf{Purpose}}\\%
\multicolumn{1}{@{}p{\XLingPapercolawidth}}{Charis SIL}&\multicolumn{1}{p{\XLingPapercolbwidth}@{}}{Main document type face}\\%
\multicolumn{1}{@{}p{\XLingPapercolawidth}}{Doulos SIL}&\multicolumn{1}{p{\XLingPapercolbwidth}@{}}{}\\%
\multicolumn{1}{@{}p{\XLingPapercolawidth}}{GentiumAlt}&\multicolumn{1}{p{\XLingPapercolbwidth}@{}}{Font Examples in figures}\\%
\multicolumn{1}{@{}p{\XLingPapercolawidth}}{Times New Roman}&\multicolumn{1}{p{\XLingPapercolbwidth}@{}}{Font Examples in figures}\\%
\multicolumn{1}{@{}p{\XLingPapercolawidth}}{Helvetica}&\multicolumn{1}{p{\XLingPapercolbwidth}@{}}{Font Examples in figures}\\%
\multicolumn{1}{@{}p{\XLingPapercolawidth}}{Noto Sans}&\multicolumn{1}{p{\XLingPapercolbwidth}@{}}{Font Examples in figures}\\%
\multicolumn{1}{@{}p{\XLingPapercolawidth}}{Source Han Serif}&\multicolumn{1}{p{\XLingPapercolbwidth}@{}}{Japanese Characters}\\%
\multicolumn{1}{@{}p{\XLingPapercolawidth}}{\href{https://collab.its.virginia.edu/wiki/tibetan-script/Jomolhari\%20ID.html}{\textcolor[rgb]{0,0,0}{Jomolhari}}}&\multicolumn{1}{p{\XLingPapercolbwidth}@{}}{Tibetan Script}\\%
\multicolumn{1}{@{}p{\XLingPapercolawidth}}{}&\multicolumn{1}{p{\XLingPapercolbwidth}@{}}{Ethiopic Script}\\%
\multicolumn{1}{@{}p{\XLingPapercolawidth}}{}&\multicolumn{1}{p{\XLingPapercolbwidth}@{}}{Methodology Chart And Google images}\\%
\multicolumn{1}{@{}p{\XLingPapercolawidth}}{}&\multicolumn{1}{p{\XLingPapercolbwidth}@{}}{Google Spread sheet}\\%
\multicolumn{1}{@{}p{\XLingPapercolawidth}}{xkcd Script}&\multicolumn{1}{p{\XLingPapercolbwidth}@{}}{HandWriting}\\%
\multicolumn{1}{@{}p{\XLingPapercolawidth}}{Droid Serif}&\multicolumn{1}{p{\XLingPapercolbwidth}@{}}{Unicode Characters and Angle Brackets}\\%
\end{longtable}
}
}}}}\vspace{11pt plus 2pt minus 1pt}\vspace{11pt plus 2pt minus 1pt}\XLingPaperneedspace{3\baselineskip}\protect\hypertarget{ntProgramingTools}{}\XLingPaperaddtocontents{ntProgramingTools}\XLingPaperCharisZSILFontFamily{\fontsize{8}{9.6}\selectfont \textup{\textup{{\protect\raggedright{\singlespacing
{Table }{39.}{  \setcounter{footnote}{0}Programming tools used\footnotemark{}\\}}}\vspace{0pt}{\singlespacing
\hspace*{.25in}{\setcounter{footnote}{1}
\XLingPaperminmaxcellincolumn{Program}{\XLingPapermincola}{\textbf{Program}}{\XLingPapermaxcola}{+0\tabcolsep}
\XLingPaperminmaxcellincolumn{Purpose}{\XLingPapermincolb}{\textbf{Purpose}}{\XLingPapermaxcolb}{+0\tabcolsep}
\XLingPaperminmaxcellincolumn{Bash}{\XLingPapermincola}{Bash}{\XLingPapermaxcola}{+0\tabcolsep}
\XLingPaperminmaxcellincolumn{sources}{\XLingPapermincolb}{Corpus generation from raw sources}{\XLingPapermaxcolb}{+0\tabcolsep}
\XLingPaperminmaxcellincolumn{Git}{\XLingPapermincola}{Git}{\XLingPapermaxcola}{+0\tabcolsep}
\XLingPaperminmaxcellincolumn{Project}{\XLingPapermincolb}{Project versioning}{\XLingPapermaxcolb}{+0\tabcolsep}
\XLingPaperminmaxcellincolumn{Perl}{\XLingPapermincola}{Perl}{\XLingPapermaxcola}{+0\tabcolsep}
\XLingPaperminmaxcellincolumn{character}{\XLingPapermincolb}{Typographical character cleanup}{\XLingPapermaxcolb}{+0\tabcolsep}
\XLingPaperminmaxcellincolumn{Sed}{\XLingPapermincola}{Sed}{\XLingPapermaxcola}{+0\tabcolsep}
\XLingPaperminmaxcellincolumn{character}{\XLingPapermincolb}{Typographical character cleanup}{\XLingPapermaxcolb}{+0\tabcolsep}
\XLingPaperminmaxcellincolumn{Join}{\XLingPapermincola}{Join}{\XLingPapermaxcola}{+0\tabcolsep}
\XLingPaperminmaxcellincolumn{Tracking}{\XLingPapermincolb}{Tracking changes in the corpus}{\XLingPapermaxcolb}{+0\tabcolsep}
\XLingPaperminmaxcellincolumn{CCTables}{\XLingPapermincola}{CCTables}{\XLingPapermaxcola}{+0\tabcolsep}
\XLingPaperminmaxcellincolumn{corpora}{\XLingPapermincolb}{Making corpora compatible with {\XLingPaperCharisZSILFontFamily{\textit{Typing}}}}{\XLingPapermaxcolb}{+0\tabcolsep}
\XLingPaperminmaxcellincolumn{Typing}{\XLingPapermincola}{Typing}{\XLingPapermaxcola}{+0\tabcolsep}
\XLingPaperminmaxcellincolumn{Proposing}{\XLingPapermincolb}{Proposing new keyboards and scoring existing keyboards}{\XLingPapermaxcolb}{+0\tabcolsep}
\XLingPaperminmaxcellincolumn{Keyboard}{\XLingPapermincola}{Keyboard Layout Analyzer}{\XLingPapermaxcola}{+0\tabcolsep}
\XLingPaperminmaxcellincolumn{Proposing}{\XLingPapermincolb}{Proposing new keyboards and scoring existing keyboards}{\XLingPapermaxcolb}{+0\tabcolsep}
\XLingPaperminmaxcellincolumn{Charry}{\XLingPapermincola}{Charry}{\XLingPapermaxcola}{+0\tabcolsep}
\XLingPaperminmaxcellincolumn{Creating}{\XLingPapermincolb}{Creating digrams}{\XLingPapermaxcolb}{+0\tabcolsep}
\XLingPaperminmaxcellincolumn{Numphemes}{\XLingPapermincola}{\href{https://github.com/nqthqn/numphemes}{\textcolor[rgb]{0,0,0}{Numphemes}}}{\XLingPapermaxcola}{+0\tabcolsep}
\XLingPaperminmaxcellincolumn{Counting}{\XLingPapermincolb}{Counting grapheme clusters}{\XLingPapermaxcolb}{+0\tabcolsep}
\XLingPaperminmaxcellincolumn{XLingPaper}{\XLingPapermincola}{\href{https://software.sil.org/xlingpaper/}{\textcolor[rgb]{0,0,0}{XLingPaper}}}{\XLingPapermaxcola}{+0\tabcolsep}
\XLingPaperminmaxcellincolumn{Typesetting}{\XLingPapermincolb}{Typesetting}{\XLingPapermaxcolb}{+0\tabcolsep}
\XLingPaperminmaxcellincolumn{Zotero}{\XLingPapermincola}{Zotero}{\XLingPapermaxcola}{+0\tabcolsep}
\XLingPaperminmaxcellincolumn{Citation}{\XLingPapermincolb}{PDF and Citation management, Research note management}{\XLingPapermaxcolb}{+0\tabcolsep}
\XLingPaperminmaxcellincolumn{EndnoteOn}{\XLingPapermincola}{Endnote}{\XLingPapermaxcola}{+0\tabcolsep}
\XLingPaperminmaxcellincolumn{Citation}{\XLingPapermincolb}{PDF and Citation management, Research note management}{\XLingPapermaxcolb}{+0\tabcolsep}
\XLingPaperminmaxcellincolumn{Google:}{\XLingPapermincola}{Google: Docs, Spreadsheets, Drive, Draw}{\XLingPapermaxcola}{+0\tabcolsep}
\XLingPaperminmaxcellincolumn{Drafting,}{\XLingPapermincolb}{Drafting, peer review, advisor and committee review}{\XLingPapermaxcolb}{+0\tabcolsep}
\XLingPaperminmaxcellincolumn{Omnigraffle}{\XLingPapermincola}{Omnigraffle\footnote{{See footnote }\hyperlink{nEndnoteOS}{2} in chapter .}}{\XLingPapermaxcola}{+0\tabcolsep}
\XLingPaperminmaxcellincolumn{images}{\XLingPapermincolb}{Diagramming and Font images}{\XLingPapermaxcolb}{+0\tabcolsep}
\XLingPaperminmaxcellincolumn{CairoSVG}{\XLingPapermincola}{CairoSVG}{\XLingPapermaxcola}{+0\tabcolsep}
\XLingPaperminmaxcellincolumn{graphs}{\XLingPapermincolb}{Bar graphs}{\XLingPapermaxcolb}{+0\tabcolsep}
\XLingPaperminmaxcellincolumn{PyGal}{\XLingPapermincola}{PyGal}{\XLingPapermaxcola}{+0\tabcolsep}
\XLingPaperminmaxcellincolumn{graphs}{\XLingPapermincolb}{Bar graphs}{\XLingPapermaxcolb}{+0\tabcolsep}
\XLingPaperminmaxcellincolumn{Keyboard}{\XLingPapermincola}{Keyboard Layout Editor}{\XLingPapermaxcola}{+0\tabcolsep}
\XLingPaperminmaxcellincolumn{Keyboards}{\XLingPapermincolb}{Visualizations of Keyboards}{\XLingPapermaxcolb}{+0\tabcolsep}
\setlength{\XLingPaperavailabletablewidth}{433.62pt}
\setlength{\XLingPapertableminwidth}{\XLingPapermincola+\XLingPapermincolb}
\setlength{\XLingPapertablemaxwidth}{\XLingPapermaxcola+\XLingPapermaxcolb}
\XLingPapercalculatetablewidthratio{}
\XLingPapersetcolumnwidth{\XLingPapercolawidth}{\XLingPapermincola}{\XLingPapermaxcola}{-0\tabcolsep}
\XLingPapersetcolumnwidth{\XLingPapercolbwidth}{\XLingPapermincolb}{\XLingPapermaxcolb}{-2\tabcolsep}\setcounter{footnote}{1}\singlespacing\vspace*{-3\baselineskip}
\begin{longtable}
[l]{@{}p{\XLingPapercolawidth}p{\XLingPapercolbwidth}@{}}\multicolumn{1}{@{}p{\XLingPapercolawidth}}{\textbf{Program}}&\multicolumn{1}{p{\XLingPapercolbwidth}@{}}{\textbf{Purpose}}\\%
\multicolumn{1}{@{}p{\XLingPapercolawidth}}{\protect\footnotetext[1]{{\leftskip0pt\parindent1em\raisebox{\baselineskip}[0pt]{\protect\hypertarget{nGNUTools}{}} The default operating system used during this study was Ubuntu 16.04. All *NIX utilities should be understood to be the GNU variety.}}Bash}&\multicolumn{1}{p{\XLingPapercolbwidth}@{}}{Corpus generation from raw sources}\\%
\multicolumn{1}{@{}p{\XLingPapercolawidth}}{Git}&\multicolumn{1}{p{\XLingPapercolbwidth}@{}}{Project versioning}\\%
\multicolumn{1}{@{}p{\XLingPapercolawidth}}{Perl}&\multicolumn{1}{p{\XLingPapercolbwidth}@{}}{Typographical character cleanup}\\%
\multicolumn{1}{@{}p{\XLingPapercolawidth}}{Sed}&\multicolumn{1}{p{\XLingPapercolbwidth}@{}}{Typographical character cleanup}\\%
\multicolumn{1}{@{}p{\XLingPapercolawidth}}{Join}&\multicolumn{1}{p{\XLingPapercolbwidth}@{}}{Tracking changes in the corpus}\\%
\multicolumn{1}{@{}p{\XLingPapercolawidth}}{CCTables}&\multicolumn{1}{p{\XLingPapercolbwidth}@{}}{Making corpora compatible with {\XLingPaperCharisZSILFontFamily{\textit{Typing}}}}\\%
\multicolumn{1}{@{}p{\XLingPapercolawidth}}{Typing}&\multicolumn{1}{p{\XLingPapercolbwidth}@{}}{Proposing new keyboards and scoring existing keyboards}\\%
\multicolumn{1}{@{}p{\XLingPapercolawidth}}{Keyboard Layout Analyzer}&\multicolumn{1}{p{\XLingPapercolbwidth}@{}}{Proposing new keyboards and scoring existing keyboards}\\%
\multicolumn{1}{@{}p{\XLingPapercolawidth}}{Charry}&\multicolumn{1}{p{\XLingPapercolbwidth}@{}}{Creating digrams}\\%
\multicolumn{1}{@{}p{\XLingPapercolawidth}}{\href{https://github.com/nqthqn/numphemes}{\textcolor[rgb]{0,0,0}{Numphemes}}}&\multicolumn{1}{p{\XLingPapercolbwidth}@{}}{Counting grapheme clusters}\\%
\multicolumn{1}{@{}p{\XLingPapercolawidth}}{\href{https://software.sil.org/xlingpaper/}{\textcolor[rgb]{0,0,0}{XLingPaper}}}&\multicolumn{1}{p{\XLingPapercolbwidth}@{}}{Typesetting}\\%
\multicolumn{1}{@{}p{\XLingPapercolawidth}}{Zotero}&\multicolumn{1}{p{\XLingPapercolbwidth}@{}}{PDF and Citation management, Research note management}\\%
\multicolumn{1}{@{}p{\XLingPapercolawidth}}{Endnote\protect\footnote{{\leftskip0pt\parindent1em\raisebox{\baselineskip}[0pt]{\protect\hypertarget{nEndnoteOS}{}} On MacOS.}}}&\multicolumn{1}{p{\XLingPapercolbwidth}@{}}{PDF and Citation management, Research note management}\\%
\multicolumn{1}{@{}p{\XLingPapercolawidth}}{Google: Docs, Spreadsheets, Drive, Draw}&\multicolumn{1}{p{\XLingPapercolbwidth}@{}}{Drafting, peer review, advisor and committee review}\\%
\multicolumn{1}{@{}p{\XLingPapercolawidth}}{Omnigraffle\footnote{{See footnote }\hyperlink{nEndnoteOS}{2} in chapter .}}&\multicolumn{1}{p{\XLingPapercolbwidth}@{}}{Diagramming and Font images}\\%
\multicolumn{1}{@{}p{\XLingPapercolawidth}}{CairoSVG}&\multicolumn{1}{p{\XLingPapercolbwidth}@{}}{Bar graphs}\\%
\multicolumn{1}{@{}p{\XLingPapercolawidth}}{PyGal}&\multicolumn{1}{p{\XLingPapercolbwidth}@{}}{Bar graphs}\\%
\multicolumn{1}{@{}p{\XLingPapercolawidth}}{Keyboard Layout Editor}&\multicolumn{1}{p{\XLingPapercolbwidth}@{}}{Visualizations of Keyboards}\\%
\end{longtable}
}
}}}}\vspace{11pt plus 2pt minus 1pt}\vspace{11pt plus 2pt minus 1pt}\XLingPaperneedspace{3\baselineskip}\protect\hypertarget{ntOpenSourceToolsForOptimization}{}\XLingPaperaddtocontents{ntOpenSourceToolsForOptimization}{\protect\raggedright{\singlespacing
{Table }{40.}{  Open source tools for keyboard layout analysis\\}}}\vspace{0pt}{\singlespacing
\hspace*{.25in}{\XLingPaperCharisZSILFontFamily{\fontsize{8}{9.6}\selectfont \textup{\textup{
\XLingPaperminmaxcellincolumn{Program}{\XLingPapermincola}{\textbf{Program}}{\XLingPapermaxcola}{+0\tabcolsep}
\XLingPaperminmaxcellincolumn{Language}{\XLingPapermincolb}{\textbf{Programing Language}}{\XLingPapermaxcolb}{+0\tabcolsep}
\XLingPaperminmaxcellincolumn{Link}{\XLingPapermincolc}{\textbf{Link}}{\XLingPapermaxcolc}{+0\tabcolsep}
\XLingPaperminmaxcellincolumn{Typing}{\XLingPapermincola}{Typing}{\XLingPapermaxcola}{+0\tabcolsep}
\XLingPaperminmaxcellincolumn{C}{\XLingPapermincolb}{C}{\XLingPapermaxcolb}{+0\tabcolsep}
\XLingPaperminmaxcellincolumn{https://github.com/michaeldickens/Typing}{\XLingPapermincolc}{\href{https://github.com/michaeldickens/Typing}{\textcolor[rgb]{0,0,0}{https://github.com/michaeldickens/Typing}}}{\XLingPapermaxcolc}{+0\tabcolsep}
\XLingPaperminmaxcellincolumn{Keyboard}{\XLingPapermincola}{Keyboard layout analyzer}{\XLingPapermaxcola}{+0\tabcolsep}
\XLingPaperminmaxcellincolumn{JavaScript}{\XLingPapermincolb}{JavaScript}{\XLingPapermaxcolb}{+0\tabcolsep}
\XLingPaperminmaxcellincolumn{https://github.com/patorjk/keyboard-layout-analyzer}{\XLingPapermincolc}{\href{https://github.com/patorjk/keyboard-layout-analyzer}{\textcolor[rgb]{0,0,0}{https://github.com/patorjk/keyboard-layout-analyzer}}}{\XLingPapermaxcolc}{+0\tabcolsep}
\XLingPaperminmaxcellincolumn{optimizer}{\XLingPapermincola}{German optimizer with three state levels}{\XLingPapermaxcola}{+0\tabcolsep}
\XLingPaperminmaxcellincolumn{C}{\XLingPapermincolb}{C}{\XLingPapermaxcolb}{+0\tabcolsep}
\XLingPaperminmaxcellincolumn{http://509.ch/opt.7z}{\XLingPapermincolc}{\href{http://509.ch/opt.7z}{\textcolor[rgb]{0,0,0}{http://509.ch/opt.7z}}}{\XLingPapermaxcolc}{+0\tabcolsep}
\XLingPaperminmaxcellincolumn{optimizer}{\XLingPapermincola}{Integer programming optimizer}{\XLingPapermaxcola}{+0\tabcolsep}
\XLingPaperminmaxcellincolumn{C++}{\XLingPapermincolb}{C++}{\XLingPapermaxcolb}{+0\tabcolsep}
\XLingPaperminmaxcellincolumn{http://resources.mpi-inf.mpg.de/keyboardoptimization/}{\XLingPapermincolc}{\href{http://resources.mpi-inf.mpg.de/keyboardoptimization/}{\textcolor[rgb]{0,0,0}{http://resources.mpi-inf.mpg.de/keyboardoptimization/}}}{\XLingPapermaxcolc}{+0\tabcolsep}
\XLingPaperminmaxcellincolumn{}{\XLingPapermincola}{}{\XLingPapermaxcola}{+0\tabcolsep}
\XLingPaperminmaxcellincolumn{Python}{\XLingPapermincolb}{Python}{\XLingPapermaxcolb}{+0\tabcolsep}
\XLingPaperminmaxcellincolumn{}{\XLingPapermincolc}{}{\XLingPapermaxcolc}{+0\tabcolsep}
\XLingPaperminmaxcellincolumn{}{\XLingPapermincola}{}{\XLingPapermaxcola}{+0\tabcolsep}
\XLingPaperminmaxcellincolumn{Ruby}{\XLingPapermincolb}{Ruby}{\XLingPapermaxcolb}{+0\tabcolsep}
\XLingPaperminmaxcellincolumn{}{\XLingPapermincolc}{}{\XLingPapermaxcolc}{+0\tabcolsep}
\XLingPaperminmaxcellincolumn{Carpalx}{\XLingPapermincola}{Carpalx}{\XLingPapermaxcola}{+0\tabcolsep}
\XLingPaperminmaxcellincolumn{Perl}{\XLingPapermincolb}{Perl}{\XLingPapermaxcolb}{+0\tabcolsep}
\XLingPaperminmaxcellincolumn{http://mkweb.bcgsc.ca/carpalx/?download}{\XLingPapermincolc}{\href{http://mkweb.bcgsc.ca/carpalx/?download}{\textcolor[rgb]{0,0,0}{http://mkweb.bcgsc.ca/carpalx/?download}}}{\XLingPapermaxcolc}{+0\tabcolsep}
\XLingPaperminmaxcellincolumn{Keyboard}{\XLingPapermincola}{White Keyboard Layout}{\XLingPapermaxcola}{+0\tabcolsep}
\XLingPaperminmaxcellincolumn{Rust}{\XLingPapermincolb}{Rust}{\XLingPapermaxcolb}{+0\tabcolsep}
\XLingPaperminmaxcellincolumn{https://github.com/mw8/white\_keyboard\_layout}{\XLingPapermincolc}{\href{https://github.com/mw8/white\_keyboard\_layout}{\textcolor[rgb]{0,0,0}{https://github.com/mw8/white\_keyboard\_layout}}}{\XLingPapermaxcolc}{+0\tabcolsep}
\XLingPaperminmaxcellincolumn{optimizer}{\XLingPapermincola}{"Yak" keyboard layout and optimizer}{\XLingPapermaxcola}{+0\tabcolsep}
\XLingPaperminmaxcellincolumn{JavaScript}{\XLingPapermincolb}{JavaScript}{\XLingPapermaxcolb}{+0\tabcolsep}
\XLingPaperminmaxcellincolumn{https://github.com/wincent/yak-layout}{\XLingPapermincolc}{\href{https://github.com/wincent/yak-layout}{\textcolor[rgb]{0,0,0}{https://github.com/wincent/yak-layout}}}{\XLingPapermaxcolc}{+0\tabcolsep}
\XLingPaperminmaxcellincolumn{French}{\XLingPapermincola}{French Data}{\XLingPapermaxcola}{+0\tabcolsep}
\XLingPaperminmaxcellincolumn{iPython}{\XLingPapermincolb}{iPython}{\XLingPapermaxcolb}{+0\tabcolsep}
\XLingPaperminmaxcellincolumn{https://github.com/annafeit/optimizing-the-french-keyboard}{\XLingPapermincolc}{\href{https://github.com/annafeit/optimizing-the-french-keyboard}{\textcolor[rgb]{0,0,0}{https://github.com/annafeit/optimizing-the-french-keyboard}}}{\XLingPapermaxcolc}{+0\tabcolsep}
\XLingPaperminmaxcellincolumn{Algorithm}{\XLingPapermincola}{Genetic Algorithm}{\XLingPapermaxcola}{+0\tabcolsep}
\XLingPaperminmaxcellincolumn{Java}{\XLingPapermincolb}{Java}{\XLingPapermaxcolb}{+0\tabcolsep}
\XLingPaperminmaxcellincolumn{https://github.com/travisjayday/GeneticKeyboard}{\XLingPapermincolc}{\href{https://github.com/travisjayday/GeneticKeyboard}{\textcolor[rgb]{0,0,0}{https://github.com/travisjayday/GeneticKeyboard}}}{\XLingPapermaxcolc}{+0\tabcolsep}
\setlength{\XLingPaperavailabletablewidth}{433.62pt}
\setlength{\XLingPapertableminwidth}{\XLingPapermincola+\XLingPapermincolb+\XLingPapermincolc}
\setlength{\XLingPapertablemaxwidth}{\XLingPapermaxcola+\XLingPapermaxcolb+\XLingPapermaxcolc}
\XLingPapercalculatetablewidthratio{}
\XLingPapersetcolumnwidth{\XLingPapercolawidth}{\XLingPapermincola}{\XLingPapermaxcola}{-0\tabcolsep}
\XLingPapersetcolumnwidth{\XLingPapercolbwidth}{\XLingPapermincolb}{\XLingPapermaxcolb}{-2\tabcolsep}
\XLingPapersetcolumnwidth{\XLingPapercolcwidth}{\XLingPapermincolc}{\XLingPapermaxcolc}{-2\tabcolsep}\singlespacing\vspace*{-3\baselineskip}
\begin{longtable}
[l]{@{}p{\XLingPapercolawidth}p{\XLingPapercolbwidth}p{\XLingPapercolcwidth}@{}}\multicolumn{1}{@{}p{\XLingPapercolawidth}}{\textbf{Program}}&\multicolumn{1}{p{\XLingPapercolbwidth}}{\textbf{Programing Language}}&\multicolumn{1}{p{\XLingPapercolcwidth}@{}}{\textbf{Link}}\\%
\multicolumn{1}{@{}p{\XLingPapercolawidth}}{Typing}&\multicolumn{1}{p{\XLingPapercolbwidth}}{C}&\multicolumn{1}{p{\XLingPapercolcwidth}@{}}{\href{https://github.com/michaeldickens/Typing}{\textcolor[rgb]{0,0,0}{https://github.com/michaeldickens/Typing}}}\\%
\multicolumn{1}{@{}p{\XLingPapercolawidth}}{Keyboard layout analyzer}&\multicolumn{1}{p{\XLingPapercolbwidth}}{JavaScript}&\multicolumn{1}{p{\XLingPapercolcwidth}@{}}{\href{https://github.com/patorjk/keyboard-layout-analyzer}{\textcolor[rgb]{0,0,0}{https://github.com/patorjk/keyboard-layout-analyzer}}}\\%
\multicolumn{1}{@{}p{\XLingPapercolawidth}}{German optimizer with three state levels}&\multicolumn{1}{p{\XLingPapercolbwidth}}{C}&\multicolumn{1}{p{\XLingPapercolcwidth}@{}}{\href{http://509.ch/opt.7z}{\textcolor[rgb]{0,0,0}{http://509.ch/opt.7z}}}\\%
\multicolumn{1}{@{}p{\XLingPapercolawidth}}{Integer programming optimizer}&\multicolumn{1}{p{\XLingPapercolbwidth}}{C++}&\multicolumn{1}{p{\XLingPapercolcwidth}@{}}{\href{http://resources.mpi-inf.mpg.de/keyboardoptimization/}{\textcolor[rgb]{0,0,0}{http://resources.mpi-inf.mpg.de/keyboardoptimization/}}}\\%
\multicolumn{1}{@{}p{\XLingPapercolawidth}}{}&\multicolumn{1}{p{\XLingPapercolbwidth}}{Python}&\multicolumn{1}{p{\XLingPapercolcwidth}@{}}{}\\%
\multicolumn{1}{@{}p{\XLingPapercolawidth}}{}&\multicolumn{1}{p{\XLingPapercolbwidth}}{Ruby}&\multicolumn{1}{p{\XLingPapercolcwidth}@{}}{}\\%
\multicolumn{1}{@{}p{\XLingPapercolawidth}}{Carpalx}&\multicolumn{1}{p{\XLingPapercolbwidth}}{Perl}&\multicolumn{1}{p{\XLingPapercolcwidth}@{}}{\href{http://mkweb.bcgsc.ca/carpalx/?download}{\textcolor[rgb]{0,0,0}{http://mkweb.bcgsc.ca/carpalx/?download}}}\\%
\multicolumn{1}{@{}p{\XLingPapercolawidth}}{White Keyboard Layout}&\multicolumn{1}{p{\XLingPapercolbwidth}}{Rust}&\multicolumn{1}{p{\XLingPapercolcwidth}@{}}{\href{https://github.com/mw8/white\_keyboard\_layout}{\textcolor[rgb]{0,0,0}{https://github.com/mw8/white\_keyboard\_layout}}}\\%
\multicolumn{1}{@{}p{\XLingPapercolawidth}}{"Yak" keyboard layout and optimizer}&\multicolumn{1}{p{\XLingPapercolbwidth}}{JavaScript}&\multicolumn{1}{p{\XLingPapercolcwidth}@{}}{\href{https://github.com/wincent/yak-layout}{\textcolor[rgb]{0,0,0}{https://github.com/wincent/yak-layout}}}\\%
\multicolumn{1}{@{}p{\XLingPapercolawidth}}{French Data}&\multicolumn{1}{p{\XLingPapercolbwidth}}{iPython}&\multicolumn{1}{p{\XLingPapercolcwidth}@{}}{\href{https://github.com/annafeit/optimizing-the-french-keyboard}{\textcolor[rgb]{0,0,0}{https://github.com/annafeit/optimizing-the-french-keyboard}}}\\%
\multicolumn{1}{@{}p{\XLingPapercolawidth}}{Genetic Algorithm}&\multicolumn{1}{p{\XLingPapercolbwidth}}{Java}&\multicolumn{1}{p{\XLingPapercolcwidth}@{}}{\href{https://github.com/travisjayday/GeneticKeyboard}{\textcolor[rgb]{0,0,0}{https://github.com/travisjayday/GeneticKeyboard}}}\\%
\end{longtable}
}}}}
}\clearpage
\thispagestyle{bodyfirstpage}\markboth{A place to test temporary typesetting configurations}{A place to test temporary typesetting configurations}
\XLingPaperaddtocontents{aAplace}{\vspace*{.65in}\XLingPaperneedspace{3\baselineskip}\noindent
{\centering
APPENDIX \raisebox{\baselineskip}[0pt]{\protect\hypertarget{aAplace}{}}\raisebox{\baselineskip}[0pt]{\pdfbookmark[1]{E  A place to test temporary typesetting configurations}{aAplace}}E\protect\\}}\par{}
{\XLingPaperneedspace{3\baselineskip}\noindent
{\centering
A place to test temporary typesetting configurations\protect\\}}\par{}
\vspace{16pt}{\vspace{15pt}\XLingPaperneedspace{3\baselineskip}\noindent
\fontsize{13}{15.6}\selectfont \textbf{{\noindent
\raisebox{\baselineskip}[0pt]{\pdfbookmark[2]{{E.1 } TextFonts}{sTextFont}}\raisebox{\baselineskip}[0pt]{\protect\hypertarget{sTextFont}{}}{E.1 }TextFonts}}\markboth{TextFonts}{A place to test temporary typesetting configurations}\XLingPaperaddtocontents{sTextFont}}\par{}
\penalty10000\vspace{10pt}\penalty10000\indent The ultimate geek font from Randall Munroe\par{}\indent {\XLingPaperxkcdZScriptFontFamily{some text in a handwriting we can all love.}}\par{}{\vspace{15pt}\XLingPaperneedspace{3\baselineskip}\noindent
\fontsize{13}{15.6}\selectfont \textbf{{\noindent
\raisebox{\baselineskip}[0pt]{\pdfbookmark[2]{{E.2 } Keyboard fonts}{sKeyboardFonts}}\raisebox{\baselineskip}[0pt]{\protect\hypertarget{sKeyboardFonts}{}}{E.2 }Keyboard fonts}}\markboth{Keyboard fonts}{A place to test temporary typesetting configurations}\XLingPaperaddtocontents{sKeyboardFonts}}\par{}
\penalty10000\vspace{10pt}\penalty10000\indent Three non-unicode compliant fonts for keyboards\par{}\indent Option1\par{}\indent {\XLingPaperKeyboardZKeysExZExpandedFontFamily{\textup{\textmd{\textasciitilde{}!@\#\textdollar{}\%\^{}\&*()\_+}}}}\par{}\indent {\XLingPaperKeyboardZKeysExZExpandedFontFamily{\textup{\textmd{QERTYUIOP\{\}}}}}\par{}\indent {\XLingPaperKeyboardZKeysExZExpandedFontFamily{\textup{\textmd{ASDFGHJKL:"\textbar{}\textbackslash{}}}}}\par{}\indent {\XLingPaperKeyboardZKeysExZExpandedFontFamily{\textup{\textmd{ZXCVBNM\textless{}\textgreater{}?}}}}\par{}\indent {\XLingPaperKeyboardZKeysExZExpandedFontFamily{\textup{\textmd{`1234567890-=}}}}\par{}\indent {\XLingPaperKeyboardZKeysExZExpandedFontFamily{\textup{\textmd{qwertyuiop\textsquarebracketleft{}\textsquarebracketright{}}}}}\par{}\indent {\XLingPaperKeyboardZKeysExZExpandedFontFamily{\textup{\textmd{asdfghjkl;'\textbackslash{}}}}}\par{}\indent {\XLingPaperKeyboardZKeysExZExpandedFontFamily{\textup{\textmd{zxcvbnm,./}}}}\par{}\indent Problems with option 1: (1) it uses uppercase letters. This creates ambiguity on if {\XLingPaperKeyboardZKeysExZExpandedFontFamily{\textup{\textmd{q}}}} is pressed or not. (2) there is no altgr. (3) It only provides ASCII keys and not all Unicode characters (4) doesn't respect Unicode characters, used lowercase ASCII for keys with names.\par{}\indent option2\par{}\indent {\XLingPaperaaZQWERTZXTastenFontFamily{\textup{\textmd{\textasciitilde{}!@\#\textdollar{}\%\^{}\&*()\_+}}}}\par{}\indent {\XLingPaperaaZQWERTZXTastenFontFamily{\textup{\textmd{QERTYUIOP\{\}}}}}\par{}\indent {\XLingPaperaaZQWERTZXTastenFontFamily{\textup{\textmd{ASDFGHJKL:"\textbar{}\textbackslash{}}}}}\par{}\indent {\XLingPaperaaZQWERTZXTastenFontFamily{\textup{\textmd{ZXCVBNM\textless{}\textgreater{}?}}}}\par{}\indent {\XLingPaperaaZQWERTZXTastenFontFamily{\textup{\textmd{`1234567890-=}}}}\par{}\indent {\XLingPaperaaZQWERTZXTastenFontFamily{\textup{\textmd{qwertyuiop\textsquarebracketleft{}\textsquarebracketright{}}}}}\par{}\indent {\XLingPaperaaZQWERTZXTastenFontFamily{\textup{\textmd{asdfghjkl;'\textbackslash{}}}}}\par{}\indent {\XLingPaperaaZQWERTZXTastenFontFamily{\textup{\textmd{zxcvbnm,./}}}}\par{}\indent Problems with option 2: (1) It doesn't abide by line height norms. for instance return key and enter key and plus sign. (2) doesn't respect Unicode for modifier keys. (3) keys have the french default values on them.\par{}\indent option 3\par{}\indent {\XLingPaperLinuxZBiolinumZKeyboardFontFamily{\textup{\textmd{This is a keyboard 3 test.?QERTHSFBVS}}}}\par{}\indent Problems with option 3: (1) It doesn't have modifier keys.\par{}\indent what I really want is a stylized text and I can choose what I want in the text. This way all Unicode characters can become a "key".\par{}\indent \vspace*{2.5ex}\leavevmode{}{\XeTeXpicfile "../keypress.png" scaled 500}\par{}{\vspace{15pt}\XLingPaperneedspace{3\baselineskip}\noindent
\fontsize{13}{15.6}\selectfont \textbf{{\noindent
\raisebox{\baselineskip}[0pt]{\pdfbookmark[2]{{E.3 } Images}{sImages}}\raisebox{\baselineskip}[0pt]{\protect\hypertarget{sImages}{}}{E.3 }Images}}\markboth{Images}{A place to test temporary typesetting configurations}\XLingPaperaddtocontents{sImages}}\par{}
\penalty10000\vspace{10pt}\penalty10000\vspace{11pt plus 2pt minus 1pt}\setbox0=\vbox{\protect\centering \leavevmode
\vspace*{0pt}{\XeTeXpdffile "../Resources/Helvetica-CharisSIL.pdf" scaled 700}\\[0pt]\protect\hypertarget{fHelvetica-CSIL2}{}\XLingPaperaddtocontents{fHelvetica-CSIL2}{\singlespacing
{Figure }{26.}{ Helvetica and Charis SIL\\}}}\box0\par{}\vspace{11pt plus 2pt minus 1pt}\setbox0=\vbox{\protect\centering \leavevmode
\vspace*{0pt}{\XeTeXpdffile "../Resources/TNR-CSIL.pdf" scaled 700}\\[0pt]\protect\hypertarget{fNTR-CSIL}{}\XLingPaperaddtocontents{fNTR-CSIL}{\singlespacing
{Figure }{27.}{ New Times Roman and Charis SIL\\}}}\box0\par{}\vspace{11pt plus 2pt minus 1pt}\setbox0=\vbox{\protect\centering \leavevmode
\vspace*{0pt}{\XeTeXpicfile "../Resources/modern-word-combinations-urban-dictionary-18__880.png" scaled 400}\\[0pt]\protect\hypertarget{fUnkeyboardinated}{}\XLingPaperaddtocontents{fUnkeyboardinated}{\singlespacing
{Figure }{28.}{ Unkeyboardinated\\}}}\box0\par{}\vspace{11pt plus 2pt minus 1pt}\setbox0=\vbox{\protect\raggedright\leavevmode
\vspace*{0pt}{\XeTeXpicfile "../Resources/diacritics.png" scaled 750}\\[0pt]\protect\hypertarget{fZalgo}{}\XLingPaperaddtocontents{fZalgo}{\singlespacing
{Figure }{29.}{ \setcounter{footnote}{0}An example of \hyperlink{gtZalgo}{{\textit{Zalgo}}}\footnotemark{}\\}}\protect\footnotetext[1]{{\leftskip0pt\parindent1em\raisebox{\baselineskip}[0pt]{\protect\hypertarget{nZalgoFootnote}{}} Commic curisey of Randall Munroe licesed under the Creative Commons Attribution-NonCommercial 2.5 License. Image available at \href{https://xkcd.com/1647}{\textcolor[rgb]{0,0,0}{https://xkcd.com/1647}}}}}\box0\protect\footnotetext[1]{{\leftskip0pt\parindent1em\raisebox{\baselineskip}[0pt]{\protect\hypertarget{nZalgoFootnote}{}} Commic curisey of Randall Munroe licesed under the Creative Commons Attribution-NonCommercial 2.5 License. Image available at \href{https://xkcd.com/1647}{\textcolor[rgb]{0,0,0}{https://xkcd.com/1647}}}}\par{}\vspace{11pt plus 2pt minus 1pt}{\clearpage
\thispagestyle{bodyfirstpage}\vspace*{.65in}\XLingPaperneedspace{3\baselineskip}\noindent
\raisebox{\baselineskip}[0pt]{\protect\hypertarget{rXLingPapReferences}{}}\raisebox{\baselineskip}[0pt]{\pdfbookmark[1]{References}{rXLingPapReferences}}{\centering
References\protect\\}\XLingPaperaddtocontents{rXLingPapReferences}\markboth{References}{References}
}\par{}
\vspace{10.8pt}{\singlespacing
\raggedright
\hangindent.25in\relax
\hangafter1\relax
\fontsize{12}{14.399999999999999}\selectfont \raisebox{\baselineskip}[0pt]{\protect\hypertarget{rAdomi2007Overn}{}}{Adomi, Eshaenana E.  }{2007.  }\textup{Overnight Internet Browsing Among Cybercafe Users in Abraka, Nigeria.  }\textit{The Journal of Community Informatics }\textup{3}(2).    \href{http://ci-journal.net/index.php/ciej/article/view/322}{\textcolor[rgb]{0,0,0}{http://ci-journal.net/index.php/ciej/article/view/322}}\par
\vspace{11pt plus 2pt minus 1pt}\hangindent.25in\relax
\hangafter1\relax
\fontsize{12}{14.399999999999999}\selectfont \raisebox{\baselineskip}[0pt]{\protect\hypertarget{rAlKhatib2008Langu}{}}{Al-Khatib, Mahmoud A. \& Enaq H. Sabbah.  }{2008.  }\textup{Language Choice in Mobile Text Messages among Jordanian University Students.  }\textit{SKY Journal of Linguistics }\textup{21. }\textup{37-65.  }  \href{http://www.linguistics.fi/julkaisut/sky2008.shtml}{\textcolor[rgb]{0,0,0}{http://www.linguistics.fi/julkaisut/sky2008.shtml}}\par
\vspace{11pt plus 2pt minus 1pt}\hangindent.25in\relax
\hangafter1\relax
\fontsize{12}{14.399999999999999}\selectfont \raisebox{\baselineskip}[0pt]{\protect\hypertarget{Albright-Paper}{}}{Albright, Eric Scott.  }{2000.  }Design of an electronic method for describing writing systems.  Workshop on Web-based Language Documentation and Description, 12-15 December 2000.  Philadelphia, USA.  \href{https://www.sil.org/resources/archives/5751}{\textcolor[rgb]{0,0,0}{https://www.sil.org/resources/archives/5751}}\par
\vspace{11pt plus 2pt minus 1pt}\hangindent.25in\relax
\hangafter1\relax
\fontsize{12}{14.399999999999999}\selectfont \raisebox{\baselineskip}[0pt]{\protect\hypertarget{Albright-MA}{}}{Albright, Eric Scott.  }{2001.  }\textit{Design of an Electronic Method for describing writing systems.  }Dallas, Texas: Graduate Institute of Applied Linguistics thesis.    \href{http://www.tei-c.org/Vault/Workgroups/CE/Design\_of\_an\_Electronic.pdf}{\textcolor[rgb]{0,0,0}{http://www.tei-c.org/Vault/Workgroups/CE/Design\_of\_an\_Electronic.pdf}}\par
\vspace{11pt plus 2pt minus 1pt}\hangindent.25in\relax
\hangafter1\relax
\fontsize{12}{14.399999999999999}\selectfont \raisebox{\baselineskip}[0pt]{\protect\hypertarget{Anderson}{}}{Anderson, Paul.  }{2012.  }A universal Amazigh keyboard for Latin script and Tifinagh.  In Ouguengay, Youssef Ait \& Siham Boulaknadel, eds. \textit{Les ressources langagières : construction et exploitation : le 4ème atelier international sur les technologies d'information et de communication pour l'amazighe. Rabat, 24-25 février 2011, }(Colloques et séminaires (Institut royal de la culture amazighe) 27), 165-79. Rabat, Morocco: Institut Royal de la Culture Amazighe, Centre des Etudes Informatiques, des Systèmes d'Information et de Communication.  \href{http://tal.ircam.ma/conference/docs/ticam2011/a\%20universal\%20amazigh\%20keyboard\%20for\%20latin\%20script.pdf}{\textcolor[rgb]{0,0,0}{http://tal.ircam.ma/conference/docs/ticam2011/​a\%20universal\%20amazigh\%20keyboard\%20for\%20latin\%20script.pdf}}\par
\vspace{11pt plus 2pt minus 1pt}\hangindent.25in\relax
\hangafter1\relax
\fontsize{12}{14.399999999999999}\selectfont \raisebox{\baselineskip}[0pt]{\protect\hypertarget{rAnderson2013IsuOr}{}}{Anderson, Stephen C.  }{2013.  }Isu Orthography Guide.  Yaounde, Cameroon: SIL International, ms.  \href{https://www.sil.org/resources/archives/49012}{\textcolor[rgb]{0,0,0}{https://www.sil.org/resources/archives/49012}}\par
\vspace{11pt plus 2pt minus 1pt}\hangindent.25in\relax
\hangafter1\relax
\fontsize{12}{14.399999999999999}\selectfont \raisebox{\baselineskip}[0pt]{\protect\hypertarget{rApple}{}}{Apple, Inc.  }{2018.  }TrueType™ Reference Manual: About Apple Advanced Typography Fonts.    \href{https://developer.apple.com/fonts/TrueType-Reference-Manual/RM06/Chap6AATIntro.html}{\textcolor[rgb]{0,0,0}{https://developer.apple.com/fonts/TrueType-Reference-Manual/RM06/Chap6AATIntro.html}}  (9 November 2018.)\par
\vspace{11pt plus 2pt minus 1pt}\hangindent.25in\relax
\hangafter1\relax
\fontsize{12}{14.399999999999999}\selectfont \raisebox{\baselineskip}[0pt]{\protect\hypertarget{rBaba1978Yaobh}{}}{Baba, Tiémoko Sébastien.  }{1978.  }\textit{Yaobhaa -wo bhe pe -se -ya ʼgu ({\textit{Receuil de contes yacouba, ʼGwetaa -wo}}).  }Abidjan, Ivory Coast: Société Internationale de Linguistique (SIL International).  \href{https://www.sil.org/resources/archives/34532}{\textcolor[rgb]{0,0,0}{https://www.sil.org/resources/archives/34532}}\par
\vspace{11pt plus 2pt minus 1pt}\hangindent.25in\relax
\hangafter1\relax
\fontsize{12}{14.399999999999999}\selectfont \raisebox{\baselineskip}[0pt]{\protect\hypertarget{rBailey2007}{}}{Bailey, Dwayne.  }{2007.  }\textup{Creating a Single South African Keyboard Layout to Promote Language.  }\textit{LEXIKOS }\textup{17}(1). \textup{212-25.  }  doi:\href{http://doai.io/10.5788/17-0-587}{10.5788/17-0-587}\par
\vspace{11pt plus 2pt minus 1pt}\hangindent.25in\relax
\hangafter1\relax
\fontsize{12}{14.399999999999999}\selectfont \raisebox{\baselineskip}[0pt]{\protect\hypertarget{rBaker2007Digit}{}}{Baker, Nancy A., Rakié Cham, Erin Hale Cidboy, James Cook \& Mark S. Redfern.  }{2007a.  }\textup{Digit kinematics during typing with standard and ergonomic keyboard configurations.  }\textit{International Journal of Industrial Ergonomics }\textup{37}(4). \textup{345-55.  }  doi:\href{http://doai.io/10.1016/j.ergon.2006.12.004}{10.1016/j.ergon.2006.12.004}\par
\vspace{11pt plus 2pt minus 1pt}\hangindent.25in\relax
\hangafter1\relax
\fontsize{12}{14.399999999999999}\selectfont \raisebox{\baselineskip}[0pt]{\protect\hypertarget{rBaker2007Kinamatics}{}}{Baker, Nancy A., Rakié Cham, Erin Hale Cidboy, James Cook \& Mark S. Redfern.  }{2007b.  }\textup{Kinematics of the fingers and hands during computer keyboard use.  }\textit{Clinical Biomechanics }\textup{22}(1). \textup{34-43.  }  doi:\href{http://doai.io/10.1016/j.clinbiomech.2006.08.008}{10.1016/j.clinbiomech.2006.08.008}\par
\vspace{11pt plus 2pt minus 1pt}\hangindent.25in\relax
\hangafter1\relax
\fontsize{12}{14.399999999999999}\selectfont \raisebox{\baselineskip}[0pt]{\protect\hypertarget{Bakkali}{}}{Bakkali, Jaafar EL, EL Mehdi Stouti \& Tarek EL Bardouni.  }{2015.  }\textup{Design and Implementation of an Android SMS Virtual Keyboard for the Berber Language.  }\textit{International Journal of Education and Management Engineering }\textup{5}(1). \textup{1-7.  }  doi:\href{http://doai.io/10.5815/ijeme.2015.01.01}{10.5815/ijeme.2015.01.01}\par
\vspace{11pt plus 2pt minus 1pt}\hangindent.25in\relax
\hangafter1\relax
\fontsize{12}{14.399999999999999}\selectfont \raisebox{\baselineskip}[0pt]{\protect\hypertarget{rBehbahan2011Optim}{}}{Behbahan, Navid Samimi.  }{2011.  }Optimization of Farsi Letter Arrangement on Keyboard by Simulated Annealing and Genetic Algorithms.  5th SASTech 2011, Khavaran Higher-education Institute.  Mashhad, Iran. May 12-14.  \href{http://khi.ac.ir/Research\_Portal/5thsastech/Computer/Computer\_38.pdf}{\textcolor[rgb]{0,0,0}{http://khi.ac.ir/Research\_Portal/5thsastech/Computer/Computer\_38.pdf}}\par
\vspace{11pt plus 2pt minus 1pt}\hangindent.25in\relax
\hangafter1\relax
\fontsize{12}{14.399999999999999}\selectfont \raisebox{\baselineskip}[0pt]{\protect\hypertarget{rBermentVincent2004ME9tho}{}}{Berment, Vincent.  }{2004.  }\textit{{\textit{Méthodes pour informatiser les langues et les groupes de langues “peu dotées”}}.  }Grenoble, France: Université Joseph-Fourier dissertation.    \href{https://tel.archives-ouvertes.fr/tel-00006313}{\textcolor[rgb]{0,0,0}{https://tel.archives-ouvertes.fr/tel-00006313}}\par
\vspace{11pt plus 2pt minus 1pt}\hangindent.25in\relax
\hangafter1\relax
\fontsize{12}{14.399999999999999}\selectfont \raisebox{\baselineskip}[0pt]{\protect\hypertarget{Bernard}{}}{Bernard, H. Russell.  }{1992.  }\textup{Preserving Language Diversity.  }\textit{Human Organization }\textup{51}(1). \textup{82-9.  }  doi:\href{http://doai.io/10.17730/humo.51.1.bp4765g377q32761}{10.17730/humo.51.1.bp4765g377q32761}\par
\vspace{11pt plus 2pt minus 1pt}\hangindent.25in\relax
\hangafter1\relax
\fontsize{12}{14.399999999999999}\selectfont \raisebox{\baselineskip}[0pt]{\protect\hypertarget{rURISyntax2005}{}}{Berners-Lee, Tim, R. Fielding \& L. Masinter.  }{2005.  }\textit{RFC 3986 – Uniform Resource Identifier (URI): Generic Syntax.  }Online: Internet Engineering Task Force.  \href{https://tools.ietf.org/html/rfc3986}{\textcolor[rgb]{0,0,0}{https://tools.ietf.org/html/rfc3986}}\par
\vspace{11pt plus 2pt minus 1pt}\hangindent.25in\relax
\hangafter1\relax
\fontsize{12}{14.399999999999999}\selectfont \raisebox{\baselineskip}[0pt]{\protect\hypertarget{rBesacier2014}{}}{Besacier, Laurent, Etienne Barnard, Alexey Karpov \& Tanja Schultz.  }{2014.  }\textup{Automatic speech recognition for under-resourced languages: A survey.  }\textit{Speech Communication }\textup{56. }\textup{85-100.  }  doi:\href{http://doai.io/10.1016/j.specom.2013.07.008}{10.1016/j.specom.2013.07.008}\par
\vspace{11pt plus 2pt minus 1pt}\hangindent.25in\relax
\hangafter1\relax
\fontsize{12}{14.399999999999999}\selectfont \raisebox{\baselineskip}[0pt]{\protect\hypertarget{rBi2012Multi}{}}{Bi, Xiaojun, Barton A. Smith \& Shumin Zhai.  }{2012.  }\textup{Multilingual Touchscreen Keyboard Design and Optimization.  }\textit{Human–Computer Interaction }\textup{27}(4). \textup{352-82.  }  doi:\href{http://doai.io/10.1080/07370024.2012.678241}{10.1080/07370024.2012.678241}\par
\vspace{11pt plus 2pt minus 1pt}\hangindent.25in\relax
\hangafter1\relax
\fontsize{12}{14.399999999999999}\selectfont \raisebox{\baselineskip}[0pt]{\protect\hypertarget{rBodomoetal2006}{}}{Bodomo, Adams, Charles Marfo, Andrew Cunningham \& Sally Y.K. Mok.  }{2006.  }\textup{A Unicode Keyboard for African Languages: The Case of Dagaare and Twi.  }\textit{International Journal of Technology and Human Interaction }\textup{2}(1). \textup{1-20.  }  doi:\href{http://doai.io/10.4018/jthi.2006010101}{10.4018/jthi.2006010101}\par
\vspace{11pt plus 2pt minus 1pt}\hangindent.25in\relax
\hangafter1\relax
\fontsize{12}{14.399999999999999}\selectfont \raisebox{\baselineskip}[0pt]{\protect\hypertarget{rBoerger1996}{}}{Boerger, Brenda H.  }{1996.  }\textup{When c, q, r, x, and z are vowels: An informal report on Natqgu orthography.  }\textit{Notes on Literacy }\textup{22}(4). \textup{39-44.  }  \href{https://www.sil.org/resources/archives/5326}{\textcolor[rgb]{0,0,0}{https://www.sil.org/resources/archives/5326}}\par
\vspace{11pt plus 2pt minus 1pt}\hangindent.25in\relax
\hangafter1\relax
\fontsize{12}{14.399999999999999}\selectfont \raisebox{\baselineskip}[0pt]{\protect\hypertarget{rBoerger2007Natqg}{}}{Boerger, Brenda H.  }{2007.  }\textup{Natqgu Literacy: Capturing Three Domains for Written Language Use.  }\textit{Language Documentation \& Conservation }\textup{1}(2). \textup{126–153.  }  \href{http://hdl.handle.net/10125/1601}{\textcolor[rgb]{0,0,0}{http://hdl.handle.net/10125/1601}}\par
\vspace{11pt plus 2pt minus 1pt}\hangindent.25in\relax
\hangafter1\relax
\fontsize{12}{14.399999999999999}\selectfont \raisebox{\baselineskip}[0pt]{\protect\hypertarget{rBolli1978Writi}{}}{Bolli, Margrit.  }{1978.  }\textup{Writing tone with punctuation marks.  }\textit{Notes on Literacy }\textup{23. }\textup{16-18.  }  \href{https://www.sil.org/resources/archives/5336}{\textcolor[rgb]{0,0,0}{https://www.sil.org/resources/archives/5336}}\par
\vspace{11pt plus 2pt minus 1pt}\hangindent.25in\relax
\hangafter1\relax
\fontsize{12}{14.399999999999999}\selectfont \raisebox{\baselineskip}[0pt]{\protect\hypertarget{rBolli1980Progr}{}}{Bolli, Margrit.  }{1980a.  }\textup{Progress in literacy in Yacouba country.  }\textit{Notes on Literacy }\textup{31. }\textup{1-6.  }  \href{https://www.sil.org/resources/archives/5388}{\textcolor[rgb]{0,0,0}{https://www.sil.org/resources/archives/5388}}\par
\vspace{11pt plus 2pt minus 1pt}\hangindent.25in\relax
\hangafter1\relax
\fontsize{12}{14.399999999999999}\selectfont \raisebox{\baselineskip}[0pt]{\protect\hypertarget{rBolli1980Yacou}{}}{Bolli, Margrit.  }{1980b.  }\textup{Yacouba literacy report II: March 1977–February 1979.  }\textit{Notes on Literacy }\textup{31. }\textup{7-14.  }  \href{https://www.sil.org/resources/archives/5364}{\textcolor[rgb]{0,0,0}{https://www.sil.org/resources/archives/5364}}\par
\vspace{11pt plus 2pt minus 1pt}\hangindent.25in\relax
\hangafter1\relax
\fontsize{12}{14.399999999999999}\selectfont \raisebox{\baselineskip}[0pt]{\protect\hypertarget{rBolli1983TheVi}{}}{Bolli, Margrit.  }{1983a.  }\textup{The Victor Hugoes in Dan country: Developing a mother-tongue body of literature in a Neoliterate society.  }\textit{Notes on Scripture in Use }\textup{5. }\textup{3-14.  }  \href{https://www.sil.org/resources/archives/5982}{\textcolor[rgb]{0,0,0}{https://www.sil.org/resources/archives/5982}}\par
\vspace{11pt plus 2pt minus 1pt}\hangindent.25in\relax
\hangafter1\relax
\fontsize{12}{14.399999999999999}\selectfont \raisebox{\baselineskip}[0pt]{\protect\hypertarget{rBolliMargrit1983jofr}{}}{Bolli, Margrit.  }{1983b.  }\textup{The Victor Hugos in Dan Country: Developing a Mother-Tongue Body of Literature in a Neoliterate Society.  }\textit{Journal of Reading }\textup{27}(1). \textup{16-21.  }  \href{http://www.jstor.org/stable/40029291}{\textcolor[rgb]{0,0,0}{http://www.jstor.org/stable/40029291}}\par
\vspace{11pt plus 2pt minus 1pt}\hangindent.25in\relax
\hangafter1\relax
\fontsize{12}{14.399999999999999}\selectfont \raisebox{\baselineskip}[0pt]{\protect\hypertarget{rBolli1991Ortho}{}}{Bolli, Margrit.  }{1991.  }\textup{Orthography difficulties to be overcome by Dan people literate in French.  }\textit{Notes on Literacy }\textup{65. }\textup{25-34.  }  \href{https://www.sil.org/resources/archives/5611}{\textcolor[rgb]{0,0,0}{https://www.sil.org/resources/archives/5611}}\par
\vspace{11pt plus 2pt minus 1pt}\hangindent.25in\relax
\hangafter1\relax
\fontsize{12}{14.399999999999999}\selectfont \raisebox{\baselineskip}[0pt]{\protect\hypertarget{rBF1982}{}}{Bolli, Margrit \& Eva Flik.  }{1982.  }\textit{Guide d’orthographe pour la langue dan (dialecte gwɛtaawo).  }Abidjan, Ivory Coast: Société Internationale de Linguistique (SIL International).  \href{https://www.sil.org/resources/archives/34713}{\textcolor[rgb]{0,0,0}{https://www.sil.org/resources/archives/34713}}\par
\vspace{11pt plus 2pt minus 1pt}\hangindent.25in\relax
\hangafter1\relax
\fontsize{12}{14.399999999999999}\selectfont \raisebox{\baselineskip}[0pt]{\protect\hypertarget{rBolli1994Cours}{}}{Bolli, Margrit \& Eva Flik.  }{1994.  }\textit{{\textit{Cours-éclair de lecture pour les lecteurs du français apprenant à lire le dan gwɛɛtaawo}} (Quick reading course for readers of French learning to reading Eastern Dan).  }Abidjan, Ivory Coast: Société Internationale de Linguistique (SIL International).  \href{https://www.sil.org/resources/archives/34670}{\textcolor[rgb]{0,0,0}{https://www.sil.org/resources/archives/34670}}\par
\vspace{11pt plus 2pt minus 1pt}\hangindent.25in\relax
\hangafter1\relax
\fontsize{12}{14.399999999999999}\selectfont \raisebox{\baselineskip}[0pt]{\protect\hypertarget{rBolli2000RutF6}{}}{Bolli, Margrit \& Eva Flik.  }{2000a.  }\textit{Rutö (Ruth).  }Abidjan, Ivory Coast: Société Internationale de Linguistique (SIL International).SIL Language and Culture Archive ID: 40701.  \par
\vspace{11pt plus 2pt minus 1pt}\hangindent.25in\relax
\hangafter1\relax
\fontsize{12}{14.399999999999999}\selectfont \raisebox{\baselineskip}[0pt]{\protect\hypertarget{rBolli2000Jonah}{}}{Bolli, Margrit \& Eva Flik.  }{2000b.  }\textit{Zonasö (Jonah).  }Abidjan, Ivory Coast: Société Internationale de Linguistique (SIL International).SIL Language and Culture Archive ID: 40712.  \par
\vspace{11pt plus 2pt minus 1pt}\hangindent.25in\relax
\hangafter1\relax
\fontsize{12}{14.399999999999999}\selectfont \raisebox{\baselineskip}[0pt]{\protect\hypertarget{rBF1982WDEDICEF}{}}{Bolli, Margrit, Flik Eva, Jean D. Oumple \& Jean-Paul Zongo.  }{1982.  }\textit{{\textit{Syllabaire Dan}}.  }Pascal D. Kokora. , ed. (Collection Je lis ma langue.) Abidjan, Côte d'Ivoire : Paris, France: Les Nouvelles Éditions Africaines ; EDICEF.  \href{https://www.sil.org/resources/archives/34569}{\textcolor[rgb]{0,0,0}{https://www.sil.org/resources/archives/34569}}\par
\vspace{11pt plus 2pt minus 1pt}\hangindent.25in\relax
\hangafter1\relax
\fontsize{12}{14.399999999999999}\selectfont \raisebox{\baselineskip}[0pt]{\protect\hypertarget{rBurmeister1980Toneo}{}}{Burmeister, Jonathan L.  }{1980.  }Tone orthography for the Ivory Coast languages.  \textit{14th Congress of the West African Linguistic Society April 14-18, 1980, }Université du Bénin: Cotonou, République Populaire du Bénin.  \href{https://www.sil.org/resources/archives/5091}{\textcolor[rgb]{0,0,0}{https://www.sil.org/resources/archives/5091}}\par
\vspace{11pt plus 2pt minus 1pt}\hangindent.25in\relax
\hangafter1\relax
\fontsize{12}{14.399999999999999}\selectfont \raisebox{\baselineskip}[0pt]{\protect\hypertarget{rBurmeisterJonathanL.1987Numbe}{}}{Burmeister, Jonathan L.  }{1987.  }Numbers before literacy (Ivory Coast Literacy Programme).  In Gilles Gagne, F. Daems, J. Sturm \& E. Tarrab, \textit{Selected papers in mother tongue literacy — Études en pédagogie de la langue maternelle, }19-25. Dordrecht, Holland \& Montreal, Canada — Faculté des Sciences de l'éducation, Université de Montréal: Foris Publications \& Centre de Diffusion P.P.M.F.\par
\vspace{11pt plus 2pt minus 1pt}\hangindent.25in\relax
\hangafter1\relax
\fontsize{12}{14.399999999999999}\selectfont \raisebox{\baselineskip}[0pt]{\protect\hypertarget{rCahillKaren2008}{}}{Cahill, Michael C. \& Elke Karan.  }{2008.  }\textup{Factors in Designing Effective Orthographies for Unwritten Languages.  }\textit{SIL Electronic Working Papers }\textup{2008-001. }\textup{1-15.  }  \href{http://www.sil.org/silewp/abstract.asp?ref=2008-001}{\textcolor[rgb]{0,0,0}{http://www.sil.org/silewp/abstract.asp?ref=2008-001}}\par
\vspace{11pt plus 2pt minus 1pt}\hangindent.25in\relax
\hangafter1\relax
\fontsize{12}{14.399999999999999}\selectfont \raisebox{\baselineskip}[0pt]{\protect\hypertarget{rChelliahShobhanaLakshmiWillemJosephdeReuse2011Handb}{}}{Chelliah, Shobhana Lakshmi, Willem Joseph de Reuse.  }{2011.  }\textit{Handbook of descriptive linguistic fieldwork.  }Dordrecht, Netherlands; New York: Springer.  doi:\href{http://doai.io/10.1007/978-90-481-9026-3}{10.1007/978-90-481-9026-3}\par
\vspace{11pt plus 2pt minus 1pt}\hangindent.25in\relax
\hangafter1\relax
\fontsize{12}{14.399999999999999}\selectfont \raisebox{\baselineskip}[0pt]{\protect\hypertarget{rCieri2016}{}}{Cieri, Christopher, Mike Maxwell, Stephanie Strassel \& Jennifer Tracey.  }{2016.  }Selection Criteria for Low Resource Language Programs.  \textit{Proceedings of the Tenth International Conference on Language Resources and Evaluation (LREC 2016).  }4543-49.  Paris, France: European Language Resources Association (ELRA).  \href{http://www.lrec-conf.org/proceedings/lrec2016/summaries/1254.html}{\textcolor[rgb]{0,0,0}{http://www.lrec-conf.org/proceedings/lrec2016/summaries/​1254.html}}\par
\vspace{11pt plus 2pt minus 1pt}\hangindent.25in\relax
\hangafter1\relax
\fontsize{12}{14.399999999999999}\selectfont \raisebox{\baselineskip}[0pt]{\protect\hypertarget{rConstableonCharacters}{}}{Constable, Peter G.  }{2001.  }Understanding characters, keystrokes, codepoints and glyphs: encoding and working with multilingual text.  In Melinda Lyons, \textit{Implementing Writing Systems: an Introduction, }9-33. Dallas, Texas: SIL International.  \href{http://scripts.sil.org/IWS-Chapter02}{\textcolor[rgb]{0,0,0}{http://scripts.sil.org/IWS-Chapter02}}\par
\vspace{11pt plus 2pt minus 1pt}\hangindent.25in\relax
\hangafter1\relax
\fontsize{12}{14.399999999999999}\selectfont \raisebox{\baselineskip}[0pt]{\protect\hypertarget{Constable}{}}{Constable, Peter G.  }{2002.  }\textup{Toward a Model for Language Identification Defining an ontology of language-related categories.  }\textit{SIL Electronic Working Papers }\textup{2002-003. }\textup{1-45.  }Dallas, Texas: SIL International.  \href{https://www.sil.org/resources/publications/entry/7853}{\textcolor[rgb]{0,0,0}{https://www.sil.org/resources/publications/entry/7853}}\par
\vspace{11pt plus 2pt minus 1pt}\hangindent.25in\relax
\hangafter1\relax
\fontsize{12}{14.399999999999999}\selectfont \raisebox{\baselineskip}[0pt]{\protect\hypertarget{rConstable2003}{}}{Constable, Peter G.  }{2003.  }Technical Details: Characters, Codepoints, Glyphs: Part 1.  In Victor Gaultney, \textit{Guidelines for Writing System Support, }28-9. Dallas, Texas: UNESCO and SIL International.  \href{https://scripts.sil.org/WSI\_Guidelines}{\textcolor[rgb]{0,0,0}{https://scripts.sil.org/WSI\_Guidelines}}\par
\vspace{11pt plus 2pt minus 1pt}\hangindent.25in\relax
\hangafter1\relax
\fontsize{12}{14.399999999999999}\selectfont \raisebox{\baselineskip}[0pt]{\protect\hypertarget{rCooper1998}{}}{Cooper, Alan.  }{1998.  }\textit{The inmates are running the asylum.  }1st edn.  Indianapolis, Indiana: Sams.\par
\vspace{11pt plus 2pt minus 1pt}\hangindent.25in\relax
\hangafter1\relax
\fontsize{12}{14.399999999999999}\selectfont \raisebox{\baselineskip}[0pt]{\protect\hypertarget{Cooper}{}}{Cooper, Gregory.  }{2005.  }\textit{Issues in the Development of a Writing System for the Kalasha Language.  }Sydney, Australia: Macquarie University dissertation.  \par
\vspace{11pt plus 2pt minus 1pt}\hangindent.25in\relax
\hangafter1\relax
\fontsize{12}{14.399999999999999}\selectfont \raisebox{\baselineskip}[0pt]{\protect\hypertarget{rCrystal2008}{}}{Crystal, David.  }{2008.  }\textit{A Dictionary of Linguistics and Phonetics.  }6th edn.  Oxford, UK: Blackwell Publishing Ltd.\par
\vspace{11pt plus 2pt minus 1pt}\hangindent.25in\relax
\hangafter1\relax
\fontsize{12}{14.399999999999999}\selectfont \raisebox{\baselineskip}[0pt]{\protect\hypertarget{rDavis2018Whistler}{}}{Davis, Mark \& Ken Whistler.  }{2018.  }Unicode Standard Annex \#15: Unicode Normalization Forms.  \textit{Unicode Technical Reports, }Revision 47 edn.  Mountain View, California: Unicode Consortium.  \href{http://unicode.org/reports/tr15}{\textcolor[rgb]{0,0,0}{http://unicode.org/reports/tr15}}\par
\vspace{11pt plus 2pt minus 1pt}\hangindent.25in\relax
\hangafter1\relax
\fontsize{12}{14.399999999999999}\selectfont \raisebox{\baselineskip}[0pt]{\protect\hypertarget{rDavisetal2018LDML}{}}{Davis, Mark \& Others.  }{2018.  }Unicode Technical Standard \#35: Unicode locale data markup language (LDML).  \textit{Unicode Technical Reports, }Revision 53 edn.  Mountain View, California: Unicode Consortium.  \href{http://unicode.org/reports/tr35}{\textcolor[rgb]{0,0,0}{http://unicode.org/reports/tr35}}\par
\vspace{11pt plus 2pt minus 1pt}\hangindent.25in\relax
\hangafter1\relax
\fontsize{12}{14.399999999999999}\selectfont \raisebox{\baselineskip}[0pt]{\protect\hypertarget{rDavisetal2018}{}}{Davis, Mark, Laurențiu Iancu \& Ken Whistler.  }{2018.  }Unicode Standard Annex \#44: Unicode Character Database.  \textit{Unicode Technical Reports, }Revision 22 edn.  Mountain View, California: Unicode Consortium.  \href{http://unicode.org/reports/tr44}{\textcolor[rgb]{0,0,0}{http://unicode.org/reports/tr44}}\par
\vspace{11pt plus 2pt minus 1pt}\hangindent.25in\relax
\hangafter1\relax
\fontsize{12}{14.399999999999999}\selectfont \raisebox{\baselineskip}[0pt]{\protect\hypertarget{rDavison2009Wehor}{}}{Davison, Phil.  }{2009.  }Weh orthography guide.  Yaounde, Cameroon: SIL International, ms.  \href{https://www.sil.org/resources/archives/47887}{\textcolor[rgb]{0,0,0}{https://www.sil.org/resources/archives/47887}}\par
\vspace{11pt plus 2pt minus 1pt}\hangindent.25in\relax
\hangafter1\relax
\fontsize{12}{14.399999999999999}\selectfont \raisebox{\baselineskip}[0pt]{\protect\hypertarget{rDeshwal2006Ergon}{}}{Deshwal, Priyendra S. \& Kalyanmoy Deb.  }{2006.  }Ergonomic Design of an Optimal Hindi Keyboard for Convenient Use.  \textit{IEEE Congress on Evolutionary Computation, 2006. CEC 2006.  }2187-94.    doi:\href{http://doai.io/10.1109/CEC.2006.1688577}{10.1109/CEC.2006.1688577}\par
\vspace{11pt plus 2pt minus 1pt}\hangindent.25in\relax
\hangafter1\relax
\fontsize{12}{14.399999999999999}\selectfont \raisebox{\baselineskip}[0pt]{\protect\hypertarget{rDeskthority-keyprofile}{}}{Deskthority Wiki.  }{2012.  }KeyProfile image by user Findecanor.    \href{https://deskthority.net/wiki/File:KeyProfile.gif}{\textcolor[rgb]{0,0,0}{https://deskthority.net/wiki/File:KeyProfile.gif}}  (1. November 2018.)\par
\vspace{11pt plus 2pt minus 1pt}\hangindent.25in\relax
\hangafter1\relax
\fontsize{12}{14.399999999999999}\selectfont \raisebox{\baselineskip}[0pt]{\protect\hypertarget{rDeumert2017}{}}{Deumert, Ana.  }{2017.  }Text-Messaging in Africa.  \textit{Oxford Research Encyclopedia of Linguistics, }Oxford, England: Oxford University Press.  doi:\href{http://doai.io/10.1093/acrefore/9780199384655.013.229}{10.1093/acrefore/9780199384655.013.229}\par
\vspace{11pt plus 2pt minus 1pt}\hangindent.25in\relax
\hangafter1\relax
\fontsize{12}{14.399999999999999}\selectfont \raisebox{\baselineskip}[0pt]{\protect\hypertarget{rDeumert2013}{}}{Deumert, Ana \& Kristin Vold Lexander.  }{2013.  }\textup{Texting Africa: Writing as performance.  }\textit{Journal of Sociolinguistics }\textup{17}(3). \textup{522-46.  }  doi:\href{http://doai.io/10.1111/josl.12043}{10.1111/josl.12043}\par
\vspace{11pt plus 2pt minus 1pt}\hangindent.25in\relax
\hangafter1\relax
\fontsize{12}{14.399999999999999}\selectfont \raisebox{\baselineskip}[0pt]{\protect\hypertarget{rDickens2016}{}}{Dickens, Michael.  }{2016.  }Typing: A simulated annealing program for optimizing keyboards.    \href{https://github.com/michaeldickens/Typing}{\textcolor[rgb]{0,0,0}{https://github.com/michaeldickens/Typing}}  (21 October 2018.)\par
\vspace{11pt plus 2pt minus 1pt}\hangindent.25in\relax
\hangafter1\relax
\fontsize{12}{14.399999999999999}\selectfont \raisebox{\baselineskip}[0pt]{\protect\hypertarget{Diki-Kidiri}{}}{Diki-Kidiri, Marcel.  }{2011.  }How to Guarantee the Presence and the Life of a Language in Cyberspace.  In Kuzmin, Evgeny, Ekaterina Plys \& Anastasia Parshakova, \textit{Linguistic and Cultural Diversity in Cyberspace. Proceedings of the International Conference (Yakutsk, Russian Federation, 2-4 July, 2008), }230-2. Moscow: Russia: Interregional Library Cooperation Centre.\par
\vspace{11pt plus 2pt minus 1pt}\hangindent.25in\relax
\hangafter1\relax
\fontsize{12}{14.399999999999999}\selectfont \raisebox{\baselineskip}[0pt]{\protect\hypertarget{rDuitsmanJohn1981APlus}{}}{Duitsman, John.  }{1981.  }\textup{A Plus for Plurals in writing Liberian Krahn.  }\textit{Notes on Literacy }\textup{36. }\textup{26-29.  }  \href{https://www.sil.org/resources/archives/5491}{\textcolor[rgb]{0,0,0}{https://www.sil.org/resources/archives/5491}}\par
\vspace{11pt plus 2pt minus 1pt}\hangindent.25in\relax
\hangafter1\relax
\fontsize{12}{14.399999999999999}\selectfont \raisebox{\baselineskip}[0pt]{\protect\hypertarget{rDurn2018}{}}{Durdin, Marc.  }{2018.  }The Case for Keyman.  Personal Blogpost.    \href{https://marc.durdin.net/2018/03/the-case-for-keyman/}{\textcolor[rgb]{0,0,0}{https://marc.durdin.net/2018/03/the-case-for-keyman/}}  (18. Novmenber 2018.)\par
\vspace{11pt plus 2pt minus 1pt}\hangindent.25in\relax
\hangafter1\relax
\fontsize{12}{14.399999999999999}\selectfont \raisebox{\baselineskip}[0pt]{\protect\hypertarget{rEggersetalPreprint2003}{}}{Eggers, Jan, Dominique Feillet, Steffen Kehl, Marc Oliver Wagner \& Bernard Yannou.  }{2003a.  }An Ant Colony Optimization Algorithm for the Optimization of the Keyboard Arrangement Problem.   HAL: archives-ouvertes.fr, ms.  \href{https://hal.archives-ouvertes.fr/hal-00748742}{\textcolor[rgb]{0,0,0}{https://hal.archives-ouvertes.fr/hal-00748742}}  (24. November 2018.)\par
\vspace{11pt plus 2pt minus 1pt}\hangindent.25in\relax
\hangafter1\relax
\fontsize{12}{14.399999999999999}\selectfont \raisebox{\baselineskip}[0pt]{\protect\hypertarget{rEggers2003Optim}{}}{Eggers, Jan, Dominique Feillet, Steffen Kehl, Marc Oliver Wagner \& Bernard Yannou.  }{2003b.  }\textup{Optimization of the keyboard arrangement problem using an Ant Colony algorithm.  }\textit{European Journal of Operational Research }\textup{148}(3). \textup{672-86.  }  doi:\href{http://doai.io/10.1016/S0377-2217(02)00489-7}{10.1016/S0377-2217(02)00489-7}\par
\vspace{11pt plus 2pt minus 1pt}\hangindent.25in\relax
\hangafter1\relax
\fontsize{12}{14.399999999999999}\selectfont \raisebox{\baselineskip}[0pt]{\protect\hypertarget{rEhrensbergerDowMaureenSharonOBrien2015Ergon}{}}{Ehrensberger-Dow, Maureen \& Sharon O'Brien.  }{2015.  }\textup{Ergonomics of the Translation Workplace: Potential for Cognitive Friction.  }\textit{Translation Spaces }\textup{4}(1). \textup{98-118.  }  doi:\href{http://doai.io/10.1075/ts.4.1.05ehr}{10.1075/ts.4.1.05ehr}\par
\vspace{11pt plus 2pt minus 1pt}\hangindent.25in\relax
\hangafter1\relax
\fontsize{12}{14.399999999999999}\selectfont \raisebox{\baselineskip}[0pt]{\protect\hypertarget{rEngland2008}{}}{England, Nora C.  }{2008.  }\textit{ISO 639-3 Change Request 2008-063.  }(Approved.) Online: ISO 639-3 Registrar (SIL International).  \href{https://iso639-3.sil.org/request/2008-063}{\textcolor[rgb]{0,0,0}{https://iso639-3.sil.org/request/2008-063}}\par
\vspace{11pt plus 2pt minus 1pt}\hangindent.25in\relax
\hangafter1\relax
\fontsize{12}{14.399999999999999}\selectfont \raisebox{\baselineskip}[0pt]{\protect\hypertarget{rEvertype1994}{}}{Everson, Michael \& Baldur Sigurðsson.  }{1994.  }On the Status of the Latin Letter Þorn and of Its Sorting Order.  \textit{Report to CEN/TC304 Presented in Reykjavík 1994-06-07, }Reykjavík, Iceland: Evertype.  \href{https://web.archive.org/web/20090207040419/http://evertype.com:80/standards/wynnyogh/thorn.html}{\textcolor[rgb]{0,0,0}{https://web.archive.org/web/20090207040419/http://evertype.com:80​/standards/wynnyogh/thorn.html}}\par
\vspace{11pt plus 2pt minus 1pt}\hangindent.25in\relax
\hangafter1\relax
\fontsize{12}{14.399999999999999}\selectfont \raisebox{\baselineskip}[0pt]{\protect\hypertarget{rFauconnierGilles.1985Menta}{}}{Fauconnier, Gilles.  }{1985.  }\textit{Mental spaces: aspects of meaning construction in natural language.  }Cambridge, Massachusetts: MIT Press.\par
\vspace{11pt plus 2pt minus 1pt}\hangindent.25in\relax
\hangafter1\relax
\fontsize{12}{14.399999999999999}\selectfont \raisebox{\baselineskip}[0pt]{\protect\hypertarget{rFauconnierGilles1994Menta}{}}{Fauconnier, Gilles.  }{1994.  }\textit{Mental spaces: aspects of meaning construction in natural language.  }Cambridge ; New York, NY, USA: Cambridge University Press.\par
\vspace{11pt plus 2pt minus 1pt}\hangindent.25in\relax
\hangafter1\relax
\fontsize{12}{14.399999999999999}\selectfont \raisebox{\baselineskip}[0pt]{\protect\hypertarget{Galla}{}}{Galla, Candace.  }{2009.  }Indigenous language revitalization and technology from traditional to contemporary domains.  In Reyhner, Jon \& Louise Lockard, \textit{Indigenous Language Revitalization: Encouragement, Guidance \& Lessons Learned, }167-82. Flagstaff, Arizona: Northern Arizona University.\par
\vspace{11pt plus 2pt minus 1pt}\hangindent.25in\relax
\hangafter1\relax
\fontsize{12}{14.399999999999999}\selectfont \raisebox{\baselineskip}[0pt]{\protect\hypertarget{rGFM2017}{}}{GitHub Engineering.  }{2017.  }\textit{GitHub Flavored Markdown Spec.  }Online: GitHub.  \href{https://github.github.com/gfm}{\textcolor[rgb]{0,0,0}{https://github.github.com/gfm}}\par
\vspace{11pt plus 2pt minus 1pt}\hangindent.25in\relax
\hangafter1\relax
\fontsize{12}{14.399999999999999}\selectfont \raisebox{\baselineskip}[0pt]{\protect\hypertarget{rGloverFredManuelLagunaRafaelMartED2000Funda}{}}{Glover, Fred, Manuel Laguna \& Rafael Martí.  }{2000.  }\textup{Fundamentals of Scatter Search and Path Relinking.  }\textit{Control and Cybernetics }\textup{29}(3). \textup{653-84.  }  \href{http://control.ibspan.waw.pl:3000/contents/show/70?year=2000}{\textcolor[rgb]{0,0,0}{http://control.ibspan.waw.pl:3000/contents/show/70?year=2000}}\par
\vspace{11pt plus 2pt minus 1pt}\hangindent.25in\relax
\hangafter1\relax
\fontsize{12}{14.399999999999999}\selectfont \raisebox{\baselineskip}[0pt]{\protect\hypertarget{rGosdenChrisYvonneMarshall1999TheCu}{}}{Gosden, Chris \& Yvonne Marshall.  }{1999.  }\textup{The Cultural Biography of Objects.  }\textit{World Archaeology }\textup{31}(2). \textup{169-78.  }  doi:\href{http://doai.io/10.1080/00438243.1999.9980439}{10.1080/00438243.1999.9980439}\par
\vspace{11pt plus 2pt minus 1pt}\hangindent.25in\relax
\hangafter1\relax
\fontsize{12}{14.399999999999999}\selectfont \raisebox{\baselineskip}[0pt]{\protect\hypertarget{rGranberry1991}{}}{Granberry, Julian.  }{1991.  }\textit{Essential Swedish grammar.  }New York: Dover Publications.\par
\vspace{11pt plus 2pt minus 1pt}\hangindent.25in\relax
\hangafter1\relax
\fontsize{12}{14.399999999999999}\selectfont \raisebox{\baselineskip}[0pt]{\protect\hypertarget{rGrenobleLenoreALindsayJWhaley2006Ortho}{}}{Grenoble, Lenore A. \& Lindsay J Whaley.  }{2006.  }Orthography.  \textit{Saving languages: an introduction to language revitalization, }137-59.  Cambridge, UK: Cambridge University Press.  doi:\href{http://doai.io/10.1017/CBO9780511615931.007}{10.1017/CBO9780511615931.007}\par
\vspace{11pt plus 2pt minus 1pt}\hangindent.25in\relax
\hangafter1\relax
\fontsize{12}{14.399999999999999}\selectfont \raisebox{\baselineskip}[0pt]{\protect\hypertarget{rGSMA2018}{}}{GSM Association.  }{2018.  }Mobile Money Metrics: Deployment Tracker.    \href{https://www.gsma.com/mobilemoneymetrics/\#deployment-tracker}{\textcolor[rgb]{0,0,0}{https://www.gsma.com/mobilemoneymetrics/\#deployment-tracker}}  (24. November 2018.)\par
\vspace{11pt plus 2pt minus 1pt}\hangindent.25in\relax
\hangafter1\relax
\fontsize{12}{14.399999999999999}\selectfont \raisebox{\baselineskip}[0pt]{\protect\hypertarget{GuE9rin}{}}{Guérin, Valérie.  }{2008.  }\textup{Writing an endangered language.  }\textit{Language Documentation \& Conservation }\textup{2}(1). \textup{47–67.  }  \href{http://hdl.handle.net/10125/1804}{\textcolor[rgb]{0,0,0}{http://hdl.handle.net/10125/1804}}\par
\vspace{11pt plus 2pt minus 1pt}\hangindent.25in\relax
\hangafter1\relax
\fontsize{12}{14.399999999999999}\selectfont \raisebox{\baselineskip}[0pt]{\protect\hypertarget{rNazam1983}{}}{Halaoui, Nazam, Kalilou Tera \& Monique Trabi.  }{1983.  }\textit{{\textit{Atlas des langues Mandé-sud de Côte d'Ivoire}}.  }Abidjan: Institut de linguistique appliquée \& Agence de coopération culturelle et technique.\par
\vspace{11pt plus 2pt minus 1pt}\hangindent.25in\relax
\hangafter1\relax
\fontsize{12}{14.399999999999999}\selectfont \raisebox{\baselineskip}[0pt]{\protect\hypertarget{rHartell1993}{}}{Hartell, Rhonda L.  }{1993.  }Alphabets of Côte d'Ivoire.  In Rhonda L. Hartell, \textit{Alphabets of Africa, }124-48. Dakar: UNESCO and Summer Institute of Linguistics.\par
\vspace{11pt plus 2pt minus 1pt}\hangindent.25in\relax
\hangafter1\relax
\fontsize{12}{14.399999999999999}\selectfont \raisebox{\baselineskip}[0pt]{\protect\hypertarget{Holm1971}{}}{Holm, Wayne.  }{1971.  }\textup{Navajo Reading Study: Grapheme and unit frequencies in Navajo.  }\textit{Reading Studies progress report № }\textup{12.  }University of New Mexico.  \href{https://eric.ed.gov/?id=ED059806}{\textcolor[rgb]{0,0,0}{https://eric.ed.gov/?id=ED059806}}\par
\vspace{11pt plus 2pt minus 1pt}\hangindent.25in\relax
\hangafter1\relax
\fontsize{12}{14.399999999999999}\selectfont \raisebox{\baselineskip}[0pt]{\protect\hypertarget{rSoketal2017}{}}{Horton, Joshua, Makara Sok, Marc Durdin \& Rasmey Ty.  }{2017.  }Spoof-Vulnerable Rendering in Khmer Unicode Implementations.  \textit{Proceedings of Pacific Asia Confrence on Information Systems, }Langkawi – Malaysia.  \par
\vspace{11pt plus 2pt minus 1pt}\hangindent.25in\relax
\hangafter1\relax
\fontsize{12}{14.399999999999999}\selectfont \raisebox{\baselineskip}[0pt]{\protect\hypertarget{Hosken}{}}{Hosken, Martin.  }{2003.  }Creating an Orthography Description.   SIL International – Non-Roman Script Initiative, ms.  \href{https://scripts.sil.org/WP-Encoding}{\textcolor[rgb]{0,0,0}{https://scripts.sil.org/WP-Encoding}}\par
\vspace{11pt plus 2pt minus 1pt}\hangindent.25in\relax
\hangafter1\relax
\fontsize{12}{14.399999999999999}\selectfont \raisebox{\baselineskip}[0pt]{\protect\hypertarget{rHosken2001}{}}{Hosken, Martin.  }{2001.  }An introduction to keyboard design theory: What goes where?  In Melinda Lyons, \textit{Implementing Writing Systems: an Introduction, }121-37. Dallas, Texas: SIL International.  \href{https://scripts.sil.org/KeybrdDesign}{\textcolor[rgb]{0,0,0}{https://scripts.sil.org/KeybrdDesign}}\par
\vspace{11pt plus 2pt minus 1pt}\hangindent.25in\relax
\hangafter1\relax
\fontsize{12}{14.399999999999999}\selectfont \raisebox{\baselineskip}[0pt]{\protect\hypertarget{rHosken2003Victor}{}}{Hosken, Martin \& Victor Gaultney.  }{2003.  }Section 7: Technical Details: Data Entry and Editing.  In Victor Gaultney, \textit{Guidelines for Writing System Support, }47-52. Dallas, Texas: UNESCO and SIL International.  \href{https://scripts.sil.org/WSI\_Guidelines}{\textcolor[rgb]{0,0,0}{https://scripts.sil.org/WSI\_Guidelines}}\par
\vspace{11pt plus 2pt minus 1pt}\hangindent.25in\relax
\hangafter1\relax
\fontsize{12}{14.399999999999999}\selectfont \raisebox{\baselineskip}[0pt]{\protect\hypertarget{rHymanLary2014}{}}{Hyman, Larry M.  }{2014.  }\textup{How To Study a Tone Language, with exemplification from Oku (Grassfields Bantu, Cameroon).  }\textit{Language Documentation \& Conservation }\textup{8. }\textup{525-62.  }  \href{http://hdl.handle.net/10125/24624}{\textcolor[rgb]{0,0,0}{http://hdl.handle.net/10125/24624}}\par
\vspace{11pt plus 2pt minus 1pt}\hangindent.25in\relax
\hangafter1\relax
\fontsize{12}{14.399999999999999}\selectfont \raisebox{\baselineskip}[0pt]{\protect\hypertarget{rISO9541-1}{}}{International Organization for Standardization (ISO).  }{1991.  }\textit{ISO/IEC 9541-1:2012: Information technology -- Font information interchange -- Part 1: Architecture.  }Geneva, Switzerland: International Standards Organization.  \href{https://www.iso.org/standard/54796.html}{\textcolor[rgb]{0,0,0}{https://www.iso.org/standard/54796.html}}\par
\vspace{11pt plus 2pt minus 1pt}\hangindent.25in\relax
\hangafter1\relax
\fontsize{12}{14.399999999999999}\selectfont \raisebox{\baselineskip}[0pt]{\protect\hypertarget{rISO15924}{}}{International Organization for Standardization (ISO).  }{2004.  }\textit{ISO 15924:2004 Information and documentation -- Codes for the representation of names of scripts.  }Geneva, Switzerland: International Standards Organization.  \href{http://www.unicode.org/iso15924/}{\textcolor[rgb]{0,0,0}{http://www.unicode.org/iso15924/}}  (21. December 2014.)\par
\vspace{11pt plus 2pt minus 1pt}\hangindent.25in\relax
\hangafter1\relax
\fontsize{12}{14.399999999999999}\selectfont \raisebox{\baselineskip}[0pt]{\protect\hypertarget{rISO639-3}{}}{International Organization for Standardization (ISO).  }{2007.  }\textit{ISO 639-3:2007 : Codes for the representation of names of languages -- Part 3: Alpha-3 code for comprehensive coverage of languages.  }Geneva, Switzerland: International Standards Organization.  \href{https://www.iso.org/standard/39534.html}{\textcolor[rgb]{0,0,0}{https://www.iso.org/standard/39534.html}}\par
\vspace{11pt plus 2pt minus 1pt}\hangindent.25in\relax
\hangafter1\relax
\fontsize{12}{14.399999999999999}\selectfont \raisebox{\baselineskip}[0pt]{\protect\hypertarget{rISO9995}{}}{International Organization for Standardization (ISO).  }{2009.  }\textit{ISO/IEC 9995-1:2009: Information technology -- Keyboard layouts for text and office systems -- Part 1: General principles governing keyboard layouts.  }Geneva, Switzerland: International Standards Organization.  \href{https://www.iso.org/standard/51645.html}{\textcolor[rgb]{0,0,0}{https://www.iso.org/standard/51645.html}}\par
\vspace{11pt plus 2pt minus 1pt}\hangindent.25in\relax
\hangafter1\relax
\fontsize{12}{14.399999999999999}\selectfont \raisebox{\baselineskip}[0pt]{\protect\hypertarget{rISO10646}{}}{International Organization for Standardization (ISO).  }{2010.  }\textit{ISO 10646:2010 : Information technology – Universal Coded Character Set (UCS).  }(E) Final Committee Draft (FCD) edn.  Geneva, Switzerland: International Standards Organization.  \href{http://unicode.org/L2/L2010/10038-fcd10646-main.pdf}{\textcolor[rgb]{0,0,0}{http://unicode.org/L2/L2010/10038-fcd10646-main.pdf}}\par
\vspace{11pt plus 2pt minus 1pt}\hangindent.25in\relax
\hangafter1\relax
\fontsize{12}{14.399999999999999}\selectfont \raisebox{\baselineskip}[0pt]{\protect\hypertarget{ISO639-3-2018}{}}{International Organization for Standardization (ISO) - SIL International.  }{2018.  }\textit{ISO 639-3 Standard Registrar.  }Dallas, Texas: SIL International.  \href{https://iso639-3.sil.org/}{\textcolor[rgb]{0,0,0}{https://iso639-3.sil.org/}}\par
\vspace{11pt plus 2pt minus 1pt}\hangindent.25in\relax
\hangafter1\relax
\fontsize{12}{14.399999999999999}\selectfont \raisebox{\baselineskip}[0pt]{\protect\hypertarget{rIWS}{}}{Internet World Stats.  }{2018.  }Africa Internet User Stats and 2018 Population by Country.    \href{https://www.internetworldstats.com/africa.htm\#ci}{\textcolor[rgb]{0,0,0}{https://www.internetworldstats.com/africa.htm\#ci}}  (21 October 2018.)\par
\vspace{11pt plus 2pt minus 1pt}\hangindent.25in\relax
\hangafter1\relax
\fontsize{12}{14.399999999999999}\selectfont \raisebox{\baselineskip}[0pt]{\protect\hypertarget{Jany}{}}{Jany, Carmen.  }{2010.  }\textup{Orthography Design for Chuxnabán Mixe.  }\textit{Language Documentation \& Conservation }\textup{4}(1). \textup{231-53.  }  \href{https://scholarspace.manoa.hawaii.edu/handle/10125/4481}{\textcolor[rgb]{0,0,0}{https://scholarspace.manoa.hawaii.edu/handle/10125/4481}}\par
\vspace{11pt plus 2pt minus 1pt}\hangindent.25in\relax
\hangafter1\relax
\fontsize{12}{14.399999999999999}\selectfont \raisebox{\baselineskip}[0pt]{\protect\hypertarget{rJumia2015}{}}{Jumia.  }{2015.  }\textit{Growth of the smartphone market in Africa 2015 - Côte d’Ivoire.  }Côte d’Ivoire: Jumia.  \href{http://africabusiness.com/wp-content/uploads/2015/06/THE-GROWTH-OF-THE-SMARTPHONE-MARKET-IN-AFRICA-COTE-DIVOIRE.pdf}{\textcolor[rgb]{0,0,0}{http://africabusiness.com/wp-content/uploads/2015/06/THE-GROWTH-OF-THE-SMARTPHONE-MARKET-IN-AFRICA-COTE-DIVOIRE.pdf}}\par
\vspace{11pt plus 2pt minus 1pt}\hangindent.25in\relax
\hangafter1\relax
\fontsize{12}{14.399999999999999}\selectfont \raisebox{\baselineskip}[0pt]{\protect\hypertarget{rKacmarcik2017}{}}{Kacmarcik, Gary \& Travis Leithead.  }{2017.  }UI Events KeyboardEvent code Values.  \textit{W3C Candidate Recommendation, }Online: W3C.  \href{https://www.w3.org/TR/uievents-code}{\textcolor[rgb]{0,0,0}{https://www.w3.org/TR/uievents-code}}\par
\vspace{11pt plus 2pt minus 1pt}\hangindent.25in\relax
\hangafter1\relax
\fontsize{12}{14.399999999999999}\selectfont \raisebox{\baselineskip}[0pt]{\protect\hypertarget{rKaigwaMark2017FromC}{}}{Kaigwa, Mark.  }{2017.  }From Cyber Café to Smartphone: Kenya’s Social Media Lens Zooms In on the Country and Out to the World.  In Bitange Ndemo and Tim Weiss, \textit{Digital Kenya, }187-222. London: Palgrave Macmillan UK.  doi:\href{http://doai.io/10.1057/978-1-137-57878-5\_7}{10.1057/978-1-137-57878-5\_7}\par
\vspace{11pt plus 2pt minus 1pt}\hangindent.25in\relax
\hangafter1\relax
\fontsize{12}{14.399999999999999}\selectfont \raisebox{\baselineskip}[0pt]{\protect\hypertarget{rKaranE2014}{}}{Karan, Elke.  }{2014.  }Standardization: What's the hurry?  In Mike Cahill \& Karen Rice, eds. \textit{Developing Orthographies for Unwritten Languages, }107-38. Publications in Language Use and Education 6.  Dallas, Texas: SIL International.\par
\vspace{11pt plus 2pt minus 1pt}\hangindent.25in\relax
\hangafter1\relax
\fontsize{12}{14.399999999999999}\selectfont \raisebox{\baselineskip}[0pt]{\protect\hypertarget{rKarrenbauerAndreasAnttiOulasvirta2014Impro}{}}{Karrenbauer, Andreas \& Antti Oulasvirta.  }{2014.  }Improvements to keyboard optimization with integer programming.  \textit{Proceedings of the 27th annual ACM symposium on User interface software and technology - UIST '14.  }621-26.  Honolulu, Hawaii, USA: ACM Press.  \href{http://resources.mpi-inf.mpg.de/keyboardoptimization}{\textcolor[rgb]{0,0,0}{http://resources.mpi-inf.mpg.de/keyboardoptimization}}  doi:\href{http://doai.io/10.1145/2642918.2647382}{10.1145/2642918.2647382}\par
\vspace{11pt plus 2pt minus 1pt}\hangindent.25in\relax
\hangafter1\relax
\fontsize{12}{14.399999999999999}\selectfont \raisebox{\baselineskip}[0pt]{\protect\hypertarget{rKessE9gbeu2007}{}}{Kességbeu, Mongnan Alphonse.  }{2007.  }\textit{{\textit{"Sanni kö =dhɔtrɔɔ -yö nu}} \textsquarebracketleft{}{\textit{En attendant l'arrivée du médecin}}; While waiting for a medical doctor\textsquarebracketright{}.  }Abidjan, Côte d'Ivoire: Self published.\par
\vspace{11pt plus 2pt minus 1pt}\hangindent.25in\relax
\hangafter1\relax
\fontsize{12}{14.399999999999999}\selectfont \raisebox{\baselineskip}[0pt]{\protect\hypertarget{rKhorshid2010}{}}{Khorshid, Emad, Abdulaziz Alfadli \& Majed Majeed.  }{2010.  }\textup{A new optimal Arabic keyboard layout using genetic algorithm.  }\textit{International Journal of Design Engineering }\textup{3}(1). \textup{25-25.  }  doi:\href{http://doai.io/10.1504/IJDE.2010.032821}{10.1504/IJDE.2010.032821}  \href{http://www.inderscience.com/link.php?id=32821}{\textcolor[rgb]{0,0,0}{http://www.inderscience.com/link.php?id=32821}}\par
\vspace{11pt plus 2pt minus 1pt}\hangindent.25in\relax
\hangafter1\relax
\fontsize{12}{14.399999999999999}\selectfont \raisebox{\baselineskip}[0pt]{\protect\hypertarget{rKoffi1995Indig}{}}{Koffi, Ettien N.  }{1995.  }\textup{Indigenizing punctuation marks.  }\textit{Notes on Literacy }\textup{21}(2). \textup{1-11.  }  \href{http://www.sil.org/resources/archives/5459}{\textcolor[rgb]{0,0,0}{http://www.sil.org/resources/archives/5459}}\par
\vspace{11pt plus 2pt minus 1pt}\hangindent.25in\relax
\hangafter1\relax
\fontsize{12}{14.399999999999999}\selectfont \raisebox{\baselineskip}[0pt]{\protect\hypertarget{Kornai}{}}{Kornai, András.  }{2013.  }\textup{Digital Language Death.  }\textit{PLOS ONE }\textup{8:10. }\textup{e77056.  }  \href{https://doi.org/10.1371/journal.pone.0077056}{\textcolor[rgb]{0,0,0}{https://doi.org/10.1371/journal.pone.0077056}}  doi:\href{http://doai.io/10.1371/journal.pone.0077056}{10.1371/journal.pone.0077056}\par
\vspace{11pt plus 2pt minus 1pt}\hangindent.25in\relax
\hangafter1\relax
\fontsize{12}{14.399999999999999}\selectfont \raisebox{\baselineskip}[0pt]{\protect\hypertarget{rKrauwerSteven2003TheBa}{}}{Krauwer, Steven.  }{2003.  }The Basic Language Resource Kit (BLARK) as the First Milestone for the Language Resources Roadmap.  \textit{Proceedings of the 2003 International Workshop Speech and Computer (SPECOM-2003), Moscow, Russia.  }8-15.  \par
\vspace{11pt plus 2pt minus 1pt}\hangindent.25in\relax
\hangafter1\relax
\fontsize{12}{14.399999999999999}\selectfont \raisebox{\baselineskip}[0pt]{\protect\hypertarget{rKutschLojenga1993Thewr}{}}{Kutsch Lojenga, Constance.  }{1993.  }\textup{The writing and reading of tone in Bantu languages.  }\textit{Notes on Literacy }\textup{19}(1). \textup{1-19.  }  \href{https://www.sil.org/resources/archives/5233}{\textcolor[rgb]{0,0,0}{https://www.sil.org/resources/archives/5233}}\par
\vspace{11pt plus 2pt minus 1pt}\hangindent.25in\relax
\hangafter1\relax
\fontsize{12}{14.399999999999999}\selectfont \raisebox{\baselineskip}[0pt]{\protect\hypertarget{rKutschLojenga1986}{}}{Kutsch Lojenga, Constance.  }{1986.  }\textup{Some Experiences in Writing and teaching Tone in Africa.  }\textit{Notes on Literacy }\textup{Special Issue №1 (Papers Presented at the First international Literacy Consultant Seminar, November 11-20, 1985). }\textup{59-65.  }  \href{}{\textcolor[rgb]{0,0,0}{}}\par
\vspace{11pt plus 2pt minus 1pt}\hangindent.25in\relax
\hangafter1\relax
\fontsize{12}{14.399999999999999}\selectfont \raisebox{\baselineskip}[0pt]{\protect\hypertarget{rLarson1976Punct}{}}{Larson, Mildred L.  }{1976.  }\textup{Punctuating the translation for ease of reading.  }\textit{Notes on Translation }\textup{60. }\textup{23-26.  }\par
\vspace{11pt plus 2pt minus 1pt}\hangindent.25in\relax
\hangafter1\relax
\fontsize{12}{14.399999999999999}\selectfont \raisebox{\baselineskip}[0pt]{\protect\hypertarget{rLauber1982Thein}{}}{Lauber, Edward.  }{1982.  }\textup{The indigenisation of literacy in Dan (Yacouba).  }\textit{READ }\textup{17}(2). \textup{1-21.  }  \href{http://www.sil.org/resources/archives/23380}{\textcolor[rgb]{0,0,0}{http://www.sil.org/resources/archives/23380}}\par
\vspace{11pt plus 2pt minus 1pt}\hangindent.25in\relax
\hangafter1\relax
\fontsize{12}{14.399999999999999}\selectfont \raisebox{\baselineskip}[0pt]{\protect\hypertarget{rLauber1983Thein}{}}{Lauber, Edward.  }{1983.  }\textup{The indigenisation of literacy in Dan (Yacouba).  }\textit{Notes on Literacy }\textup{37. }\textup{16-21.  }\par
\vspace{11pt plus 2pt minus 1pt}\hangindent.25in\relax
\hangafter1\relax
\fontsize{12}{14.399999999999999}\selectfont \raisebox{\baselineskip}[0pt]{\protect\hypertarget{rLevshinaNatalia2016Verbs}{}}{Levshina, Natalia.  }{2016.  }\textup{Verbs of letting in Germanic and Romance languages: A quantitative investigation based on a parallel corpus of film subtitles.  }\textit{Languages in Contrast }\textup{16}(1). \textup{84-117.  }  doi:\href{http://doai.io/10.1075/lic.16.1.04lev}{10.1075/lic.16.1.04lev}\par
\vspace{11pt plus 2pt minus 1pt}\hangindent.25in\relax
\hangafter1\relax
\fontsize{12}{14.399999999999999}\selectfont \raisebox{\baselineskip}[0pt]{\protect\hypertarget{rLewis2010Asses}{}}{Lewis, Melvyn Paul \& Gary F. Simons.  }{2010.  }\textup{Assessing endangerment: Expanding Fishman's GIDS.  }\textit{Revue Roumaine de Linguistique }\textup{55}(2). \textup{103–120.  }  \href{https://www.lingv.ro/RRL-2010.html}{\textcolor[rgb]{0,0,0}{https://www.lingv.ro/RRL-2010.html}}\par
\vspace{11pt plus 2pt minus 1pt}\hangindent.25in\relax
\hangafter1\relax
\fontsize{12}{14.399999999999999}\selectfont \raisebox{\baselineskip}[0pt]{\protect\hypertarget{rLiao2013Chine}{}}{Liao, Chen \& Pilsung Choe.  }{2013.  }\textup{Chinese Keyboard Layout Design Based on Polyphone Disambiguation and a Genetic Algorithm.  }\textit{International Journal of Human-Computer Interaction }\textup{29}(6). \textup{391-403.  }  doi:\href{http://doai.io/10.1080/10447318.2013.777827}{10.1080/10447318.2013.777827}\par
\vspace{11pt plus 2pt minus 1pt}\hangindent.25in\relax
\hangafter1\relax
\fontsize{12}{14.399999999999999}\selectfont \raisebox{\baselineskip}[0pt]{\protect\hypertarget{rLievrouw2014}{}}{Lievrouw, Leah A.  }{2014.  }Materiality and Media in Communication and Technology Studies: An Unfinished Project.  In Gillespie, Tarleton, Pablo J. Boczkowski \& Kirsten A. Foot, \textit{Media technologies: essays on communication, materiality, and society, }21–51. Cambridge, Massachusetts: The MIT Press.  doi:\href{http://doai.io/10.7551/mitpress/9780262525374.003.0002}{10.7551/mitpress/9780262525374.003.0002}\par
\vspace{11pt plus 2pt minus 1pt}\hangindent.25in\relax
\hangafter1\relax
\fontsize{12}{14.399999999999999}\selectfont \raisebox{\baselineskip}[0pt]{\protect\hypertarget{Light-Anderson}{}}{Light, Lissa W. \& Peter G. Anderson.  }{1993.  }\textup{Typewriter Keyboards via simulated Annealing.  }\textit{AI Expert }\textup{8}(9). \textup{21-7.  }  \href{http://scholarworks.rit.edu/article/727}{\textcolor[rgb]{0,0,0}{http://scholarworks.rit.edu/article/727}}  (11. April 2015.)\par
\vspace{11pt plus 2pt minus 1pt}\hangindent.25in\relax
\hangafter1\relax
\fontsize{12}{14.399999999999999}\selectfont \raisebox{\baselineskip}[0pt]{\protect\hypertarget{rLFCpke}{}}{Lüpke, Friederike.  }{2011.  }Orthography development.  In Peter Austin \& Julia Sallabank, \textit{Handbook of endangered languages, }312-36. Cambridge, UK: Cambridge University Press.  doi:\href{http://doai.io/10.1017/CBO9780511975981.016}{10.1017/CBO9780511975981.016}\par
\vspace{11pt plus 2pt minus 1pt}\hangindent.25in\relax
\hangafter1\relax
\fontsize{12}{14.399999999999999}\selectfont \raisebox{\baselineskip}[0pt]{\protect\hypertarget{NRSIGlossary}{}}{Lyons, Melinda.  }{2001.  }Glossary.  In Lyons, Melinda, \textit{Implementing Writing Systems: an Introduction, }223-32. Dallas, texas: SIL International.  \href{https://scripts.sil.org/Glossary}{\textcolor[rgb]{0,0,0}{https://scripts.sil.org/Glossary}}  (26. October 2018.)\par
\vspace{11pt plus 2pt minus 1pt}\hangindent.25in\relax
\hangafter1\relax
\fontsize{12}{14.399999999999999}\selectfont \raisebox{\baselineskip}[0pt]{\protect\hypertarget{rMacKenzie1992}{}}{MacKenzie, I. Scott.  }{1992.  }\textup{Fitts' law as a research and design tool in human-computer interaction.  }\textit{Human-Computer Interaction }\textup{7. }\textup{91-139.  }\par
\vspace{11pt plus 2pt minus 1pt}\hangindent.25in\relax
\hangafter1\relax
\fontsize{12}{14.399999999999999}\selectfont \raisebox{\baselineskip}[0pt]{\protect\hypertarget{rMalas2008Towar}{}}{Malas, Tareq M., Sinan S. Taifour \& Gheith Ali Abandah.  }{2008.  }Toward Optimal Arabic Keyboard Layout Using Genetic Algorithm.  \textit{Proceedings of 9th Int'l Middle Eastern Multiconference on Simulation and Modeling (MESM 2008), Aug 26-28, Amman, Jordan., }Philadelphia University, Amman, Jordan.    \href{http://www2.ju.edu.jo/sites/Academic/abandah/Lists/Published\%20Research/Attachments/6/Abstract.pdf}{\textcolor[rgb]{0,0,0}{http://www2.ju.edu.jo/sites/Academic/abandah/Lists/Published​\%20Research/Attachments/6/Abstract.pdf}}\par
\vspace{11pt plus 2pt minus 1pt}\hangindent.25in\relax
\hangafter1\relax
\fontsize{12}{14.399999999999999}\selectfont \raisebox{\baselineskip}[0pt]{\protect\hypertarget{rMarinaras1993Desig}{}}{Marinaras, Nicholas \& Kostas Lyritzis.  }{1993.  }\textup{Design of an alternative keyboard layout for the Greek language.  }\textit{International Journal of Human-Computer Interaction }\textup{5}(3). \textup{289-310.  }  doi:\href{http://doai.io/10.1080/10447319309526069}{10.1080/10447319309526069}\par
\vspace{11pt plus 2pt minus 1pt}\hangindent.25in\relax
\hangafter1\relax
\fontsize{12}{14.399999999999999}\selectfont \raisebox{\baselineskip}[0pt]{\protect\hypertarget{rMayzner1965}{}}{Mayzner, Mark S. \& Margaret Elizabeth Tresselt.  }{1965.  }\textup{Tables of single-letter and digram frequency counts for various word-length and letter-position combinations.  }\textit{Psychonomic Monograph Supplements }\textup{1}(2). \textup{13-32.  }  \href{http://psycnet.apa.org/record/1967-02842-001}{\textcolor[rgb]{0,0,0}{http://psycnet.apa.org/record/1967-02842-001}}\par
\vspace{11pt plus 2pt minus 1pt}\hangindent.25in\relax
\hangafter1\relax
\fontsize{12}{14.399999999999999}\selectfont \raisebox{\baselineskip}[0pt]{\protect\hypertarget{rME9aryDavidCatherineCharyRichardPalluelGermainJeanPierreOrliaguet2005Visua}{}}{Méary, David, Catherine Chary, Richard Palluel-Germain, Jean-Pierre Orliaguet.  }{2005.  }\textup{Visual Perception of Writing and Pointing Movements.  }\textit{Perception }\textup{34}(9). \textup{1061-67.  }  doi:\href{http://doai.io/10.1068/p3388}{10.1068/p3388}\par
\vspace{11pt plus 2pt minus 1pt}\hangindent.25in\relax
\hangafter1\relax
\fontsize{12}{14.399999999999999}\selectfont \raisebox{\baselineskip}[0pt]{\protect\hypertarget{rMorrisonRaynerSacca}{}}{Morrison, Robert E. \& Keith Rayner.  }{1981.  }\textup{Saccade size in reading depends upon character spaces and not visual angle.  }\textit{Perception \& Psychophysics }\textup{30}(4). \textup{395-96.  }  doi:\href{http://doai.io/10.3758/BF03206156}{10.3758/BF03206156}\par
\vspace{11pt plus 2pt minus 1pt}\hangindent.25in\relax
\hangafter1\relax
\fontsize{12}{14.399999999999999}\selectfont \raisebox{\baselineskip}[0pt]{\protect\hypertarget{rMustafa2011SMSCo}{}}{Mustafa, Ruba Ahmad Abdullah.  }{2011.  }\textit{SMS Code-switching among Teenagers in Jordan.  }Amman, Jordan: Middle East University dissertation.    \href{http://www.meu.edu.jo/ar/images/research\%20papers/-\%20SMS\%20Code-switching\%20among\%20Teenagers\%20in\%20Jordan.pdf}{\textcolor[rgb]{0,0,0}{http://www.meu.edu.jo/ar/images/research\%20papers/-\%20SMS\%20Code-switching\%20among\%20Teenagers\%20in\%20Jordan.pdf}}\par
\vspace{11pt plus 2pt minus 1pt}\hangindent.25in\relax
\hangafter1\relax
\fontsize{12}{14.399999999999999}\selectfont \raisebox{\baselineskip}[0pt]{\protect\hypertarget{rNorvig2012}{}}{Norvig, Peter.  }{2012.  }English Letter Frequency Counts: Mayzner Revisited or ETAOIN SRHLDCU.    \href{http://norvig.com/mayzner.html}{\textcolor[rgb]{0,0,0}{http://norvig.com/mayzner.html}}  (15. November 2018.)\par
\vspace{11pt plus 2pt minus 1pt}\hangindent.25in\relax
\hangafter1\relax
\fontsize{12}{14.399999999999999}\selectfont \raisebox{\baselineskip}[0pt]{\protect\hypertarget{rD6stlingRobert2014Bayes}{}}{Östling, Robert.  }{2014.  }Bayesian Word Alignment for Massively Parallel Texts.  \textit{Proceedings of the 14th Conference of the European Chapter of the Association for Computational Linguistics.  }123-27.  Gothenburg, Sweden: Association for Computational Linguistics.\par
\vspace{11pt plus 2pt minus 1pt}\hangindent.25in\relax
\hangafter1\relax
\fontsize{12}{14.399999999999999}\selectfont \raisebox{\baselineskip}[0pt]{\protect\hypertarget{rD6stlingRobert2015Bayes}{}}{Östling, Robert.  }{2015.  }\textit{Bayesian Models for Multilingual Word Alignment.  }Stockholm, Sweden: Stockholm University dissertation.    \href{https://www.diva-portal.org/smash/get/diva2:798117/FULLTEXT01.pdf}{\textcolor[rgb]{0,0,0}{https://www.diva-portal.org/smash/get/diva2:798117/FULLTEXT01.pdf}}\par
\vspace{11pt plus 2pt minus 1pt}\hangindent.25in\relax
\hangafter1\relax
\fontsize{12}{14.399999999999999}\selectfont \raisebox{\baselineskip}[0pt]{\protect\hypertarget{rParks1993}{}}{Parkes, M. B.  }{1993.  }\textit{Pause and Effect: An Introduction to the History of Punctuation in the West.  }Berkeley; Los Angeles: University of California Press.\par
\vspace{11pt plus 2pt minus 1pt}\hangindent.25in\relax
\hangafter1\relax
\fontsize{12}{14.399999999999999}\selectfont \raisebox{\baselineskip}[0pt]{\protect\hypertarget{Paterson}{}}{Paterson, Hugh J. III.  }{2014.  }Keyboard layouts: Lessons from the Meꞌphaa and Sochiapam Chinantec designs.  In Mari C. Jones, \textit{Endangered Languages and New Technologies, }49-66. Cambridge, UK: Cambridge University Press.  doi:\href{http://doai.io/10.1017/CBO9781107279063.006}{10.1017/CBO9781107279063.006}\par
\vspace{11pt plus 2pt minus 1pt}\hangindent.25in\relax
\hangafter1\relax
\fontsize{12}{14.399999999999999}\selectfont \raisebox{\baselineskip}[0pt]{\protect\hypertarget{rPaul2008}{}}{Paul, Cristian Kit.  }{2008.  }Romanian diacritic marks: How did we end up looking half-illiterate?    \href{http://kitblog.com/2008/10/romanian\_diacritic\_marks.html}{\textcolor[rgb]{0,0,0}{http://kitblog.com/2008/10/romanian\_diacritic\_marks.html}}  (9. Novermber 2018.)\par
\vspace{11pt plus 2pt minus 1pt}\hangindent.25in\relax
\hangafter1\relax
\fontsize{12}{14.399999999999999}\selectfont \raisebox{\baselineskip}[0pt]{\protect\hypertarget{rPavalanathan}{}}{Pavalanathan, Umashanthi \& Jacob Eisenstein.  }{2016.  }\textit{More \vspace*{0pt}{\XeTeXpicfile "../Resources/smile.png" scaled 70}, less :) The Competition for Paralinguistic Function in Microblog writing. {\XLingPaperCambriaZMathFontFamily{\textit{ƒ i ® s {\XLingPaperDejaVuZSansFontFamily{✝}} m {\XLingPaperCharisZSILFontFamily{¤}} ñ d @ ¥}}}. {\XLingPaperCharisZSILFontFamily{\textup{\textup{\textmd{Vol. 21. Iss. 7-November}}}}}.  }Online: University of Chicago.  doi:\href{http://doai.io/10.5210/fm.v21i11.6879}{10.5210/fm.v21i11.6879}\par
\vspace{11pt plus 2pt minus 1pt}\hangindent.25in\relax
\hangafter1\relax
\fontsize{12}{14.399999999999999}\selectfont \raisebox{\baselineskip}[0pt]{\protect\hypertarget{rPhilips2009Tagsf}{}}{Philips, Addison. \& Mark Davis.  }{2009.  }\textup{Tags for Identifying Languages.  }\textit{Unknown Journal Title }\textup{Unkown Journal Volume.  }Online: Internet Engineering Task Force (IETF).  \href{https://tools.ietf.org/html/bcp47}{\textcolor[rgb]{0,0,0}{https://tools.ietf.org/html/bcp47}}\par
\vspace{11pt plus 2pt minus 1pt}\hangindent.25in\relax
\hangafter1\relax
\fontsize{12}{14.399999999999999}\selectfont \raisebox{\baselineskip}[0pt]{\protect\hypertarget{rPinetSvetlanaJohannesC.ZieglerF.XavierAlario122016Typin}{}}{Pinet, Svetlana, Johannes C. Ziegler \& F.-Xavier Alario.  }{2016.  }\textup{Typing is writing: Linguistic properties modulate typing execution.  }\textit{Psychonomic Bulletin \& Review }\textup{23}(6). \textup{1898-1906.  }  doi:\href{http://doai.io/10.3758/s13423-016-1044-3}{10.3758/s13423-016-1044-3}  \href{http://link.springer.com/10.3758/s13423-016-1044-3}{\textcolor[rgb]{0,0,0}{http://link.springer.com/10.3758/s13423-016-1044-3}}\par
\vspace{11pt plus 2pt minus 1pt}\hangindent.25in\relax
\hangafter1\relax
\fontsize{12}{14.399999999999999}\selectfont \raisebox{\baselineskip}[0pt]{\protect\hypertarget{rRaafi2011}{}}{Raafi, M.A.C.M. \& H. M. Nasir.  }{2011.  }A Transliteration Keyboard Configuration with Tamil Unicode Characters.  In R.K. Choudhary, Manoj Verma \& Sanjeev Saini, \textit{Proceedings of the 5th International Conference on ADVANCED COMPUTING \& COMMUNICATION TECHNOLOGIES \textsquarebracketleft{}ICACCT-2011\textsquarebracketright{}, }135-38. Karnal, India: ABC Group of Publication.  \href{http://www.apiit.edu.in/downloads/all\%20chapters/CHAPTER-29.pdf}{\textcolor[rgb]{0,0,0}{http://www.apiit.edu.in/downloads/all\%20chapters/CHAPTER-29.pdf}}\par
\vspace{11pt plus 2pt minus 1pt}\hangindent.25in\relax
\hangafter1\relax
\fontsize{12}{14.399999999999999}\selectfont \raisebox{\baselineskip}[0pt]{\protect\hypertarget{rRastogiKavita20151231RajiO}{}}{Rastogi, Kavita.  }{2015.  }\textup{Raji Orthography Development.  }\textit{Himalayan Linguistics }\textup{14}(2). \textup{41-48.  }  doi:\href{http://doai.io/10.5070/H914224947}{10.5070/H914224947}\par
\vspace{11pt plus 2pt minus 1pt}\hangindent.25in\relax
\hangafter1\relax
\fontsize{12}{14.399999999999999}\selectfont \raisebox{\baselineskip}[0pt]{\protect\hypertarget{rRehki2017}{}}{Rehki, Sachin.  }{2017.  }The Hierarchy of User Friction.    \href{https://web.archive.org/web/*/https://medium.com/@sachinrekhi/the-hierarchy-of-user-friction-e99113b77d78}{\textcolor[rgb]{0,0,0}{https://web.archive.org/web/*/https://medium.com/@sachinrekhi/the-hierarchy-of-user-friction-e99113b77d78}}  (18. November 2018.)\par
\vspace{11pt plus 2pt minus 1pt}\hangindent.25in\relax
\hangafter1\relax
\fontsize{12}{14.399999999999999}\selectfont \raisebox{\baselineskip}[0pt]{\protect\hypertarget{rResnikPhilipMariBromanOlsenMonaDiab1999TheBi}{}}{Resnik, Philip, Mari Broman Olsen, Mona Diab.  }{1999.  }\textup{The Bible as a Parallel Corpus: Annotating the 'Book of 2000 Tongues'.  }\textit{Computers and the Humanities }\textup{33}(1-2). \textup{129-53.  }  doi:\href{http://doai.io/10.1023/A:1001798929185}{10.1023/A:1001798929185}\par
\vspace{11pt plus 2pt minus 1pt}\hangindent.25in\relax
\hangafter1\relax
\fontsize{12}{14.399999999999999}\selectfont \raisebox{\baselineskip}[0pt]{\protect\hypertarget{rRobertssubmittedChapt}{}}{Roberts, David \& Valentin Vydrin.  }{(submitted).  }Chapter 10: Eastern Dan.  In Roberts, David, \textit{Tone orthography and reading fluency: the voice of evidence in ten Niger-Congo languages. }John Benjamins.\par
\vspace{11pt plus 2pt minus 1pt}\hangindent.25in\relax
\hangafter1\relax
\fontsize{12}{14.399999999999999}\selectfont \raisebox{\baselineskip}[0pt]{\protect\hypertarget{rRobertsn.d.Marki}{}}{Roberts, David, Dana Basnight-Brown \& Valentin Vydrin.  }{(forthcoming).  }Marking tone with punctuation: and orthography experiment in Eastern Dan (Côte d’Ivoire).   Unknown institution, ms.\par
\vspace{11pt plus 2pt minus 1pt}\hangindent.25in\relax
\hangafter1\relax
\fontsize{12}{14.399999999999999}\selectfont \raisebox{\baselineskip}[0pt]{\protect\hypertarget{rSairosse2004Useof}{}}{Sairosse, Tomas Mauta \& Stephen M. Mutula.  }{2004.  }\textup{Use of cybercafés: study of Gaborone City, Botswana.  }\textit{Program: electronic library and information systems }\textup{38}(1). \textup{60-66.  }  doi:\href{http://doai.io/10.1108/00330330410519206}{10.1108/00330330410519206}\par
\vspace{11pt plus 2pt minus 1pt}\hangindent.25in\relax
\hangafter1\relax
\fontsize{12}{14.399999999999999}\selectfont \raisebox{\baselineskip}[0pt]{\protect\hypertarget{rSalvo2016}{}}{Salvo, Jose Miguel R., Christian Jay B. Raagas, Maria Tatjana Claudeene M. Medina \& Alyssa Jean A. Portus.  }{2016.  }Ergonomic Keyboard Layout Designed for the Filipino Language.  In Goonetilleke, Ravindra S. \& Waldemar Karwowski., eds. \textit{Advances in Physical Ergonomics and Human Factors, }Advances in Intelligent Systems and Computing, vol 489, 407-16. Cham: Springer.  doi:\href{http://doai.io/10.1007/978-3-319-41694-6\_41}{10.1007/978-3-319-41694-6\_41}\par
\vspace{11pt plus 2pt minus 1pt}\hangindent.25in\relax
\hangafter1\relax
\fontsize{12}{14.399999999999999}\selectfont \raisebox{\baselineskip}[0pt]{\protect\hypertarget{rSarbapriyaRay2016}{}}{Sarbapriya, Ray.  }{2016.  }\textup{Revisiting the Evolution and Application of Assignment Problem: A Brief Overview.  }\textit{Industrial Engineering Letters }\textup{6}(10). \textup{16-38.  }  \href{https://iiste.org/Journals/index.php/IEL/article/viewFile/34509/35511}{\textcolor[rgb]{0,0,0}{https://iiste.org/Journals/index.php/IEL/article/viewFile/34509/35511}}\par
\vspace{11pt plus 2pt minus 1pt}\hangindent.25in\relax
\hangafter1\relax
\fontsize{12}{14.399999999999999}\selectfont \raisebox{\baselineskip}[0pt]{\protect\hypertarget{rScannell2009Dan}{}}{Scannell, Kevin P.  }{2009.  }Dan.  \textit{An Crúbadán, }Online: Saint Louis University, Saint Louis, USA.  \href{http://crubadan.org/languages/dnj}{\textcolor[rgb]{0,0,0}{http://crubadan.org/languages/dnj}}\par
\vspace{11pt plus 2pt minus 1pt}\hangindent.25in\relax
\hangafter1\relax
\fontsize{12}{14.399999999999999}\selectfont \raisebox{\baselineskip}[0pt]{\protect\hypertarget{rScannell2011}{}}{Scannell, Kevin P.  }{2011.  }\textup{Statistical unicodification of African languages.  }\textit{Language Resources and Evaluation }\textup{45}(3). \textup{375–386.  }  doi:\href{http://doai.io/10.1007/s10579-011-9150-3}{10.1007/s10579-011-9150-3}\par
\vspace{11pt plus 2pt minus 1pt}\hangindent.25in\relax
\hangafter1\relax
\fontsize{12}{14.399999999999999}\selectfont \raisebox{\baselineskip}[0pt]{\protect\hypertarget{rSchneider2010Langu}{}}{Schneider, Cynthia.  }{2010.  }Language Learning and Literacy Development in the Field.  In Treis, Yvonne, De Busser, Rik, \textit{Selected Papers from the 2009 Conference of the Australian Linguistic Society, }1-27. Australian Linguistic Society.  \href{www.als.asn.au/proceedings/als2009/schneider.pdf}{\textcolor[rgb]{0,0,0}{www.als.asn.au/proceedings/als2009/schneider.pdf}}\par
\vspace{11pt plus 2pt minus 1pt}\hangindent.25in\relax
\hangafter1\relax
\fontsize{12}{14.399999999999999}\selectfont \raisebox{\baselineskip}[0pt]{\protect\hypertarget{rSeifartFrank2006Ortho}{}}{Seifart, Frank.  }{2006.  }Orthography development.  In Jost Gippert, Nikolaus Himmelmann \& Ulrike Mosel, \textit{Essentials of language documentation. (Trends in Linguistics. Studies and Monographs \textsquarebracketleft{}TiLSM\textsquarebracketright{} 178), }275-301. Berlin; New York: Mouton de Gruyter.\par
\vspace{11pt plus 2pt minus 1pt}\hangindent.25in\relax
\hangafter1\relax
\fontsize{12}{14.399999999999999}\selectfont \raisebox{\baselineskip}[0pt]{\protect\hypertarget{rSIL2018non-alphabetic}{}}{SIL International.  }{2018.  }Best practice when using non-alphabetic characters in orthographies: Helping languages succeed in the modern world.   SIL International, ms.  \href{https://www.sil.org/orthography/fonts-and-technical-issues}{\textcolor[rgb]{0,0,0}{https://www.sil.org/orthography/fonts-and-technical-issues}}  (17. June 2018.)\par
\vspace{11pt plus 2pt minus 1pt}\hangindent.25in\relax
\hangafter1\relax
\fontsize{12}{14.399999999999999}\selectfont \raisebox{\baselineskip}[0pt]{\protect\hypertarget{rInternational1996Dan}{}}{SIL International.  }{1996.  }Dan.  In Barbara F. Grimes., \textit{Ethnologue: Languages of the world, }13th edn.  247. Dallas, Texas: Summer Institute of Linguistics, Inc.\par
\vspace{11pt plus 2pt minus 1pt}\hangindent.25in\relax
\hangafter1\relax
\fontsize{12}{14.399999999999999}\selectfont \raisebox{\baselineskip}[0pt]{\protect\hypertarget{rKeyman2018}{}}{SIL International.  }{2018a.  }Keyman Training, Day 1: Video from SIL International's Language Technology Vimeo Channel, uploaded March 31, 2018.    \href{https://vimeo.com/262624566}{\textcolor[rgb]{0,0,0}{https://vimeo.com/262624566}}  (18. November 2018.)\par
\vspace{11pt plus 2pt minus 1pt}\hangindent.25in\relax
\hangafter1\relax
\fontsize{12}{14.399999999999999}\selectfont \raisebox{\baselineskip}[0pt]{\protect\hypertarget{rInternational2018Dan}{}}{SIL International.  }{2018b.  }Dan.  In Gary F. Simons \& Charles D. Fennig, \textit{Ethnologue: Languages of the world. }21st edn.  Dallas, Texas: SIL International.  \href{https://www.ethnologue.com/language/dnj}{\textcolor[rgb]{0,0,0}{https://www.ethnologue.com/language/dnj}}\par
\vspace{11pt plus 2pt minus 1pt}\hangindent.25in\relax
\hangafter1\relax
\fontsize{12}{14.399999999999999}\selectfont \raisebox{\baselineskip}[0pt]{\protect\hypertarget{rSILInternationalBolli1982Guide}{}}{SIL International (Bolli, Margrit \& Eva Flik).  }{1982.  }\textit{Guide d'orthographe pour la langue dan (dialecte gwetaawo).  }Abidjan, Ivory Coast: Société Internationale de Linguistique (SIL International).  \href{https://www.sil.org/resources/archives/34713}{\textcolor[rgb]{0,0,0}{https://www.sil.org/resources/archives/34713}}\par
\vspace{11pt plus 2pt minus 1pt}\hangindent.25in\relax
\hangafter1\relax
\fontsize{12}{14.399999999999999}\selectfont \raisebox{\baselineskip}[0pt]{\protect\hypertarget{Silva-Donaghy}{}}{Silva, Kalena \& Keola Donaghy.  }{2004.  }Ke A‘o Ho‘okeleka‘a‘ike: Hawaiian Language Instruction On The Internet.  In Yoshiko Saito-Abbott, Richard Donovan \& Thomas Abbott, \textit{Language on the Edge: Implications for Teaching Foreign Languages and Cultures, Proceedings of Digital Stream 2003 (Emerging Technologies in Teaching Language and Culture IV). }San Diego, California: Montezuma Publishing.\par
\vspace{11pt plus 2pt minus 1pt}\hangindent.25in\relax
\hangafter1\relax
\fontsize{12}{14.399999999999999}\selectfont \raisebox{\baselineskip}[0pt]{\protect\hypertarget{rSimonsAndFennig2018}{}}{Simons, Gary F. \& Charles D. Fennig,  }{Eds.  }{2018.  }\textit{Ethnologue: Languages of the world.  }21st edn.  Dallas, Texas: SIL International.  \href{https://www.ethnologue.com}{\textcolor[rgb]{0,0,0}{https://www.ethnologue.com}}\par
\vspace{11pt plus 2pt minus 1pt}\hangindent.25in\relax
\hangafter1\relax
\fontsize{12}{14.399999999999999}\selectfont \raisebox{\baselineskip}[0pt]{\protect\hypertarget{rSinglerJohnV1990}{}}{Singler, John V.  }{1990.  }\textup{Linguistics and Liberian Languages in the 1970s and 1980s: A Bibliography.  }\textit{Liberian Studies Journal }\textup{15}(1). \textup{108-26.  }  \href{https://scholarworks.iu.edu/journals/index.php/lsj/article/view/4124}{\textcolor[rgb]{0,0,0}{https://scholarworks.iu.edu/journals/index.php/lsj/article/view/4124}}\par
\vspace{11pt plus 2pt minus 1pt}\hangindent.25in\relax
\hangafter1\relax
\fontsize{12}{14.399999999999999}\selectfont \raisebox{\baselineskip}[0pt]{\protect\hypertarget{rSmalley1959}{}}{Smalley, William A.  }{1959.  }\textup{Orthography Conference for French West Africa.  }\textit{The Bible Translator }\textup{10}(4). \textup{181-87.  }  doi:\href{http://doai.io/10.1177/000608445901000407}{10.1177/000608445901000407}\par
\vspace{11pt plus 2pt minus 1pt}\hangindent.25in\relax
\hangafter1\relax
\fontsize{12}{14.399999999999999}\selectfont \raisebox{\baselineskip}[0pt]{\protect\hypertarget{rSmalley1963}{}}{Smalley, William A.  }{1963.  }Orthography Conference for French West Africa.  In Smalley, William A., eds. \textit{Helps for Translators: Orthography Studies, }VI, 120-26. London, England: United Bible Societies.\par
\vspace{11pt plus 2pt minus 1pt}\hangindent.25in\relax
\hangafter1\relax
\fontsize{12}{14.399999999999999}\selectfont \raisebox{\baselineskip}[0pt]{\protect\hypertarget{rSnider2014}{}}{Snider, Keith L.  }{2014.  }\textup{On Establishing Underlying Tonal Contrast.  }\textit{Language Documentation \& Conservation }\textup{8. }\textup{707-37.  }  \href{http://scholarspace.manoa.hawaii.edu/handle/10125/24622}{\textcolor[rgb]{0,0,0}{http://scholarspace.manoa.hawaii.edu/handle/10125/24622}}\par
\vspace{11pt plus 2pt minus 1pt}\hangindent.25in\relax
\hangafter1\relax
\fontsize{12}{14.399999999999999}\selectfont \raisebox{\baselineskip}[0pt]{\protect\hypertarget{rSotoetal2016}{}}{Soto, Ricardo, Broderick Crawford, Boris Almonacid, Stefanie Niklander \& Eduardo Olguín.  }{2016.  }Optimization for UI Design via Metaheuristics.  In Constantine Stephanidis, ed. \textit{HCI International 2016 – Posters' Extended Abstracts, }150-54. Communications in Computer and Information Science 617.  Cham: Springer International Publishing.  \href{https://link.springer.com/chapter/10.1007\%2F978-3-319-40548-3\_25}{\textcolor[rgb]{0,0,0}{https://link.springer.com/chapter/10.1007\%2F978-3-319-40548-3\_25}}  doi:\href{http://doai.io/10.1007/978-3-319-40548-3\_25}{10.1007/978-3-319-40548-3\_25}\par
\vspace{11pt plus 2pt minus 1pt}\hangindent.25in\relax
\hangafter1\relax
\fontsize{12}{14.399999999999999}\selectfont \raisebox{\baselineskip}[0pt]{\protect\hypertarget{Statista}{}}Number of mobile cellular subscriptions per 100 inhabitants in Côte d'Ivoire from 2000 to 2017.  2018.    \href{https://www.statista.com/statistics/502014/mobile-cellular-subscriptions-per-100-inhabitants-in-cote-d-ivoire/}{\textcolor[rgb]{0,0,0}{https://www.statista.com/statistics/502014/mobile-cellular-subscriptions-per-100-inhabitants-in-cote-d-ivoire/}}  (21. October 2018.)\par
\vspace{11pt plus 2pt minus 1pt}\hangindent.25in\relax
\hangafter1\relax
\fontsize{12}{14.399999999999999}\selectfont \raisebox{\baselineskip}[0pt]{\protect\hypertarget{rSteinbergeretal}{}}{Steinberger, Ralf, Mohamed Ebrahim, Alexandros Poulis, Manuel Carrasco-Benitez, Patrick Schlüter, Marek Przybyszewski, Signe Gilbro.  }{2014.  }\textup{An overview of the European Union’s highly multilingual parallel corpora.  }\textit{Language Resources and Evaluation }\textup{48}(4). \textup{679-707.  }  doi:\href{http://doai.io/10.1007/s10579-014-9277-0}{10.1007/s10579-014-9277-0}\par
\vspace{11pt plus 2pt minus 1pt}\hangindent.25in\relax
\hangafter1\relax
\fontsize{12}{14.399999999999999}\selectfont \raisebox{\baselineskip}[0pt]{\protect\hypertarget{rSternstein2008}{}}{Sternstein, Martin.  }{2008.  }\textup{Mathematics and the Dan Culture.  }\textit{The Journal of Mathematics and Culture }\textup{3}(1). \textup{1-13.  }\par
\vspace{11pt plus 2pt minus 1pt}\hangindent.25in\relax
\hangafter1\relax
\fontsize{12}{14.399999999999999}\selectfont \raisebox{\baselineskip}[0pt]{\protect\hypertarget{rSvenskaAkademien2006Svens}{}}{Svenska Akademien.  }{2006.  }\textit{{\textit{Svenska Akademiens ordlista: över svenska språket}}.  }13th edn.  Stockholm: Norstedts Akademiska Förlag.\par
\vspace{11pt plus 2pt minus 1pt}\hangindent.25in\relax
\hangafter1\relax
\fontsize{12}{14.399999999999999}\selectfont \raisebox{\baselineskip}[0pt]{\protect\hypertarget{rSzameitat2009}{}}{Szameitat, André J., Jan Rummel, Diana P. Szameitat, Annette Sterr.  }{2009.  }\textup{Behavioral and emotional consequences of brief delays in human–computer interaction.  }\textit{International Journal of Human-Computer Studies }\textup{67}(7). \textup{561-70.  }  doi:\href{http://doai.io/10.1016/j.ijhcs.2009.02.004}{10.1016/j.ijhcs.2009.02.004}  \href{http://linkinghub.elsevier.com/retrieve/pii/S1071581909000329}{\textcolor[rgb]{0,0,0}{http://linkinghub.elsevier.com/retrieve/pii/S1071581909000329}}\par
\vspace{11pt plus 2pt minus 1pt}\hangindent.25in\relax
\hangafter1\relax
\fontsize{12}{14.399999999999999}\selectfont \raisebox{\baselineskip}[0pt]{\protect\hypertarget{rAlgorithms}{}}{The Linux Information Project.  }{2005.  }Algorithms: A Very Brief Introduction.  The Linux Information Project.    \href{http://www.linfo.org/algorithm.html}{\textcolor[rgb]{0,0,0}{http://www.linfo.org/algorithm.html}}  (4. November 2018.)\par
\vspace{11pt plus 2pt minus 1pt}\hangindent.25in\relax
\hangafter1\relax
\fontsize{12}{14.399999999999999}\selectfont \raisebox{\baselineskip}[0pt]{\protect\hypertarget{rString}{}}{The Linux Information Project.  }{2007.  }String Definition.  The Linux Information Project.    \href{http://www.linfo.org/string.html}{\textcolor[rgb]{0,0,0}{http://www.linfo.org/string.html}}  (4. November 2018.)\par
\vspace{11pt plus 2pt minus 1pt}\hangindent.25in\relax
\hangafter1\relax
\fontsize{12}{14.399999999999999}\selectfont \raisebox{\baselineskip}[0pt]{\protect\hypertarget{rUnicodeGlossary}{}}{The Unicode Consortium.  }{2017.  }\textit{Glossary of Unicode Terms.  }Version 8/18/2017 edn.  Mountain View, California: The Unicode Consortium.  \href{https://www.unicode.org/glossary}{\textcolor[rgb]{0,0,0}{https://www.unicode.org/glossary}}  (8. November 2018.)\par
\vspace{11pt plus 2pt minus 1pt}\hangindent.25in\relax
\hangafter1\relax
\fontsize{12}{14.399999999999999}\selectfont \raisebox{\baselineskip}[0pt]{\protect\hypertarget{rUnicode11Standard}{}}{The Unicode Consortium.  }{2018.  }\textit{The Unicode Standard.  }Version 11.0.0 edn.  Mountain View, California: The Unicode Consortium.  \href{http://www.unicode.org/versions/Unicode11.0.0}{\textcolor[rgb]{0,0,0}{http://www.unicode.org/versions/Unicode11.0.0}}\par
\vspace{11pt plus 2pt minus 1pt}\hangindent.25in\relax
\hangafter1\relax
\fontsize{12}{14.399999999999999}\selectfont \raisebox{\baselineskip}[0pt]{\protect\hypertarget{Thoms}{}}{Thomas, Abbey \& Gary F. Simons.  }{2017.  }Measuring Digital Language Support (Poster).  9th DFW Metroplex Linguistics Conference.  Arlington, TX, 20 October 2017.  \href{https://scholars.sil.org/sites/scholars/files/gary\_f\_simons/poster/digital\_support\_poster.pdf}{\textcolor[rgb]{0,0,0}{https://scholars.sil.org/sites/scholars/files/gary\_f\_simons/​poster/digital\_support\_poster.pdf}}  (24. October 2018.)\par
\vspace{11pt plus 2pt minus 1pt}\hangindent.25in\relax
\hangafter1\relax
\fontsize{12}{14.399999999999999}\selectfont \raisebox{\baselineskip}[0pt]{\protect\hypertarget{Trosterud}{}}{Trosterud, Trond.  }{2012.  }\textup{A restricted freedom of choice: Linguistic diversity in the digital landscape.  }\textit{Nordlyd (Tromsø University Working Papers on Language and Linguistics) }\textup{39}(2). \textup{89-104.  }  \href{http://septentrio.uit.no/index.php/ nordlyd/article/view/2474/0}{\textcolor[rgb]{0,0,0}{http://septentrio.uit.no/index.php/ nordlyd/article/view/2474/0}}\par
\vspace{11pt plus 2pt minus 1pt}\hangindent.25in\relax
\hangafter1\relax
\fontsize{12}{14.399999999999999}\selectfont \raisebox{\baselineskip}[0pt]{\protect\hypertarget{rTwardoch}{}}{Twardoch, Adam.  }{2009.  }Polish Diacritics: how to?    \href{http://www.twardoch.com/download/polishhowto/intro.html}{\textcolor[rgb]{0,0,0}{http://www.twardoch.com/download/polishhowto/intro.html}}  (9. November 2018.)\par
\vspace{11pt plus 2pt minus 1pt}\hangindent.25in\relax
\hangafter1\relax
\fontsize{12}{14.399999999999999}\selectfont \raisebox{\baselineskip}[0pt]{\protect\hypertarget{rUSDHHS-UX}{}}{U.S. Department of Health \& Human Services.  }{2018.  }Interaction Design Basics.    \href{https://www.usability.gov/what-and-why/interaction-design.html}{\textcolor[rgb]{0,0,0}{https://www.usability.gov/what-and-why/interaction-design.html}}  (2. March 2018.)\par
\vspace{11pt plus 2pt minus 1pt}\hangindent.25in\relax
\hangafter1\relax
\fontsize{12}{14.399999999999999}\selectfont \raisebox{\baselineskip}[0pt]{\protect\hypertarget{rUmaoka}{}}{Umaoka, Yoshito \& Others.  }{2018.  }Unicode Technical Standard \#35: Unicode locale data markup language (LDML) part 2: general.  \textit{Unicode Technical Reports, }Revision 34 edn.  Mountain View, California: Unicode Consortium.  \href{https://www.unicode.org/reports/tr35/tr35-general.html}{\textcolor[rgb]{0,0,0}{https://www.unicode.org/reports/tr35/tr35-general.html}}  (10. October 2018.)\par
\vspace{11pt plus 2pt minus 1pt}\hangindent.25in\relax
\hangafter1\relax
\fontsize{12}{14.399999999999999}\selectfont \raisebox{\baselineskip}[0pt]{\protect\hypertarget{Venezky1967}{}}{Venezky, Richard.  }{1967.  }\textup{English Orthography: Its graphical structure and its Relation to sound.  }\textit{Reading Research Quarterly }\textup{2}(3). \textup{75-105.  }  \href{https://www.jstor.org/stable/747031}{\textcolor[rgb]{0,0,0}{https://www.jstor.org/stable/747031}}  doi:\href{http://doai.io/10.2307/747031}{10.2307/747031}\par
\vspace{11pt plus 2pt minus 1pt}\hangindent.25in\relax
\hangafter1\relax
\fontsize{12}{14.399999999999999}\selectfont \raisebox{\baselineskip}[0pt]{\protect\hypertarget{Venezky1970}{}}{Venezky, Richard.  }{1970.  }\textit{The structure of English Orthography.  }(Janua linguarum., Series minor 82.) The Hague: Mouton.  \href{https://www.degruyter.com/view/product/5127}{\textcolor[rgb]{0,0,0}{https://www.degruyter.com/view/product/5127}}\par
\vspace{11pt plus 2pt minus 1pt}\hangindent.25in\relax
\hangafter1\relax
\fontsize{12}{14.399999999999999}\selectfont \raisebox{\baselineskip}[0pt]{\protect\hypertarget{rVitrano-Wilson2018}{}}{Vitrano-Wilson, Seth, Ryan Gehrmann, Carolyn Miller \& Cheung Xaiyavong.  }{2018.  }\textup{Tone marks as vowel diacritics in two scripts: repurposing tone marks for non-tonal phenomena in Cado and other Southeast Asian languages.  }\textit{Writing Systems Research }\textup{10}(1). \textup{43–67.  }  doi:\href{http://doai.io/10.1080/17586801.2018.1493409}{10.1080/17586801.2018.1493409}\par
\vspace{11pt plus 2pt minus 1pt}\hangindent.25in\relax
\hangafter1\relax
\fontsize{12}{14.399999999999999}\selectfont \raisebox{\baselineskip}[0pt]{\protect\hypertarget{Vits153}{}}{Vitsœ.  }{2012.  }Dieter Rams: Ten principles for good design.    \href{http://www.vitsoe.com/en/gb/about/dieterrams/gooddesign}{\textcolor[rgb]{0,0,0}{http://www.vitsoe.com/en/gb/about/dieterrams/gooddesign}}  (27 June 2012.)\par
\vspace{11pt plus 2pt minus 1pt}\hangindent.25in\relax
\hangafter1\relax
\fontsize{12}{14.399999999999999}\selectfont \raisebox{\baselineskip}[0pt]{\protect\hypertarget{rVoldLexander2011Texti}{}}{Vold Lexander, Kristin.  }{2011.  }\textup{Texting and African language literacy.  }\textit{New Media \& Society }\textup{13}(3). \textup{427-43.  }  doi:\href{http://doai.io/10.1177/1461444810393905}{10.1177/1461444810393905}\par
\vspace{11pt plus 2pt minus 1pt}\hangindent.25in\relax
\hangafter1\relax
\fontsize{12}{14.399999999999999}\selectfont \raisebox{\baselineskip}[0pt]{\protect\hypertarget{rVydrindnjECorpus}{}}{Vydrin, Valentin.  }{2018.  }Corpus dan de l'Est.    \href{http://cormand.huma-num.fr/dan/}{\textcolor[rgb]{0,0,0}{http://cormand.huma-num.fr/dan/}}  (2. December 2018.)\par
\vspace{11pt plus 2pt minus 1pt}\hangindent.25in\relax
\hangafter1\relax
\fontsize{12}{14.399999999999999}\selectfont \raisebox{\baselineskip}[0pt]{\protect\hypertarget{rVydrin2010Lepie}{}}{Vydrin, Valentin.  }{2010.  }{\textit{Le pied métrique dans les langues mandé}} \textsquarebracketleft{}The metrical foot in Mande languages\textsquarebracketright{}.  In Franck Floricic, \textit{Essais de typologie et de linguistique générale: Mélanges offerts à Denis Creissels, }53-62. Lyon, France: ENS Éditions.\par
\vspace{11pt plus 2pt minus 1pt}\hangindent.25in\relax
\hangafter1\relax
\fontsize{12}{14.399999999999999}\selectfont \raisebox{\baselineskip}[0pt]{\protect\hypertarget{rVydrin2012ISO63}{}}{Vydrin, Valentin.  }{2012.  }\textit{ISO 639-3 Change Request 2012-083.  }(Approved.) Online: ISO 639-3 Registrar (SIL International).  \href{https://iso639-3.sil.org/request/2012-083}{\textcolor[rgb]{0,0,0}{https://iso639-3.sil.org/request/2012-083}}\par
\vspace{11pt plus 2pt minus 1pt}\hangindent.25in\relax
\hangafter1\relax
\fontsize{12}{14.399999999999999}\selectfont \raisebox{\baselineskip}[0pt]{\protect\hypertarget{rVydrin2016Tonal}{}}{Vydrin, Valentin.  }{2016.  }Tonal inflection in Mande languages: the cases of Bamana and Dan-Gwɛɛtaa.  In Enrique L. Palancar \& Jean Léo Léonard, eds. \textit{Tone and Inflection: New facts and new perspectives, }83-105. Trends in Linguistics. Studies and Monographs \textsquarebracketleft{}TiLSM\textsquarebracketright{} 296.  Berlin, Boston: De Gruyter — Mouton.  doi:\href{http://doai.io/10.1515/9783110452754-005}{10.1515/9783110452754-005}\par
\vspace{11pt plus 2pt minus 1pt}\hangindent.25in\relax
\hangafter1\relax
\fontsize{12}{14.399999999999999}\selectfont \raisebox{\baselineskip}[0pt]{\protect\hypertarget{rVandRChapter}{}}{Vydrin, Valentin \& David Roberts.  }{(submitted).  }Tonal oral reading errors in the orthography of Eastern Dan ({\textit{Côte d’Ivoire}}).  In David Roberts, \textit{Tone orthography and reading fluency: the voice of evidence in ten Niger-Congo languages. }John Benjamins.\par
\vspace{11pt plus 2pt minus 1pt}\hangindent.25in\relax
\hangafter1\relax
\fontsize{12}{14.399999999999999}\selectfont \raisebox{\baselineskip}[0pt]{\protect\hypertarget{rVydrinValentin200842F41742B41A414}{}}{Vydrin, Valentin \textsquarebracketleft{}Выдрин, Валентин Феодосьевич\textsquarebracketright{}.  }{2008.  }{\textit{ЯЗЫК ДАН}}.  Санкт-Петербург, Википедия \textsquarebracketleft{}St. Petersburg, Russia\textsquarebracketright{}: Кунсткамера \textsquarebracketleft{}Kunst Kamera\textsquarebracketright{}, ms.  \href{http://mandelang.kunstkamera.ru/files/mandelang/dan.pdf}{\textcolor[rgb]{0,0,0}{http://mandelang.kunstkamera.ru/files/mandelang/dan.pdf}}  (5. December 2018.)\par
\vspace{11pt plus 2pt minus 1pt}\hangindent.25in\relax
\hangafter1\relax
\fontsize{12}{14.399999999999999}\selectfont \raisebox{\baselineskip}[0pt]{\protect\hypertarget{rVydrin2008EDDictionary}{}}{Vydrine \textsquarebracketleft{}Vydrin\textsquarebracketright{}, Valentin \& Mongnan Alphonse Kességbeu.  }{2008.  }\textit{{\textit{Dictionnaire Dan-Français (dan de l'Est) avec une esquisse de grammaire du dan de l'Est et un index français-dan}}.  }1ère edn.  St Pétersbourg, Russia: Musée d'anthropologie et d'ethnographie, Académie des sciences de la Russie — Nestor-Istoria.  \href{https://halshs.archives-ouvertes.fr/halshs-00715560}{\textcolor[rgb]{0,0,0}{https://halshs.archives-ouvertes.fr/halshs-00715560}}\par
\vspace{11pt plus 2pt minus 1pt}\hangindent.25in\relax
\hangafter1\relax
\fontsize{12}{14.399999999999999}\selectfont \raisebox{\baselineskip}[0pt]{\protect\hypertarget{rVydrine2003Map}{}}{Vydrine \textsquarebracketleft{}Vydrin\textsquarebracketright{}, Valentin, Ted G. Bergman \& Matthew Benjamin.  }{2003.  }\textup{Mandé Language Family of West Africa: Location and Genetic Classification.  }\textit{SIL Electronic Survey Reports (SILER) }\textup{2000-003.  }SIL International.  \href{https://www.sil.org/resources/archives/9019}{\textcolor[rgb]{0,0,0}{https://www.sil.org/resources/archives/9019}}\par
\vspace{11pt plus 2pt minus 1pt}\hangindent.25in\relax
\hangafter1\relax
\fontsize{12}{14.399999999999999}\selectfont \raisebox{\baselineskip}[0pt]{\protect\hypertarget{rV12Btoli1461612011}{}}{Vītoliņš, Valdis.  }{2011.  }Modernized Latvian Ergonomic Keyboard.  In  Zenija Roja, Henrijs Kalkis \& Valdis Kalkis (eds), \textit{Contemporary Ergonomics and Business 2011: Proceedings of the First International Scientific-Practical Conference of the Latvian Ergonomics Society.  }21-27.  Riga, Latvia: University of Latvia Press.  \href{https://arxiv.org/pdf/1707.03753.pdf}{\textcolor[rgb]{0,0,0}{https://arxiv.org/pdf/1707.03753.pdf}}\par
\vspace{11pt plus 2pt minus 1pt}\hangindent.25in\relax
\hangafter1\relax
\fontsize{12}{14.399999999999999}\selectfont \raisebox{\baselineskip}[0pt]{\protect\hypertarget{rHTML5}{}}{W3C.  }{2017.  }\textit{HTML5. Recommendation.  }Online: World Wide Web Consortium.  \href{https://www.w3.org/TR/html5}{\textcolor[rgb]{0,0,0}{https://www.w3.org/TR/html5}}  (1. December 2018.)\par
\vspace{11pt plus 2pt minus 1pt}\hangindent.25in\relax
\hangafter1\relax
\fontsize{12}{14.399999999999999}\selectfont \raisebox{\baselineskip}[0pt]{\protect\hypertarget{rWagner2003}{}}{Wagner, Marc Oliver, Bernard Yannou, Steffen Kehl, Dominique Feillet \& Jan Eggers.  }{2003.  }\textup{Ergonomic Modeling and Optimization of the Keyboard Arrangement with an Ant Colony Optimization Algorithm.  }\textit{Journal of Engineering Design }\textup{14}(2). \textup{187-208.  }  doi:\href{http://doai.io/10.1080/0954482031000091509}{10.1080/0954482031000091509}\par
\vspace{11pt plus 2pt minus 1pt}\hangindent.25in\relax
\hangafter1\relax
\fontsize{12}{14.399999999999999}\selectfont \raisebox{\baselineskip}[0pt]{\protect\hypertarget{rWajcmanJones2012}{}}{Wajcman, Judy \& Paul K. Jones.  }{2012.  }\textup{Border communication: media sociology and STS.  }\textit{Media, Culture \& Society }\textup{34}(6). \textup{673-90.  }  doi:\href{http://doai.io/10.1177/0163443712449496}{10.1177/0163443712449496}\par
\vspace{11pt plus 2pt minus 1pt}\hangindent.25in\relax
\hangafter1\relax
\fontsize{12}{14.399999999999999}\selectfont \raisebox{\baselineskip}[0pt]{\protect\hypertarget{rWalker2003Evolv}{}}{Walker, Christopher P.  }{2003.  }Evolving a More Optimal Keyboard.  Missouri University of Science and Technology: Missouri University of Science and Technology, ms.  \href{http://web.mst.edu/~tauritzd/courses/ec/fs2003/project/Walker.pdf}{\textcolor[rgb]{0,0,0}{http://web.mst.edu/\textasciitilde{}tauritzd/courses/ec/fs2003/project/Walker.pdf}}  (26. November 2018.)\par
\vspace{11pt plus 2pt minus 1pt}\hangindent.25in\relax
\hangafter1\relax
\fontsize{12}{14.399999999999999}\selectfont \raisebox{\baselineskip}[0pt]{\protect\hypertarget{rWarschauer2002Langu}{}}{Warschauer, Mark, Ghada R. El Said \& Ayman G. Zohry.  }{2002.  }\textup{Language Choice Online: Globalization and Identity in Egypt.  }\textit{Journal of Computer-Mediated Communication }\textup{7}(4). \textup{0-0.  }Blackwell Publishing Ltd.  doi:\href{http://doai.io/10.1111/j.1083-6101.2002.tb00157.x}{10.1111/j.1083-6101.2002.tb00157.x}\par
\vspace{11pt plus 2pt minus 1pt}\hangindent.25in\relax
\hangafter1\relax
\fontsize{12}{14.399999999999999}\selectfont \raisebox{\baselineskip}[0pt]{\protect\hypertarget{rWeberetal}{}}{Weber, Diana Dahlin, Diane Wroge \& Joan Bomberger Yoder.  }{2007.  }\textup{Writers’ Workshops: A Strategy for Developing Indigenous Writers.  }\textit{Language Documentation \& Conservation }\textup{1}(1). \textup{77-93.  }  \href{http://hdl.handle.net/10125/1728}{\textcolor[rgb]{0,0,0}{http://hdl.handle.net/10125/1728}}\par
\vspace{11pt plus 2pt minus 1pt}\hangindent.25in\relax
\hangafter1\relax
\fontsize{12}{14.399999999999999}\selectfont \raisebox{\baselineskip}[0pt]{\protect\hypertarget{rWhistler}{}}{Whistler, Ken \& Asmus Freytag.  }{2015.  }Unicode Standard Annex \#23: Unicode The Character Property Model.  \textit{Unicode Technical Reports, }Revision 11 edn.  Mountain View, California: Unicode Consortium.  \href{http://www.unicode.org/reports/tr23}{\textcolor[rgb]{0,0,0}{http://www.unicode.org/reports/tr23}}  (27. May 2015.)\par
\vspace{11pt plus 2pt minus 1pt}\hangindent.25in\relax
\hangafter1\relax
\fontsize{12}{14.399999999999999}\selectfont \raisebox{\baselineskip}[0pt]{\protect\hypertarget{rWikimediaISO9995}{}}{Wikimedia Commons contributors (User Karl432).  }{2012.  }File:Keyboard-sections-zones-grid-ISOIEC-9995-1.jpg.  \textit{Wikimedia Commons, the free media repository, }Online: Wikimedia Foundation.  \href{https://commons.wikimedia.org/w/index.php?title=File:Keyboard-sections-zones-grid-ISOIEC-9995-1.jpg\&oldid=235940156}{\textcolor[rgb]{0,0,0}{https://commons.wikimedia.org/w/index.php?title=File:Keyboard-sections-zones-grid-ISOIEC-9995-1.jpg\&oldid=235940156}}  (1. November 2018.)\par
\vspace{11pt plus 2pt minus 1pt}\hangindent.25in\relax
\hangafter1\relax
\fontsize{12}{14.399999999999999}\selectfont \raisebox{\baselineskip}[0pt]{\protect\hypertarget{rWikiNumeroSign}{}}{Wikipedia contributors.  }{2018.  }Numero sign.  \textit{Wikipedia, The Free Encyclopedia, }Online: Wikimedia Foundation.  \href{https://en.wikipedia.org/w/index.php?title=Numero\_sign\&oldid=842034015}{\textcolor[rgb]{0,0,0}{https://en.wikipedia.org/w/index.php?title=Numero\_sign\&oldid=842034015}}  (27. November 2018.)\par
\vspace{11pt plus 2pt minus 1pt}\hangindent.25in\relax
\hangafter1\relax
\fontsize{12}{14.399999999999999}\selectfont \raisebox{\baselineskip}[0pt]{\protect\hypertarget{rWandT2016}{}}{Wołosik, Michał, \& Marek Tabędzki.  }{2016.  }\textup{Screen keyboard arrangement optimization for polish language \textsquarebracketleft{}{\textit{Optymalizacja układu klawiatury ekranowej dla języka polskiego}}\textsquarebracketright{}.  }\textit{Advances in Computer Science Research }\textup{13. }\textup{75-93.  }  \href{http://pbc.biaman.pl/Content/47610/Advances\%20in\%20Computer\%20Science\%20Research\%20Nr\%2013\%20(2017).pdf}{\textcolor[rgb]{0,0,0}{http://pbc.biaman.pl/Content/47610/Advances\%20in\%20Computer​\%20Science\%20Research\%20Nr\%2013\%20(2017).pdf}}\par
\vspace{11pt plus 2pt minus 1pt}\hangindent.25in\relax
\hangafter1\relax
\fontsize{12}{14.399999999999999}\selectfont \raisebox{\baselineskip}[0pt]{\protect\hypertarget{rYacob2016}{}}{Yacob, Daniel \& Richard Ishida.  }{2016.  }\textit{Ethiopic Layout Requirements.  }(W3C Working Draft 15 November 2016.) www.w3.org: W3C.  \href{https://www.w3.org/TR/2016/WD-elreq-20161115}{\textcolor[rgb]{0,0,0}{https://www.w3.org/TR/2016/WD-elreq-20161115}}\par
\vspace{11pt plus 2pt minus 1pt}\hangindent.25in\relax
\hangafter1\relax
\fontsize{12}{14.399999999999999}\selectfont \raisebox{\baselineskip}[0pt]{\protect\hypertarget{rYepes2013}{}}{Yepes, Antonio Jimeno, Elise Prieur-Gaston \& Aurélie Névéol.  }{2013.  }\textup{Combining MEDLINE and publisher data to create parallel corpora for the automatic translation of biomedical text.  }\textit{BMC bioinformatics }\textup{14. }\textup{146-46.  }  doi:\href{http://doai.io/10.1186/1471-2105-14-146}{10.1186/1471-2105-14-146}\par
\vspace{11pt plus 2pt minus 1pt}\hangindent.25in\relax
\hangafter1\relax
\fontsize{12}{14.399999999999999}\selectfont \raisebox{\baselineskip}[0pt]{\protect\hypertarget{rYinAndSu}{}}{Yin, Peng-Yeng \& En-Ping Su.  }{2011.  }\textup{Cyber Swarm optimization for general keyboard arrangement problem.  }\textit{International Journal of Industrial Ergonomics }\textup{41}(1). \textup{43-52.  }  doi:\href{http://doai.io/10.1016/j.ergon.2010.11.007}{10.1016/j.ergon.2010.11.007}\par
\vspace{11pt plus 2pt minus 1pt}\hangindent.25in\relax
\hangafter1\relax
\fontsize{12}{14.399999999999999}\selectfont \raisebox{\baselineskip}[0pt]{\protect\hypertarget{rYinetAl2010}{}}{Yin, Peng-Yeng, Fred Glover, Manuel Laguna \& Jia-Xian Zhu.  }{2010.  }\textup{Cyber swarm algorithms – improving particle swarm optimization using adaptive memory strategies.  }\textit{European Journal of Operational Research }\textup{201}(2). \textup{377–389.  }  doi:\href{http://doai.io/10.1016/j.ejor.2009.03.035}{10.1016/j.ejor.2009.03.035}\par
\vspace{11pt plus 2pt minus 1pt}\hangindent.25in\relax
\hangafter1\relax
\fontsize{12}{14.399999999999999}\selectfont \raisebox{\baselineskip}[0pt]{\protect\hypertarget{rZecevic2000Aneva}{}}{Zecevic, A., D. I. Miller \& K. Harburn.  }{2000.  }\textup{An evaluation of the ergonomics of three computer keyboards.  }\textit{Ergonomics }\textup{43}(1). \textup{55-72.  }  doi:\href{http://doai.io/10.1080/001401300184666}{10.1080/001401300184666}\par
\vspace{11pt plus 2pt minus 1pt}\hangindent.25in\relax
\hangafter1\relax
\fontsize{12}{14.399999999999999}\selectfont \raisebox{\baselineskip}[0pt]{\protect\hypertarget{rZhang2014}{}}{Zhang, Shikun, Wang Ling \& Chris Dyer.  }{2014.  }Dual Subtitles as Parallel Corpora.  \textit{Proceedings of the Ninth International Conference on Language Resources and Evaluation (LREC-2014).  }Reykjavik, Iceland.  \par
\vspace{11pt plus 2pt minus 1pt}\hangindent.25in\relax
\hangafter1\relax
\fontsize{12}{14.399999999999999}\selectfont \raisebox{\baselineskip}[0pt]{\protect\hypertarget{Zhozhikov-et-al}{}}{Zhozhikov, Anatoli, Yakov Aleksandrov \& Alexander Varlamov.  }{‎2011.  }The Type Fonts of the Yakut Alphabet and Those of the Minority Peoples Residing in the Republic of Sakha (Yakutia): Challenges of Applying in Operating Systems.  In Kuzmin, Evgeny, Ekaterina Plys \& Anastasia Parshakova, \textit{Linguistic and Cultural Diversity in Cyberspace. Proceedings of the International Conference (Yakutsk, Russian Federation, 2-4 July, 2008), }250-3. Moscow: Russia: Interregional Library Cooperation Centre.\par
\vspace{11pt plus 2pt minus 1pt}\hangindent.25in\relax
\hangafter1\relax
\fontsize{12}{14.399999999999999}\selectfont \raisebox{\baselineskip}[0pt]{\protect\hypertarget{r62A62763164A62E62F63164A62764162An.d.64A64368664A646}{}}{تاريخ دريافت, تاريخ پذيرش.  }{1394 (2015).  }\textup{{\XLingPaperLateefFontFamily{\textit{تكاملي استراتژي از استفاده با كليد صفحه برروي انگليسي حروف براي جديد چينش يك}}}{\textit{\textsquarebracketleft{}A New Layout for English Letters on the Keyboard Using Evolutionary Strategy\textsquarebracketright{}}}.  }\textit{JIPET }\textup{6}(23). \textup{21-28.  }  \href{http://jipet.iaun.ac.ir/article\_12477.html}{\textcolor[rgb]{0,0,0}{http://jipet.iaun.ac.ir/article\_12477.html}}\par
\vspace{11pt plus 2pt minus 1pt}}\fontsize{11}{13.2}\selectfont \clearpage\XLingPaperendtableofcontents
\pagebreak\end{MainFont}
\end{document}
